\documentclass[12pt]{article}

% Import preambles and macros for homework
% Essential packages
\usepackage{amsmath, amsfonts, amssymb, amsthm}
\usepackage{mathtools}
\usepackage{enumitem}
\usepackage{graphicx}
\usepackage{wrapfig}
\usepackage{systeme}
\usepackage{caption}
\usepackage{soul}
\usepackage[dvipsnames]{xcolor}
\usepackage{fancyhdr}
\allowdisplaybreaks

% Page layout
\usepackage[
  top=2cm,
  bottom=2cm,
  left=2cm,
  right=2cm,
  headheight=17pt,
  includehead,includefoot,
  heightrounded,
]{geometry}


% pgfornament for title page decorations
\usepackage[object=vectorian]{pgfornament}

% Fancy header/footer setup
\pagestyle{fancy}
\setlength{\headheight}{14.49998pt}
\addtolength{\topmargin}{-2.49998pt}
\renewcommand{\footrulewidth}{0.4pt}
\setlength\parindent{15pt}
% Math notation shortcuts
\newcommand{\R}{\mathbb{R}}
\newcommand{\Q}{\mathbb{Q}}
\newcommand{\Z}{\mathbb{Z}}
\newcommand{\N}{\mathbb{N}}
\newcommand{\C}{\mathbb{C}}
\newcommand{\X}{\mathcal{X}}

% Theorem environments
\newtheorem{mainthm}{Theorem}[section]
\newtheorem{theorem}{Theorem}[section]  
\newtheorem{lemma}[theorem]{Lemma}
\newtheorem{proposition}[theorem]{Proposition}
\newtheorem{corollary}[theorem]{Corollary}
\newtheorem{definition}[theorem]{Definition}
\newtheorem{claim}[theorem]{Claim}

% Calculus
\newcommand{\diff}{\mathop{}\!\mathrm{d}}
\newcommand{\deriv}[2]{\frac{\mathrm{d}#1}{\mathrm{d}#2}}
\newcommand{\pderiv}[2]{\frac{\partial #1}{\partial #2}}

% Linear Algebra
\newcommand{\inner}[2]{\langle #1, #2 \rangle}
\newcommand{\norm}[1]{\| #1 \|}
\newcommand{\tr}{\operatorname{tr}}
\newcommand{\spn}{\operatorname{span}}
\newcommand{\rank}{\operatorname{rank}}
\newcommand{\nullity}{\operatorname{nullity}}

% Logic
\newcommand{\contra}{\Rightarrow\Leftarrow}

% Custom commands for notes
\newcommand{\todo}[1]{\textcolor{red}{[TODO: #1]}}
\newcommand{\important}[1]{\textbf{\textcolor{blue}{#1}}}

%Number Theory
\DeclareMathOperator{\Li}{Li}
\newcommand{\floor}[1]{\left\lfloor #1 \right\rfloor}
\newcommand{\fract}[1]{\left\{ #1 \right\}}




\newcommand{\maketitlepage}{
    \begin{titlepage}
        \centering
        \vspace*{2.0cm}
        \pgfornament{84}\\
        {\LARGE \textsc{\coursename}\par}
        \vspace{0.5cm}
        {\large\coursecode\par}
        \vspace{0.5cm}
        {\large\instructor\par}
        \vspace{1.5cm}
        {\huge\bfseries\assignment\par}
        \vspace{1cm}
        {\LARGE\itshape\author\par}
        \vspace{2cm}
        {\large\bfseries Due Date:\par}
        \vspace{0.5cm}
        {\Large \duedate}\\
        \pgfornament{84}
    \end{titlepage}
}
% =============================================
% HOMEWORK CONFIGURATION - EDIT THESE VALUES!
% =============================================

% Your personal info
\renewcommand{\author}{Deepak Jassal}
\newcommand{\authorlast}{Jassal}

% Course info
\newcommand{\coursename}{Course Name}
\newcommand{\coursecode}{Course code}
\newcommand{\instructor}{Instructor}

% Assignment-specific info (CHANGE THESE FOR EACH HOMEWORK)
\newcommand{\assignment}{Assignment }
\newcommand{\duedate}{Month Day\textsuperscript{th}, 20XX}

% Header configuration
\fancyhead[l]{\assignment}
\fancyhead[c]{\coursecode}
\fancyhead[r]{\monthyear}
\fancyfoot[c]{\authorlast{ }\thepage}

\renewcommand{\author}{Deepak Jassal}
\renewcommand{\authorlast}{Jassal}
\renewcommand{\coursename}{Multivariable Calculus I}
\renewcommand{\coursecode}{MATH 202}
\renewcommand{\assignment}{Assignment 5}
\renewcommand{\instructor}{Dr. Stanley Yao Xiao}
\renewcommand{\duedate}{November 15\textsuperscript{th}, 2025}

\begin{document}
\begin{titlepage}
	\centering
	\vspace*{2.0cm}	
	\pgfornament{84}\\
	{\LARGE \textsc{\coursename}\par}
	\vspace{0.5cm}
	{\large\coursecode\par}
    \vspace{0.5cm}
    {\large\instructor\par}
	\vspace{1.5cm}
	{\huge\bfseries\assignment\par}
	\vspace{1cm}
	{\LARGE\itshape\author\par}
    \vspace{2cm}
	{\large\bfseries Due Date:\par}
	\vspace{0.5cm}
	{\Large \duedate}\\
	\pgfornament{84}
\end{titlepage}
\stepcounter{section}
\section*{Question 1}
Compute the volume of a right cylindrical cone where the base radius is \( R \) and the height of the cone is \( h \), using cylindrical coordinates.\\
The Jacobian determinant for cylindrical coordinates has already been computed in class, it is $r$. The triple integral for the volume of the right cylindrical cone $\mathcal{C}$ is 
\[
	V=\iiint_\mathcal{C}1\,dV.
\]
Converting to cylindrical coordinates 
\[
	(x,y,z)\rightarrow (r\cos\theta,r\sin\theta,z)
\]
yields
\[
	V=\int_{0}^{2\pi}\int_{0}^{h}\int_{0}^{\frac{R}{h}z}r\,dr\,dz\,d\theta.
\]
The bounds for the integral in $r$ are obtained by considering the slant of the boundary of the cone from $(0,0,0)$ to a point on the upper edge which would be $\frac{R}{h}z$. Evaluating the integral
\[
	V=\int_{0}^{2\pi}\int_{0}^{h}\int_{0}^{\frac{R}{h}z}r\,dr\,dz\,d\theta=2\pi\int_{0}^{h}\int_{0}^{\frac{R}{h}z}r\,dr\,dz=2\pi\int_{0}^{h}\left[ \frac{r^2}{2} \right]_{0}^{\frac{R}{h}z}\,dz=\pi\frac{R^2}{h^2}\int_{0}^{h}z^2\,dz=\frac{\pi\frac{R^2}{h^2}}{3}\left[ z^3 \right]_0^h
\]
\[
	V=\frac{\pi hR^2}{3}.
\]
\stepcounter{section}
\section*{Question 2}
A spherical cap is the piece of a sphere (ball) sliced off by a plane. Suppose that a sphere has radius 10, and the height of the spherical cap is 6. Determine the volume of the spherical cap.\\
It can be seen that the volume of half a sphere is
\[
	V_{\text{half sphere}}=V_{\text{cap}}+\frac{1}{2}V_{\text{cylinder}}+\frac{1}{2}V_{\text{napkin ring}}.
\]
Rearranging we get 
\[
	V_{\text{cap}}=V_{sphere}-\frac{1}{2}V_{\text{cylinder}}-\frac{1}{2}V_{\text{napkin ring}}.
\]
\[
	V_{\text{half sphere}}=\frac{2}{3}\pi R^3=\frac{2000\pi}{3},
\]
\[
	\frac{1}{2}V_{\text{cylinder}}=\frac{1}{2}\pi r^2h=\pi(\sqrt{10^2-4^2}^2)(6)=\frac{1}{2}\pi(\sqrt{84}^2)(8)=336\pi,
\]
\[
	V_{\text{napkin ring}}=\frac{\pi h^3}{3}=\frac{128\pi}{3}.
\]
Putting all of these together
\[
	V_{\text{cap}}=\frac{2000\pi}{3}-336\pi-\frac{128\pi}{3}=288\pi.
\]
\stepcounter{section}
\section*{Question 3}
A client spends \( X \) minutes in an insurance agent's waiting room and \( Y \) minutes meeting with the agent (That is, both \( X, Y \) are random variables). The joint probability density function of \( X \) and \( Y \) can be modelled by

\[
f(x, y) = 
\begin{cases} 
ke^{-\frac{x+2y}{40}} & \text{for } x > 0, y > 0 \\
0 & \text{otherwise}.
\end{cases}
\]

Determine the value of \( k \) and the probability that a client spends less than 60 minutes at the agent's office.\\
\textit{Solution.}
\[
	\iint_{\R^2}f(x,y)\,dA=\int_{-\infty}^{\infty}\int_{-\infty}^{\infty}ke^{-\frac{x+2y}{40}}\,dy\,dx=k\int_{-\infty}^{\infty}e^{\frac{-x}{40}}\int_{-\infty}^{\infty}e^{\frac{-y}{20}}\,dy\,dx
\]
Since the probability density function is non-zero only for non-zero values of $x$ and $y$ we can further simplify the double integral
\[
	k\int_{0}^{\infty}e^{\frac{-x}{40}}\int_{0}^{\infty}e^{\frac{-y}{20}}\,dy\,dx
\]
\[
	u=\frac{y}{20},\: 20\,du=dy,\: y=0\Rightarrow u=0,\: y=\infty \Rightarrow u=\infty
\]
\[
	k\int_{0}^{\infty}e^{\frac{-x}{40}}\int_{0}^{\infty}20e^{-u}\,du\,dx=k\int_{0}^{\infty}e^{\frac{-x}{40}}\left[ 20e^{-u} \right]^\infty_0\,dx=-20k\int_{0}^{\infty}e^{\frac{-x}{40}}\,dx
\]
\[
	v=\frac{x}{40},\: 40\,dv=dx,\: x=0\Rightarrow v=0,\: x=\infty \Rightarrow v=\infty
\]
\[
	-20k\int_{0}^{\infty}40e^{-v}\,dv=800k\left[ e^{-v} \right]^\infty_0=800k
\]
We want the total probability to be 1, so we have 
\[
	k=\frac{1}{800}.
\]
We want to find $\mathcal{P}(X+Y<60)$, which can be done by evaluating the following integral
\[
	\mathcal{P}(X+Y<60)=\frac{1}{800}\int_{X=0}^{60}\int_{Y=0}^{60-x}e^{-\frac{x+2y}{40}}\,dy\,dx.
\]
Using the same substitutions as the previous part we get
\[
	\frac{1}{800}\int_{0}^{60}e^{\frac{-x}{40}}\int_{0}^{60-x}e^{\frac{-y}{20}}\,dy\,dx=\frac{1}{800}\int_{0}^{60}e^{\frac{-x}{40}}\left[ -20e^{-u} \right]^{\frac{60-x}{20}}_0\,dx=\frac{1}{800}\int_{0}^{60}e^{\frac{-x}{40}}\left(20 - 20e^{\frac{x-60}{20}}  \right)\,dx
\]
Factoring and multiplying through
\[
	\frac{1}{40}\int_{0}^{60} e^{\frac{-x}{40}} -e^{-3}e^{\frac{x}{40}}\,dx=\frac{1}{40}\left[  \int_{0}^{60}e^{\frac{-x}{40}} - e^{-3}\int_{0}^{60}e^{\frac{x}{40}} \right]
\]
\[
	\frac{1}{40}\left[  -40\left[ e^{\frac{-x}{40}} \right]_0^{60} - 40e^{-3}\left[ e^{\frac{x}{40}} \right]_0^{60} \right]=\left[ -\left[ e^{\frac{-x}{40}} \right]_0^{60} - e^{-3}\left[ e^{\frac{x}{40}} \right]_0^{60} \right]=-e^{\frac{3}{2}}+1-e^{\frac{3}{2}}+e^{-3}
\]
\[
	\mathcal{P}(X+Y<60)=1-e^{-\frac{3}{2}}+e^{-3}.
\]
\stepcounter{section}
\section*{Question 4}
(Worth 20 points). In this question, we derive some formulas for volumes of higher dimensional balls. Let \( V_n(R) \) denote the volume of a \( n \)-ball (in \( \mathbb{R}^{n+1} \)) of radius \( R \).

A 0-ball is just an interval. This allows us to compute the volume of a 2-ball, defined by the inequality

\[
x^2 + y^2 + z^2 \leq R^2,
\]

using cylindrical coordinates, we deduce:

\[
V_2(R) = \int_{0}^{2\pi} \int_{0}^{R} \int_{-\sqrt{R^2-r^2}}^{\sqrt{R^2-r^2}} r \, dh \, dr \, d\theta = \int_{0}^{2\pi} \int_{0}^{R} 2r\sqrt{R^2 - r^2} \, dr \, d\theta.
\]

\begin{enumerate}[label=(\alph*)]
\item Verify that \( V_2(R) = \frac{4\pi R^3}{3} \), by finishing with the iterated integral above.
\[
	V_2(R)=\int_{0}^{2\pi} \int_{0}^{R} 2\sqrt{R^2 - r^2} \, dr \, d\theta.
\]
Using the substitution
\[
	u=R^2-r^2,\,du=-2r\,dr,\,r=1\Rightarrow u=R^2,\, r=R\Rightarrow u=0
\]
\[
	V_2(R)=\int_{0}^{2\pi} \int_{R^2}^{0} -\sqrt{u} \, du \, d\theta
\]
\[
	V_2(R)=\int_{0}^{2\pi} \int_{0}^{R^2} \sqrt{u} \, du \, d\theta
\]
\[
	V_2(R)=2\pi\left[ \frac{2u^{\frac{3}{2}}}{3} \right]^{r^2}_0=\frac{4\pi R^2}{3}.
\]
\item Using the same idea, deduce that

\[
	V_3(R) = \int_{0}^{2\pi} \int_{0}^{R} \pi(R^2 - r^2)r \, dr \, d\theta.
\]\\
In $\R^4$ a ball with radius $R$ is the set of points 
\[
	x_1^2+x_2^2+x_3^2+x_4^2\leq R^2.
\]
Converting to cylindrical coordinates in $\R^4$ we get $r^2=x_1^2+x_2^2$, which then gives
\[
	r^2+x_3^2+x_4^2\leq R^2.
\]
When $r$ is fixed the $(x_3,x_4)$ coordinates become a disk in $\R^2$, this disk has a radius of $\sqrt{R^2-r^2}$ and area of $\pi(R^2-r^2)$. The volume from the $(x_1,x_2)$ coordinates is given by $r\,dr\,d\theta$. All of this gives
\[
	V_3(R) = \int_{0}^{2\pi} \int_{0}^{R} \pi(R^2 - r^2)r \, dr \, d\theta.
\]
\item Generalize this to obtain the formula

\[
	V_{n+2}(R) = \int_{0}^{2\pi} \int_{0}^{R} V_n(\sqrt{R^2 - r^2})r \, dr \, d\theta.
\]\\
A ball in $\R^{n+2}$ with radius $R$ set of coordinates
\[
	x_1^2+x_2^2+\cdots+x_{n+2}^2\leq R^2.
\]
Converting to cylindrical coordinates we get $r^2=x_1^2+x_2^2$. Let $y=(x_3,\dots,x_{n+2})\in\R^{n}$. We then get 
\[
	r^2+\left\|y\right\|^2\leq R^2.
\]
Rearranging
\[
	\left\|y\right\|^2\leq R^2-r^2,
\]
we see that $y$ is within an $n-ball$ with radius $\sqrt{R^2-r^2}$, and the volume of this ball is $V_n(\sqrt{R^2-r^2})$. The $(x_1,x_2)$ coordinates give $r\,dr\,d\theta$. This all yields
\[
	V_{n+2}(R) = \int_{0}^{2\pi} \int_{0}^{R} V_n(\sqrt{R^2 - r^2})r \, dr \, d\theta.
\]
\newpage
\item Using the recursion in part (c), give formulas for \( V_4(R), V_5(R), V_6(R) \).\\
From the prvious part we know that for $V_{n+2}(R)$ the component $V_n(R)$ has a radius of $\sqrt{R^2-r^2}$ and that $V_n(R)\propto \left(\sqrt{R^2-r^2}\right)^{n+1}$, this will help simplfy the following integrals.
\[
	V_4(R)=\int_{0}^{2\pi} \int_{0}^{R} \left(\frac{4\pi R^2}{3}\right)(\sqrt{R^2 - r^2})r \, dr \, d\theta=2\pi\int_{0}^{R} \left(\frac{4\pi}{3}\right)\left(R^2 - r^2\right)^{\frac{3}{2}}r \, dr
\]
\[
	u=R^2-r^2,\,du=-2r\,dr,\,r=1\Rightarrow u=R^2,\, r=R\Rightarrow u=0
\]
\[
	V_4(R)=\frac{4\pi^2}{3}\int_{R^2}^{0}-u^{\frac{3}{2}}\,du=\frac{4\pi^2}{3}\int_{0}^{R^2}u^{\frac{3}{2}}\,du=\frac{8\pi^2}{15}\left[ u^{\frac{5}{2}} \right]^{R^2}_0=\frac{8\pi^2}{15}R^5
\]
\[
	V_4(R)=\frac{8\pi^2}{15}R^5.
\]
\[
	V_3(R) = \int_{0}^{2\pi} \int_{0}^{R} \pi(R^2 - r^2)r \, dr \, d\theta=2\pi\int_{0}^{R} \pi(R^2 - r^2)r \, dr
\]
\[
	u=R^2-r^2,\,du=-2r\,dr,\,r=1\Rightarrow u=R^2,\, r=R\Rightarrow u=0
\]
\[
	V_3(R) =2\pi^2\int_{0}^{R}R^2r-r^3\,dr=2\pi^2\left[ \frac{R^2r^2}{2}-\frac{r^4}{4} \right]_{0}^{R}=2\pi^2\left( \frac{R^4}{2}-\frac{R^4}{4} \right)=\frac{\pi^2R^4}{2}.
\]
\[
	V_{5}(R) = \int_{0}^{2\pi} \int_{0}^{R} V_3(\sqrt{R^2 - r^2})r \, dr \, d\theta=\int_{0}^{2\pi} \int_{0}^{R} \left(\frac{\pi^2R^4}{2}\right)(\sqrt{R^2 - r^2})r \, dr \, d\theta
\]
\[
	V_{5}(R) =2\pi^3 \int_{0}^{R} (R^2 - r^2)^2r \, dr=2\pi^3 \int_{0}^{R} R^4r -2R^2r^3 +r^5 \, dr=2\pi^3\left[ \frac{R^4r^2}{2} -\frac{2R^2r^4}{4} + \frac{r^6}{6} \right]_0^{R^2},
\]
\[
	V_{5}(R)=\pi^3\left( \frac{R^6}{2} -\frac{R^6}{2} + \frac{R^6}{6} \right)=\frac{\pi^3}{6}R^6.
\]
\[
	V_6(R)=\int_{0}^{2\pi} \int_{0}^{R} V_4(\sqrt{R^2 - r^2})r \, dr \, d\theta=2\pi\int_{0}^{R} \frac{8\pi^2}{15}R^5(\sqrt{R^2 - r^2})r \, dr=\frac{8\pi^3}{15}\int_{0}^{R}\left(R^2-r^2\right)^{\frac{5}{2}}r\,dr.
\]
\[
	u=R^2-r^2,\,du=-2r\,dr,\,r=1\Rightarrow u=R^2,\, r=R\Rightarrow u=0
\]
\[
	V_6(R)=\frac{8\pi^3}{15}\int_{0}^{R^2}u^{\frac{5}{2}}\,du=\frac{16\pi^3}{105}\left[ u^{\frac{7}{2}} \right]^{R^2}_0,
\]
\[
	V_6(R)=\frac{16\pi^3}{105}R^7.
\]
\newpage
\item Evaluate the limit
\[
	\lim_{n \to \infty} \frac{V_n(R)}{R^{n+1}}.
\]
Since $V_n(R)\propto \left(\sqrt{R^2-r^2}\right)^{n+1}$ we have 
\[
	\frac{V_n(R)}{R^{n+1}}=\alpha \frac{\left(\sqrt{R^2-r^2}\right)^{n+1}}{R^{n+1}},
\]
for some $\alpha$ which is independant of $n$. So in the limit we obtain
\[
	\lim_{n \to \infty} \frac{V_n(R)}{R^{n+1}}=\lim_{n \to \infty} \alpha \frac{\left(\sqrt{R^2-r^2}\right)^{n+1}}{R^{n+1}}=\alpha\lim_{n \to \infty}  \frac{\left(\sqrt{R^2-r^2}\right)^{n+1}}{R^{n+1}}.
\]
And 
\[
	\lim_{n \to \infty}  \frac{\left(\sqrt{R^2-r^2}\right)^{n+1}}{R^{n+1}}=0,
\]
since the denominator grows faster than the numerator. So,
\[
	\lim_{n \to \infty} \frac{V_n(R)}{R^{n+1}}=0.
\]
\end{enumerate}

\end{document}