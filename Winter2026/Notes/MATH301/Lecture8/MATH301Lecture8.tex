\documentclass[12pt]{article}
\usepackage{color,soul}
% Import preambles and macros for notes
% Essential packages for notes
\usepackage{amsmath, amssymb, amsthm}
\usepackage{mathtools}  % for \coloneqq, etc.
\usepackage{geometry}   % Better page margins
\usepackage{parskip}    % Better paragraph spacing
\usepackage{microtype}  % Better typography
\usepackage{enumitem}   % Customize lists
\usepackage{hyperref}   % Clickable links
\usepackage{booktabs}   % Better tables
\usepackage{tcolorbox}  % For colored boxes/theorems

% Page layout for notes
\geometry{a4paper, margin=1in}
\setlength{\parskip}{0.8em}

% Theorem environments with shared numbering
\newtheorem{theorem}{Theorem}[subsection]  % Number within sections: 2.3.1, 2.3.2, etc.

% All other environments share the same counter as theorem
\newtheorem{lemma}[theorem]{Lemma}
\newtheorem{proposition}[theorem]{Proposition}
\newtheorem{corollary}[theorem]{Corollary}
\newtheorem{definition}[theorem]{Definition}
\newtheorem{example}[theorem]{Example}
\newtheorem{remark}[theorem]{Remark}
\newtheorem{claim}[theorem]{Claim}

% Custom colors for notes
\usepackage{xcolor}
\definecolor{note-blue}{RGB}{220, 230, 255}
\definecolor{theorem-green}{RGB}{220, 255, 220}
% Math notation shortcuts for notes
\newcommand{\R}{\mathbb{R}}
\newcommand{\C}{\mathbb{C}}
\newcommand{\Q}{\mathbb{Q}}
\newcommand{\Z}{\mathbb{Z}}
\newcommand{\N}{\mathbb{N}}

% Calculus
\newcommand{\diff}{\mathop{}\!\mathrm{d}}
\newcommand{\deriv}[2]{\frac{\mathrm{d}#1}{\mathrm{d}#2}}
\newcommand{\pderiv}[2]{\frac{\partial #1}{\partial #2}}

% Linear Algebra
\newcommand{\inner}[2]{\langle #1, #2 \rangle}
\newcommand{\norm}[1]{\| #1 \|}
\newcommand{\tr}{\operatorname{tr}}
\newcommand{\spn}{\operatorname{span}}
\newcommand{\rank}{\operatorname{rank}}
\newcommand{\nullity}{\operatorname{nullity}}

% Logic
\newcommand{\contra}{\Rightarrow\Leftarrow}

% Custom commands for notes
\newcommand{\todo}[1]{\textcolor{red}{[TODO: #1]}}
\newcommand{\important}[1]{\textbf{\textcolor{blue}{#1}}}
% Theorem system 
% Theorem System original by https://github.com/kcajc/math-notes-template, this is modified 
% The following boxes are provided:
%   Definition:     \defn 
%   Theorem:        \thm 
%   Lemma:          \lem
%   Corollary:      \cor
%   Proposition:    \prop   
%   Claim:          \clm
%   Fact:           \fact
%   Proof:          \pf
%   Example:        \ex
%   Remark:         \rmk (sentence), \rmkb (block)
% Suffix
%   r:              Allow Theorem/Definition to be referenced, e.g. thmr
%   p:              Add a short proof block for Lemma, Corollary, Proposition or Claim, e.g. lemp
%                   For theorems, use \pf for proof blocks

% Definition - subsection numbering for 1.1.1, 1.1.2
\newtcbtheorem[number within=subsection]{mydefinition}{Definition}
{
    colbacktitle=green!20!white,
    colback=green!10!white,
    coltitle=black,
    fonttitle=\bfseries\large,
}{defn}

\NewDocumentCommand{\defn}{m+m}{
    \begin{mydefinition}{#1}{}
        #2
    \end{mydefinition}
}

\NewDocumentCommand{\defnr}{mm+m}{
    \begin{mydefinition}{#1}{#2}
        #3
    \end{mydefinition}
}

% Theorem - subsection numbering
\newtcbtheorem[use counter from=mydefinition]{mytheorem}{Theorem}
{
    colbacktitle=cyan!20!white,
    colback=cyan!10!white,
    coltitle=black,
    fonttitle=\bfseries\large,
}{thm}

\NewDocumentCommand{\thm}{m+m}{
    \begin{mytheorem}{#1}{}
        #2
    \end{mytheorem}
}

\NewDocumentCommand{\thmr}{mm+m}{
    \begin{mytheorem}{#1}{#2}
        #3
    \end{mytheorem}
}

% Lemma - subsection numbering
\newtcbtheorem[use counter from=mydefinition]{mylemma}{Lemma}
{
    colbacktitle=violet!20!white,
    colback=violet!10!white,
    coltitle=black,
    fonttitle=\bfseries\large,
}{lem}

\NewDocumentCommand{\lem}{m+m}{
    \begin{mylemma}{#1}{}
        #2
    \end{mylemma}
}

% Improved proof environments with consistent QED placement
\newenvironment{lempf}{
    \par\noindent{\it \textbf{Proof for Lemma.}}\par\nopagebreak
    \begin{list}{}{\setlength\leftmargin{1em}\setlength\rightmargin{0em}}
    \item\relax
}{
    \hfill$\qed$\end{list}
}

\NewDocumentCommand{\lemp}{m+m+m}{
    \begin{mylemma}{#1}{}
        #2
    \end{mylemma}
    \begin{lempf}
        #3
    \end{lempf}
}

% Corollary - subsection numbering
\newtcbtheorem[use counter from=mydefinition]{mycorollary}{Corollary}
{
    colbacktitle=orange!20!white,
    colback=orange!10!white,
    coltitle=black,
    fonttitle=\bfseries\large,
}{cor}

\NewDocumentCommand{\cor}{+m}{
    \begin{mycorollary}{}{}
        #1
    \end{mycorollary}
}

\newenvironment{corpf}{
    \par\noindent{\it \textbf{Proof for Corollary.}}\par\nopagebreak
    \begin{list}{}{\setlength\leftmargin{1em}\setlength\rightmargin{0em}}
    \item\relax
}{
    \hfill$\qed$\end{list}
}

\NewDocumentCommand{\corp}{m+m+m}{
    \begin{mycorollary}{}{}
        #1
    \end{mycorollary}
    \begin{corpf}
        #2
    \end{corpf}
}

% Proposition - subsection numbering
\newtcbtheorem[use counter from=mydefinition]{myproposition}{Proposition}
{
    colbacktitle=yellow!30!white,
    colback=yellow!20!white,
    coltitle=black,
    fonttitle=\bfseries\large,
}{prop}

\NewDocumentCommand{\prop}{+m}{
    \begin{myproposition}{}{}
        #1
    \end{myproposition}
}

\newenvironment{proppf}{
    \par\noindent{\it \textbf{Proof for Proposition.}}\par\nopagebreak
    \begin{list}{}{\setlength\leftmargin{1em}\setlength\rightmargin{0em}}
    \item\relax
}{
    \hfill$\qed$\end{list}
}

\NewDocumentCommand{\propp}{+m+m}{
    \begin{myproposition}{}{}
        #1
    \end{myproposition}
    \begin{proppf}
        #2
    \end{proppf}
}

% Claim - subsection numbering
\newtcbtheorem[use counter from=mydefinition]{myclaim}{Claim}
{
    colbacktitle=pink!30!white,
    colback=pink!20!white,
    coltitle=black,
    fonttitle=\bfseries\large,
}{clm}

\NewDocumentCommand{\clm}{m+m}{
    \begin{myclaim}{#1}{}
        #2
    \end{myclaim}
}

\newenvironment{clmpf}{
    \par\noindent{\it \textbf{Proof for Claim.}}\par\nopagebreak
    \begin{list}{}{\setlength\leftmargin{1em}\setlength\rightmargin{0em}}
    \item\relax
}{
    \hfill$\qed$\end{list}
}

\NewDocumentCommand{\clmp}{m+m+m}{
    \begin{myclaim}{#1}{}
        #2
    \end{myclaim}
    \begin{clmpf}
        #3
    \end{clmpf}
}

% Fact - subsection numbering
\newtcbtheorem[use counter from=mydefinition]{myfact}{Fact}
{
    colbacktitle=purple!20!white,
    colback=purple!10!white,
    coltitle=black,
    fonttitle=\bfseries\large,
}{fact}

\NewDocumentCommand{\fact}{+m}{
    \begin{myfact}{}{}
        #1
    \end{myfact}
}

% Proof - customizable name
\NewDocumentEnvironment{custompf}{m}
{
    \par\noindent{\it \textbf{#1}}\par\nopagebreak
    \begin{list}{}{\setlength\leftmargin{1em}\setlength\rightmargin{0em}}
    \item\relax
}
{
    \hfill$\qed$\end{list}
}

\NewDocumentCommand{\pf}{O{Proof}+m}{
    \begin{custompf}{#1.}
        #2
    \end{custompf}
}

% Example - improved environment
\newenvironment{example}{
    \par\vspace{5pt}
    \noindent\textbf{Example.}\par\nopagebreak
    \begin{list}{}{\setlength\leftmargin{1em}\setlength\rightmargin{0em}}
    \item\relax
}{
    \end{list}\vspace{5pt}
}

\NewDocumentCommand{\ex}{+m}{
    \begin{example}
        #1
    \end{example}
}

% Remark
\NewDocumentCommand{\rmk}{+m}{
    {\it \color{blue!50!white}#1}
}

% Remark block - improved environment
\newenvironment{remark}{
    \par\vspace{5pt}
    \noindent\textbf{Remark.}\par\nopagebreak
    \begin{list}{}{\setlength\leftmargin{1em}\setlength\rightmargin{0em}}
    \item\relax
}{
    \end{list}\vspace{5pt}
}

\NewDocumentCommand{\rmkb}{+m}{
    \begin{remark}
        #1
    \end{remark}
}

\title{MATH 301 Lecture 8}
\author{Deepak Jassal}
\date{February 4\textsuperscript{th}, 2026}

\begin{document}
\maketitle
\section{Complex Differentiability}
\defn{Complex Derivatives}{
    \begin{enumerate}
        \item Let $A$ be an open subsest of $\C$ and $z_0\in A$ and let $f:A\to\C$. Then $f$ is differentiable at $z_0$ if the limit
            \[
                f'(z_0)=\lim_{f(z)-f(z_0)}{z-z_0} \tag{$\ast$}
            \]
            exists.
        \item $f$ is called analytic at $z_0$ if $f'(z)$ in some nieghbourhood of $f(z_0)$. This means that the limit is the same if $z\to z_0$ by any path.
    \end{enumerate}
}
For the limit $\ast$ to exist is means that. For all $\varepsilon>0$ there exists $\delta>0$ such that 
\[
    0<|z-z_0|<\delta\Rightarrow \left|\frac{f(z)-f(z_0)}{z-z_0}\right|=|f'(z_0)|<\varepsilon.
\]
Writing $\ast$ as $F(z)$ we say the above means
\[
    \lim_{z\to z_0}F(z)=f'(z_0).
\]
\textit{Note:} Analytic and holomorphic are synonymous in this context.
\prop{
    Suppose that $f,g$ are analytic functions on an open set $A\subseteq\C$. Then
    \begin{enumerate}
        \item $af+bg$ is analytic on $A$, $\forall a,b\in\C$.
        \item $(fg)'=f'g+fg'$
        \item $\left(\dfrac{f}{g}\right)'=\dfrac{f'g-g'f}{g^2}$ provided $g\neq0$ on $A$.
        \item The chain rule is also valid:\\
        If $f:A\to\C$ with $A$ open is analytic, and $g:B\to\C$ with $B$ open is analytic and $f(A)\subseteq B$
        \[
            (f(g(z)))'=f'(g(z))g'(z).
        \]
        \item If $\gamma:(a,b)\to\C$ with $(a,b)\subseteq\R$ if differentiable, and $f:A\to\C$ with $A$ open is analytic. Then
        \[
            \frac{d}{dt}f(\gamma(t))=f'(\gamma(t))\gamma'(t).
        \]
        provided that $\gamma((a,b))\subseteq\C$.
    \end{enumerate}
 }
 \subsection{Conformal Mapping Theorem}
\thm{Conformal Mapping Theorem}{
    Let $A\subseteq\C$ be an open set, and $f:A\to\C$ is analytic on $A$. Further suppose that $\gamma:I\to A$, where $I$ is an open interval containing 0, is differentiable, $z_0\in A$, $\gamma(0)=z_0,$ and $\gamma'(0)=v\neq0$. Also suppose that $f'(z_0)\neq0$. Then there exists $0<r\in\R$ and $\theta\in[0,2\pi)$ such that if
    \[
        u=\frac{d}{dt}f(\gamma(t))\vert_0
    \] 
    then 
    \begin{enumerate}
        \item $|u|=r|v|$;
        \item arg$u=$arg$v+\theta$ mod$2\pi$.
    \end{enumerate}
    We say that $f$ is conformal at $z_0$
}
\textit{Note:} also that a conformal map preserves the orientation of tangent vectors. It preserves angles but not always lengths.
\pf[Proof of Theorem 1.1.1]{
    \[
        u=\frac{d}{dt}f(\gamma(t))\vert_0=f'(\gamma(0))\gamma'(0)=f'(z_0)v.
    \]
    \[
        \Rightarrow |u|=|f'(z_0)||v|
    \]
    \[
        \arg(u)=\arg(v)+\arg(f'(z_0))
    \]
    Now define $r=|f'(z_0)|>0$ and $\theta=\arg(f(z_0))$.
}
\subsection{Cauchy-Riemann Equation}
Let $f:A\to\C$, $\mu:\C\to\R^2$ be defined as \[\mu(z)=\mu(x+iy)=(x,y).\] Then $f(x+iy)$ can be written as 
\[
    f(x+iy)=u(x,y)+v(x,y)
\]
where 
\[
    u:\mu(A)\to\R
\]
\[
    v:\mu(A)\to\R.
\]
\ex{
    $f(z)=z^2=(x+iy)^2=x^2-y^2+2xy$ where $u(x,y)=x^2-y^2$ and $v(x,y)=2xy$.
}
\thm{}{
    Let $f:A\to\C,\,z_0\in A$ and $z=x+iy$
    \[
        f(z)=u(x,y)+iv(x,y)
    \]
    \begin{enumerate}
        \item If $f'(z_0)$ exists Then
        $u_x(x_0,y_0)=u_y(x_0,y_0)$ and $v_x(x_0,y_0)=-v_y(x_0,y_0)$.\\
        These are called the Cauchy-Riemann equations.
        \item If $u_x,u_y,v_x,v_y$ are continuous on $\mu(A)$ and satisfy the Cauchy-Riemann equations then $f$ is analytic on $A$. 
    \end{enumerate}
}
\pf[Proof of Theorem 1.2.1]{
    \begin{enumerate}
        \item 
        \[
            \lim_{z\to z_0}\frac{f(z)-f(z_0)}{z-z_0}
        \]
        exists. We calculate it along two paths
        \begin{enumerate}[label=(\roman*)]
            \item $L:x+iy_0\to x_0+iy_0$ where $y_0$ is a constant.
            \[
                \lim_{z\to z_0}f'(z_0)=\lim_{x\to x_0}f^\ast(z)=u_x(x_0,y_0)+iv_x(x_0,y_0).
            \]
            \item $L:x_0+iy\to x_0+iy_0$
            \[
                \lim_{z\to z_0}f'(z_0)=\lim_{y\to y_0}f^\ast(z)=u_y(x_0,y_0)+iv_y(x_0,y_0).
            \]
        \end{enumerate}
        Since $f'(z_0)$ exists we must have that $u_y(x_0,y_0)+iv_y(x_0,y_0)=u_x(x_0,y_0)+iv_x(x_0,y_0)$.
    \end{enumerate}
}

\end{document}