\documentclass[12pt]{article}
\usepackage{color,soul}
% Import preambles and macros for notes
% Essential packages for notes
\usepackage{amsmath, amssymb, amsthm}
\usepackage{mathtools}  % for \coloneqq, etc.
\usepackage{geometry}   % Better page margins
\usepackage{parskip}    % Better paragraph spacing
\usepackage{microtype}  % Better typography
\usepackage{enumitem}   % Customize lists
\usepackage{hyperref}   % Clickable links
\usepackage{booktabs}   % Better tables
\usepackage{tcolorbox}  % For colored boxes/theorems

% Page layout for notes
\geometry{a4paper, margin=1in}
\setlength{\parskip}{0.8em}

% Theorem environments with shared numbering
\newtheorem{theorem}{Theorem}[subsection]  % Number within sections: 2.3.1, 2.3.2, etc.

% All other environments share the same counter as theorem
\newtheorem{lemma}[theorem]{Lemma}
\newtheorem{proposition}[theorem]{Proposition}
\newtheorem{corollary}[theorem]{Corollary}
\newtheorem{definition}[theorem]{Definition}
\newtheorem{example}[theorem]{Example}
\newtheorem{remark}[theorem]{Remark}
\newtheorem{claim}[theorem]{Claim}

% Custom colors for notes
\usepackage{xcolor}
\definecolor{note-blue}{RGB}{220, 230, 255}
\definecolor{theorem-green}{RGB}{220, 255, 220}
% Math notation shortcuts for notes
\newcommand{\R}{\mathbb{R}}
\newcommand{\C}{\mathbb{C}}
\newcommand{\Q}{\mathbb{Q}}
\newcommand{\Z}{\mathbb{Z}}
\newcommand{\N}{\mathbb{N}}

% Calculus
\newcommand{\diff}{\mathop{}\!\mathrm{d}}
\newcommand{\deriv}[2]{\frac{\mathrm{d}#1}{\mathrm{d}#2}}
\newcommand{\pderiv}[2]{\frac{\partial #1}{\partial #2}}

% Linear Algebra
\newcommand{\inner}[2]{\langle #1, #2 \rangle}
\newcommand{\norm}[1]{\| #1 \|}
\newcommand{\tr}{\operatorname{tr}}
\newcommand{\spn}{\operatorname{span}}
\newcommand{\rank}{\operatorname{rank}}
\newcommand{\nullity}{\operatorname{nullity}}

% Logic
\newcommand{\contra}{\Rightarrow\Leftarrow}

% Custom commands for notes
\newcommand{\todo}[1]{\textcolor{red}{[TODO: #1]}}
\newcommand{\important}[1]{\textbf{\textcolor{blue}{#1}}}
% Theorem system 
% Theorem System original by https://github.com/kcajc/math-notes-template, this is modified 
% The following boxes are provided:
%   Definition:     \defn 
%   Theorem:        \thm 
%   Lemma:          \lem
%   Corollary:      \cor
%   Proposition:    \prop   
%   Claim:          \clm
%   Fact:           \fact
%   Proof:          \pf
%   Example:        \ex
%   Remark:         \rmk (sentence), \rmkb (block)
% Suffix
%   r:              Allow Theorem/Definition to be referenced, e.g. thmr
%   p:              Add a short proof block for Lemma, Corollary, Proposition or Claim, e.g. lemp
%                   For theorems, use \pf for proof blocks

% Definition - subsection numbering for 1.1.1, 1.1.2
\newtcbtheorem[number within=subsection]{mydefinition}{Definition}
{
    colbacktitle=green!20!white,
    colback=green!10!white,
    coltitle=black,
    fonttitle=\bfseries\large,
}{defn}

\NewDocumentCommand{\defn}{m+m}{
    \begin{mydefinition}{#1}{}
        #2
    \end{mydefinition}
}

\NewDocumentCommand{\defnr}{mm+m}{
    \begin{mydefinition}{#1}{#2}
        #3
    \end{mydefinition}
}

% Theorem - subsection numbering
\newtcbtheorem[use counter from=mydefinition]{mytheorem}{Theorem}
{
    colbacktitle=cyan!20!white,
    colback=cyan!10!white,
    coltitle=black,
    fonttitle=\bfseries\large,
}{thm}

\NewDocumentCommand{\thm}{m+m}{
    \begin{mytheorem}{#1}{}
        #2
    \end{mytheorem}
}

\NewDocumentCommand{\thmr}{mm+m}{
    \begin{mytheorem}{#1}{#2}
        #3
    \end{mytheorem}
}

% Lemma - subsection numbering
\newtcbtheorem[use counter from=mydefinition]{mylemma}{Lemma}
{
    colbacktitle=violet!20!white,
    colback=violet!10!white,
    coltitle=black,
    fonttitle=\bfseries\large,
}{lem}

\NewDocumentCommand{\lem}{m+m}{
    \begin{mylemma}{#1}{}
        #2
    \end{mylemma}
}

% Improved proof environments with consistent QED placement
\newenvironment{lempf}{
    \par\noindent{\it \textbf{Proof for Lemma.}}\par\nopagebreak
    \begin{list}{}{\setlength\leftmargin{1em}\setlength\rightmargin{0em}}
    \item\relax
}{
    \hfill$\qed$\end{list}
}

\NewDocumentCommand{\lemp}{m+m+m}{
    \begin{mylemma}{#1}{}
        #2
    \end{mylemma}
    \begin{lempf}
        #3
    \end{lempf}
}

% Corollary - subsection numbering
\newtcbtheorem[use counter from=mydefinition]{mycorollary}{Corollary}
{
    colbacktitle=orange!20!white,
    colback=orange!10!white,
    coltitle=black,
    fonttitle=\bfseries\large,
}{cor}

\NewDocumentCommand{\cor}{+m}{
    \begin{mycorollary}{}{}
        #1
    \end{mycorollary}
}

\newenvironment{corpf}{
    \par\noindent{\it \textbf{Proof for Corollary.}}\par\nopagebreak
    \begin{list}{}{\setlength\leftmargin{1em}\setlength\rightmargin{0em}}
    \item\relax
}{
    \hfill$\qed$\end{list}
}

\NewDocumentCommand{\corp}{m+m+m}{
    \begin{mycorollary}{}{}
        #1
    \end{mycorollary}
    \begin{corpf}
        #2
    \end{corpf}
}

% Proposition - subsection numbering
\newtcbtheorem[use counter from=mydefinition]{myproposition}{Proposition}
{
    colbacktitle=yellow!30!white,
    colback=yellow!20!white,
    coltitle=black,
    fonttitle=\bfseries\large,
}{prop}

\NewDocumentCommand{\prop}{+m}{
    \begin{myproposition}{}{}
        #1
    \end{myproposition}
}

\newenvironment{proppf}{
    \par\noindent{\it \textbf{Proof for Proposition.}}\par\nopagebreak
    \begin{list}{}{\setlength\leftmargin{1em}\setlength\rightmargin{0em}}
    \item\relax
}{
    \hfill$\qed$\end{list}
}

\NewDocumentCommand{\propp}{+m+m}{
    \begin{myproposition}{}{}
        #1
    \end{myproposition}
    \begin{proppf}
        #2
    \end{proppf}
}

% Claim - subsection numbering
\newtcbtheorem[use counter from=mydefinition]{myclaim}{Claim}
{
    colbacktitle=pink!30!white,
    colback=pink!20!white,
    coltitle=black,
    fonttitle=\bfseries\large,
}{clm}

\NewDocumentCommand{\clm}{m+m}{
    \begin{myclaim}{#1}{}
        #2
    \end{myclaim}
}

\newenvironment{clmpf}{
    \par\noindent{\it \textbf{Proof for Claim.}}\par\nopagebreak
    \begin{list}{}{\setlength\leftmargin{1em}\setlength\rightmargin{0em}}
    \item\relax
}{
    \hfill$\qed$\end{list}
}

\NewDocumentCommand{\clmp}{m+m+m}{
    \begin{myclaim}{#1}{}
        #2
    \end{myclaim}
    \begin{clmpf}
        #3
    \end{clmpf}
}

% Fact - subsection numbering
\newtcbtheorem[use counter from=mydefinition]{myfact}{Fact}
{
    colbacktitle=purple!20!white,
    colback=purple!10!white,
    coltitle=black,
    fonttitle=\bfseries\large,
}{fact}

\NewDocumentCommand{\fact}{+m}{
    \begin{myfact}{}{}
        #1
    \end{myfact}
}

% Proof - customizable name
\NewDocumentEnvironment{custompf}{m}
{
    \par\noindent{\it \textbf{#1}}\par\nopagebreak
    \begin{list}{}{\setlength\leftmargin{1em}\setlength\rightmargin{0em}}
    \item\relax
}
{
    \hfill$\qed$\end{list}
}

\NewDocumentCommand{\pf}{O{Proof}+m}{
    \begin{custompf}{#1.}
        #2
    \end{custompf}
}

% Example - improved environment
\newenvironment{example}{
    \par\vspace{5pt}
    \noindent\textbf{Example.}\par\nopagebreak
    \begin{list}{}{\setlength\leftmargin{1em}\setlength\rightmargin{0em}}
    \item\relax
}{
    \end{list}\vspace{5pt}
}

\NewDocumentCommand{\ex}{+m}{
    \begin{example}
        #1
    \end{example}
}

% Remark
\NewDocumentCommand{\rmk}{+m}{
    {\it \color{blue!50!white}#1}
}

% Remark block - improved environment
\newenvironment{remark}{
    \par\vspace{5pt}
    \noindent\textbf{Remark.}\par\nopagebreak
    \begin{list}{}{\setlength\leftmargin{1em}\setlength\rightmargin{0em}}
    \item\relax
}{
    \end{list}\vspace{5pt}
}

\NewDocumentCommand{\rmkb}{+m}{
    \begin{remark}
        #1
    \end{remark}
}

\title{MATH 301 Lecture 1}
\author{Deepak Jassal}
\date{January 9\textsuperscript{th}, 2026}

\begin{document}
\maketitle  
\section*{16\textsuperscript{th} Century} 
Gerolamo Cardano solved the cubic equation\\
Ludovico Ferrari solved the quadratic equation.
\setcounter{section}{1}
\subsection{Casus Irreducibles}
\[
    x^3+bx^2+x+d=0
\]
If the polynomial does not have rational roots, but does have three real roots then the Cardano formula must use imaginary numbers.\\
\defn{$\C$, (W.R.Hamilton 1833)}{
    $\C=\{(a,b):a,b\in\R\}$, $z=a+bi,$ $a,b\in\R$.
}
\defn{Complex Addition and Multiplication}{
    \begin{itemize}
        \item $(a,b)+(c,d)=(a+c,b+d)$;
        \item $(a,b)\times(c,d)=(ac-bd,ad+bc)$.
    \end{itemize}
}
\[
    i=(0,1),\quad 1=(1,0),\quad \R\subseteq\C. 
\]
Given that $z=a+bi$ is a complex number, then $\operatorname{Re}(z)=a$ and $\operatorname{Im}(z)=b$.\\
$(\C,+,\times)$ is a field.\\
Given that $z=a+bi\neq0$, then we have $z^{-1}=\dfrac{1}{a+bi}=\dfrac{1}{a+bi}\times\dfrac{a-bi}{a-bi}=\underbrace{\dfrac{a}{a^2+b^2}}_A+\underbrace{\left(\dfrac{-b}{a^2+b^2}\right)}_{Bi}$.
\newpage
\ex{
    \begin{align*}
        z&=\frac{(2+i)^2}{3-i}=\frac{3+4i}{3-i}\times\frac{3+i}{3+i}\\
        &=\frac{9+3i+12i-4}{9-1}\\
        &=\frac{5}{10}+\frac{15}{10}i.
    \end{align*}
}
\ex{
    \begin{align*}
        (a+bi)^2&=12-5i\\
        a^2-b^2+2abi&=12-5i
    \end{align*}
    \[
        \begin{cases}
            a^2-b^2=12\\
            2ab=-5
        \end{cases}\Rightarrow z=\frac{5\sqrt{2}}{2}-\frac{\sqrt{2}}{2}i.
    \]
}
\subsection{Geometric Representation of $\C$}
\begin{figure}[h!]
    \centering
    \begin{tikzpicture}[scale=1.5]
        % Complex plane axes
        \draw[->, thick] (-2.5,0) -- (2.5,0) node[right] {$\operatorname{Re}$ (real axis)};
        \draw[->, thick] (0,-2.5) -- (0,2.5) node[above] {$\operatorname{Im}$ (imaginary axis)};
        
        % Grid lines
        \draw[gray!30, thin] (-2.5,-2.5) grid (2.5,2.5);
        
        % Tick marks
        \foreach \x in {-2,-1,1,2}
            \draw (\x,0.05) -- (\x,-0.05) node[below] {$\x$};
        \foreach \y in {-2,-1,1,2}
            \draw (0.05,\y) -- (-0.05,\y) node[left] {$\y i$};
        
        % Vector from origin to point (a,b)
        \def\a{1.5}  % real part
        \def\b{1.2}  % imaginary part
        
        % Vector with arrow
        \draw[->, very thick, blue] (0,0) -- (\a,\b) 
            node[midway, above, sloped] {$\vec{r}$};
        
        % Point at (a,b)
        \fill[red] (\a,\b) circle (2pt) node[right] {$z = a + bi$};
        
        % Dashed lines to axes
        \draw[dashed, gray] (\a,\b) -- (\a,0) node[below] {$a$};
        \draw[dashed, gray] (\a,\b) -- (0,\b) node[left] {$bi$};
        
        % Angle arc from positive real axis to vector
        \draw[thick, red] (0.5,0) arc (0:{atan(\b/\a)}:0.5);
        
        % Angle label
        \node[red] at ({0.6*cos(atan(\b/\a)/2)},{0.6*sin(atan(\b/\a)/2)}) {$\theta$};
        
        % Unit circle (optional, shows relationship)
        \draw[orange, dashed, thick] (0,0) circle ({sqrt(\a*\a + \b*\b)});
        
        % Polar form label
        \node[below left] at (0,0) {$0$};

    \end{tikzpicture}
    \caption{Complex plane $\C$, with a complex number $z=a+bi$ represented as a vector.}
\end{figure}
Here $\vec{r}=|z|=\sqrt{a^2+b^2}$, $\theta=\mathrm{arg}(z)$, and for convenience we restrict $0\leq \theta<2\pi.$\\
\prop{Given $z_1=a_1+b_1i,z_2=a_2+b_2i\in\C$ 
    \begin{enumerate}
        \item $|z_1z_2|=|z_1|\times|z_2|$
        \item $\mathrm{arg}(z_1z_2)=\mathrm{arg}(z_1)\mathrm{arg}(z_2) \mod{2\pi}$ 
    \end{enumerate}
    }
\begin{proof}
    \begin{enumerate}
        \item Proof of part 1.
        \begin{align*}
            z_1z_2&=(a_1+b_1i)\times(a_2+b_2i)\\
            &=a_1a_2-b_1b_2+(a_1b_2+a_2b_1)i\\
            |z_1z_2|&=\sqrt{(a_1a_2-b_1b_2)^2+(a_1b_2+a_2b_1)^2}\\
            &=\sqrt{(a_1^2a_2^2-2a_1a_2b_1b_2+b_1^2b_2^2)+(a_1^2b_2^2+2a_1a_2b_1b_2+a_2^2b_1^2)}\\
            &=\sqrt{a_1^2a_2^2+b_1^2b_2^2+a_1^2b_2^2+a_2^2b_1^2}\\
            &= \sqrt{a_1^2(a_2^2 + b_2^2) + b_1^2(a_2^2 + b_2^2)} \\
            &= \sqrt{(a_1^2 + b_1^2)(a_2^2 + b_2^2)} \\
            &= \sqrt{a_1^2 + b_1^2} \cdot \sqrt{a_2^2 + b_2^2} \\
            &= |z_1| \cdot |z_2|
        \end{align*}
        \item Proof of part 2.
        \begin{align*}
            z_1&=|z_1|(\cos(\alpha)+i\sin(\alpha))\\
            z_2&=|z_2|(\cos(\beta)+i\sin(\beta))\\
            z_1z_2&=|z_1z_2|\left[(\cos(\alpha)\cos(\beta)-\sin(\alpha)\sin(\beta))+i(\cos(\alpha)\sin(\beta)+\cos(\beta)\sin(\alpha))\right]\\
            &=|z_1|\times|z_2|(\cos(\alpha+\beta)+i\sin(\alpha+\beta))\\
            \cos(\alpha+\beta)&=\cos(\alpha+\beta\mod(2\pi))\\
            \sin(\alpha+\beta)&=\sin(\alpha+\beta\mod(2\pi))
        \end{align*}
        Therefore, $\mathrm{arg}(z_1z_2)=\mathrm{arg}(z_1)\times\mathrm(z_2)\mod2\pi$.\qedhere
    \end{enumerate}
\end{proof}
\subsection{De Moivre's Formula}
If $z=r(\cos(\theta)+i\sin(\theta))$ then $z^n=r^n(\cos(n\theta)+i\sin(n\theta))$. This can be proven by induction after showing the first case for $n=2.$
\ex{
    Use De Moivre's formula to express $\cos(5\theta)$ and $\sin(5\theta)$ in terms of $\cos(\theta)$ and $\sin(\theta)$.\\
    Let $z=\cos(\theta)+i\sin(\theta)$ ($|z|=1$). Then
    \begin{align*}
        z^5&=\cos(5\theta)+i\sin(5\theta)\\
        &=(\cos(\theta)+i\sin(\theta))^5\\
        &=\cos^5(\theta)+5i\cos^4(\theta)\sin(\theta)-10\cos^3(\theta)\sin^2(\theta)-10i\cos^2(\theta)\sin^3(\theta)+\cos(\theta)\sin^4(\theta)+i\sin^5(\theta).
    \end{align*}
    Comparing coefficients we obtain
    \[
        \cos(5\theta)=\cos^5(\theta)-10\cos^3(\theta)sin^2(\theta)+5\cos(\theta)\sin^4(\theta),
    \]
    \[
        \sin(5\theta)=-5\cos^4(\theta)-10\cos^2\sin^3(\theta)+\sin^5(\theta).
    \]
}
\ex{Find all $n^{\text{th}}$ roots of 1 ($n\in\N$).\\ We have $z^n=1$ where $z\in\C$, and we want to find the value of $z$.\\
Let $z=\cos(\theta)+i\sin(\theta)$, then 
\[
    z^n=\cos(n\theta)+i\sin(n\theta)=1.
\]
From here we can see that $\cos(n\theta)=1$ and $\sin(n\theta)=0$. From here we get
\[
    n\theta=2k\pi\Rightarrow\theta=\frac{2k\pi}{n},\quad k=1,2,3,\dots,n-1.
\]
Therefore, the solution is 
\[
    z_k=\cos\left(\frac{2k\pi}{n}\right)+i\sin\left(\frac{2k\pi}{n}\right).
\]
To obtain the other roots: let $\omega=\cos\left(\frac{2\pi}{n}\right)+i\sin\left(\frac{2\pi}{n}\right)$. Then, the other roots are
\[
    \left\{1,\omega,\omega^2,\cdots,\omega^{n-1} \right\}.
\]
}
\subsection{Complex Conjugation}
\defn{Complex Conjugate}{
    If $z=a+bi$ with $a,b\in\R$, then we define the complex conjugate of $z$ as $\bar{z}=a-bi$.
}
\props{
    \begin{enumerate}
        \item $\overline{z_1+z_2}=\overline{z_1}+\overline{z_1}$
        \item $\overline{z_1\times z_2}=\overline{z_1}\times\overline{z_1}$
        \item $z\overline{z}=|z|^2$
        \item $\operatorname{Re}(z)=\frac{z+\overline{z}}{2}$\\
              $\operatorname{Im}(z)=\frac{z-\overline{z}}{2i}$
        \item $\overline{\overline{z}}=z$.
    \end{enumerate}
}
\rmkb{$\mathrm{arg}(z)=\mathrm{arg}(\overline{z})$.}
\props{Properties of the modulus of $z$ $|z|$.
\begin{enumerate}
    \item $|z_1z_2|=|z_1|\times|z_2|$
    \item $-z\leq\operatorname{Re}(z)\leq z$\\
          $|-z|\leq\operatorname{Im}(z)\leq |z|$
    \item $|\overline{z}|=|z|$
    \item $|z_1+z_2|\leq |z_1|+|z_2|$
    \item $||z_1|-|z_2||\leq |z_1-z_2|$
    \item $\left|\displaystyle\sum_{i=1}^{n}z_i\omega_i\right|\leq\sqrt{\displaystyle\sum_{i=1}^{n}|z_i^2|}\times\sqrt{\displaystyle\sum_{i=1}^{n}|\omega_i^2|}$\\\\
        This inequality is Cauchy-Schwarz in $\C$.
\end{enumerate}
}
\end{document}