\documentclass[12pt]{article}
\usepackage{color,soul}
% Import preambles and macros for notes
% Essential packages for notes
\usepackage{amsmath, amssymb, amsthm}
\usepackage{mathtools}  % for \coloneqq, etc.
\usepackage{geometry}   % Better page margins
\usepackage{parskip}    % Better paragraph spacing
\usepackage{microtype}  % Better typography
\usepackage{enumitem}   % Customize lists
\usepackage{hyperref}   % Clickable links
\usepackage{booktabs}   % Better tables
\usepackage{tcolorbox}  % For colored boxes/theorems

% Page layout for notes
\geometry{a4paper, margin=1in}
\setlength{\parskip}{0.8em}

% Theorem environments with shared numbering
\newtheorem{theorem}{Theorem}[subsection]  % Number within sections: 2.3.1, 2.3.2, etc.

% All other environments share the same counter as theorem
\newtheorem{lemma}[theorem]{Lemma}
\newtheorem{proposition}[theorem]{Proposition}
\newtheorem{corollary}[theorem]{Corollary}
\newtheorem{definition}[theorem]{Definition}
\newtheorem{example}[theorem]{Example}
\newtheorem{remark}[theorem]{Remark}
\newtheorem{claim}[theorem]{Claim}

% Custom colors for notes
\usepackage{xcolor}
\definecolor{note-blue}{RGB}{220, 230, 255}
\definecolor{theorem-green}{RGB}{220, 255, 220}
% Math notation shortcuts for notes
\newcommand{\R}{\mathbb{R}}
\newcommand{\C}{\mathbb{C}}
\newcommand{\Q}{\mathbb{Q}}
\newcommand{\Z}{\mathbb{Z}}
\newcommand{\N}{\mathbb{N}}

% Calculus
\newcommand{\diff}{\mathop{}\!\mathrm{d}}
\newcommand{\deriv}[2]{\frac{\mathrm{d}#1}{\mathrm{d}#2}}
\newcommand{\pderiv}[2]{\frac{\partial #1}{\partial #2}}

% Linear Algebra
\newcommand{\inner}[2]{\langle #1, #2 \rangle}
\newcommand{\norm}[1]{\| #1 \|}
\newcommand{\tr}{\operatorname{tr}}
\newcommand{\spn}{\operatorname{span}}
\newcommand{\rank}{\operatorname{rank}}
\newcommand{\nullity}{\operatorname{nullity}}

% Logic
\newcommand{\contra}{\Rightarrow\Leftarrow}

% Custom commands for notes
\newcommand{\todo}[1]{\textcolor{red}{[TODO: #1]}}
\newcommand{\important}[1]{\textbf{\textcolor{blue}{#1}}}
% Theorem system 
% Theorem System original by https://github.com/kcajc/math-notes-template, this is modified 
% The following boxes are provided:
%   Definition:     \defn 
%   Theorem:        \thm 
%   Lemma:          \lem
%   Corollary:      \cor
%   Proposition:    \prop   
%   Claim:          \clm
%   Fact:           \fact
%   Proof:          \pf
%   Example:        \ex
%   Remark:         \rmk (sentence), \rmkb (block)
% Suffix
%   r:              Allow Theorem/Definition to be referenced, e.g. thmr
%   p:              Add a short proof block for Lemma, Corollary, Proposition or Claim, e.g. lemp
%                   For theorems, use \pf for proof blocks

% Definition - subsection numbering for 1.1.1, 1.1.2
\newtcbtheorem[number within=subsection]{mydefinition}{Definition}
{
    colbacktitle=green!20!white,
    colback=green!10!white,
    coltitle=black,
    fonttitle=\bfseries\large,
}{defn}

\NewDocumentCommand{\defn}{m+m}{
    \begin{mydefinition}{#1}{}
        #2
    \end{mydefinition}
}

\NewDocumentCommand{\defnr}{mm+m}{
    \begin{mydefinition}{#1}{#2}
        #3
    \end{mydefinition}
}

% Theorem - subsection numbering
\newtcbtheorem[use counter from=mydefinition]{mytheorem}{Theorem}
{
    colbacktitle=cyan!20!white,
    colback=cyan!10!white,
    coltitle=black,
    fonttitle=\bfseries\large,
}{thm}

\NewDocumentCommand{\thm}{m+m}{
    \begin{mytheorem}{#1}{}
        #2
    \end{mytheorem}
}

\NewDocumentCommand{\thmr}{mm+m}{
    \begin{mytheorem}{#1}{#2}
        #3
    \end{mytheorem}
}

% Lemma - subsection numbering
\newtcbtheorem[use counter from=mydefinition]{mylemma}{Lemma}
{
    colbacktitle=violet!20!white,
    colback=violet!10!white,
    coltitle=black,
    fonttitle=\bfseries\large,
}{lem}

\NewDocumentCommand{\lem}{m+m}{
    \begin{mylemma}{#1}{}
        #2
    \end{mylemma}
}

% Improved proof environments with consistent QED placement
\newenvironment{lempf}{
    \par\noindent{\it \textbf{Proof for Lemma.}}\par\nopagebreak
    \begin{list}{}{\setlength\leftmargin{1em}\setlength\rightmargin{0em}}
    \item\relax
}{
    \hfill$\qed$\end{list}
}

\NewDocumentCommand{\lemp}{m+m+m}{
    \begin{mylemma}{#1}{}
        #2
    \end{mylemma}
    \begin{lempf}
        #3
    \end{lempf}
}

% Corollary - subsection numbering
\newtcbtheorem[use counter from=mydefinition]{mycorollary}{Corollary}
{
    colbacktitle=orange!20!white,
    colback=orange!10!white,
    coltitle=black,
    fonttitle=\bfseries\large,
}{cor}

\NewDocumentCommand{\cor}{+m}{
    \begin{mycorollary}{}{}
        #1
    \end{mycorollary}
}

\newenvironment{corpf}{
    \par\noindent{\it \textbf{Proof for Corollary.}}\par\nopagebreak
    \begin{list}{}{\setlength\leftmargin{1em}\setlength\rightmargin{0em}}
    \item\relax
}{
    \hfill$\qed$\end{list}
}

\NewDocumentCommand{\corp}{m+m+m}{
    \begin{mycorollary}{}{}
        #1
    \end{mycorollary}
    \begin{corpf}
        #2
    \end{corpf}
}

% Proposition - subsection numbering
\newtcbtheorem[use counter from=mydefinition]{myproposition}{Proposition}
{
    colbacktitle=yellow!30!white,
    colback=yellow!20!white,
    coltitle=black,
    fonttitle=\bfseries\large,
}{prop}

\NewDocumentCommand{\prop}{+m}{
    \begin{myproposition}{}{}
        #1
    \end{myproposition}
}

\newenvironment{proppf}{
    \par\noindent{\it \textbf{Proof for Proposition.}}\par\nopagebreak
    \begin{list}{}{\setlength\leftmargin{1em}\setlength\rightmargin{0em}}
    \item\relax
}{
    \hfill$\qed$\end{list}
}

\NewDocumentCommand{\propp}{+m+m}{
    \begin{myproposition}{}{}
        #1
    \end{myproposition}
    \begin{proppf}
        #2
    \end{proppf}
}

% Claim - subsection numbering
\newtcbtheorem[use counter from=mydefinition]{myclaim}{Claim}
{
    colbacktitle=pink!30!white,
    colback=pink!20!white,
    coltitle=black,
    fonttitle=\bfseries\large,
}{clm}

\NewDocumentCommand{\clm}{m+m}{
    \begin{myclaim}{#1}{}
        #2
    \end{myclaim}
}

\newenvironment{clmpf}{
    \par\noindent{\it \textbf{Proof for Claim.}}\par\nopagebreak
    \begin{list}{}{\setlength\leftmargin{1em}\setlength\rightmargin{0em}}
    \item\relax
}{
    \hfill$\qed$\end{list}
}

\NewDocumentCommand{\clmp}{m+m+m}{
    \begin{myclaim}{#1}{}
        #2
    \end{myclaim}
    \begin{clmpf}
        #3
    \end{clmpf}
}

% Fact - subsection numbering
\newtcbtheorem[use counter from=mydefinition]{myfact}{Fact}
{
    colbacktitle=purple!20!white,
    colback=purple!10!white,
    coltitle=black,
    fonttitle=\bfseries\large,
}{fact}

\NewDocumentCommand{\fact}{+m}{
    \begin{myfact}{}{}
        #1
    \end{myfact}
}

% Proof - customizable name
\NewDocumentEnvironment{custompf}{m}
{
    \par\noindent{\it \textbf{#1}}\par\nopagebreak
    \begin{list}{}{\setlength\leftmargin{1em}\setlength\rightmargin{0em}}
    \item\relax
}
{
    \hfill$\qed$\end{list}
}

\NewDocumentCommand{\pf}{O{Proof}+m}{
    \begin{custompf}{#1.}
        #2
    \end{custompf}
}

% Example - improved environment
\newenvironment{example}{
    \par\vspace{5pt}
    \noindent\textbf{Example.}\par\nopagebreak
    \begin{list}{}{\setlength\leftmargin{1em}\setlength\rightmargin{0em}}
    \item\relax
}{
    \end{list}\vspace{5pt}
}

\NewDocumentCommand{\ex}{+m}{
    \begin{example}
        #1
    \end{example}
}

% Remark
\NewDocumentCommand{\rmk}{+m}{
    {\it \color{blue!50!white}#1}
}

% Remark block - improved environment
\newenvironment{remark}{
    \par\vspace{5pt}
    \noindent\textbf{Remark.}\par\nopagebreak
    \begin{list}{}{\setlength\leftmargin{1em}\setlength\rightmargin{0em}}
    \item\relax
}{
    \end{list}\vspace{5pt}
}

\NewDocumentCommand{\rmkb}{+m}{
    \begin{remark}
        #1
    \end{remark}
}

\title{MATH 301 Lecture 6}
\author{Deepak Jassal}
\date{January 28\textsuperscript{rd}, 2026}

\begin{document}
\setcounter{section}{1}
\maketitle
\defn{One More Characterization of Disconnected Sets}{
    Suppose that $A+B\cup C$ and $B\neq\phi$ $C\neq\phi$. Then $A$ is disconnected is both $B$ and $C$ are open and closed relative to $A$.
}
\textit{Recall.} Definition 1 and 2 are equivalent.\\
Equivalence of Definition 1 and 2:\\
Suppose that $A$ is disconnected accoriding to definition 1. So there are open sets $U$ and $V$ sucj that $B\subseteq U$ and $C\subseteq C$ and $U\cap V=\phi$. Hence, $B=A\cap U$, so $B$ is open relative to $A$. Similarily $C$ is open relative to $A$. So $B$ and $C$ are open in $A$. Also $V\subseteq U^c$ and $U^c$ is closed. $C\subseteq V\subseteq U^c$ but $A\cap U^c=C$ so $C$ is closed relative to $A$, same arguemnt for $B$.\\
The converse: Suppose that $B$ and $C$ are clopen relative to $A$. Then there exists $A$ open in $\C$ such that $B=U\cap A$ and $U\cap C=\phi$. So $C\subseteq U^c$ is closed. Hence $\bar{C}\subseteq U$, therefore $\bar{C}\cap U=\phi$, this implies that $B\subseteq U$. Similarily $C\cap \bar{B}=\phi$. Therefore, $A$ is disconnected by definition 2.
Equivalence of Definition 1 and 2:\\
Definition 1 to 2:\\
$B\subseteq U$ and $C\subseteq V$ with $U\cap V=\phi$. This implies that $B\subseteq V^c$ and $C\subseteq U^c$ wich are both closed. Thus $\bar{B}\cap C=\phi$ similarily $\bar{C}\cap B=\phi$. This gives $1\Rightarrow 2$.\\
Definition 2 to 1:\\
$\bar{B}\cap C=\phi$ and $\bar{C}\cap B=\phi$.\\
\begin{claim}
    For every $z\in\C$ there is a disk $D(z,\delta_z)$ such that $D(z,\delta_z)\cap B=\phi$.
\end{claim}
\pf[Proof of Claim]{
    Suppose not, that means that no matter how small $\delta_z$ is we have $D(z,\delta_z)\cap B\neq\phi$. So there exists a sequence $(b_n)_{n=1}^\infty$ with $\lim_{n\to\infty}b)n=z$. So $z\in\bar{B}$ adn away to $C$. But $\bar{B}\cap C=\phi$. This is a contradiction.
}
By symmetry $\forall z\in B, \exists\delta_z>0,D(z,\delta_z)\cap C=\phi.$ Now $B\subseteq \bigcup_{z\in B}D(z,\frac{\delta_z}{2})=U$ is open $B\subseteq U$ and $C\subseteq \bigcup_{z\in C}D(z,\frac{\delta_z}{2})=V$ is open $C\subseteq V$.
\begin{claim}
    $U\cap V=\phi$ so definition 1 is true.
\end{claim}
\pf[Proof of Claim]{
    Suppose for contradiction that $w\in U\cap V$. So $w\in U$ and $w\in V\Rightarrow$
    \[
        w\in D(z_1,\frac{\delta_{z_1}}{2})
    \]
    \[
        w\in D(Z_2,\frac{\delta_{z_2}}{2})
    \]
    W.L.O.G suppose that $\delta_{z_1}\leq \delta_{z_2}$, then 
    \[
        D(z_2,\delta_{z_2})\cap B=\phi
    \]  
    but
    \[
        |z_2-z_1|\leq |z_2-w|+|w-z_2|\leq\frac{\delta_{z_2}}{2}+\frac{\delta_{z_2}}{2}\leq \delta_{z_2}.
    \]
    Therefore $z_1\in D(z_2,\delta_{z_2})$. This is a contradiction.
}
\thm{Mapping of Compact Sets}{
    Let $K$ be a compact set in $\C$ and $f:K\to \C$ or $\R$ a continuous function, then $f(K)$ is compact. 
}
\pf[Proof of Theorem 1.0.2]{
    Suppose that $K$ is compact with 
    \[
        K=\bigcup_{\alpha\in\mathcal{I}}U_\alpha
    \]
    with $U_\alpha$ being open. Therefore,
    \[
        K\subseteq\bigcup_{\alpha\in\mathcal{I}}f^{-1}(U_\alpha)
    \]
    is an open cover of $K$. By compactness this cover has a finite subcover. Thus $f(K)$ has a finite subcover.
}
\thm{Extreme Value Theorem}{
    If $K$ is compact and $f:K\to\R$ is continuous then $f$ attains its maximum and minimum at some points $a,b\in K$.
}
\thm{Mean Value Theorem}{
    If $A$ is a conencted set in $\C$ and $f:A\to\C$ is continuous then $f(A)$ is connected.
}
\subsection{The Reimann Sphere}
This is just one point compactification of $\C$ with the unit sphere in $\R^3$
\defn{Extended Complex Plane}{
    Wtih the Reimann sphere we defined $\bar{\C}=\C\cup\{\infty\}$.\\
    $D(\infty,R)=\{z\in\C:|z|\leq>R\}$.
}
\[
    z+\infty=\infty,\quad z\cdot\infty=\infty.
\]
are defined.
\[
    0\times\infty,\quad \infty-\infty,\quad\frac{\infty}{\infty}
\]
are not defined.
\defn{Limits with the Reimann Sphere}{
    $\lim_{z\to\infty}f(z)=z_0$ $\forall z>0$ there exists $N$ with $|z|>N$ such that $f()$
}
\end{document}