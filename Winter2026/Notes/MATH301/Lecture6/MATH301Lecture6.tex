\documentclass[12pt]{article}
\usepackage{color,soul}
% Import preambles and macros for notes
% Essential packages for notes
\usepackage{amsmath, amssymb, amsthm}
\usepackage{mathtools}  % for \coloneqq, etc.
\usepackage{geometry}   % Better page margins
\usepackage{parskip}    % Better paragraph spacing
\usepackage{microtype}  % Better typography
\usepackage{enumitem}   % Customize lists
\usepackage{hyperref}   % Clickable links
\usepackage{booktabs}   % Better tables
\usepackage{tcolorbox}  % For colored boxes/theorems

% Page layout for notes
\geometry{a4paper, margin=1in}
\setlength{\parskip}{0.8em}

% Theorem environments with shared numbering
\newtheorem{theorem}{Theorem}[subsection]  % Number within sections: 2.3.1, 2.3.2, etc.

% All other environments share the same counter as theorem
\newtheorem{lemma}[theorem]{Lemma}
\newtheorem{proposition}[theorem]{Proposition}
\newtheorem{corollary}[theorem]{Corollary}
\newtheorem{definition}[theorem]{Definition}
\newtheorem{example}[theorem]{Example}
\newtheorem{remark}[theorem]{Remark}
\newtheorem{claim}[theorem]{Claim}

% Custom colors for notes
\usepackage{xcolor}
\definecolor{note-blue}{RGB}{220, 230, 255}
\definecolor{theorem-green}{RGB}{220, 255, 220}
% Math notation shortcuts for notes
\newcommand{\R}{\mathbb{R}}
\newcommand{\C}{\mathbb{C}}
\newcommand{\Q}{\mathbb{Q}}
\newcommand{\Z}{\mathbb{Z}}
\newcommand{\N}{\mathbb{N}}

% Calculus
\newcommand{\diff}{\mathop{}\!\mathrm{d}}
\newcommand{\deriv}[2]{\frac{\mathrm{d}#1}{\mathrm{d}#2}}
\newcommand{\pderiv}[2]{\frac{\partial #1}{\partial #2}}

% Linear Algebra
\newcommand{\inner}[2]{\langle #1, #2 \rangle}
\newcommand{\norm}[1]{\| #1 \|}
\newcommand{\tr}{\operatorname{tr}}
\newcommand{\spn}{\operatorname{span}}
\newcommand{\rank}{\operatorname{rank}}
\newcommand{\nullity}{\operatorname{nullity}}

% Logic
\newcommand{\contra}{\Rightarrow\Leftarrow}

% Custom commands for notes
\newcommand{\todo}[1]{\textcolor{red}{[TODO: #1]}}
\newcommand{\important}[1]{\textbf{\textcolor{blue}{#1}}}
% Theorem system 
% Theorem System original by https://github.com/kcajc/math-notes-template, this is modified 
% The following boxes are provided:
%   Definition:     \defn 
%   Theorem:        \thm 
%   Lemma:          \lem
%   Corollary:      \cor
%   Proposition:    \prop   
%   Claim:          \clm
%   Fact:           \fact
%   Proof:          \pf
%   Example:        \ex
%   Remark:         \rmk (sentence), \rmkb (block)
% Suffix
%   r:              Allow Theorem/Definition to be referenced, e.g. thmr
%   p:              Add a short proof block for Lemma, Corollary, Proposition or Claim, e.g. lemp
%                   For theorems, use \pf for proof blocks

% Definition - subsection numbering for 1.1.1, 1.1.2
\newtcbtheorem[number within=subsection]{mydefinition}{Definition}
{
    colbacktitle=green!20!white,
    colback=green!10!white,
    coltitle=black,
    fonttitle=\bfseries\large,
}{defn}

\NewDocumentCommand{\defn}{m+m}{
    \begin{mydefinition}{#1}{}
        #2
    \end{mydefinition}
}

\NewDocumentCommand{\defnr}{mm+m}{
    \begin{mydefinition}{#1}{#2}
        #3
    \end{mydefinition}
}

% Theorem - subsection numbering
\newtcbtheorem[use counter from=mydefinition]{mytheorem}{Theorem}
{
    colbacktitle=cyan!20!white,
    colback=cyan!10!white,
    coltitle=black,
    fonttitle=\bfseries\large,
}{thm}

\NewDocumentCommand{\thm}{m+m}{
    \begin{mytheorem}{#1}{}
        #2
    \end{mytheorem}
}

\NewDocumentCommand{\thmr}{mm+m}{
    \begin{mytheorem}{#1}{#2}
        #3
    \end{mytheorem}
}

% Lemma - subsection numbering
\newtcbtheorem[use counter from=mydefinition]{mylemma}{Lemma}
{
    colbacktitle=violet!20!white,
    colback=violet!10!white,
    coltitle=black,
    fonttitle=\bfseries\large,
}{lem}

\NewDocumentCommand{\lem}{m+m}{
    \begin{mylemma}{#1}{}
        #2
    \end{mylemma}
}

% Improved proof environments with consistent QED placement
\newenvironment{lempf}{
    \par\noindent{\it \textbf{Proof for Lemma.}}\par\nopagebreak
    \begin{list}{}{\setlength\leftmargin{1em}\setlength\rightmargin{0em}}
    \item\relax
}{
    \hfill$\qed$\end{list}
}

\NewDocumentCommand{\lemp}{m+m+m}{
    \begin{mylemma}{#1}{}
        #2
    \end{mylemma}
    \begin{lempf}
        #3
    \end{lempf}
}

% Corollary - subsection numbering
\newtcbtheorem[use counter from=mydefinition]{mycorollary}{Corollary}
{
    colbacktitle=orange!20!white,
    colback=orange!10!white,
    coltitle=black,
    fonttitle=\bfseries\large,
}{cor}

\NewDocumentCommand{\cor}{+m}{
    \begin{mycorollary}{}{}
        #1
    \end{mycorollary}
}

\newenvironment{corpf}{
    \par\noindent{\it \textbf{Proof for Corollary.}}\par\nopagebreak
    \begin{list}{}{\setlength\leftmargin{1em}\setlength\rightmargin{0em}}
    \item\relax
}{
    \hfill$\qed$\end{list}
}

\NewDocumentCommand{\corp}{m+m+m}{
    \begin{mycorollary}{}{}
        #1
    \end{mycorollary}
    \begin{corpf}
        #2
    \end{corpf}
}

% Proposition - subsection numbering
\newtcbtheorem[use counter from=mydefinition]{myproposition}{Proposition}
{
    colbacktitle=yellow!30!white,
    colback=yellow!20!white,
    coltitle=black,
    fonttitle=\bfseries\large,
}{prop}

\NewDocumentCommand{\prop}{+m}{
    \begin{myproposition}{}{}
        #1
    \end{myproposition}
}

\newenvironment{proppf}{
    \par\noindent{\it \textbf{Proof for Proposition.}}\par\nopagebreak
    \begin{list}{}{\setlength\leftmargin{1em}\setlength\rightmargin{0em}}
    \item\relax
}{
    \hfill$\qed$\end{list}
}

\NewDocumentCommand{\propp}{+m+m}{
    \begin{myproposition}{}{}
        #1
    \end{myproposition}
    \begin{proppf}
        #2
    \end{proppf}
}

% Claim - subsection numbering
\newtcbtheorem[use counter from=mydefinition]{myclaim}{Claim}
{
    colbacktitle=pink!30!white,
    colback=pink!20!white,
    coltitle=black,
    fonttitle=\bfseries\large,
}{clm}

\NewDocumentCommand{\clm}{m+m}{
    \begin{myclaim}{#1}{}
        #2
    \end{myclaim}
}

\newenvironment{clmpf}{
    \par\noindent{\it \textbf{Proof for Claim.}}\par\nopagebreak
    \begin{list}{}{\setlength\leftmargin{1em}\setlength\rightmargin{0em}}
    \item\relax
}{
    \hfill$\qed$\end{list}
}

\NewDocumentCommand{\clmp}{m+m+m}{
    \begin{myclaim}{#1}{}
        #2
    \end{myclaim}
    \begin{clmpf}
        #3
    \end{clmpf}
}

% Fact - subsection numbering
\newtcbtheorem[use counter from=mydefinition]{myfact}{Fact}
{
    colbacktitle=purple!20!white,
    colback=purple!10!white,
    coltitle=black,
    fonttitle=\bfseries\large,
}{fact}

\NewDocumentCommand{\fact}{+m}{
    \begin{myfact}{}{}
        #1
    \end{myfact}
}

% Proof - customizable name
\NewDocumentEnvironment{custompf}{m}
{
    \par\noindent{\it \textbf{#1}}\par\nopagebreak
    \begin{list}{}{\setlength\leftmargin{1em}\setlength\rightmargin{0em}}
    \item\relax
}
{
    \hfill$\qed$\end{list}
}

\NewDocumentCommand{\pf}{O{Proof}+m}{
    \begin{custompf}{#1.}
        #2
    \end{custompf}
}

% Example - improved environment
\newenvironment{example}{
    \par\vspace{5pt}
    \noindent\textbf{Example.}\par\nopagebreak
    \begin{list}{}{\setlength\leftmargin{1em}\setlength\rightmargin{0em}}
    \item\relax
}{
    \end{list}\vspace{5pt}
}

\NewDocumentCommand{\ex}{+m}{
    \begin{example}
        #1
    \end{example}
}

% Remark
\NewDocumentCommand{\rmk}{+m}{
    {\it \color{blue!50!white}#1}
}

% Remark block - improved environment
\newenvironment{remark}{
    \par\vspace{5pt}
    \noindent\textbf{Remark.}\par\nopagebreak
    \begin{list}{}{\setlength\leftmargin{1em}\setlength\rightmargin{0em}}
    \item\relax
}{
    \end{list}\vspace{5pt}
}

\NewDocumentCommand{\rmkb}{+m}{
    \begin{remark}
        #1
    \end{remark}
}

\title{MATH 301 Lecture 6}
\author{Deepak Jassal}
\date{January 28\textsuperscript{rd}, 2026}

\begin{document}
\setcounter{section}{1}
\maketitle
\defn{Closure}{
    Let $A\subseteq\C$. Then the closure of $A$ denoted as $\bar{A}$ is the set of all limits of convergent sequences in $A$.\\
    \textit{Note.} $A\subseteq \bar{A}$ because if $z\in A$ we can take constant sequence $z_n=z$ for all $n\in\N$. So $\lim_{n\to\infty}z_n=z\in A$. 
}
\ex{
    $A=\left\{\frac{1}{n}:n\in\N\right\}$, $\bar{A}=A\cup\left\{0\right\}$.
}
\propp{$\bar{A}$ is closed}{
    Let $(z_n)_{n=1}^\infty$ be a sequence in $\bar{A}$ we need to show that if $\lim_{n\to\infty}z_n=c$ then $c\in\bar{A}$.
    \begin{enumerate}
        \item[Case 1.] If $(z_n)_{n=1}^\infty z_n\subseteq A$ then $\lim_{n\to\infty}z_n\in\bar{A}$ by definition.
        \item[Case 2.] Otherwise define a sequence $(z_n')_{n=1}^\infty\in\bar{A}$ by
        \[
            z_n'=
            \begin{cases}
                z_n&\text{if } z_n\in A\\
                w_n&\text{such that } |w_n-z_n|<\frac{1}{n},\,w_n\in A,\,z_n\in\bar{A}\setminus A.    
            \end{cases}
        \] 
    \end{enumerate}
    So
    \begin{align*}
        (z_n')_{n=1}^\infty&\subseteq\\
        \lim_{n\to\infty}z_n'&=\lim_{n\to\infty}(z_n+(z_n'-z_n))\\
        &\lim_{n\to\infty}z_n+\lim_{n\to\infty}(z_n'-z_n)\\
        &=c+0=c.
    \end{align*}
    Therefore, $\lim_{n\to\infty}z_n'=c\in\bar{A}$.
}
\cor{$\bar{\bar{A}}=\bar{A}$}
\prop{
    $\bar{A}$ is the smallest closed set containing $A$. So,
    \[
        \bar{A}=\bigcap_{c\subseteq C}C
    \]
    such that $A\subseteq C$ and $C$ is closed.
}
\subsection{More About Continuous Functions}
So far we know that $f:\C\to\C$ is continuous if
\begin{enumerate}
    \item $z_n\to z\Rightarrow f(z_n)\to f(z)$
    \item \textit{iff} $f^{-1}(U)$ is open for all $U\subseteq\C$ that are open
    \item \textit{iff} $f^{-1}(U)$ is closed for all $U\subseteq\C$ that are closed
\end{enumerate}
This has to be changed if $f:A\to\C$ with $A\neq\C$.
\defn{Inherited Topology}{
    Topology on a subset of $\C$ is inherited from $\C$
    \begin{enumerate}
        \item Let $A\subseteq\C$ and $A_1\subseteq A$. Then $A_1$ os ``open relative to $A$'' if there eists an open set $U$ in $\C$ such that $A_1=A\cap U$,
        \item Let $A\subseteq\C$ and $A_1\subseteq A$. Then $A_1$ os ``closed relative to $A$'' if there eists a closed set $C$ in $\C$ such that $A_1=A\cap C$.
    \end{enumerate}
    $A$ with this topology is considered to be a space. Note that $A$ is opena and closed in the inherited topology.
}
Now suppose that
\[
    f:A\to\C.
\]
The following are equivalent statements
\begin{enumerate}
    \item $f$ is continuous on $A$ (limit definition)
    \item $f^{-1}(U)$ is open relative to $A$ for all open $U$ in $\C$
    \item $f^{-1}(U)$ is closed relative to $A$ for all closed $U$ in $\C$
\end{enumerate}
\subsection{Connected Sets in $\C$}
\defn{Connected Sets}{
    A set $A\subseteq\C$ is connected if $A=B\cup C$ with $B\cap C=\phi$, $b\neq\phi$ and $c\neq\phi$ and there are open sets $u,V\subseteq\C$ such that $U\cap V=\phi$ and $B\subseteq U$ and $C\subseteq V$. We say that $U,V$ separate $B,C$. 
}
\defn{Rudin's Definition}{
    A set $A$ is separated (disconnected) if $A=B\cup C$ with $B\neq\phi$ and $C\neq\phi$ and
    \begin{enumerate}
        \item $\bar{B}\cap C=\phi$
        \item $B\cap\bar{C}=\phi$.
    \end{enumerate}
}
These two definitions are equivalent. Proof will be shown later
\defn{Path-Connected}{
    A set $A$ is path-connected if for every two points $a$ and $b$ in $A$ if there extists a continuous path
    \[
        \gamma:[0,1]\to A
    \]
    such that $\gamma(0)=a$ and $\gamma(1)=b$
}
\propp{If $A$ is path connected then it is connected (according to the first two definition)}{
    Suppose for contradiction that $A$ is path-connected and also disconnected. So $A=B\cup C$, $B\neq\phi$, $C\neq\phi$, $B\subseteq U$, $C\subseteq V$ and $U\cap V=\phi$. Assume that $a\in B$ and $b\in C$. $\gamma^{-1}(U)$ is open and $a\in\gamma^{-1}(U)$ and $\gamma^{-1}(V)$ is open and $b\in\gamma^{-1}(V)$ (open relative to [0,1]). So $[0,1]=\gamma^{-1}(U)\cup\gamma^{-1}(V)$ with $\gamma^{-1}(U)\cap\gamma^{-1}(V)$. This implies that $[0,1]\subseteq\R$ is disconnected
}
The converse is not true.
\ex{
    Let $B=\{0+iy:-1\leq y\leq1\}$ and $C=\{(t,\sin\left(\frac{1}{t}\right)):t\in(0,1)\}$.
    \begin{claim}
        $\bar{C}=C\cup B$ is connected do $\bar{C}\cap B\neq\phi$, but not path-connected
    \end{claim}
    \pf[Proof of Claim]{
        Suppose there exists $\gamma:[0,1]\to $ is continuous such that $\gamma(0)=Q$ and $\gamma(1)=A$.\\
        For all $\varepsilon>0$ there exists $\delta>0$ such that
        \[
            |t-1|<\delta\Rightarrow|\gamma(t)-\gamma(1)|<\varepsilon.
        \]
        The graph of $1-\delta<t\leq Q$ should be in the disk $D(Q,\varepsilon)$.
    }
}
\thm{Path Connectedness}{
    If $A$ is open and connected then $A$ is path-connected. Furthermore, we can always find a differentiable path $\gamma:[0,1]\to A$ connecting points $a,b$.
}
\pf[Proof of the Theorem]{
    Show that the set of all points of $A$ that can be connected by differentiable paths of $A$ is open.
}
\end{document}