\documentclass[12pt]{article}
\usepackage{color,soul}
% Import preambles and macros for notes
% Essential packages for notes
\usepackage{amsmath, amssymb, amsthm}
\usepackage{mathtools}  % for \coloneqq, etc.
\usepackage{geometry}   % Better page margins
\usepackage{parskip}    % Better paragraph spacing
\usepackage{microtype}  % Better typography
\usepackage{enumitem}   % Customize lists
\usepackage{hyperref}   % Clickable links
\usepackage{booktabs}   % Better tables
\usepackage{tcolorbox}  % For colored boxes/theorems

% Page layout for notes
\geometry{a4paper, margin=1in}
\setlength{\parskip}{0.8em}

% Theorem environments with shared numbering
\newtheorem{theorem}{Theorem}[subsection]  % Number within sections: 2.3.1, 2.3.2, etc.

% All other environments share the same counter as theorem
\newtheorem{lemma}[theorem]{Lemma}
\newtheorem{proposition}[theorem]{Proposition}
\newtheorem{corollary}[theorem]{Corollary}
\newtheorem{definition}[theorem]{Definition}
\newtheorem{example}[theorem]{Example}
\newtheorem{remark}[theorem]{Remark}
\newtheorem{claim}[theorem]{Claim}

% Custom colors for notes
\usepackage{xcolor}
\definecolor{note-blue}{RGB}{220, 230, 255}
\definecolor{theorem-green}{RGB}{220, 255, 220}
% Math notation shortcuts for notes
\newcommand{\R}{\mathbb{R}}
\newcommand{\C}{\mathbb{C}}
\newcommand{\Q}{\mathbb{Q}}
\newcommand{\Z}{\mathbb{Z}}
\newcommand{\N}{\mathbb{N}}

% Calculus
\newcommand{\diff}{\mathop{}\!\mathrm{d}}
\newcommand{\deriv}[2]{\frac{\mathrm{d}#1}{\mathrm{d}#2}}
\newcommand{\pderiv}[2]{\frac{\partial #1}{\partial #2}}

% Linear Algebra
\newcommand{\inner}[2]{\langle #1, #2 \rangle}
\newcommand{\norm}[1]{\| #1 \|}
\newcommand{\tr}{\operatorname{tr}}
\newcommand{\spn}{\operatorname{span}}
\newcommand{\rank}{\operatorname{rank}}
\newcommand{\nullity}{\operatorname{nullity}}

% Logic
\newcommand{\contra}{\Rightarrow\Leftarrow}

% Custom commands for notes
\newcommand{\todo}[1]{\textcolor{red}{[TODO: #1]}}
\newcommand{\important}[1]{\textbf{\textcolor{blue}{#1}}}
% Theorem system 
% Theorem System original by https://github.com/kcajc/math-notes-template, this is modified 
% The following boxes are provided:
%   Definition:     \defn 
%   Theorem:        \thm 
%   Lemma:          \lem
%   Corollary:      \cor
%   Proposition:    \prop   
%   Claim:          \clm
%   Fact:           \fact
%   Proof:          \pf
%   Example:        \ex
%   Remark:         \rmk (sentence), \rmkb (block)
% Suffix
%   r:              Allow Theorem/Definition to be referenced, e.g. thmr
%   p:              Add a short proof block for Lemma, Corollary, Proposition or Claim, e.g. lemp
%                   For theorems, use \pf for proof blocks

% Definition - subsection numbering for 1.1.1, 1.1.2
\newtcbtheorem[number within=subsection]{mydefinition}{Definition}
{
    colbacktitle=green!20!white,
    colback=green!10!white,
    coltitle=black,
    fonttitle=\bfseries\large,
}{defn}

\NewDocumentCommand{\defn}{m+m}{
    \begin{mydefinition}{#1}{}
        #2
    \end{mydefinition}
}

\NewDocumentCommand{\defnr}{mm+m}{
    \begin{mydefinition}{#1}{#2}
        #3
    \end{mydefinition}
}

% Theorem - subsection numbering
\newtcbtheorem[use counter from=mydefinition]{mytheorem}{Theorem}
{
    colbacktitle=cyan!20!white,
    colback=cyan!10!white,
    coltitle=black,
    fonttitle=\bfseries\large,
}{thm}

\NewDocumentCommand{\thm}{m+m}{
    \begin{mytheorem}{#1}{}
        #2
    \end{mytheorem}
}

\NewDocumentCommand{\thmr}{mm+m}{
    \begin{mytheorem}{#1}{#2}
        #3
    \end{mytheorem}
}

% Lemma - subsection numbering
\newtcbtheorem[use counter from=mydefinition]{mylemma}{Lemma}
{
    colbacktitle=violet!20!white,
    colback=violet!10!white,
    coltitle=black,
    fonttitle=\bfseries\large,
}{lem}

\NewDocumentCommand{\lem}{m+m}{
    \begin{mylemma}{#1}{}
        #2
    \end{mylemma}
}

% Improved proof environments with consistent QED placement
\newenvironment{lempf}{
    \par\noindent{\it \textbf{Proof for Lemma.}}\par\nopagebreak
    \begin{list}{}{\setlength\leftmargin{1em}\setlength\rightmargin{0em}}
    \item\relax
}{
    \hfill$\qed$\end{list}
}

\NewDocumentCommand{\lemp}{m+m+m}{
    \begin{mylemma}{#1}{}
        #2
    \end{mylemma}
    \begin{lempf}
        #3
    \end{lempf}
}

% Corollary - subsection numbering
\newtcbtheorem[use counter from=mydefinition]{mycorollary}{Corollary}
{
    colbacktitle=orange!20!white,
    colback=orange!10!white,
    coltitle=black,
    fonttitle=\bfseries\large,
}{cor}

\NewDocumentCommand{\cor}{+m}{
    \begin{mycorollary}{}{}
        #1
    \end{mycorollary}
}

\newenvironment{corpf}{
    \par\noindent{\it \textbf{Proof for Corollary.}}\par\nopagebreak
    \begin{list}{}{\setlength\leftmargin{1em}\setlength\rightmargin{0em}}
    \item\relax
}{
    \hfill$\qed$\end{list}
}

\NewDocumentCommand{\corp}{m+m+m}{
    \begin{mycorollary}{}{}
        #1
    \end{mycorollary}
    \begin{corpf}
        #2
    \end{corpf}
}

% Proposition - subsection numbering
\newtcbtheorem[use counter from=mydefinition]{myproposition}{Proposition}
{
    colbacktitle=yellow!30!white,
    colback=yellow!20!white,
    coltitle=black,
    fonttitle=\bfseries\large,
}{prop}

\NewDocumentCommand{\prop}{+m}{
    \begin{myproposition}{}{}
        #1
    \end{myproposition}
}

\newenvironment{proppf}{
    \par\noindent{\it \textbf{Proof for Proposition.}}\par\nopagebreak
    \begin{list}{}{\setlength\leftmargin{1em}\setlength\rightmargin{0em}}
    \item\relax
}{
    \hfill$\qed$\end{list}
}

\NewDocumentCommand{\propp}{+m+m}{
    \begin{myproposition}{}{}
        #1
    \end{myproposition}
    \begin{proppf}
        #2
    \end{proppf}
}

% Claim - subsection numbering
\newtcbtheorem[use counter from=mydefinition]{myclaim}{Claim}
{
    colbacktitle=pink!30!white,
    colback=pink!20!white,
    coltitle=black,
    fonttitle=\bfseries\large,
}{clm}

\NewDocumentCommand{\clm}{m+m}{
    \begin{myclaim}{#1}{}
        #2
    \end{myclaim}
}

\newenvironment{clmpf}{
    \par\noindent{\it \textbf{Proof for Claim.}}\par\nopagebreak
    \begin{list}{}{\setlength\leftmargin{1em}\setlength\rightmargin{0em}}
    \item\relax
}{
    \hfill$\qed$\end{list}
}

\NewDocumentCommand{\clmp}{m+m+m}{
    \begin{myclaim}{#1}{}
        #2
    \end{myclaim}
    \begin{clmpf}
        #3
    \end{clmpf}
}

% Fact - subsection numbering
\newtcbtheorem[use counter from=mydefinition]{myfact}{Fact}
{
    colbacktitle=purple!20!white,
    colback=purple!10!white,
    coltitle=black,
    fonttitle=\bfseries\large,
}{fact}

\NewDocumentCommand{\fact}{+m}{
    \begin{myfact}{}{}
        #1
    \end{myfact}
}

% Proof - customizable name
\NewDocumentEnvironment{custompf}{m}
{
    \par\noindent{\it \textbf{#1}}\par\nopagebreak
    \begin{list}{}{\setlength\leftmargin{1em}\setlength\rightmargin{0em}}
    \item\relax
}
{
    \hfill$\qed$\end{list}
}

\NewDocumentCommand{\pf}{O{Proof}+m}{
    \begin{custompf}{#1.}
        #2
    \end{custompf}
}

% Example - improved environment
\newenvironment{example}{
    \par\vspace{5pt}
    \noindent\textbf{Example.}\par\nopagebreak
    \begin{list}{}{\setlength\leftmargin{1em}\setlength\rightmargin{0em}}
    \item\relax
}{
    \end{list}\vspace{5pt}
}

\NewDocumentCommand{\ex}{+m}{
    \begin{example}
        #1
    \end{example}
}

% Remark
\NewDocumentCommand{\rmk}{+m}{
    {\it \color{blue!50!white}#1}
}

% Remark block - improved environment
\newenvironment{remark}{
    \par\vspace{5pt}
    \noindent\textbf{Remark.}\par\nopagebreak
    \begin{list}{}{\setlength\leftmargin{1em}\setlength\rightmargin{0em}}
    \item\relax
}{
    \end{list}\vspace{5pt}
}

\NewDocumentCommand{\rmkb}{+m}{
    \begin{remark}
        #1
    \end{remark}
}

\title{MATH 301 Lecture 3}
\author{Deepak Jassal}
\date{January 16\textsuperscript{th}, 2026}

\begin{document}
\maketitle
\stepcounter{section}
\subsection{$\log z$}
$e^z$ is a periodic function with period $2\pi i$ ($e^{z+2\pi i}=e^z$). 
\[
    \C\overset{e^z}{\mapsto}\C\setminus \{0\}.
\]
However this is not one-to-one.\\
In the real case 
\[
    \R\underset{\ln x}{\overset{e^x}{\xleftrightharpoons{\hspace{1cm}}}}\R_{>0},
\]
\[
    e^{\ln x}=x,\quad x>0,
\]
\[
    \ln e^x=x,\quad x>0.
\]

Let $A_{y_0}=\{x+iy:x,y\in\R,y_o\leq y<y_0+2\pi\}$. In this case we have 
\[
    A_{y_0}\underset{\ln x}{\overset{e^x}{\xleftrightharpoons{\hspace{1cm}}}}\C\setminus \{0\},
\]
this function is one-to-one and onto.
\defn{Complex Logarithm}{
    Given $y_0\in\R$ we can define
    \[
        \log: \C\setminus \{0\}\mapsto A_{y_0}.
    \]
    We say that this is a particulae branch of the logarithm, corresponding to $A_{y_0}$ or $y_0$.\\
    If we don't indicate the branch, then we can consider $\log z$ as a multivalued ``function''.
}
\defn{Branch of the Logarithm}{
    If \[y_o\leq \mathrm{arg}z< y_0+2\pi,\] then \[\log z =\log|z|+i\mathrm{arg}z.\] If the branch is not specified then \[\log z =\log|z|+i\mathrm{arg}z+2\pi in,\quad n\in\Z.\]
}
\ex{
    State all the values of $\log1$.
    \[
        \log1=0+2\pi in,\quad n\in\Z
    \]
}
\ex{
    State all the values of $\log1$.
    \[
        \log i=\log|i|+\frac{i\pi}{2}+2\pi in,\quad n\in\Z
    \]
    \[
        \log i=\frac{i\pi}{2}+2\pi in,\quad n\in\Z.
    \]
}
\textit{Note.} $\log z$ is the ``inverse'' of $e^z$ in the following way
\begin{enumerate}
    \item $e^{\log z}=z$ no matter which branch of $\log$ we take.
    \item $\log e^z=z$ \textit{if and only if} $z$ is the branch of a $\log$ $y_0\leq \mathrm{arg}z< y_0+2\pi$.
\end{enumerate} 
The branch of a $\log z$ is the image of $\log z$ to the range to which it maps.
\prop{
    $\log(z_1z_2)=\log z_1+\log z_2$ modulo $2\pi i$.
}
\defn{Complex $a^b$}{
    $a^b=e^{b\log a}$ is a multivalued function
    \begin{enumerate}
        \item $a^b$ is single valued \textit{if and only if} $b\in\Z$.
        \[
            e^{b\log a}=e^{b(\log|a|+\mathrm{arg}a+2\pi im)}=e^{b\log|a|+\mathrm{arg}a}e^{2\pi in}=e^{b\log|a|+\mathrm{arg}a}.
        \]
        \item If $b\in\R\setminus\Q$ then $e^{b\log a}$ has many values.
        \item If $\Im b\neq0$
        \[
            e^{b2\pi ni}=e^{(b_1+b_2i)2\pi ni}=e^{2b_1\pi ni-2b_2\pi n}.
        \]
        \item If $b=\frac{p}{q}$, $(p,q)=1$, $q>0$. Then $a^b$ has exactly $q$ distinct values
        \[
            e^{b\log a}=e^{b(\log a +2\pi ni)}=e^{b\log a + \frac{2\pi niq}{p}}.
        \]
        Since we have $(p,q)=1$ and $n\in \Z$ $np$ will give all remainders mod $q$.
        \[
            np=kq+r,\quad r=1,2,3,\dots,q-1.
        \]
        \[
            e^{\frac{2\pi niq}{p}}=e^{2\pi ik+\frac{2\pi riq}{p}}.
        \]
    \end{enumerate}
}
$e^{ba}$ and $(e^a)^b$ are not necessarily the same thing.
\[
    (e^a)^b=\begin{cases}
        e^{ab} & \text{if $a$ is in the image of log},\\
        e^{ba+2\pi inb} & \text{otherwise}.
    \end{cases}
\]
\section{The Function $\sqrt[n]{z}=z^\frac{1}{n}$}
\begin{align*}
    z^{\frac{1}{n}}=e^{\frac{1}{n}\log z}&=e^{\frac{1}{n}\left(\log |z|+i\arg z +2\pi mi\right)}\\
    &=e^{\frac{1}{n}\log |z|}e^{\frac{i\arg z}{n}}e^{\frac{2\pi mi}{n}}\\
    &=|z|^\frac{1}{n}e^\frac{i\theta}{n}e^\frac{2\pi ki}{n}
\end{align*}
$k=0,1,\dots,n-1$, and $m\in\Z$.
\end{document}