\documentclass[12pt]{article}
\usepackage{color,soul}
% Import preambles and macros for notes
% Essential packages for notes
\usepackage{amsmath, amssymb, amsthm}
\usepackage{mathtools}  % for \coloneqq, etc.
\usepackage{geometry}   % Better page margins
\usepackage{parskip}    % Better paragraph spacing
\usepackage{microtype}  % Better typography
\usepackage{enumitem}   % Customize lists
\usepackage{hyperref}   % Clickable links
\usepackage{booktabs}   % Better tables
\usepackage{tcolorbox}  % For colored boxes/theorems

% Page layout for notes
\geometry{a4paper, margin=1in}
\setlength{\parskip}{0.8em}

% Theorem environments with shared numbering
\newtheorem{theorem}{Theorem}[subsection]  % Number within sections: 2.3.1, 2.3.2, etc.

% All other environments share the same counter as theorem
\newtheorem{lemma}[theorem]{Lemma}
\newtheorem{proposition}[theorem]{Proposition}
\newtheorem{corollary}[theorem]{Corollary}
\newtheorem{definition}[theorem]{Definition}
\newtheorem{example}[theorem]{Example}
\newtheorem{remark}[theorem]{Remark}
\newtheorem{claim}[theorem]{Claim}

% Custom colors for notes
\usepackage{xcolor}
\definecolor{note-blue}{RGB}{220, 230, 255}
\definecolor{theorem-green}{RGB}{220, 255, 220}
% Math notation shortcuts for notes
\newcommand{\R}{\mathbb{R}}
\newcommand{\C}{\mathbb{C}}
\newcommand{\Q}{\mathbb{Q}}
\newcommand{\Z}{\mathbb{Z}}
\newcommand{\N}{\mathbb{N}}

% Calculus
\newcommand{\diff}{\mathop{}\!\mathrm{d}}
\newcommand{\deriv}[2]{\frac{\mathrm{d}#1}{\mathrm{d}#2}}
\newcommand{\pderiv}[2]{\frac{\partial #1}{\partial #2}}

% Linear Algebra
\newcommand{\inner}[2]{\langle #1, #2 \rangle}
\newcommand{\norm}[1]{\| #1 \|}
\newcommand{\tr}{\operatorname{tr}}
\newcommand{\spn}{\operatorname{span}}
\newcommand{\rank}{\operatorname{rank}}
\newcommand{\nullity}{\operatorname{nullity}}

% Logic
\newcommand{\contra}{\Rightarrow\Leftarrow}

% Custom commands for notes
\newcommand{\todo}[1]{\textcolor{red}{[TODO: #1]}}
\newcommand{\important}[1]{\textbf{\textcolor{blue}{#1}}}
% Theorem system 
% Theorem System original by https://github.com/kcajc/math-notes-template, this is modified 
% The following boxes are provided:
%   Definition:     \defn 
%   Theorem:        \thm 
%   Lemma:          \lem
%   Corollary:      \cor
%   Proposition:    \prop   
%   Claim:          \clm
%   Fact:           \fact
%   Proof:          \pf
%   Example:        \ex
%   Remark:         \rmk (sentence), \rmkb (block)
% Suffix
%   r:              Allow Theorem/Definition to be referenced, e.g. thmr
%   p:              Add a short proof block for Lemma, Corollary, Proposition or Claim, e.g. lemp
%                   For theorems, use \pf for proof blocks

% Definition - subsection numbering for 1.1.1, 1.1.2
\newtcbtheorem[number within=subsection]{mydefinition}{Definition}
{
    colbacktitle=green!20!white,
    colback=green!10!white,
    coltitle=black,
    fonttitle=\bfseries\large,
}{defn}

\NewDocumentCommand{\defn}{m+m}{
    \begin{mydefinition}{#1}{}
        #2
    \end{mydefinition}
}

\NewDocumentCommand{\defnr}{mm+m}{
    \begin{mydefinition}{#1}{#2}
        #3
    \end{mydefinition}
}

% Theorem - subsection numbering
\newtcbtheorem[use counter from=mydefinition]{mytheorem}{Theorem}
{
    colbacktitle=cyan!20!white,
    colback=cyan!10!white,
    coltitle=black,
    fonttitle=\bfseries\large,
}{thm}

\NewDocumentCommand{\thm}{m+m}{
    \begin{mytheorem}{#1}{}
        #2
    \end{mytheorem}
}

\NewDocumentCommand{\thmr}{mm+m}{
    \begin{mytheorem}{#1}{#2}
        #3
    \end{mytheorem}
}

% Lemma - subsection numbering
\newtcbtheorem[use counter from=mydefinition]{mylemma}{Lemma}
{
    colbacktitle=violet!20!white,
    colback=violet!10!white,
    coltitle=black,
    fonttitle=\bfseries\large,
}{lem}

\NewDocumentCommand{\lem}{m+m}{
    \begin{mylemma}{#1}{}
        #2
    \end{mylemma}
}

% Improved proof environments with consistent QED placement
\newenvironment{lempf}{
    \par\noindent{\it \textbf{Proof for Lemma.}}\par\nopagebreak
    \begin{list}{}{\setlength\leftmargin{1em}\setlength\rightmargin{0em}}
    \item\relax
}{
    \hfill$\qed$\end{list}
}

\NewDocumentCommand{\lemp}{m+m+m}{
    \begin{mylemma}{#1}{}
        #2
    \end{mylemma}
    \begin{lempf}
        #3
    \end{lempf}
}

% Corollary - subsection numbering
\newtcbtheorem[use counter from=mydefinition]{mycorollary}{Corollary}
{
    colbacktitle=orange!20!white,
    colback=orange!10!white,
    coltitle=black,
    fonttitle=\bfseries\large,
}{cor}

\NewDocumentCommand{\cor}{+m}{
    \begin{mycorollary}{}{}
        #1
    \end{mycorollary}
}

\newenvironment{corpf}{
    \par\noindent{\it \textbf{Proof for Corollary.}}\par\nopagebreak
    \begin{list}{}{\setlength\leftmargin{1em}\setlength\rightmargin{0em}}
    \item\relax
}{
    \hfill$\qed$\end{list}
}

\NewDocumentCommand{\corp}{m+m+m}{
    \begin{mycorollary}{}{}
        #1
    \end{mycorollary}
    \begin{corpf}
        #2
    \end{corpf}
}

% Proposition - subsection numbering
\newtcbtheorem[use counter from=mydefinition]{myproposition}{Proposition}
{
    colbacktitle=yellow!30!white,
    colback=yellow!20!white,
    coltitle=black,
    fonttitle=\bfseries\large,
}{prop}

\NewDocumentCommand{\prop}{+m}{
    \begin{myproposition}{}{}
        #1
    \end{myproposition}
}

\newenvironment{proppf}{
    \par\noindent{\it \textbf{Proof for Proposition.}}\par\nopagebreak
    \begin{list}{}{\setlength\leftmargin{1em}\setlength\rightmargin{0em}}
    \item\relax
}{
    \hfill$\qed$\end{list}
}

\NewDocumentCommand{\propp}{+m+m}{
    \begin{myproposition}{}{}
        #1
    \end{myproposition}
    \begin{proppf}
        #2
    \end{proppf}
}

% Claim - subsection numbering
\newtcbtheorem[use counter from=mydefinition]{myclaim}{Claim}
{
    colbacktitle=pink!30!white,
    colback=pink!20!white,
    coltitle=black,
    fonttitle=\bfseries\large,
}{clm}

\NewDocumentCommand{\clm}{m+m}{
    \begin{myclaim}{#1}{}
        #2
    \end{myclaim}
}

\newenvironment{clmpf}{
    \par\noindent{\it \textbf{Proof for Claim.}}\par\nopagebreak
    \begin{list}{}{\setlength\leftmargin{1em}\setlength\rightmargin{0em}}
    \item\relax
}{
    \hfill$\qed$\end{list}
}

\NewDocumentCommand{\clmp}{m+m+m}{
    \begin{myclaim}{#1}{}
        #2
    \end{myclaim}
    \begin{clmpf}
        #3
    \end{clmpf}
}

% Fact - subsection numbering
\newtcbtheorem[use counter from=mydefinition]{myfact}{Fact}
{
    colbacktitle=purple!20!white,
    colback=purple!10!white,
    coltitle=black,
    fonttitle=\bfseries\large,
}{fact}

\NewDocumentCommand{\fact}{+m}{
    \begin{myfact}{}{}
        #1
    \end{myfact}
}

% Proof - customizable name
\NewDocumentEnvironment{custompf}{m}
{
    \par\noindent{\it \textbf{#1}}\par\nopagebreak
    \begin{list}{}{\setlength\leftmargin{1em}\setlength\rightmargin{0em}}
    \item\relax
}
{
    \hfill$\qed$\end{list}
}

\NewDocumentCommand{\pf}{O{Proof}+m}{
    \begin{custompf}{#1.}
        #2
    \end{custompf}
}

% Example - improved environment
\newenvironment{example}{
    \par\vspace{5pt}
    \noindent\textbf{Example.}\par\nopagebreak
    \begin{list}{}{\setlength\leftmargin{1em}\setlength\rightmargin{0em}}
    \item\relax
}{
    \end{list}\vspace{5pt}
}

\NewDocumentCommand{\ex}{+m}{
    \begin{example}
        #1
    \end{example}
}

% Remark
\NewDocumentCommand{\rmk}{+m}{
    {\it \color{blue!50!white}#1}
}

% Remark block - improved environment
\newenvironment{remark}{
    \par\vspace{5pt}
    \noindent\textbf{Remark.}\par\nopagebreak
    \begin{list}{}{\setlength\leftmargin{1em}\setlength\rightmargin{0em}}
    \item\relax
}{
    \end{list}\vspace{5pt}
}

\NewDocumentCommand{\rmkb}{+m}{
    \begin{remark}
        #1
    \end{remark}
}

\title{MATH 301 Lecture 4}
\author{Deepak Jassal}
\date{January 21\textsuperscript{th}, 2026}

\begin{document}
\maketitle
\section{Basic Topology of $\C$}
\setcounter{subsection}{1}
\defn{Open Disk}{
    An open disk $D(z_0,r)=\{z\in\C: |z-z_0|<r\}$ where $r>0$ is called an $r$ neighbourhood of $z_0$.
}
\defn{Punctured Disk}{
    $D_0(z_0,r)=D(z_0,r)\setminus\{z_0\}$
}
\defn{Open Sets}{
    A subset $U$ of $\C$ is called open f for every $z\in U$ there exists $r>0$ such that $D(z,r)\subseteq U$.
}
\defn{Closed Set}{
    A subset $C$ of $\C$ is called a closed set if $C^c=\C\setminus C$ is open. Equivalently, $C$ is closed if it contains the limits points of all convergent sequences in $C$.
}
\defn{Limit of a Sequence}{
    Let $(z_n)_{n=1}^\infty$ be a sequence in $\C$. Then
    \[
        \lim_{n\to\infty}z_n=w
    \]
    means that for evert $\varepsilon>0$ there exists $N\in\N$ such that if $n\geq N$ then $|z_n-w|<\varepsilon$. Or, for every $\varepsilon>0$ there exists $N\in\N$ such that $z_n\in D(w,\varepsilon)$.
}
\pf[Proof of Equivalence for 1.1.5]{
    Suppose that $C$ is a closed subset of $\C$, and supposed that $(z_n)_{n=1}^\infty$ is a sequence in $C$. Suppose for contradiction suppose that $w\not\in C$, so $w\in C^c$, but this set is open, so there exists a disk $D(w,r)$, $r>0$ such that $D(w,r)\subseteq C^c$. This implies that $\lim_{n\to\infty}z_n\neq w$. Thus, we have a contradiction. Now, suppose that for every sequence $(z_n)_{n=1}^\infty$, $\lim_{n\to\infty}z_n=w$, then $w\in C$. We want to show that $C^c$ is open. For a contradiction, suppose that $C^c$ is not open. That is there exists $z_0\in C^c$ such that no matter how small $r>0$ is $D(z_0,r)\not\subseteq C^c$. The disk contains at least one point from $C$. Then $\lim_{n\to\infty}z_n=z_0$, with our assumption $z_0\in C$. So $z_0\not\in C^c$. Thus we have a contradiction, the two statements are equivalent.
}
\prop{
    Open Sets 
    \begin{enumerate}
        \item $\varphi$ and $\C$ are open;
        \item an arbitrary union of open sets is open;
        \item an intersection of a finite number of open sets is open.
    \end{enumerate}
}

\prop{
    Closed Sets
    \begin{enumerate}
        \item $\varphi$ and $\C$ are closed;
        \item an arbitrary intersection of closed sets is closed;
        \item a union of a finite number of closed sets is closed.
    \end{enumerate}
}
\pf[Proof of Open Set 4]{
    Suppose $U_1,U_2,\dots,U_n$ are open sets. Let $z_0\in U_1\cap U_2\cap\cdots\cap U_n$. Then $z_0\in U_i$ implies that there exists $r_i>0$ such that $D(z_0,r_i)\subseteq U_i$ (for all $1\leq i\leq n$). Let $\rho=\min\{r_1,r_2,\dots,r_n\}$, then $D(z_0,\rho)\subseteq D(z_0,r_i)$ for $1\leq i\leq n$. So $D(z_0,\rho)\subset U_1\cap U_2\cap\cdots\cap U_n$.
}
\subsection{Continuity}
Let $f:A\to\C$ be a function where $A$ is an open subset of $\C$.
\defn{Point Continuity}{
    $f$ is a continuous at $z_0\in A$ if for every sequence $(z_n)_{n=1}^\infty\in A$ if 
    \[
        \lim_{n\to\infty}z_n=z_0
    \]
    implies that
    \[
        \lim_{n\to\infty}f(z_n)=f(z_0).
    \]
}
\defn{Continuity}{
    $f$ is continuous on $A$ if it is continuous at every point of $A$.
}
\propp{
    Let $f:\C\to\C$, then the following statements are equivalent
    \begin{enumerate}
        \item $f$ is continuous on $\C$;
        \item For every open subset $U$ of $\C$, $f^{-1}(U)$ is open;
        \item for every closed subset $C$ of $\C$, $f^{-1}(C)$ is closed.
    \end{enumerate}
}{
    ($2\Leftrightarrow 3$) If $U$ is open, then $\C\setminus U=U^c$ is closed, let $C=U^c$. $\C=U\cup C$, and $U\cap C=\varphi$. $f^{-1}(\C)=\C$ and $f^{-1}(\C)=f^{-1}(U)\cup f^{-1}(C)$. Since these unions are disjoint, we have that $f^{_1}(U)$ is open \textit{if and only if} $f^{-1}(C)$ is closed.\\
    ($1\Rightarrow 2$) $z_n\to z_0\Rightarrow f(z_n)\to f(z_0)$. Let $U$ be an open set in $\C$. Let $z_0\in f^{-1}(U)$, so $f(z_0)\in U$. Suppose that $f^{-1}(U)$ is not ope. Then $z_0\in f^{-1}(U)$ such that for all $r>0$ $D(z_0,r)\not\subseteq f^{-1}(U)$, but $f(z_0)\in U$. This implies that $D(z_),r$ contains a point form outside of $(f^{-1}(U))^c$. this is true for all $n\in \N$ $D\left(z_0,\frac{1}{n}\right)$. There exists a sequence $(z_n)_{n=1}^\infty$ putisde of $f^{-1}(U)$ with $\lim_{n\to\infty}z_n=z_0$ and $z_0\in f^{-1}(U)$. So consider $(f(z_n))_{n=1}^\infty$ it is totally outside of $U$, so $(f(z_n))_{n=1}^\infty\subseteq U^c$ which is closed. This implies that $\lim_{n\to\infty}f(z_n)\subseteq U^c$ so $z_n\to z_0$, but since $f$ is continuous $\lim_{n\to\infty}f(z_n)=f(z_0)\subseteq U$, which is a contradiction.\\
    ($2\Rightarrow1$) Now suppose that $f^{-1}(U)$ is open for every open set in $\C$. We want to show that $z_n\to z_0$ implies that $f(z_n)\to f(z_0)$. Suppose that $z_n\to z_0$ but $f(z_n)\not\to f(z_0)$. $f(z_n)\not\to f(z_0)\Rightarrow \exists r>0, \text{ such that the disk } D(z_0,r)$ does not contain $f(z_n)$ for every $n$ and $f^{-1}(D(f(z_0),r))$ is open and contains $z_0$. $f(z_n)\not\in D(f(z_0),r)$ implies that $z_n\not\in D(z_0,r)$ implies that $z_n\not\to z_0.$
}

\end{document}