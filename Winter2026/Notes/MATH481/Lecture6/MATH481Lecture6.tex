\documentclass[12pt]{article}
\usepackage{color,soul}
% Import preambles and macros for notes
% Essential packages for notes
\usepackage{amsmath, amssymb, amsthm}
\usepackage{mathtools}  % for \coloneqq, etc.
\usepackage{geometry}   % Better page margins
\usepackage{parskip}    % Better paragraph spacing
\usepackage{microtype}  % Better typography
\usepackage{enumitem}   % Customize lists
\usepackage{hyperref}   % Clickable links
\usepackage{booktabs}   % Better tables
\usepackage{tcolorbox}  % For colored boxes/theorems

% Page layout for notes
\geometry{a4paper, margin=1in}
\setlength{\parskip}{0.8em}

% Theorem environments with shared numbering
\newtheorem{theorem}{Theorem}[subsection]  % Number within sections: 2.3.1, 2.3.2, etc.

% All other environments share the same counter as theorem
\newtheorem{lemma}[theorem]{Lemma}
\newtheorem{proposition}[theorem]{Proposition}
\newtheorem{corollary}[theorem]{Corollary}
\newtheorem{definition}[theorem]{Definition}
\newtheorem{example}[theorem]{Example}
\newtheorem{remark}[theorem]{Remark}
\newtheorem{claim}[theorem]{Claim}

% Custom colors for notes
\usepackage{xcolor}
\definecolor{note-blue}{RGB}{220, 230, 255}
\definecolor{theorem-green}{RGB}{220, 255, 220}
% Math notation shortcuts for notes
\newcommand{\R}{\mathbb{R}}
\newcommand{\C}{\mathbb{C}}
\newcommand{\Q}{\mathbb{Q}}
\newcommand{\Z}{\mathbb{Z}}
\newcommand{\N}{\mathbb{N}}

% Calculus
\newcommand{\diff}{\mathop{}\!\mathrm{d}}
\newcommand{\deriv}[2]{\frac{\mathrm{d}#1}{\mathrm{d}#2}}
\newcommand{\pderiv}[2]{\frac{\partial #1}{\partial #2}}

% Linear Algebra
\newcommand{\inner}[2]{\langle #1, #2 \rangle}
\newcommand{\norm}[1]{\| #1 \|}
\newcommand{\tr}{\operatorname{tr}}
\newcommand{\spn}{\operatorname{span}}
\newcommand{\rank}{\operatorname{rank}}
\newcommand{\nullity}{\operatorname{nullity}}

% Logic
\newcommand{\contra}{\Rightarrow\Leftarrow}

% Custom commands for notes
\newcommand{\todo}[1]{\textcolor{red}{[TODO: #1]}}
\newcommand{\important}[1]{\textbf{\textcolor{blue}{#1}}}
\input{../../../../preambles/theorem-system-section.tex}

\title{MATH 481 Lecture 6}
\author{Deepak Jassal}
\date{January 27\textsuperscript{th}, 2026}

\begin{document}
\setcounter{section}{2}
\setcounter{tcb@cnt@mydefinition}{13}
\maketitle
\thm{Convergence of Dirichlet Series}{
    Let $f:\N\to\C$ be an arithmetic function, $F(s)$ the associated Dirichlet series and $s_f(x)=\sum_{n\leq x}f(n)$ the associated summatory function. Then,
    \begin{enumerate}[label=(\roman*)]
        \item For any $s\in\C$ with $\Re(s)>0$ such that $F(s)$ converges we have
        \[
            F(s)=s\int_{1}^{\infty}\frac{s_f(x)}{x^{s+1}}\,dx.\tag{$\ast$} \label{ast}
        \]
        \item If $S_f(x)=\Oh(x^\alpha)$ for some $\alpha\geq0$, then $F(s)$ converges then for all $s$ with $\Re(s)>\alpha$ and \eqref{ast} holds for all such $s$.
    \end{enumerate}
}
\begin{remark}
    \begin{enumerate}
        \item Let $\chi$ be a Dirichlet character ($|\chi(n)|=0$ or 1, $\chi$ is periodic).\\
        For non-trivial $\chi$, we can prove that
        \[
            \sum_{n\leq x}\chi(n)=\Oh(1).
        \]
        Let $L(s,\chi)$ be the Dirichlet series associated with $\chi$.
        \[
            L(s,\chi)=\sum_{n=1}^{\infty}\frac{\chi(n)}{n^s}.
        \]
        This is absolutly convergent for $\Re(s)>1$.\\
        By theorem 2.14(ii) and using the bound $\sum_{n\leq x}\chi(n)=\Oh(1)$ we see that $L(s,\chi)$ converges when $\Re(s)>0$.
        \item Theorem 2.14(i) can be expressed in terms of Mellin transforms.\\
        Given $\phi:\R^+\to\C$. The Mellin transform of $\phi$ is given by
        \[
            \overset{\sim}{\phi}(s)=\int_{0}^{\infty}\phi(x)x^{-s}\,dx
        \]
        ($s\in\C$) provided the integral converges.\\
        Theorem 2.14(i) can then be restated as $\dfrac{F(s)}{s}$ is the Mellin transform of $\dfrac{S_f(x)}{x}$.
    \end{enumerate}
\end{remark}
\pf[Proof of Theorem 2.14]{
    Let $s\in\C$ be such that $\Re(s)>0$ and $F(s)$ converges.\\
    (i) Consdier 
    \[
        F_N(s)=\sum\frac{f(n)}{n^s}.
    \]
    By Abel's summation formula we have
    \begin{align*}
        F_N(s)&=\left(\sum_{n\leq N}f(n)\right)N^{-s}+s\int_{1}^{N}S_f(t)t^{-s-1}\,dt\\
        &=S_f(N)N^{-s}+s\int_{1}^{\infty}S_f(N)T^{-s-1}\,dt.
    \end{align*}
    As $N\to\infty$ $F_N(s)\to F(s)$, and by \textit{theorem 2.12} $\dfrac{S_f(N)}{N^s}\to0$.\\
    So as $N\to\infty$ we get
    \[
        F(s)=s\int_{1}^{\infty}S_f(t)t^{-s-1}\,dt.
    \]
    (ii) We have
    \[
        F_N(s)=\frac{S_f(N)}{N^s}+s\int_{1}^{N}S_f(t)t^{-s-1}\,dt.
    \]
    \[
        \left|\frac{S_f(N)}{N^s}\right|=\left|\frac{S_f(N)}{N^{\Re(s)}}\right|=\Oh\left(N^{\alpha-\Re(s)}\right),
    \]
    so for $\Re(s)>\alpha$, we have $\dfrac{S_f(N)}{N^s}\to 0$ as $N\to\infty$ we also have 
    \[
        \left|\frac{S_f(t)}{t^{s+1}}\right|=\Oh(t^{\alpha-\Re(s)-1}).
    \]
    It follows that
    \begin{align*}
        \left|\int_{1}^{N}\frac{S_f(t)}{t^{s+1}}\,dt\right|&\leq\int_{1}^{N}\left|\frac{S_f(t)}{t^{s+1}}\right|\,dt\\
        &=\Oh\left(\int_{1}^{N}t^{\alpha-\Re(s)-1}\,dt\right)\\
        &=\Oh\left(N^{\alpha-\Re(s)-1}-1\right).
    \end{align*}
    Hence, $\int_{1}^{\infty}\frac{S_f(t)}{t^{s-1}}\,dt$ is convergent for $\Re(s)>\alpha$. Taking $N\to\infty$ we get $F(s)$ is convergent for $\Re(s)>\alpha$ and 
    \[
        F(s)=s\int_{1}^{\infty}\frac{S_f(t)}{t^{s-1}}\,dt
    \]
    as desired.
}
\end{document}