\documentclass[12pt]{article}
\usepackage{color,soul}
% Import preambles and macros for notes
% Essential packages for notes
\usepackage{amsmath, amssymb, amsthm}
\usepackage{mathtools}  % for \coloneqq, etc.
\usepackage{geometry}   % Better page margins
\usepackage{parskip}    % Better paragraph spacing
\usepackage{microtype}  % Better typography
\usepackage{enumitem}   % Customize lists
\usepackage{hyperref}   % Clickable links
\usepackage{booktabs}   % Better tables
\usepackage{tcolorbox}  % For colored boxes/theorems

% Page layout for notes
\geometry{a4paper, margin=1in}
\setlength{\parskip}{0.8em}

% Theorem environments with shared numbering
\newtheorem{theorem}{Theorem}[subsection]  % Number within sections: 2.3.1, 2.3.2, etc.

% All other environments share the same counter as theorem
\newtheorem{lemma}[theorem]{Lemma}
\newtheorem{proposition}[theorem]{Proposition}
\newtheorem{corollary}[theorem]{Corollary}
\newtheorem{definition}[theorem]{Definition}
\newtheorem{example}[theorem]{Example}
\newtheorem{remark}[theorem]{Remark}
\newtheorem{claim}[theorem]{Claim}

% Custom colors for notes
\usepackage{xcolor}
\definecolor{note-blue}{RGB}{220, 230, 255}
\definecolor{theorem-green}{RGB}{220, 255, 220}
% Math notation shortcuts for notes
\newcommand{\R}{\mathbb{R}}
\newcommand{\C}{\mathbb{C}}
\newcommand{\Q}{\mathbb{Q}}
\newcommand{\Z}{\mathbb{Z}}
\newcommand{\N}{\mathbb{N}}

% Calculus
\newcommand{\diff}{\mathop{}\!\mathrm{d}}
\newcommand{\deriv}[2]{\frac{\mathrm{d}#1}{\mathrm{d}#2}}
\newcommand{\pderiv}[2]{\frac{\partial #1}{\partial #2}}

% Linear Algebra
\newcommand{\inner}[2]{\langle #1, #2 \rangle}
\newcommand{\norm}[1]{\| #1 \|}
\newcommand{\tr}{\operatorname{tr}}
\newcommand{\spn}{\operatorname{span}}
\newcommand{\rank}{\operatorname{rank}}
\newcommand{\nullity}{\operatorname{nullity}}

% Logic
\newcommand{\contra}{\Rightarrow\Leftarrow}

% Custom commands for notes
\newcommand{\todo}[1]{\textcolor{red}{[TODO: #1]}}
\newcommand{\important}[1]{\textbf{\textcolor{blue}{#1}}}
\input{../../../../preambles/theorem-system-section.tex}

\title{MATH 481 Lecture 3}
\author{Deepak Jassal}
\date{January 15\textsuperscript{th}, 2026}

\begin{document}
\setcounter{section}{2}
\setcounter{tcb@cnt@mydefinition}{3}
\maketitle   
\textit{Recall.} Eulers summation formula which states that for $f\in\mathcal{C}[y,x]$ we have
\[
    \sum_{y\leq n\leq x}f(n)=\int_{y}^{x}f(t)\,dt+\int_{x}^{y}\fract{t}f'(t)\,dt-\fract{t}f(x)+\fract{y}f(y).
\]
\pf[Continuation of Proof for Theorem 2.1]{
    Set $k=\floor{x}$ and $m=\floor{y}$. We observe that
    \[
        \sum_{y\leq n\leq x}f(n)=\sum_{m+1\leq n\leq k}f(n).
    \]
    We also have for $n-1,n\in[y,x]$
    \[
        \int_{n-1}^{n}\floor{t}f'(t)\,dt=nf(n)-(n-1)f(n-1)-f(n).
    \]
    Hence,
    \begin{align*}
        \int_{m+1}^{k}\floor{t}f'(t)\,dt=&\int_{m+1}^{m+2}\floor{t}f(t)\,dt+\int_{m+2}^{m+3}\floor{t}f(t)\,dt+\cdots+\int_{k-1}^{k}\floor{t}f(t)\,dt\\
        =&(m+2)f(m+2)-(m+1)f(m+1)-f(m+2)\\
        &+(m+3)f(m+3)-(m+2)f(m+1)-f(m+3)\\
        &+\;\,\vdots\\
        &+kf(k)-(k-1)f(k-1)-f(k)\\
        =&-mf(m+1)+kf(k)-\sum_{m-1\leq n\leq k}f(n).
    \end{align*}
    It follows that
    \begin{align*}
        \sum_{m-1\leq n\leq k}f(n)&=kf(k)-mf(m+1)-\int_{m+1}^{k}\floor{k}f'(t)\,dt\\
        &=kf(k)-mf(m+1)-\int_{y}^{x}\floor{t}f'(t)\,dt+\int_{y}^{m+1}\floor{t}f'(t)\,dt+\int_{k}^{x}\floor{t}f'(t)\,dt\\
        &=kf(k)-mf(m+1)-\int_{y}^{x}\floor{t}f'(t)\,dt+m\int_{y}^{m+1}f'(t)\,dt+k\int_{k}^{x}f'(t)\,dt\\
        &=kf(k)-mf(m+1)-\int_{y}^{x}\floor{t}f'(t)\,dt+m(f(m+1)-f(y))+k(f(x)-f(k))\\
        &=kf(x)-mf(y)-\int_{y}^{x}tf'(t)\,dt+\int_{y}^{x}\fract{t}f'(t)\,dt\\
        &=kf(x)-mf(y)-[tf(t)]_y^x+\int_{x}^{y}f(t)\,dt+\int_{y}^{x}\fract{t}f'(t)\,dt\\
        &=xf(x)-\fract{x}f(x)-yf(y)+\fract{y}f(y)-xf(x)+yf(y)+\int_{y}^{x}f(t)\,dt+\int_{y}^{x}\fract{t}f(t)\,dt.
    \end{align*}
    Cancelling the terms we arrive at 
    \[
                \sum_{y\leq n\leq x}f(n)=\int_{y}^{x}f(t)\,dt+\int_{x}^{y}\fract{t}f'(t)\,dt-\fract{t}f(x)+\fract{y}f(y)
    \]
    which is the desired result.
}
\thm{Sum of Logartihm's}{
    Let $N\in\N$ then we have
    \[
    \sum_{n\leq N}\log n=Nlog N-N+\frac{1}{2}\log N + c+\Oh\left(\frac{1}{N}\right).
    \]
}

Observe that the above formula can be written as
\[
    \log N!=Nlog N-N+\frac{1}{2}\log N + c+\Oh\left(\frac{1}{N}\right),
\]
exponentiating we obtain
\begin{equation}\label{eqn:stirling}
    N!=N^Ne^{-N}\sqrt{N}e^ce^{\Oh\left(\frac{1}{N}\right)}.
\end{equation}
What can be said about $e^{\Oh\left(\frac{1}{N}\right)}$?\\
Let $g(N)$ be a function such that $g(N)=\Oh\left(\frac{1}{N}\right)$. i.e., $|g(N)|\leq C\frac{1}{N}$ for some $C>0$.
\begin{align*}
    \left|e^{g(N)}\right|&=\left|a+g(N)+\frac{g(N)^2}{2!}++\frac{g(N)^3}{3!}+\cdots\right|\\
    &\leq1+\left|g(N)\right|+\left|\frac{g(N)^2}{2!}\right|+\cdots\\
    &\leq1+\frac{C}{N}+\frac{C^2}{N}\\
    &=1+\frac{C}{N}\left(1+\frac{1}{2!}\frac{C}{N}+\frac{1}{3!}\frac{C^2}{N^2}+\cdots\right)\\
    &=1+\Oh\left(\frac{1}{N}\right).
\end{align*}
It follows that $e^{\Oh\left(\frac{1}{N}\right)}=1+\Oh(\frac{1}{N})$. Furthermore,
\[
    e^x=1+\Oh\left(x\right),\quad \text{if } |x|<1.
\]
Going back to \ref{eqn:stirling} we have
\[
    N!=N^Ne^{-N}\sqrt{N}e^c\left(1+\Oh\left(\frac{1}{N}\right)\right).
\]
Hence, we arrive at stirlin's formula.
\corp{
    \[
        N!=C_1N^Ne^{-N}\sqrt{N}\left(1+\Oh\left(\frac{1}{N}\right)\right).
    \]
}{
By Euler's summation formula we have
\begin{align*}
    \sum_{n\leq N}\log n&=\int_{1}^{N}\log t\,dt+\int_{1}^{t}\fract{t}\frac{1}{t}\,dt-\fract{N}\log N\\
    &=N\log N-N+\int_{1}^{N}\frac{\fract{t}}{t}\,dt.
\end{align*}
Now let's consider $\int_{1}^{N}\frac{\fract{t}}{t}\,dt$.
\begin{align*}
    \int_{1}^{N}\frac{\fract{t}}{t}\,dt&=\int_{1}^{N}\frac{\fract{t}+\frac{1}{2}-\frac{1}{2}}{t}\,dt\\
    &=\frac{1}{2}\log N+\int_{1}^{N}\frac{\fract{t}-\frac{1}{2}}{t}\,dt.
\end{align*}
Out problem is now reduced to showing that $\int_{1}^{N}\frac{\fract{t}-\frac{1}{2}}{t}\,dt=c+\Oh\left(\frac{1}{N}\right)$, for some explicitly given constant $c$.\\
Let $\rho(t)=\frac{t}-\frac{1}{2}$, then we have
\[
    \int_{1}^{N}\frac{\rho(t)}{t}\,dt
\]
let $u=\frac{1}{t}$, $dv=\rho(t)\,dt$, $du=\frac{1}{t^2}\,dt$, $v=\int_{1}^{t}\rho(x)\,dx=R(t)$. Then the above integral becomes
\[
    \left[\frac{R(t)}{t}\right]_1^N+\int_{1}^{N}\frac{1}{t^2}\,dt.
\]
Notice a property of $R(t)$.
\begin{align*}
    R(N)=\int_{1}^{N}\rho(x)\,dx&=\int_{1}^{2}\rho(x)\,dx+\int_{2}^{3}\rho(x)\,dx+\cdots+\int_{N-1}^{N}\rho(x)\,dx\\
    &=\int_{1}^{2}\fract{x}-\frac{1}{2}\,dx+\int_{2}^{3}\fract{x}-\frac{1}{2}\,dx+\cdots+\int_{N-1}^{N}\fract{x}-\frac{1}{2}\,dx\\
    &=\int_{0}^{1}x-\frac{1}{2}\,dx+\int_{0}^{1}x-\frac{1}{2}\,dx+\cdots+\int_{0}^{1}x-\frac{1}{2}\,dx\\
    &=\left[\frac{x^2}{2}-\frac{x}{2}\right]_0^1+\left[\frac{x^2}{2}-\frac{x}{2}\right]_0^1+\cdots+\left[\frac{x^2}{2}-\frac{x}{2}\right]_0^1\\
    &=0+0+\cdots+0\\
    &=0.
\end{align*}
Hence, $R(N)=0$ for all $N\in\N$.\\
Observe that
\begin{align*}
    \left|R(t)\right|&=\left|\int_{1}^{t}\rho(x)\,dx\right|\\
    &=\left|\int_{\floor{t}}^{t}\rho(x)\,dx\right|\\
    &\leq\int_{\floor{t}}^{t}\left|\rho(x)\right|\,dx\\
    &\leq \frac{1}{2}\left(t-\floor{t}\right)=\frac{1}{2}\fract{t}\leq\frac{1}{2}.
\end{align*}
Hence, $\int_{1}^{\infty}\frac{R(t)}{t^2}\,dt=c$ for some constant $c$, and 
\[
    \int_{N}^{\infty}\frac{R(t)}{t^2}\,dt\leq \int_{N}^{\infty}\frac{\frac{1}{2}}{t^2}\,dt\ll\frac{1}{N}.
\]
Thus we have
\[
    \sum_{n\leq N}\log n =N\log N-N+\frac{1}{2}\log N+c+\Oh\left(\frac{1}{N}\right)
\]
which yields the descired result.
}{}
\newpage
\corp{Given $x\in\R$ and $x\geq1$ we have
\[
    \sum_{n\leq N}\log n=x\log x-x+\Oh(\log x).
\]}{
    The result can be derived directly from Euler's summation formula, we will instead use another approach.\\
    Applying theorem 2.4 with $N=\floor{x}$
    \begin{align*}
        \sum_{n\leq x}\log n&=\sum_{n\leq N}\log n\\
        &=\floor{x}\log\floor{x}-\floor{x}+c+\frac{1}{2}\log\floor{x}+\Oh\left(\frac{1}{\floor{x}}\right)\\
        &=(x-\fract{x})\log(x-\fract{x})-(x-\fract{x})+c+\frac{1}{2}\log(x-\fract{x})+\Oh\left(\frac{1}{(x-\fract{x})}\right)
    \end{align*}
    Given $|t|<1$
    \begin{align*}
        \log t&=-\sum_{n=1}^{\infty}\frac{t^n}{n}=1-t-\frac{t^2}{2}-\cdots\\
        &=-1-t\left(1+\frac{t}{2}+\frac{t^2}{3}+\cdots\right)=-1+\Oh(t).
    \end{align*}
    Since $|t|<1$ we further have $\log t=\Oh(1)$.\\
    Hence,
    \[
        \log(x-\fract{x})=\log x+\Oh(1).
    \]
}{}
    





\end{document}