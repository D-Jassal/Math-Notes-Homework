\documentclass[12pt]{article}
\usepackage{color,soul}
% Import preambles and macros for notes
% Essential packages for notes
\usepackage{amsmath, amssymb, amsthm}
\usepackage{mathtools}  % for \coloneqq, etc.
\usepackage{geometry}   % Better page margins
\usepackage{parskip}    % Better paragraph spacing
\usepackage{microtype}  % Better typography
\usepackage{enumitem}   % Customize lists
\usepackage{hyperref}   % Clickable links
\usepackage{booktabs}   % Better tables
\usepackage{tcolorbox}  % For colored boxes/theorems

% Page layout for notes
\geometry{a4paper, margin=1in}
\setlength{\parskip}{0.8em}

% Theorem environments with shared numbering
\newtheorem{theorem}{Theorem}[subsection]  % Number within sections: 2.3.1, 2.3.2, etc.

% All other environments share the same counter as theorem
\newtheorem{lemma}[theorem]{Lemma}
\newtheorem{proposition}[theorem]{Proposition}
\newtheorem{corollary}[theorem]{Corollary}
\newtheorem{definition}[theorem]{Definition}
\newtheorem{example}[theorem]{Example}
\newtheorem{remark}[theorem]{Remark}
\newtheorem{claim}[theorem]{Claim}

% Custom colors for notes
\usepackage{xcolor}
\definecolor{note-blue}{RGB}{220, 230, 255}
\definecolor{theorem-green}{RGB}{220, 255, 220}
% Math notation shortcuts for notes
\newcommand{\R}{\mathbb{R}}
\newcommand{\C}{\mathbb{C}}
\newcommand{\Q}{\mathbb{Q}}
\newcommand{\Z}{\mathbb{Z}}
\newcommand{\N}{\mathbb{N}}

% Calculus
\newcommand{\diff}{\mathop{}\!\mathrm{d}}
\newcommand{\deriv}[2]{\frac{\mathrm{d}#1}{\mathrm{d}#2}}
\newcommand{\pderiv}[2]{\frac{\partial #1}{\partial #2}}

% Linear Algebra
\newcommand{\inner}[2]{\langle #1, #2 \rangle}
\newcommand{\norm}[1]{\| #1 \|}
\newcommand{\tr}{\operatorname{tr}}
\newcommand{\spn}{\operatorname{span}}
\newcommand{\rank}{\operatorname{rank}}
\newcommand{\nullity}{\operatorname{nullity}}

% Logic
\newcommand{\contra}{\Rightarrow\Leftarrow}

% Custom commands for notes
\newcommand{\todo}[1]{\textcolor{red}{[TODO: #1]}}
\newcommand{\important}[1]{\textbf{\textcolor{blue}{#1}}}
\input{../../../../preambles/theorem-system-section.tex}

\title{MATH 481 Lecture 5}
\author{Deepak Jassal}
\date{January 22\textsuperscript{th}, 2026}

\begin{document}
\setcounter{section}{2}
\setcounter{tcb@cnt@mydefinition}{9}
\maketitle
\thm{Abel's Summation Formula}{
    Let $0<y\leq x$, $f\in\mathcal{C}[y,x]$ and $a(n)$ an arithmetic function, then
    \[
        \sum_{y<n\leq x}a(n)f(n)=A(n)f(x)-A(y)f(y)-\int_{y}^{x}A(t)f'(t)\,dt,
    \]
    where $A(n)$ is the summatory fuction for $a(n)$ $\left(A(n\sum_{n\leq x}a(n))\right)$.
}
\corp{
    Let $x\leq 1$, $f\in\mathcal{C}[1,x]$ and $a(n)$ be an atrithmetic function. Then,
    \[
        \sum_{n\leq x}a(n)f(n)=A(n)f(x)-\int_{y}^{x}A(t)f'(t)\,dt.
    \]
}{Apply \textit{theorem 2.10}.}{}
\ex{
    Deduce Euler's summation formula from Abel's summation formula. (Use $a(n)=1$, $A(t)=\floor{t}$.)
}

\pf[Proof of Theorem 2.10]{
    Consider 
    \[
        I=\int_{y}^{x}A(t)f't\,dt=\int_{y}^{x}\sum_{x\leq t}a(n)f't\,dt.
    \]
    Let 
    \[
        \chi(x,n)=\begin{cases}
            1&\text{if } n\leq t\\
            0&\text{otherwise}
        \end{cases}.
    \]
    Using $\chi$ we Obtain
    \begin{align*}
        I&=\int_{y}^{x}\sum_{n\leq x}a(n)\chi(n,t)f'(t)\,dt\\
        &=\sum_{n\leq x}a(n)\int_{y}^{x}\chi(n,t)f'(t)\,dt\\
        &=\sum_{n\leq x}a(n)\int_{\max{n,y}}^{x}f'(t)\,dt\\
        &=\sum_{n\leq x}a(n)\left(f(x)-f(\max{n,y})\right)\\
        &=\sum_{n\leq x}a(n)f(x)-\left(\sum_{n\leq y}a(n)f(y)+\sum_{y<n\leq x}a(n)f(n)\right)\\
        \int_{y}^{x}A(t)f'(t)\,dt&=A(x)f(x)-A(y)f(y)-\sum_{y<n\leq x}a(n)f(n)\\
        \sum_{y<n\leq x}a(n)f(n)&=A(n)f(x)-A(y)f(y)-\int_{y}^{x}A(t)f'(t)\,dt.
    \end{align*}
    Thus, we have the desired result.
}
\thm{Kronecker's Lemma}{
    Let $f:\N\to\C$ be an arithmetic function, let $s\in\C$ with $\Re(s)>0$. Suppose that
    \[
        F(s)=\sum_{n=1}^{\infty}\frac{f(n)}{n^s}
    \]
    converges. Then, 
    \[
        \frac{1}{x^s}\sum_{n\leq x}f(s)\to0\text{ as } x\to\infty.
    \]
    In particular if $\sum_{n=1}^{\infty}\frac{f(s)}{n}$ converges then we have $\frac{1}{x}\sum_{n\leq x}f(s)\to 0$ as $x\to\infty$.\\
    i.e., $\sum_{n\leq x}=\oh(x).$
}
\pf[Proof of Theorem 2.12]{
    Let 
    \[
        S_f(x)=S(x)=\sum_{n\leq x}f(n), 
    \]
    and 
    \[
        T(x)=\sum_{n\leq x}\frac{f(n)}{n^s}.
    \]
    We know that $\lim_{x\to\infty}T(x)=T(s)$. We want to prove that 
    \[
        \lim_{x\to\infty}\frac{S(x)}{x^s}=0.
    \]
    \begin{align*}
        S(x)=\sum_{n\leq x}\frac{f(n)}{n^s}n^s&=\left(\sum_{n\leq x}\frac{f(n)}{n^s}\right)x^s-s\int_{1}^{x}\left(\sum_{n\leq t}\frac{f(n)}{n^s}\right)t^{s-1}\,dt\\
        &=T(x)x^s-s\int_{0}^{x}T(t)t^{s-1}\,dt\\
        &=T(x)\int_{0}^{x}st^{s-1}\,dt-s\int_{0}^{x}T(t)t^{s-1}\,dt\\
        &=\int_{0}^{x}st^{s-1}(T(x)-T(t))\,dt
    \end{align*}
    Hence, since $\lim_{x\to\infty}T(x)=T$ we have
    \[
        |S(x)|\leq\int_{0}^{x}|s|t^{\Re(s)-1}(|T(x)-T|+|T(t)-T|)\,dt.
    \]
    Given $\varepsilon>0$, there exists $x_0>1$ such that $|T(t)-T|<\varepsilon$ whenever $x>x_0$. Given $\varepsilon>0$ we have
    \begin{align*}
        |S(x)|\leq&\int_{0}^{x_0}|s|t^{\Re(s)-1}(|T(x)-T|+|T(t)-T|)\,dt\\
        &+\int_{x_0}^{x}|s|t^{\Re(s)-1}(|T(x)-T|+|T(t)-T|)\,dt.
    \end{align*}
    So for $0<t<x_0$, we have
    \begin{align*}
        |T(x)-T|+|T(t)-T|&\leq\varepsilon +|T(t)|+|T|\\
        &\leq\varepsilon+|T|+\sum_{n\leq t}\frac{|f(n)|}{n^{\Re(s)}}
        &\leq\varepsilon+|T|+\sum_{n\leq x_0}\frac{|f(n)|}{n^{\Re(s)}}\\
        &=M(\varepsilon).
    \end{align*}
    For $x_0<t<x$
    \[
        |T(x)-T|+|T(t)-T|\leq2\varepsilon.
    \]
    Putting the two together we get
    \begin{align*}
        |S(x)|&\leq\int_{0}^{x_0}|s|t^{\Re(s)-1}M(\varepsilon)\,dt+\int_{x_0}^{x}|s|t^{\Re(s)}2\varepsilon\,dt\\
        &\leq|s|\left[M(\varepsilon)\frac{x_0\Re(s)}{\Re(s)}+2\varepsilon\left(\frac{x^{\Re(s)}-x_0^{\Re(s)}}{\Re(s)}\right)\right]
    \end{align*}
    dividing by $|x^s|=x^{\Re(s)}$
    \begin{align*}
        \frac{|S(x)|}{x^{\Re(s)}}&\leq\frac{|s|}{\Re(s)}\left[M(\varepsilon)\left(\frac{x_0}{x}\right)^{\Re(s)}+2\varepsilon-2\varepsilon\left(\frac{x_0}{x}\right)^{\Re(s)}\right].
    \end{align*}    
    Due the dependance on $\varepsilon$ for the term on the right we can make this arbitrarily small. Hence, $\lim_{x\to\infty}\frac{S(x)}{x^s}=0$.
}
Going back to studying the means of arithmetic functions:\\
Let 
\[
    M_f(x)=\frac{1}{x}\sum_{n\leq x}f(n)
\]
with $f:\N\to\C$, and 
\[
    M_f=\lim_{x\to\infty}M_f(x).
\]
\[
    L_f(x)=\frac{1}{\log x}\sum_{n\leq x}\frac{f(n)}{n},
\]
and 
\[
    L_f=\lim_{x\to\infty}L_f(x).
\]
\thm{Existence of Logarithmic Means}{
    Using the notation above, if $M_f$ exists then $L_f$ also exists and $M_f=L_f$.
}
\begin{remark}
    \[
        \frac{1}{x}\sum_{n\leq x}\Lambda(n)\to1\text{ as } x\to\infty.
    \]
    Also,
    \[
        \frac{1}{x}\sum_{n\leq x}\mu(n)\to0\text{ as } x\to\infty,
    \]
    \[
        M_\Lambda\Leftrightarrow \text{ PNT},
    \]
    computing
    \[
        M_\Lambda\& M_\mu\Rightarrow \text{ PNT}.
    \]
    However $L_\Lambda$ and $L_\mu$ are much more tractable objects and we will compute them in the next chapter.
\end{remark}
\pf[Proof of Theorem 2.13]{
    Given $f:\N\to\C$ set 
    \[
        S_f(x)=\sum_{n\leq x}f(n)\quad\quad T_f(x)=\sum_{n\leq x}\frac{f(n)}{n}
    \]
    \[
        M_f(x)=\frac{S_f(x)}{x}\quad\quad L_f(x)=\frac{T_f(x)}{\log x}.
    \]
    Suppose that
    \[
        \lim_{x\to\infty}M_f(x)=M_f.
    \]
    We need to show that $\lim_{x\to\infty}L_f(x)=M_f$.
    \begin{align*}
        T(x)&=\sum_{n\leq x}\frac{f(n)}{n}\\
        &=\frac{S(x)}{x}+\int_{1}^{x}s(t)t^{-2}\,dt\\
        \frac{T(x)}{\log x}&=\frac{S(x)}{x\log x}+\frac{1}{\log x}\int_{1}^{x}\frac{S(t)}{t^2}\,dt\\
        &=\frac{S(x)}{x\log x}+\frac{1}{\log x}I(x)\\
        &=0+\frac{1}{\log x}I(x).
    \end{align*}
    Now we need only show that $\lim_{x\to\infty}\frac{I}{\log x}=M_f$. To do this we consider 
    \begin{align*}
        I(x)-M_f\log x&=\int_{1}^{x}\frac{S(t)}{t^2}\,dt-M_f\int_{1}^{x}\frac{1}{t}\,dt\\
        &=\int_{1}^{x}\frac{\frac{S(t)}{t}-M_f}{t}\,dt
    \end{align*}
    Thus
    \[
        |I(x)-M_f\log x|\leq\int_{1}^{x}\frac{\left|\frac{S(t)}{t}-M_f\right|}{t}\,dt.
    \]
    From here proceed as in Kronecker's lemma by splitting the integral into two intervals $[1,x_0]$ and $[x_0,x]$. Doing so will give the desired result.
}
\subsection*{Dirichlet Series}
Let $f:\N\to\C$, set 
\[
    F(s)=\sum_{n=1}^{\infty}\frac{f(n)}{n^s}.
\]
Then, $F(s)$ is the Dirichlet series associated with $f$.
\thm{Convergence of Dirichlet Series}{
    Let $f:\N\to\C$ be an arithmetic function, $F(s)$ the associated Dirichlet series and $s_f(x)=\sum_{n\leq x}f(n)$ the associated summatory function. Then,
    \begin{enumerate}[label=(\roman*)]
        \item For any $s\in\C$ with $\Re(s)>0$ such that $F(s)$ converges we have
        \[
            F(s)=s\int_{1}^{\infty}\frac{s_f(x)}{x^{s+1}}\,dx.\tag{$\ast$} \label{ast}
        \]
        \item If $S_f(x)=\Oh(x^\alpha)$ for some $\alpha\geq0$, then $F(s)$ converges then for all $s$ with $\Re(s)>\alpha$ and \eqref{ast} holds for all such $s$.
    \end{enumerate}
}
\end{document}