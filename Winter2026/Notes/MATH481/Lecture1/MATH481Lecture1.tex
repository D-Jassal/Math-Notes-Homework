\documentclass[12pt]{article}
\usepackage{color,soul}
% Import preambles and macros for notes
% Essential packages for notes
\usepackage{amsmath, amssymb, amsthm}
\usepackage{mathtools}  % for \coloneqq, etc.
\usepackage{geometry}   % Better page margins
\usepackage{parskip}    % Better paragraph spacing
\usepackage{microtype}  % Better typography
\usepackage{enumitem}   % Customize lists
\usepackage{hyperref}   % Clickable links
\usepackage{booktabs}   % Better tables
\usepackage{tcolorbox}  % For colored boxes/theorems

% Page layout for notes
\geometry{a4paper, margin=1in}
\setlength{\parskip}{0.8em}

% Theorem environments with shared numbering
\newtheorem{theorem}{Theorem}[subsection]  % Number within sections: 2.3.1, 2.3.2, etc.

% All other environments share the same counter as theorem
\newtheorem{lemma}[theorem]{Lemma}
\newtheorem{proposition}[theorem]{Proposition}
\newtheorem{corollary}[theorem]{Corollary}
\newtheorem{definition}[theorem]{Definition}
\newtheorem{example}[theorem]{Example}
\newtheorem{remark}[theorem]{Remark}
\newtheorem{claim}[theorem]{Claim}

% Custom colors for notes
\usepackage{xcolor}
\definecolor{note-blue}{RGB}{220, 230, 255}
\definecolor{theorem-green}{RGB}{220, 255, 220}
% Math notation shortcuts for notes
\newcommand{\R}{\mathbb{R}}
\newcommand{\C}{\mathbb{C}}
\newcommand{\Q}{\mathbb{Q}}
\newcommand{\Z}{\mathbb{Z}}
\newcommand{\N}{\mathbb{N}}

% Calculus
\newcommand{\diff}{\mathop{}\!\mathrm{d}}
\newcommand{\deriv}[2]{\frac{\mathrm{d}#1}{\mathrm{d}#2}}
\newcommand{\pderiv}[2]{\frac{\partial #1}{\partial #2}}

% Linear Algebra
\newcommand{\inner}[2]{\langle #1, #2 \rangle}
\newcommand{\norm}[1]{\| #1 \|}
\newcommand{\tr}{\operatorname{tr}}
\newcommand{\spn}{\operatorname{span}}
\newcommand{\rank}{\operatorname{rank}}
\newcommand{\nullity}{\operatorname{nullity}}

% Logic
\newcommand{\contra}{\Rightarrow\Leftarrow}

% Custom commands for notes
\newcommand{\todo}[1]{\textcolor{red}{[TODO: #1]}}
\newcommand{\important}[1]{\textbf{\textcolor{blue}{#1}}}
% Theorem system 
% Theorem System original by https://github.com/kcajc/math-notes-template, this is modified 
% The following boxes are provided:
%   Definition:     \defn 
%   Theorem:        \thm 
%   Lemma:          \lem
%   Corollary:      \cor
%   Proposition:    \prop   
%   Claim:          \clm
%   Fact:           \fact
%   Proof:          \pf
%   Example:        \ex
%   Remark:         \rmk (sentence), \rmkb (block)
% Suffix
%   r:              Allow Theorem/Definition to be referenced, e.g. thmr
%   p:              Add a short proof block for Lemma, Corollary, Proposition or Claim, e.g. lemp
%                   For theorems, use \pf for proof blocks

% Definition - subsection numbering for 1.1.1, 1.1.2
\newtcbtheorem[number within=subsection]{mydefinition}{Definition}
{
    colbacktitle=green!20!white,
    colback=green!10!white,
    coltitle=black,
    fonttitle=\bfseries\large,
}{defn}

\NewDocumentCommand{\defn}{m+m}{
    \begin{mydefinition}{#1}{}
        #2
    \end{mydefinition}
}

\NewDocumentCommand{\defnr}{mm+m}{
    \begin{mydefinition}{#1}{#2}
        #3
    \end{mydefinition}
}

% Theorem - subsection numbering
\newtcbtheorem[use counter from=mydefinition]{mytheorem}{Theorem}
{
    colbacktitle=cyan!20!white,
    colback=cyan!10!white,
    coltitle=black,
    fonttitle=\bfseries\large,
}{thm}

\NewDocumentCommand{\thm}{m+m}{
    \begin{mytheorem}{#1}{}
        #2
    \end{mytheorem}
}

\NewDocumentCommand{\thmr}{mm+m}{
    \begin{mytheorem}{#1}{#2}
        #3
    \end{mytheorem}
}

% Lemma - subsection numbering
\newtcbtheorem[use counter from=mydefinition]{mylemma}{Lemma}
{
    colbacktitle=violet!20!white,
    colback=violet!10!white,
    coltitle=black,
    fonttitle=\bfseries\large,
}{lem}

\NewDocumentCommand{\lem}{m+m}{
    \begin{mylemma}{#1}{}
        #2
    \end{mylemma}
}

% Improved proof environments with consistent QED placement
\newenvironment{lempf}{
    \par\noindent{\it \textbf{Proof for Lemma.}}\par\nopagebreak
    \begin{list}{}{\setlength\leftmargin{1em}\setlength\rightmargin{0em}}
    \item\relax
}{
    \hfill$\qed$\end{list}
}

\NewDocumentCommand{\lemp}{m+m+m}{
    \begin{mylemma}{#1}{}
        #2
    \end{mylemma}
    \begin{lempf}
        #3
    \end{lempf}
}

% Corollary - subsection numbering
\newtcbtheorem[use counter from=mydefinition]{mycorollary}{Corollary}
{
    colbacktitle=orange!20!white,
    colback=orange!10!white,
    coltitle=black,
    fonttitle=\bfseries\large,
}{cor}

\NewDocumentCommand{\cor}{+m}{
    \begin{mycorollary}{}{}
        #1
    \end{mycorollary}
}

\newenvironment{corpf}{
    \par\noindent{\it \textbf{Proof for Corollary.}}\par\nopagebreak
    \begin{list}{}{\setlength\leftmargin{1em}\setlength\rightmargin{0em}}
    \item\relax
}{
    \hfill$\qed$\end{list}
}

\NewDocumentCommand{\corp}{m+m+m}{
    \begin{mycorollary}{}{}
        #1
    \end{mycorollary}
    \begin{corpf}
        #2
    \end{corpf}
}

% Proposition - subsection numbering
\newtcbtheorem[use counter from=mydefinition]{myproposition}{Proposition}
{
    colbacktitle=yellow!30!white,
    colback=yellow!20!white,
    coltitle=black,
    fonttitle=\bfseries\large,
}{prop}

\NewDocumentCommand{\prop}{+m}{
    \begin{myproposition}{}{}
        #1
    \end{myproposition}
}

\newenvironment{proppf}{
    \par\noindent{\it \textbf{Proof for Proposition.}}\par\nopagebreak
    \begin{list}{}{\setlength\leftmargin{1em}\setlength\rightmargin{0em}}
    \item\relax
}{
    \hfill$\qed$\end{list}
}

\NewDocumentCommand{\propp}{+m+m}{
    \begin{myproposition}{}{}
        #1
    \end{myproposition}
    \begin{proppf}
        #2
    \end{proppf}
}

% Claim - subsection numbering
\newtcbtheorem[use counter from=mydefinition]{myclaim}{Claim}
{
    colbacktitle=pink!30!white,
    colback=pink!20!white,
    coltitle=black,
    fonttitle=\bfseries\large,
}{clm}

\NewDocumentCommand{\clm}{m+m}{
    \begin{myclaim}{#1}{}
        #2
    \end{myclaim}
}

\newenvironment{clmpf}{
    \par\noindent{\it \textbf{Proof for Claim.}}\par\nopagebreak
    \begin{list}{}{\setlength\leftmargin{1em}\setlength\rightmargin{0em}}
    \item\relax
}{
    \hfill$\qed$\end{list}
}

\NewDocumentCommand{\clmp}{m+m+m}{
    \begin{myclaim}{#1}{}
        #2
    \end{myclaim}
    \begin{clmpf}
        #3
    \end{clmpf}
}

% Fact - subsection numbering
\newtcbtheorem[use counter from=mydefinition]{myfact}{Fact}
{
    colbacktitle=purple!20!white,
    colback=purple!10!white,
    coltitle=black,
    fonttitle=\bfseries\large,
}{fact}

\NewDocumentCommand{\fact}{+m}{
    \begin{myfact}{}{}
        #1
    \end{myfact}
}

% Proof - customizable name
\NewDocumentEnvironment{custompf}{m}
{
    \par\noindent{\it \textbf{#1}}\par\nopagebreak
    \begin{list}{}{\setlength\leftmargin{1em}\setlength\rightmargin{0em}}
    \item\relax
}
{
    \hfill$\qed$\end{list}
}

\NewDocumentCommand{\pf}{O{Proof}+m}{
    \begin{custompf}{#1.}
        #2
    \end{custompf}
}

% Example - improved environment
\newenvironment{example}{
    \par\vspace{5pt}
    \noindent\textbf{Example.}\par\nopagebreak
    \begin{list}{}{\setlength\leftmargin{1em}\setlength\rightmargin{0em}}
    \item\relax
}{
    \end{list}\vspace{5pt}
}

\NewDocumentCommand{\ex}{+m}{
    \begin{example}
        #1
    \end{example}
}

% Remark
\NewDocumentCommand{\rmk}{+m}{
    {\it \color{blue!50!white}#1}
}

% Remark block - improved environment
\newenvironment{remark}{
    \par\vspace{5pt}
    \noindent\textbf{Remark.}\par\nopagebreak
    \begin{list}{}{\setlength\leftmargin{1em}\setlength\rightmargin{0em}}
    \item\relax
}{
    \end{list}\vspace{5pt}
}

\NewDocumentCommand{\rmkb}{+m}{
    \begin{remark}
        #1
    \end{remark}
}

\title{MATH 481 Lecture 1}
\author{Deepak Jassal}
\date{January 8\textsuperscript{th}, 2026}

\begin{document}
\maketitle   
Motivating questions in analytic number theory.\\
What is the ``probability'' that a randomly chosen integer is 
\begin{itemize}
    \item prime?
    \item square free?
    \item has an odd number of prime factors?
\end{itemize}
Given a multiplicative function $(f:\N\to\C,\,f(mn)=f(m)f(n),\,\gcd(m,n=1))$, $\varphi(n),\,d(n),\,\sigma(n),$ etc.\\
What is the distribution of such functions on ``randomly'' chosen integers
\begin{itemize}
    \item minimum/maximum
    \item average value
    \item typical value
\end{itemize}
To make some of these questions more rigorous:\\
Let $x\geq 1$ be a real parameter\\
The answer for the questions for $n\in\{1,2,3,\dots,\left\lfloor x\right\rfloor \}$ will depend on our choice of $x$.\\
For example, consider $\dfrac{1}{x}\sum_{n\leq x}f(n)$ where $f(n)$ is a multiplicative function, and study it's behaviour in terms of $x$ for large values of $x$.
\ex{The average value $d(n)$
\[
    \lim_{x\to\infty}\frac{1}{x}\sum_{n\leq x}d(n)=\log(x), \quad \frac{1}{x}\sum_{n\leq x}d(n)\sim\log(x) 
\]}
Let $\pi(x)$ be the number of prime numbers less than or equal to $x$.
\[
    \pi(x)=\sum_{\substack{p\leq x\\ p\text{ prime}}}1.
\]
The (weak form) of the prime number theorem states that:
\[
    \pi(x)\sim\frac{x}{\log(x)}
\]
for large values of $x$.\\
In other words that is 
\[
    \lim_{x\to\infty}\frac{\pi(x)}{\frac{x}{\log(x)}}=1.
\]
A stronger version of the prime number theorem asserts that 
\[
    \pi(x)=\Li(x)+\Oh\left(e^{-c\sqrt{\log(x)}}\right)
\]
where
\[
    \Li(x)=\int_{2}^{x}\frac{dt}{\log(t)},
\]
is the logarithmic integral function.\\\
Using integration by parts we can see that 
\[
    \Li(x)=\frac{x}{\log(x)}+\Oh\left(\frac{x}{(\log(x))^3}\right).
\]
For a review of arithmetic and multiplicative functions please refer to chatper 1 of Hilderbrand's notes.\\
\setcounter{section}{1}
\section{Asymptotic Notation \& Average Values of Arithmetic Functions}
\subsection*{``Big Oh'' Notation}
Let $f:A\to\C$ be a function, and let $g(x)$ be a non-negative function.\\
We write $f(x)=\Oh(g(x))$ or $f(x)<<g(x)$ is there exists $C>0$ such that
\[
    |f(x)|\leq Cg(x)
\]
for all $x\in A$.\\
In this course we will mostly be interested in teh asymptotic behaviour of a function $f(x)$ for large values of $x$ (i.e., $x\to\infty$). In this case, the ``big oh'' notation can be understoof as follows
\[
    f(x\Oh(g(x)))
\]
whenever therre exists $C>0$ and $x_0>0$ such that
\[
    |f(x)|\leq Cg(x)
\]
for all $x\geq x_0$.
\subsection*{``Little Oh'' Notation}
We say $f(x)=\oh(g(x))$ if 
\[
    \lim_{x\to\infty}\frac{f(x)}{g(x)}=0
\] provided that $g(x)>0$ for large enough $x$.
In other words, given $\varepsilon>0$, there exists $x_0>1$ such that 
\[
    \left|\frac{f(x)}{g(x)}\right|\leq \varepsilon
\]
for all $x\geq x_0$.
Rearranging this expression we see that 
\[
    |f(x)|\leq \varepsilon g(x)\Leftrightarrow f(x)=\Oh(g(x).)
\]
\ex{
\begin{enumerate}
    \item $\log(x)=\Oh(x)$ because $\lim_{x\to\infty}\frac{\log(x)}{x}=0$.
    \item $\lim_{x\to\infty}f(x)=0\Leftrightarrow f(x)=\Oh(1)$. 
\end{enumerate}}
\subsection*{Asymptotic Equivalence}
We write
\[
    f(x)\sim g(x)
\]
if 
\[
    \lim_{x\to\infty}\frac{f(x)}{g(x)}=1.
\]
\subsection*{Order of Magnitude}
We write 
\[
    f(x)\asymp g(x)
\]
if
\[
    f(x)<<g(x),\text{ and} g(x)<<f(x).
\]
In other words, there exists $c_1,c_2\geq 0$ such that
\[
    c_1g(x)\leq f(x)\leq c_2g(x)
\]
\rmkb{$f(x)\sim g(x) \Rightarrow f(x)\asymp g(x)$ but the backwards implication is not always true.}
\ex{$x^2\asymp x^2,$ but $\lim_{x\to\infty}\frac{2x^2}{x^2}=2\neq1.$}
\subsection*{Asymptotic Formula}
\[
    f(x)=\underbrace{g(x)}_{\text{main term}}+\underbrace{\Oh(\mathcal{R}(x))}_{\text{error term}}
\]
\[
    |f(x)-g(x)|=\Oh(\mathcal{R}(x))\Leftrightarrow \exists C>0 \text{ such that} |f(x)-g(x)|\leq C\Oh(\mathcal{R}(x)).
\]
An asymptotic formula is only meaningful if 
\[
    \mathcal{R}(x)=\oh(g(x))
\]
i.e., the error term is much smaller than the main term.
\ex{$\cos(x)=1+\Oh(x^2)$ for $|x|<1$.}
\begin{proof}
    By the mean value theorem
    \[
    \left|\frac{\cos(x)-\cos(0)}{x-0}\right|=|\sin(c)|
    \]
    for some $0\leq c<|x|$. From here we can see that
    \[
    \left|\frac{1-\cos(x)}{|x|}\right|=|\sin(c)|\leq \max_{0\leq y\leq |x|}\{\sin(y)\}\leq |x|.
    \]
    Hence, 
    \begin{align*}
        1-\cos(x)&\leq x^2\\
        &=\Oh(x^2)\\
        \cos(x)&=1+\Oh(x^2).\qedhere 
    \end{align*}
\end{proof}
\subsection*{Dependance on Paramters}
Suppose that $f$ and $g$ depend on a parameter $\lambda$. Then, the notation 
\[
    f(x)=\Oh_{\lambda}(g(x))
\]
\[
    f(x)\underset{\lambda}{<<}(g(x))
\]
means that the implicit constant depends on $\lambda$.
\ex{
\begin{enumerate}
    \item $\log(x)=\Oh_\varepsilon(x^\varepsilon)$ for any $\varepsilon>0$.
    \item Given $0<\alpha<1$ then for any $A>0$ and $\varepsilon>0$ we have
    \[
        x^{-\varepsilon}\underset{\alpha,\varepsilon}{<<}e^{-c(\log(x))^\alpha}\underset{\alpha,A}{<<} (\log(x))^{-A}.
    \]
\end{enumerate}
}
\subsection*{Next Time}
Next time we will consdier 
\[
    \frac{1}{x}\sum_{n\leq x}\mu(n),
\]
and then move on to developing the summation formulas that allow us to compute average values ($\frac{1}{x}\sum{n\leq x}f(n)$) of arithmetic functions.
\end{document}
