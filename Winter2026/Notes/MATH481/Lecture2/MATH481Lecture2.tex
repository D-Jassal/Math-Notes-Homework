\documentclass[12pt]{article}
\usepackage{color,soul}
% Import preambles and macros for notes
% Essential packages for notes
\usepackage{amsmath, amssymb, amsthm}
\usepackage{mathtools}  % for \coloneqq, etc.
\usepackage{geometry}   % Better page margins
\usepackage{parskip}    % Better paragraph spacing
\usepackage{microtype}  % Better typography
\usepackage{enumitem}   % Customize lists
\usepackage{hyperref}   % Clickable links
\usepackage{booktabs}   % Better tables
\usepackage{tcolorbox}  % For colored boxes/theorems

% Page layout for notes
\geometry{a4paper, margin=1in}
\setlength{\parskip}{0.8em}

% Theorem environments with shared numbering
\newtheorem{theorem}{Theorem}[subsection]  % Number within sections: 2.3.1, 2.3.2, etc.

% All other environments share the same counter as theorem
\newtheorem{lemma}[theorem]{Lemma}
\newtheorem{proposition}[theorem]{Proposition}
\newtheorem{corollary}[theorem]{Corollary}
\newtheorem{definition}[theorem]{Definition}
\newtheorem{example}[theorem]{Example}
\newtheorem{remark}[theorem]{Remark}
\newtheorem{claim}[theorem]{Claim}

% Custom colors for notes
\usepackage{xcolor}
\definecolor{note-blue}{RGB}{220, 230, 255}
\definecolor{theorem-green}{RGB}{220, 255, 220}
% Math notation shortcuts for notes
\newcommand{\R}{\mathbb{R}}
\newcommand{\C}{\mathbb{C}}
\newcommand{\Q}{\mathbb{Q}}
\newcommand{\Z}{\mathbb{Z}}
\newcommand{\N}{\mathbb{N}}

% Calculus
\newcommand{\diff}{\mathop{}\!\mathrm{d}}
\newcommand{\deriv}[2]{\frac{\mathrm{d}#1}{\mathrm{d}#2}}
\newcommand{\pderiv}[2]{\frac{\partial #1}{\partial #2}}

% Linear Algebra
\newcommand{\inner}[2]{\langle #1, #2 \rangle}
\newcommand{\norm}[1]{\| #1 \|}
\newcommand{\tr}{\operatorname{tr}}
\newcommand{\spn}{\operatorname{span}}
\newcommand{\rank}{\operatorname{rank}}
\newcommand{\nullity}{\operatorname{nullity}}

% Logic
\newcommand{\contra}{\Rightarrow\Leftarrow}

% Custom commands for notes
\newcommand{\todo}[1]{\textcolor{red}{[TODO: #1]}}
\newcommand{\important}[1]{\textbf{\textcolor{blue}{#1}}}
\input{../../../../preambles/theorem-system-section.tex}

\title{MATH 481 Lecture 2}
\author{Deepak Jassal}
\date{January 13\textsuperscript{th}, 2026}

\begin{document}
\maketitle   
\setcounter{section}{2}
\setcounter{subsection}{1}
Many important problems in analytic number theory reduce to evaluating (partial) sums of the form
\[
    \sum_{n\leq x}a_n
\]
where $a_n$ is an arithmetic function.
\ex{
    \begin{enumerate}
        \item Let $\Psi(x)=\sum_{n\leq x}\Lambda(n)$ where $\Lambda(n)$ is the Von Mangoldt function. The prime number theorem (PNT) is equivalent to $\Psi(x)\sim x$.\\ Under the Reimann hypothesis we have
        \[
            \Psi(x)=x+\Oh\left(x^{\frac{1}{2}+\varepsilon}\right)
        \]
        for any $\varepsilon>0$.
        \item The PNT is equivalent to 
        \[
            \sum_{n\leq x}\mu(n)=\oh(x).
        \]
        i.e.,
        \[
            \lim_{x\to\infty}\frac{1}{x}\sum_{n\leq x}\mu(n)=0.
        \]
        Where $\mu(n)$ is the M\"{o}bius function.\\
        Some interesting facts about $\sum_{n\leq x}\mu(n)$. Merten's conjectured that $|\sum_{n\leq x}\mu(n)|\leq\sqrt{x}$. In 1985 Odlyzko proved that $|\sum_{n\leq x}\mu(n)|>1.06\sqrt{x}$ for infinitely many $x$.\\
        $\sum_{n\leq x}\mu(n)$ is not a straighforward or easy sum to work with and get significant results on it's asymptotic growth.\\
        Instead of considering this sum we will consider the weighted sum
        \[
        \sum_{n\leq x}\frac{\mu(n)}{n}.
        \]
        This is a common trick in analytic number theory (ANT). If $\sum_{n\leq x}f(n)$ is a difficult sum to work with, instead work with $\sum_{n\leq x}\frac{\mu(n)}{\omega(n)}$ where $\omega(n)$ is a ``nice'' weight function.
        \item Special Case of the Hardy Littlewood Conjecture
        \[
            \sum_{n\leq x}\Lambda(n)\Lambda(n+2)\sim Cx
        \]
        Where $C$ is given by \[C=\prod_{\substack{p\geq 3\\p\text{ prime}}}\left(1-\frac{1}{(p-1)^2}\right).\]
        This is a special case of the twin prime conjecture.
    \end{enumerate}
}
Let's consider some partial suns that bac be dealt with without the use of summation formulas.
\ex{
    \begin{enumerate}
        \item $\sum_{n\leq x}n=\left\lfloor x\right\rfloor$, where $\left\lfloor x\right\rfloor=x-\{x\}=x+\Oh(x)$ and $\{x\}$ is the fractional part of $x$.
        \item 
        \begin{align*}
            \sum_{n\leq x}d(n)&=\sum_{n\leq x}\sum_{d\mid n}1\\
            &=\sum_{d\leq x}\sum_{n\leq x}1\\
            &=\sum_{d\leq x}\left\lfloor\frac{x}{d}\right\rfloor\\
            &=\sum_{d\leq x}\left(\frac{x}{d}+\Oh(x)\right)\\
            &=x\sum_{d\leq x}\frac{1}{d}+\Oh(x)\\
            &=x(\log x +\Oh(x))+\Oh(x)\\
            \sum_{n\leq x}d(n)&=x\log x +\Oh(x).
        \end{align*}
        From here it is clear to see that
        \[
            \frac{1}{x}\sum_{n\leq x}d(n)=\log x + \Oh(1).
        \]
        \item We prove that
        \[
            \left|\sum_{n\leq x}\frac{\mu(n)}{n}\right|\leq 1.
        \]
        We have
        \begin{align*}
            \left|\sum_{n\leq x}\frac{\mu(n)}{n}\right|&=\frac{1}{x}\left|\sum_{n\leq x}\mu(n)\frac{x}{n}\right|\\
            &=\frac{1}{x}\left|\sum_{n\leq x}\mu(n)\left(\left\lfloor \frac{x}{n}\right\rfloor + \left\{\frac{x}{n}\right\}\right)\right|\\
            &\leq\frac{1}{x}\left|\sum_{n\leq x}\mu(n)\left\lfloor \frac{x}{n}\right\rfloor\right| + \frac{1}{x}\left|\sum_{n\leq x}\mu(n)\left\{ \frac{x}{n}\right\}\right|.
        \end{align*}
        We will first deal with the sum on the right
        \begin{align*}
            \frac{1}{x}\left|\sum_{n\leq x}\mu(n)\left\{ \frac{x}{n}\right\}\right|&\leq\frac{1}{x}\sum_{n\leq x}\left|\mu(n)\right|\left\{ \frac{x}{n}\right\}\\
            &\leq\sum_{n\leq x}\left\{\frac{x}{n}\right\}\\
            &=\sum_{n\leq \lfloor x\rfloor}\left\{\frac{x}{n}\right\}\\
            &=\sum_{n\leq \lfloor x\rfloor -1}\left\{\frac{x}{n} \right\}+\left\{\frac{x}{\lfloor x\rfloor} \right\}\\
            &\leq\floor{x}-1+\left\{\frac{\floor{x}+\{x\}}{\floor{x}}\right\}\\
            &\leq \floor{x}-1+\left\{1+\frac{\{x\}}{\floor{x}} \right\}\\
            &\leq \floor{x}-1+\frac{\fract{x}}{\floor{x}}\\
            &\leq \floor{x}-1+\fract{x}=x-1.
        \end{align*}
        Now for the left hand sum
        \begin{align*}
            \sum_{n\leq x}\mu(n)\floor{\frac{x}{n}}&=\sum_{n\leq x}\mu(n)\sum_{m\leq \frac{x}{n}\Leftrightarrow l=mn\leq x}1\\
            &\sum_{l\leq x}\sum_{n\mid l}\mu(n)\\
            &=1
        \end{align*}
        From MATH480 we know that 
        \[
            \sum_{n\mid l}\mu(n)=
            \begin{cases}
                1& \text{if } l=1\\
                0& \text{otherwise}.
            \end{cases}
        \]
        Combining these two results we arrive at
        \begin{align*}
            \left|\sum_{n\leq x}\frac{\mu(n)}{n}\right|&=\frac{1}{x}\left|\sum_{n\leq x}\mu(n)\frac{x}{n}\right|\\
            &\leq \frac{1}{x}(1+x-1)\\
            =1.
        \end{align*}
    \end{enumerate}
}
\subsection{Summation of Smooth Functions (Euler's Summation Formula)}
\thmr{Euler's Summation Formula}{eulersum}{
    Let $o<y\leq x$ be real numbers and $f\in\mathcal{C}[y,x]$. Then,
    \[
        \sum_{y<n\leq x}f(n)=\int_{y}^{x}f(t)\,dt+\int_{y}^{x}\fract{t}f'(t)dt-\fract{x}f(x)+\fract{y}f(y).
    \]
}
\corp{
    Let $x\geq 1$ be a real number, $f\in\mathcal{C}[1,x]$. Then \ref{thm:eulersum} gives
    \[
        \sum_{n\leq x}f(n)=\int_{1}^{x}f(t)\,dt+\int_{1}^{x}\fract{t}f'(t)dt-\fract{x}f(x)+f(1)
    \]
}{
    Apply theorem \ref{thm:eulersum} with $y=1$.
}
Before proving theorem \ref{thm:eulersum} we establish the following application which was stated earlier.
\thmr{Partial Sums of the Harmonic Series}{harmonic}{
    \[
        \sum_{n\leq x}\frac{1}{n}=\log x +\gamma +\Oh\left(\frac{1}{x}\right)
    \]
    where $\gamma$ is the Euler-Mascheroni constant defined asymptotic
    \[
        \gamma=\lim_{x\to\infty}\left(\sum_{n\leq x}\frac{1}{n}-\log x\right)=0.5772\dots
    \]
}
\pf{
    Applying corollary 2.2 with $f(n)=\frac{1}{n}$ we get the following
    \begin{align*}
        \sum_{n\leq x}\frac{1}{n}&=\int_{1}^{x}\frac{1}{t}\,dt+\int_{1}^{x}\fract{t}\left(\frac{-1}{t^2}\right)\,dt-\fract{x}\frac{1}{x}+1\\
        &=\log x +1 -\int_{1}^{x}\frac{\fract{t}}{t^2}\,dt+\Oh\left(\frac{1}{x}\right).
    \end{align*}
    Set $I(x)=\int_{1}^{x}\frac{\fract{t}}{t^2}\,dt$. Then,
    \begin{align*}
        I(x)&=\int_{1}^{\infty}\frac{\fract{t}}{t^2}\,dt-\int_{x}^{\infty}\frac{\fract{t}}{t^2}\,dt\\
        &=I-\int_{x}^{\infty}\frac{\fract{t}}{t^2}\,dt.
    \end{align*}    
    Observe that
    \[
        \int_{x}^{\infty}\frac{\fract{t}}{t^2}\,dt\leq \int_{x}^{\infty}\frac{1}{t^2}\,dt=\left[\frac{-1}{t}\right]^\infty_x=\frac{1}{x}.
    \]
    Thus,
    \[
        I(x)=I+\Oh\left(\frac{1}{x}\right).
    \]
    We now have
    \[
        \sum_{n\leq x}\frac{1}{n}=\log x -I + 1+\Oh\left(\frac{1}{x}\right).    
    \]
    Rearranging this we see that
    \[
    \lim_{x\to\infty}\left(\sum_{n\leq x}\frac{1}{n}-\log x\right)=\gamma=\lim_{x\to\infty}\left(1-I+\Oh\left(\frac{1}{x}\right)\right).
    \]
    Hence,
    \[
        \sum_{n\leq x}\frac{1}{n}=\log x +\gamma +\Oh\left(\frac{1}{x}\right). \qedhere
    \]
}
\pf[Proof of Theorem \ref{thm:eulersum}]{
    Set $k=\floor{x}$ and $m=\floor{y}$. Then we rewrite the sum asymptotic
    \[
        \sum_{y<n\leq x}f(n)=\sum_{m+1\leq n\leq k}f(n).
    \]
    Consider $n$ such that $n-1,n\in[y,x]$. Then,
    \begin{align*}
    \int_{n-1}^{n}\floor{t}f(t)\,dt&=\int_{n-1}^{n}(n-1)f'(t)\,dt\\
    &=(n-1)(f(n)-f(n-1))\\
    &=nf(n)-(n-1)f(n-1)-f(n)
    \end{align*}
}
\end{document}