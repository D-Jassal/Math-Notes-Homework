\documentclass[12pt]{article}
\usepackage{color,soul}
% Import preambles and macros for notes
% Essential packages for notes
\usepackage{amsmath, amssymb, amsthm}
\usepackage{mathtools}  % for \coloneqq, etc.
\usepackage{geometry}   % Better page margins
\usepackage{parskip}    % Better paragraph spacing
\usepackage{microtype}  % Better typography
\usepackage{enumitem}   % Customize lists
\usepackage{hyperref}   % Clickable links
\usepackage{booktabs}   % Better tables
\usepackage{tcolorbox}  % For colored boxes/theorems

% Page layout for notes
\geometry{a4paper, margin=1in}
\setlength{\parskip}{0.8em}

% Theorem environments with shared numbering
\newtheorem{theorem}{Theorem}[subsection]  % Number within sections: 2.3.1, 2.3.2, etc.

% All other environments share the same counter as theorem
\newtheorem{lemma}[theorem]{Lemma}
\newtheorem{proposition}[theorem]{Proposition}
\newtheorem{corollary}[theorem]{Corollary}
\newtheorem{definition}[theorem]{Definition}
\newtheorem{example}[theorem]{Example}
\newtheorem{remark}[theorem]{Remark}
\newtheorem{claim}[theorem]{Claim}

% Custom colors for notes
\usepackage{xcolor}
\definecolor{note-blue}{RGB}{220, 230, 255}
\definecolor{theorem-green}{RGB}{220, 255, 220}
% Math notation shortcuts for notes
\newcommand{\R}{\mathbb{R}}
\newcommand{\C}{\mathbb{C}}
\newcommand{\Q}{\mathbb{Q}}
\newcommand{\Z}{\mathbb{Z}}
\newcommand{\N}{\mathbb{N}}

% Calculus
\newcommand{\diff}{\mathop{}\!\mathrm{d}}
\newcommand{\deriv}[2]{\frac{\mathrm{d}#1}{\mathrm{d}#2}}
\newcommand{\pderiv}[2]{\frac{\partial #1}{\partial #2}}

% Linear Algebra
\newcommand{\inner}[2]{\langle #1, #2 \rangle}
\newcommand{\norm}[1]{\| #1 \|}
\newcommand{\tr}{\operatorname{tr}}
\newcommand{\spn}{\operatorname{span}}
\newcommand{\rank}{\operatorname{rank}}
\newcommand{\nullity}{\operatorname{nullity}}

% Logic
\newcommand{\contra}{\Rightarrow\Leftarrow}

% Custom commands for notes
\newcommand{\todo}[1]{\textcolor{red}{[TODO: #1]}}
\newcommand{\important}[1]{\textbf{\textcolor{blue}{#1}}}
\input{../../../../preambles/theorem-system-section.tex}

\title{MATH 481 Lecture 4}
\author{Deepak Jassal}
\date{January 20\textsuperscript{th}, 2026}

\begin{document}
\setcounter{section}{2}
\setcounter{tcb@cnt@mydefinition}{6}
\maketitle
Another application of Euler's summation formula.\\
The Reimann zeta function is defined as 
\[
    \zeta(s)=\sum_{n=1}^{\infty}\frac{1}{n^s},\quad s\in\C,\; \Re(s)>1.
\]
This is defined as an analytic function in the complex variable $s$ on the half plane $\Re(s)>1$. $\zeta(s)$ converges absolutely for $\Re(s)>1$.
\ex{
    \[
        \zeta(2)=\sum_{n=1}^{\infty}\frac{1}{n^2}=\frac{\pi^2}{6}
    \]
}
\ex{
    \[
        \zeta(2k)\in\Q^{2k},\quad k\in\N.
    \]
}
\thm{Integral Form of the Reimann Zeta Function}{
    \[
        \zeta(s)=\frac{s}{s-1}-s\int_{1}^{\infty}\frac{\fract{x}}{x^{s+1}}\,dx,\quad \Re(s)>0.
    \]
}
\pf[Proof of Theorem 2.7]{
    Consider for $x\geq1$
    \[
        S(x)=\sum_{n\leq x}\frac{1}{n^s},
    \]
    observe that 
    \[
        \lim_{x\to\infty}S(x)=\zeta(s).
    \]
    By Euler's summation formula we have
    \begin{align*}
        S(x)&=\sum_{n\leq x}\frac{1}{n^s}\\
        &=\int_{1}^{x}\frac{1}{t^s}\,dt-s\int_{1}^{x}\fract{t}t^{-s-1}\,dt-\fract{x}f(x)+f(1)\\
        &=\frac{1}{s-1}-\frac{x^{-s+1}}{s-1}-s\int_{1}^{\infty}\fract{t}t^{-s-1}\,dt+\int_{x}^{\infty}+\int_{x}^{\infty}\fract{t}t^{-s-1}\,dt-\fract{x}x^{-s}+1.
    \end{align*}
    Note that
    \begin{align*}
        \left|\int_{x}^{\infty}\fract{t}t^{-s-1}\,dt\right|&\leq\int_{x}^{\infty}\fract{t}|t^{-s-1}|\,dt\\
        &=\int_{x}^{\infty}\fract{t}t^{\Re(s)-1}\,dt\\
        &\leq\int_{x}^{\infty}t^{\Re(s)-1}\,dt\\
        &=\left(\frac{t^{-\Re(s)}}{-\Re(s)}\right)^\infty_x=\frac{x^{-\Re(s)}}{-\Re(s)}.
    \end{align*}
    Hence, 
    \[
        S(x)=\frac{1}{s-1}-\frac{x^{-s-1}}{s-1}-s\left(\int_{1}^{\infty}\fract{t}t^{-s-1}\,dt+\Oh\left(\frac{x^{-\Re(s)}}{-\Re(s)}\right)\right).
    \]
    As $x\to\infty$ we get $\zeta(s)$
    \begin{align*}
        \zeta(s)&=\frac{1}{s-1}-s\int_{1}^{\infty}\fract{t}t^{-s-1}\,dt+1\\
        &=\frac{s}{s-1}-s\int_{1}^{\infty}\fract{t}t^{-s-1}\,dt.
    \end{align*}
    Which is the desired result.
}
Recall PNT
    \[
        \pi(x)=
    \]

\lemp{Logarithmic Integral}{
    \[
        \mathrm{Li}_k(x)=\int_{2}^{x}\frac{1}{(\log t)^k}\,dt\ll_k\frac{x}{(\log x)^k}.
    \]
}{
    \[
        \int_{2}^{x}\frac{1}{(\log x)^k}\,dt
    \]
    we have
    \[
        \log t\geq\log2,\quad \frac{1}{\log t}\leq\frac{1}{\log2},\quad \frac{1}{(\log t)^k}<\frac{x-2}{(\log2)^k}\ll x-2\ll x.
    \]
    \[
        \underbrace{\int_{2}^{\sqrt{x}}\frac{dt}{(\log t)^k}}_{I_1}+\underbrace{\int_{\sqrt{x}}^{x}\frac{dt}{(\log t)^k}}_{I_2}.
    \]
    \[
        I_2\leq\int_{2}^{\sqrt{x}}\frac{dt}{(\log2)^k}=\frac{\sqrt{x}-2}{(\log2)^k}\ll_k\sqrt{x}\ll\frac{x}{(\log x)^k},
    \]
    for $\sqrt{x}< t< x$
    \[
        \log t\geq \log\sqrt{2}=\frac{1}{2}\log x
    \]
    \[
        \frac{1}{(\log t)^k}\leq\frac{2^k}{(\log x)^k}.
    \]
    \[
        I_2\leq \int_{\sqrt{x}}^{x}\frac{2^k}{(\log t)^k}\,dt=\frac{2^k(x-\sqrt{x})}{(\log x)^k}\ll_k\frac{x}{(\log x)^k}.
    \]
    Hence, 
    \[
        \int_{2}^{x}\frac{2^k}{(\log t)^k}\,dt=\Oh\left(\frac{x}{(\log x)^k}\right),\quad x\geq 4.
    \]
    For $2\leq x\leq 4$ we have
    \[
        \int_{2}^{x}\frac{1}{(\log t)^k}\,dt\leq\int_{2}^{x}\frac{1}{(\log 2)^k}\,dt=\frac{x-2}{(\log 2)^k}\leq\frac{x}{(\log 2)^k}\leq\left(\frac{\log x}{\log2}\right)\frac{x}{(\log x)^k}\ll\frac{x}{(\log x)^k}.
    \]
    Which is the desired result.
}
\thm{Summation of Logarithmic Integral}{
    \[
        \mathrm{Li(x)}=\int_{2}^{x}\frac{dt}{\log t}=\frac{x}{\log x}\left(\sum_{i=1}^{k-1}\frac{i!}{(\log x)^i+\Oh\left(\frac{1}{(\log x)^k}\right)}\right).
    \]
}
\pf[Proof of Theorem 2.9]{
    We have 
    \[
        \int_{2}^{x}\frac{dt}{\log t}
    \]
    \[
        u=\frac{1}{\log t},\quad dv=dt,\quad du=\frac{-1}{(\log t)^2},\quad v=t
    \]
    \begin{align*}
        \int_{2}^{x}\frac{dt}{\log t}&=\left[\frac{t}{\log t}\right]^x_2+\int_{2}^{x}\frac{dt}{t(\log t)^2}\\
        &=\frac{x}{\log x}-\frac{2}{\log2}+\frac{x}{(\log x)^2}-\frac{2}{(\log 2)^2}+2\int_{2}^{x}\frac{dt}{(\log t)^3}\\
        &=\frac{x}{\log x}+\frac{x}{(\log x)^2}+\frac{2x}{(\log x)^3}+\cdots+k\int_{2}^{x}\frac{dt}{(\log t)^{k+1}}+\Oh_k(1)\\
        &=\frac{x}{\log x}+\frac{x}{(\log x)^2}+\frac{2x}{(\log x)^3}+\cdots+(k-1)!\frac{x}{(\log x)^k}+\Oh_k\left(\frac{x}{(\log x)^{k+1}}\right)\\
        &=\frac{x}{\log x}\left(\sum_{i=1}^{k-1}\frac{i!}{(\log x)^i+\Oh\left(\frac{1}{(\log x)^k}\right)}\right).
    \end{align*}
    Thus we have arrived at the desired result.
}
We can use Euler's summation formula for sum continuous functions. Abel's Summation Formula allows for us to sum arithmetic functions.
\thm{Abel's Summation Formula}{
    Let $0<y\leq x$, $f\in\mathcal{C}[y,x]$ and $a(n)$ an arithmetic function, then
    \[
        \sum_{y<n\leq x}a(n)f(n)=A(n)f(x)-A(y)f(y)-\int_{y}^{x}A(t)f'(t)\,dt,
    \]
    where $A(n)$ is the summatory fuction for $a(n)$ $\left(A(n\sum_{n\leq x}a(n))\right)$.
}
\end{document}