\documentclass[12pt]{article}
\usepackage{color,soul}
% Import preambles and macros for notes
% Essential packages for notes
\usepackage{amsmath, amssymb, amsthm}
\usepackage{mathtools}  % for \coloneqq, etc.
\usepackage{geometry}   % Better page margins
\usepackage{parskip}    % Better paragraph spacing
\usepackage{microtype}  % Better typography
\usepackage{enumitem}   % Customize lists
\usepackage{hyperref}   % Clickable links
\usepackage{booktabs}   % Better tables
\usepackage{tcolorbox}  % For colored boxes/theorems

% Page layout for notes
\geometry{a4paper, margin=1in}
\setlength{\parskip}{0.8em}

% Theorem environments with shared numbering
\newtheorem{theorem}{Theorem}[subsection]  % Number within sections: 2.3.1, 2.3.2, etc.

% All other environments share the same counter as theorem
\newtheorem{lemma}[theorem]{Lemma}
\newtheorem{proposition}[theorem]{Proposition}
\newtheorem{corollary}[theorem]{Corollary}
\newtheorem{definition}[theorem]{Definition}
\newtheorem{example}[theorem]{Example}
\newtheorem{remark}[theorem]{Remark}
\newtheorem{claim}[theorem]{Claim}

% Custom colors for notes
\usepackage{xcolor}
\definecolor{note-blue}{RGB}{220, 230, 255}
\definecolor{theorem-green}{RGB}{220, 255, 220}
% Math notation shortcuts for notes
\newcommand{\R}{\mathbb{R}}
\newcommand{\C}{\mathbb{C}}
\newcommand{\Q}{\mathbb{Q}}
\newcommand{\Z}{\mathbb{Z}}
\newcommand{\N}{\mathbb{N}}

% Calculus
\newcommand{\diff}{\mathop{}\!\mathrm{d}}
\newcommand{\deriv}[2]{\frac{\mathrm{d}#1}{\mathrm{d}#2}}
\newcommand{\pderiv}[2]{\frac{\partial #1}{\partial #2}}

% Linear Algebra
\newcommand{\inner}[2]{\langle #1, #2 \rangle}
\newcommand{\norm}[1]{\| #1 \|}
\newcommand{\tr}{\operatorname{tr}}
\newcommand{\spn}{\operatorname{span}}
\newcommand{\rank}{\operatorname{rank}}
\newcommand{\nullity}{\operatorname{nullity}}

% Logic
\newcommand{\contra}{\Rightarrow\Leftarrow}

% Custom commands for notes
\newcommand{\todo}[1]{\textcolor{red}{[TODO: #1]}}
\newcommand{\important}[1]{\textbf{\textcolor{blue}{#1}}}
\input{../../../../preambles/theorem-system-section.tex}

\title{MATH 481 Lecture 7}
\author{Deepak Jassal}
\date{February 3\textsuperscript{rd}, 2026}

\begin{document}
\setcounter{section}{2}
\setcounter{tcb@cnt@mydefinition}{16}
\maketitle
\subsection*{Dirichlet's Hyperbola Method}
Suppose we have two arithmetic functions
\[
    f,g:\N\to\C.
\]
Let
\[
    F(x)=\sum_{n\leq x}f(n),\quad G(x)=\sum_{n\leq x}g(x).
\]
Consider $h=f\ast g$.\\
i.e.,
\[
    h(n)=\sum_{d\mid n}f(d)g\left(\frac{n}{d}\right).
\]
We have
\begin{align*}
    H(x)=\sum_{n\leq x}h(n)&=\sum_{n\leq x}\sum_{d\mid n}f(d)g\left(\frac{n}{d}\right)\\
    &=\sum_{n\leq x}\sum_{n=ab}f(a)g(b)\\
    &=\sum_{\substack{a,b\\ ab\leq x}}f(a)g(b)
    \intertext{let $y>0$}
    &=\sum_{\substack{ab\leq x\\ a\leq y}}f(a)g(b)+\sum_{\substack{ab\leq x\\ a> y}}f(a)g(b)\\
    &=\sum_{a\leq y}f(a)\sum_{b\leq\frac{x}{a}}g(b)+\sum_{b\leq \frac{x}{y}}g(b)\sum_{y<a\leq\frac{x}{b}}f(a)\\
    &=\sum_{a\leq y}f(a)G(\frac{x}{a})+\sum_{b\leq \frac{x}{y}}g(b)\left(F\left(\frac{x}{b}\right)-F(y)\right)\\\displaybreak
    &=\sum_{a\leq y}f(a)G(\frac{x}{a})+\sum_{b\leq \frac{x}{y}}g(b)F\left(\frac{x}{b}\right)-\sum_{b\leq \frac{x}{y}}g(b)F(y)\\
    &=\sum_{a\leq y}f(a)G\left(\frac{x}{a}\right)+\sum_{b\leq \frac{x}{y}}g(b)F\left(\frac{x}{b}\right)-G\left(\frac{x}{y}\right)F(y)
\end{align*}
If we choose $y=\sqrt{x}$ (this choice depends on the specific functions you are dealing with), the above can be written as 
\[
    \sum_{n\leq x}f\ast g(n)=\sum_{a\leq \sqrt{x}}f(a)G\left(\frac{x}{a}\right)+\sum_{b\leq \sqrt{x}}g(b)F\left(\frac{x}{b}\right)-G(\sqrt{x})F(\sqrt{x}).
\]
Why is this called the ``hyperbola method''? If we use this method for $\sum_{n\leq x}d(n)$ we have
\[
    \sum_{n\leq x}d(n)=\sum_{n\leq x}\sum_{ab=n}1=\sum_{\substack{a,b\\ab\leq x}}1.
\]
We can see that the first sum, sums over lattice points under hyperbola $ab=x$ of points with $a\leq\sqrt{x}$ and the second sums over points with $b\leq\sqrt{x}$ and the thid term subtracts one set of points that have been double counted.
\thm{Sum of Divisors}{
    \[
        \sum_{n\leq x}d(n)=x\log x+(2\gamma -1)x+\Oh(\sqrt{x}).
    \]
}
\pf[Proof of Theorem 2.17]{
    \begin{align*}
        \sum_{n\leq x}d(n)=\sum_{\substack{a,b\\ ab\leq x}}1&=\sum_{a\leq \sqrt{x}}\sum_{b\leq \frac{x}{a}}1+\sum_{b\leq \sqrt{x}}\sum_{a\leq \frac{x}{b}}1-\sum_{b\leq \sqrt{x}}\sum_{b\leq \sqrt{x}}1\\
        &=\sum_{a\leq \sqrt{x}}\floor{\frac{x}{a}}+\sum_{b\leq \sqrt{x}}\floor{\frac{x}{b}}-\sum_{b\leq \sqrt{x}}\floor{\sqrt{x}}\\
        &=2\sum_{a\leq \sqrt{x}}\floor{\frac{x}{a}}-\floor{\sqrt{x}}\floor{\sqrt{x}}\\
        &=2\sum_{a\leq \sqrt{x}}\left(\frac{x}{a}-\fract{\frac{x}{a}}\right)-(\sqrt{x}-\fract{\sqrt{x}})^2\\
        &=2x\sum_{a\leq \sqrt{x}}\frac{1}{a}-2\sum_{a\leq \sqrt{x}}\fract{\frac{x}{a}}-x-2\sqrt{x}\fract{\sqrt{x}}-\fract{\sqrt{x}}^2\\
        &=2x\left(\log\sqrt{x}+\gamma+\Oh\left(\frac{1}{\sqrt{x}}\right)\right)-x+\Oh(\sqrt{x})\\
        &=2\log x+(2\gamma-1)x+\Oh(\sqrt{x}).
    \end{align*}
    Which is the desired result.
}
\begin{remark}
    Dirichlet's Divisor Problem\\
    \[
        \Delta(x)=D(x)-(x\log x+(2\gamma-1)x)\tag{$\ast$}
    \]
    \[
        \text{Theorem 2.17}\Rightarrow \Delta(x)=\Oh(\sqrt{x})
    \]
    \[
        \text{Conjecture:} \Delta(x)=\Oh(x^{\frac{1}{4}}).
    \]
    Hardy et al. showed that $\Delta(x)=\Oh(x^{\frac{1}{3}})$. The state of the art is $\Delta(x)=\Oh(x^\theta)$ with $\theta=0.31\dots$.\\
    $d(n)\leq2$, $d(n)\geq\exp\left(\frac{(1+\varepsilon)\log2\log n}{\log\log n}\right)$ for any $\varepsilon>0$. $d(n)\leq C(\varepsilon)n^\varepsilon$ for any $\varepsilon>0$. $d(n)\ll_\varepsilon n^\varepsilon$.
\end{remark}
\section{Elementary Results for the Prime Coutning Function}
\end{document}