\documentclass[12pt]{article}
\usepackage{color,soul}
% Import preambles and macros for notes
% Essential packages for notes
\usepackage{amsmath, amssymb, amsthm}
\usepackage{mathtools}  % for \coloneqq, etc.
\usepackage{geometry}   % Better page margins
\usepackage{parskip}    % Better paragraph spacing
\usepackage{microtype}  % Better typography
\usepackage{enumitem}   % Customize lists
\usepackage{hyperref}   % Clickable links
\usepackage{booktabs}   % Better tables
\usepackage{tcolorbox}  % For colored boxes/theorems

% Page layout for notes
\geometry{a4paper, margin=1in}
\setlength{\parskip}{0.8em}

% Theorem environments with shared numbering
\newtheorem{theorem}{Theorem}[subsection]  % Number within sections: 2.3.1, 2.3.2, etc.

% All other environments share the same counter as theorem
\newtheorem{lemma}[theorem]{Lemma}
\newtheorem{proposition}[theorem]{Proposition}
\newtheorem{corollary}[theorem]{Corollary}
\newtheorem{definition}[theorem]{Definition}
\newtheorem{example}[theorem]{Example}
\newtheorem{remark}[theorem]{Remark}
\newtheorem{claim}[theorem]{Claim}

% Custom colors for notes
\usepackage{xcolor}
\definecolor{note-blue}{RGB}{220, 230, 255}
\definecolor{theorem-green}{RGB}{220, 255, 220}
% Math notation shortcuts for notes
\newcommand{\R}{\mathbb{R}}
\newcommand{\C}{\mathbb{C}}
\newcommand{\Q}{\mathbb{Q}}
\newcommand{\Z}{\mathbb{Z}}
\newcommand{\N}{\mathbb{N}}

% Calculus
\newcommand{\diff}{\mathop{}\!\mathrm{d}}
\newcommand{\deriv}[2]{\frac{\mathrm{d}#1}{\mathrm{d}#2}}
\newcommand{\pderiv}[2]{\frac{\partial #1}{\partial #2}}

% Linear Algebra
\newcommand{\inner}[2]{\langle #1, #2 \rangle}
\newcommand{\norm}[1]{\| #1 \|}
\newcommand{\tr}{\operatorname{tr}}
\newcommand{\spn}{\operatorname{span}}
\newcommand{\rank}{\operatorname{rank}}
\newcommand{\nullity}{\operatorname{nullity}}

% Logic
\newcommand{\contra}{\Rightarrow\Leftarrow}

% Custom commands for notes
\newcommand{\todo}[1]{\textcolor{red}{[TODO: #1]}}
\newcommand{\important}[1]{\textbf{\textcolor{blue}{#1}}}
\input{../../../../preambles/theorem-system-section.tex}

\title{MATH 420 Lecture 9}
\author{Deepak Jassal}
\date{February 4\textsuperscript{th}, 2026}

\begin{document}
\stepcounter{section}
\maketitle 
\ex{
    $R=\Z[\sqrt{-5}]$\\
    \textit{Claim:} $(2),(3),(1+\sqrt{-5}),(1-\sqrt{-5})$ are not prime (as ideals). We have that $(2)$ is prime \textit{iff} $R/(2)$ is an integral domain.\\
    $R/(2)$ is a set of equivalence classes.\\
    i.e., $[1]_{(2)}=1+(2),[0]_{(2)},[\sqrt{-5}]_{(2)},[1+\sqrt{-5}]_{(2)}$ are these all the equivalence classes?\\
    Consider $[a+b\sqrt{-5}]_{(2)}$ with $a,b\in\Z$. If $a\equiv b\equiv 0\mod2$ then $[a+b\sqrt{-5}]_{(2)}=[0]_{(2)}$. If $a$ is odd, and $b$ is even we get $[a+b\sqrt{-5}]_{(2)}=[1]_{(2)}$. If $a$ is even and $b$ is odd then $[a+b\sqrt{-5}]_{(2)}=[\sqrt{-5}]_{(2)}$. If $a$ is odd and $b$ is odd then $[a+b\sqrt{-5}]_{(2)}=[1+\sqrt{-5}]_{(2)}$. Thus we have $R/(2)=\{[0]_{(2)},[1]_{(2)},[b\sqrt{-5}]_{(2)},[1+\sqrt{-5}]_{(2)}\}$. 
    \[
        [1+\sqrt{-5}]_{(2)}[1+\sqrt{-5}]_{(2)}=[0]_{(2)}.
    \]
    Since there are zero divisors we have that $R/(2)$ is not an integral domain. Thus, $(2)$ is not prime.\\
    $R/(3)=\{[0],[1],[2],[1+\sqrt{-5}],[1+2\sqrt{-5}],[2+\sqrt{-5}],[2+2\sqrt{-5}],[\sqrt{-5}],[2\sqrt{-5}]\}$
    \[
        [1+2\sqrt{-5}][1+\sqrt{-5}]=[1+3\sqrt{-5}-10]=[0].
    \]
    $p=11$, $R/(p)$. Let $a_1,b_1\in\{0,1,\dots,10\}$. Want $[a_1+b_1\sqrt{-5}]_{(11)}$ is invertible, provided that $a_1,b_1$ are not both zero.\\
    \textit{Consider}: $[a_2+b_2\sqrt{-5}]_{(11)}$ then
    \[
       [a_1+b_1\sqrt{-5}]_{(11)}[a_2+b_2\sqrt{-5}]_{(11)}=[a_1a_2-5b_1b_2+\sqrt{-5}(a_1b_2+a_2b_1)]\tag{$\ast$}.
    \]
    Easy if one or the other is zero. Choose $(a_2,b_2)=k(a_1-b_1)$. Then ($\ast$) becomes $[k(a_1^5)+5b_1^2]_{(11)}$. Then $a_1^5+5b_1^2$ is divisible by 11 \textit{iff} $a_1,b_1$ are both divisible by 11.
}
Since $(2),(3)$ are \underline{not} prime ideals so they should be ``divisible'' by some other ideal. In particular, not prime implies not maximal. So there is some ideal $J_2,J_3$ such that $(2)\subsetneqq J_2$ and $(3)\subsetneqq J_3$. Extend (2) with $1+\sqrt{-5}$ to get $\langle 2,1+\sqrt{-5} \rangle$.\\
$R/\langle2,1+\sqrt{-5} \rangle=\{[0],[1]\}=\mathbb{F}_2$. Thus, the ideal is maximal and therefore prime. 
\[
    J_2\cdot J_2=\{\sum_{j=1}^{2}a_jb_j:a_j,b_j\in J_2\}.
\]
$a=2x+y(1+\sqrt{-5})$, $b=2u+v(1+\sqrt{-5})$. Then
\[
    a\cdots b=0+(2),
\]
Thus
\[
    J_2\cdot J_2=(2).
\]
$(3)=\langle3,1+\sqrt{-5} \rangle\langle3-\sqrt{-5} \rangle$ and neither of these ideals are the whole ring.\\
\textit{Summary:} $(2)=J_2^2$ and $(3)=J_3\overline{J_3}$, ($J_3=\langle3,1+\sqrt{-5} \rangle$, $\overline{J_3}=\langle3-\sqrt{-5} \rangle$).\\
$(5)$ is also a square. (Excersize).\\
(11) is a prime ideal.\\
$(41)=(6+\sqrt{-5})(6-\sqrt{-5})$.








\end{document}