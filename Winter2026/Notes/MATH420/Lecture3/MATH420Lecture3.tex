\documentclass[12pt]{article}
\usepackage{color,soul}
% Import preambles and macros for notes
% Essential packages for notes
\usepackage{amsmath, amssymb, amsthm}
\usepackage{mathtools}  % for \coloneqq, etc.
\usepackage{geometry}   % Better page margins
\usepackage{parskip}    % Better paragraph spacing
\usepackage{microtype}  % Better typography
\usepackage{enumitem}   % Customize lists
\usepackage{hyperref}   % Clickable links
\usepackage{booktabs}   % Better tables
\usepackage{tcolorbox}  % For colored boxes/theorems

% Page layout for notes
\geometry{a4paper, margin=1in}
\setlength{\parskip}{0.8em}

% Theorem environments with shared numbering
\newtheorem{theorem}{Theorem}[subsection]  % Number within sections: 2.3.1, 2.3.2, etc.

% All other environments share the same counter as theorem
\newtheorem{lemma}[theorem]{Lemma}
\newtheorem{proposition}[theorem]{Proposition}
\newtheorem{corollary}[theorem]{Corollary}
\newtheorem{definition}[theorem]{Definition}
\newtheorem{example}[theorem]{Example}
\newtheorem{remark}[theorem]{Remark}
\newtheorem{claim}[theorem]{Claim}

% Custom colors for notes
\usepackage{xcolor}
\definecolor{note-blue}{RGB}{220, 230, 255}
\definecolor{theorem-green}{RGB}{220, 255, 220}
% Math notation shortcuts for notes
\newcommand{\R}{\mathbb{R}}
\newcommand{\C}{\mathbb{C}}
\newcommand{\Q}{\mathbb{Q}}
\newcommand{\Z}{\mathbb{Z}}
\newcommand{\N}{\mathbb{N}}

% Calculus
\newcommand{\diff}{\mathop{}\!\mathrm{d}}
\newcommand{\deriv}[2]{\frac{\mathrm{d}#1}{\mathrm{d}#2}}
\newcommand{\pderiv}[2]{\frac{\partial #1}{\partial #2}}

% Linear Algebra
\newcommand{\inner}[2]{\langle #1, #2 \rangle}
\newcommand{\norm}[1]{\| #1 \|}
\newcommand{\tr}{\operatorname{tr}}
\newcommand{\spn}{\operatorname{span}}
\newcommand{\rank}{\operatorname{rank}}
\newcommand{\nullity}{\operatorname{nullity}}

% Logic
\newcommand{\contra}{\Rightarrow\Leftarrow}

% Custom commands for notes
\newcommand{\todo}[1]{\textcolor{red}{[TODO: #1]}}
\newcommand{\important}[1]{\textbf{\textcolor{blue}{#1}}}
\input{../../../../preambles/theorem-system-section.tex}

\title{MATH 420 Lecture 3}
\author{Deepak Jassal}
\date{January 14\textsuperscript{th}, 2026}

\begin{document}
\setcounter{section}{1}
\maketitle  
\textit{Recall.} Qoutients in ring theory were briefly mentioned.\\
The original motivation for this phenomenon came from modular arithmetic.\\
Consider the (OG ring) $(\Z,+,\times)$, We can consider an arithmetic induced by a positive integer $m$, through remainders when divied by m.\\
\ex{
    If $m=3$, every integer can be written as $3x$, $3x+1$ or $3x+2$.
    \[
      (3x+1)(3y+2)=\underbrace{9x+6x+3y}_{3z}+2.  
    \]
}
The modern way to define modular arithmetic modula $m$:
\begin{enumerate}
    \item Introduce an equivalence relation $\sim_m$ on $\Z$:\\
    $a\sim_m b$ \textit{if and only if} $a-b$ is divisible by $m$.
    \item Introduce addition and multiplication to the equivalence classes with respect to $\sim_m$
    \[
        [a]_m+[b]_m:=[a+b]_m,
    \]
    \[
        [a]_m[b]_m:=[a\times b]_m.
    \]
    \item Check this is well defined. That is if the representative of a class changes, output does not chance under addition or multiplication.
\end{enumerate}
Then we get a new ring $(\Z/m\Z,+,\times)$.\\\\
Can this be done in general?\\
If the ring is commutative we get a very nice thoery. If the ring is not commutative this becomes much more complicated; more interpretations.\\
Even for commutative rings quotients of the form $R/S$ with $S$ an arbitrary subring are not necessarily well behaved.\\
For $R/S$ to inherit the ring structure on $R$ we need that $[a]_S[b]_S=[ab]_S$ to be well defined. If there exists $s\in S$ and $r\in R$, then
\[
    [r]_S\underbrace{[s]_S}_{=[0]_S}=[rs]_S\neq0.
\]
Thus $S$ needs to satisfy a much stronger condition, that being $sr\in S$ for all $s\in S$ and $r\in R$. Such subrings are called ``ideals''.
\defn{Ideals}{
    Let $(R,+,\times)$ be a commutative ring. A subset $S\subset R$ togeth with $+,\times$ is said to be \underline{ideal} if 
    \begin{enumerate}
        \item $(S,+)$ is an Abelian group;
        \item for all $s\in S$ and $r\in R$ we have $sr\in S$. 
    \end{enumerate}
}
\propp{
    Let $(R,+,\times)$ be a commutative ring and $I\subseteq R$ an ideal. Then $R/I,+,\times$ is a ring.
}{
    \textit{Sketch}.\\
    We have to show that 
    \[
        [a_1]_I[b_1]_I=[a_2]_I[b_2]_I.
    \]
    Hypothesis implies that there exists $s_1,s_2\in I$ such that $a_2=a_1+s_1$ and $b_2=b_1+s_2$. Then,
    \begin{align*}
        [a_2]_I[b_2]_I&=[a_2b_2]_I\\
        &=[(a_1+s_1)(b_1+s_2)]_I\\
        &=[a_1b_1+s_1b_1+s_2a_1+s_1s_2]_I\\
        [a_1]_I[b_1]_I&=[a_1b_1]_I
    \end{align*}
}
To study rings, just like vector spaces we want to understand the \underline{maps} between rings.
\defn{Ring Homomorphisms}{
    Let $(R,+,\times)$,$(S,\oplus,\otimes)$ be rings.\\
    A function $f:R\mapsto S$ is said to be a homomorphism If
    \begin{enumerate}
        \item $f(0_R)=0_S$;
        \item $f(a+b)=f(a)\oplus f(b)$;
        \item $f(ab)=f(a)\otimes f(b)$.
    \end{enumerate}
}
\fact{
    If $R,S$ are commutative rings, and $f:R\mapsto S$ is a homomorphism, then the kernel fo $f$ is an ideal.
}
\end{document}