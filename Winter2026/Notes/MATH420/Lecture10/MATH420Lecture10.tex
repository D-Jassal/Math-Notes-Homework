\documentclass[12pt]{article}
\usepackage{color,soul}
% Import preambles and macros for notes
% Essential packages for notes
\usepackage{amsmath, amssymb, amsthm}
\usepackage{mathtools}  % for \coloneqq, etc.
\usepackage{geometry}   % Better page margins
\usepackage{parskip}    % Better paragraph spacing
\usepackage{microtype}  % Better typography
\usepackage{enumitem}   % Customize lists
\usepackage{hyperref}   % Clickable links
\usepackage{booktabs}   % Better tables
\usepackage{tcolorbox}  % For colored boxes/theorems

% Page layout for notes
\geometry{a4paper, margin=1in}
\setlength{\parskip}{0.8em}

% Theorem environments with shared numbering
\newtheorem{theorem}{Theorem}[subsection]  % Number within sections: 2.3.1, 2.3.2, etc.

% All other environments share the same counter as theorem
\newtheorem{lemma}[theorem]{Lemma}
\newtheorem{proposition}[theorem]{Proposition}
\newtheorem{corollary}[theorem]{Corollary}
\newtheorem{definition}[theorem]{Definition}
\newtheorem{example}[theorem]{Example}
\newtheorem{remark}[theorem]{Remark}
\newtheorem{claim}[theorem]{Claim}

% Custom colors for notes
\usepackage{xcolor}
\definecolor{note-blue}{RGB}{220, 230, 255}
\definecolor{theorem-green}{RGB}{220, 255, 220}
% Math notation shortcuts for notes
\newcommand{\R}{\mathbb{R}}
\newcommand{\C}{\mathbb{C}}
\newcommand{\Q}{\mathbb{Q}}
\newcommand{\Z}{\mathbb{Z}}
\newcommand{\N}{\mathbb{N}}

% Calculus
\newcommand{\diff}{\mathop{}\!\mathrm{d}}
\newcommand{\deriv}[2]{\frac{\mathrm{d}#1}{\mathrm{d}#2}}
\newcommand{\pderiv}[2]{\frac{\partial #1}{\partial #2}}

% Linear Algebra
\newcommand{\inner}[2]{\langle #1, #2 \rangle}
\newcommand{\norm}[1]{\| #1 \|}
\newcommand{\tr}{\operatorname{tr}}
\newcommand{\spn}{\operatorname{span}}
\newcommand{\rank}{\operatorname{rank}}
\newcommand{\nullity}{\operatorname{nullity}}

% Logic
\newcommand{\contra}{\Rightarrow\Leftarrow}

% Custom commands for notes
\newcommand{\todo}[1]{\textcolor{red}{[TODO: #1]}}
\newcommand{\important}[1]{\textbf{\textcolor{blue}{#1}}}
\input{../../../../preambles/theorem-system-section.tex}

\title{MATH 420 Lecture 10}
\author{Deepak Jassal}
\date{February 9\textsuperscript{th}, 2026}

\begin{document}
\stepcounter{section}
\maketitle 
\subsection*{Why Rings?}
We want to do computations with rings. So we want to be able to run the original algorithm: the \underline{Euclidean algorithm}.
\defn{Euclidean Norm}{
    Let $D$ be an integral domain. A Euclidean norm (valuation) is a function $\mu:D\to\Z_{\geq0}$ such that 
    \begin{enumerate}[label=(\alph*)]
        \item for all $a,b\in D$ with $b\neq0$, there exists $q,r\in D$ such that
        \[
            a=qb+r
        \]
        and either $\mu(r)=0$ or $\mu(r)<\mu(b)$
        \item and for all $a,b\in D\setminus\{0\}$, $\mu(a)\leq\mu(ab)$.
    \end{enumerate}
}
Part (a) of the definition induces a division algorithm on $D$.\\
There is a division algorithm, and Euclidean algorithm on $\R[x]$. The valuation is given by $\deg(f)$.
\defn{Eulclidean Domain}{
    A domain $D$ norm $\mu$ is a \underline{Euclidean domain}.
}
\ex{
    $\Z$, $\R[x],$ $\Z[\sqrt{-1}]$ are Eulclidean domains, and $\Z[\sqrt{-5}]$ is not a Eulclidean domain.
}
\fact{
    Impotant: Eulciean domain $\subset$ principal ideal domain (PID) $\subset$ unique factorization domains (UFD).
}
Towards Euclidean algorithm.
\defn{Divisors}{
    Let $R$ be a commutative ring. Suppose $a,b\in R$, $b\neq0$. We say $b$ divides $a$ or $b\vert a$ if there exists $q\in R$ such that $a=qb$. Otherwise we say $b\nmid a$.
}
\defn{GCD}{
    Let $a,b\in R$ we say $g\in R$ is a greatest common divisor of $a,b$ if $g\vert a$ and $g\vert b$, and for all $c\in R$ such that $c\vert a$ and $c\vert b$ we have $c\vert g$.
}
\thm{The Extended Euclidean Algorithm}{
    Let $D$ be a Euclidean domain and $a,b\in D$. Then $a,b$ has a gcd in $A$, say $g$. Further there exist $x,y\in D$ such that 
    \[
        g=ax+by.
    \]
}
\pf[Proof of the Theorem]{
    Put $I=\{ax+by:x,y\in D\}=(a)+(b)$ which is an ideal. Using the fact that $D$ being a Eulclidean domain implies that $D$ is a PID (will be proved later). There exists $d\in D$ such that $I=(d)$. Then there exists $p,q\in D$ such that $d=ap+bq$. Note that $a\in I$ and $b\in I$. Hence there exists $k_1,k_2\in I$ such that $a=dk_1$ and $b=dk_2$. Hence $d$ is a common divisor of $a,b$. Any common divisor of $a,b$ must also divide $app+bq$, so $d$ is a gcd. Suppose that $g$ is another gcd of $a,b$ then we have $g\vert d$ and $g\vert d$. Hence $d=gm$ and $g=dn$ for some $m,n\in D$. $d=(dn)m=d(mn)$
}
\end{document}