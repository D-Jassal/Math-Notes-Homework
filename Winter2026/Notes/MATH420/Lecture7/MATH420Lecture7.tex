\documentclass[12pt]{article}
\usepackage{color,soul}
% Import preambles and macros for notes
% Essential packages for notes
\usepackage{amsmath, amssymb, amsthm}
\usepackage{mathtools}  % for \coloneqq, etc.
\usepackage{geometry}   % Better page margins
\usepackage{parskip}    % Better paragraph spacing
\usepackage{microtype}  % Better typography
\usepackage{enumitem}   % Customize lists
\usepackage{hyperref}   % Clickable links
\usepackage{booktabs}   % Better tables
\usepackage{tcolorbox}  % For colored boxes/theorems

% Page layout for notes
\geometry{a4paper, margin=1in}
\setlength{\parskip}{0.8em}

% Theorem environments with shared numbering
\newtheorem{theorem}{Theorem}[subsection]  % Number within sections: 2.3.1, 2.3.2, etc.

% All other environments share the same counter as theorem
\newtheorem{lemma}[theorem]{Lemma}
\newtheorem{proposition}[theorem]{Proposition}
\newtheorem{corollary}[theorem]{Corollary}
\newtheorem{definition}[theorem]{Definition}
\newtheorem{example}[theorem]{Example}
\newtheorem{remark}[theorem]{Remark}
\newtheorem{claim}[theorem]{Claim}

% Custom colors for notes
\usepackage{xcolor}
\definecolor{note-blue}{RGB}{220, 230, 255}
\definecolor{theorem-green}{RGB}{220, 255, 220}
% Math notation shortcuts for notes
\newcommand{\R}{\mathbb{R}}
\newcommand{\C}{\mathbb{C}}
\newcommand{\Q}{\mathbb{Q}}
\newcommand{\Z}{\mathbb{Z}}
\newcommand{\N}{\mathbb{N}}

% Calculus
\newcommand{\diff}{\mathop{}\!\mathrm{d}}
\newcommand{\deriv}[2]{\frac{\mathrm{d}#1}{\mathrm{d}#2}}
\newcommand{\pderiv}[2]{\frac{\partial #1}{\partial #2}}

% Linear Algebra
\newcommand{\inner}[2]{\langle #1, #2 \rangle}
\newcommand{\norm}[1]{\| #1 \|}
\newcommand{\tr}{\operatorname{tr}}
\newcommand{\spn}{\operatorname{span}}
\newcommand{\rank}{\operatorname{rank}}
\newcommand{\nullity}{\operatorname{nullity}}

% Logic
\newcommand{\contra}{\Rightarrow\Leftarrow}

% Custom commands for notes
\newcommand{\todo}[1]{\textcolor{red}{[TODO: #1]}}
\newcommand{\important}[1]{\textbf{\textcolor{blue}{#1}}}
\input{../../../../preambles/theorem-system-section.tex}

\title{MATH 420 Lecture 6}
\author{Deepak Jassal}
\date{January 26\textsuperscript{th}, 2026}

\begin{document}
\stepcounter{section}
\maketitle 
For fields, a key invariant if the \underline{characteristic}. $\mathrm{Char}(F)=$ smallest positive integer $p$ such that $\underbrace{1+\cdots+1}_{p\text{ times}}$. If such $p$ exists, $\mathrm{Char}(F)=0$. The problem arises when one considers fields with ``mixed characteristic''.\\
Fields with the same mixed characteristic play nice with each other, enabling various \underline{cohomology theories} (\'etale, de Rham, crystalline, prismatic).\\
If a field of one element does exist, it would sync up all of these difference cohomology theories ``somehow''. That's why such an object is unlikely.\\
\defn{Prime Ideals}{
    Let $(D,+,\times)$ be a domian. We say an ideal $I\subseteq D$ is \underline{prime} if for all ideals $A,B\subseteq D$, if $A\cdot B\subseteq I$, then $A\subseteq I$ or $B\subseteq I$.
}
\defn{Generated Ideals}{
    Let $R$ be a ring and $S\subseteq R$. The ideal generated by $S$ denoted $\langle S\rangle$ is the smalled ideal containing $S$.
}
\ex{
    Let $D=\Z$. $(k)-$ ideal generated by $\{k$\}. $p\mid ab$ means every multiple of $ab$ is also a multiple of $p$. This implies that $(ab)\subseteq(p)$, the first is multplies of $ab$ and the second is all multiples of $p$.
}
\defn{Maximal Ideals}{
    An ideal $I\subsetneqq R$ is \underline{maximal} if whenever $J\subseteq R$ is an ideal and $I\subseteq J$, then $I=J$ or $J=R$.
}
\lemp{Maximal Ideals in $\Z$}{
    If $D=\Z$ then every prime ideal is maximal.
}{
    Let $p$ be a prime. Then $(p)$ is a prime ideal. We want to show that it is maximal. Suppose $(p)\subseteq J$. We may assume that $J\neq\Z$. Then $J=(k)$ (because $\Z$ is a PID) for some $k\in \Z$. Then ``every multiple of $p$ is a multiple of $k$''. In particular, $p$ is a multiple of $k$. Hence $k=\pm p,\pm1$. If $k=\pm1$, $J=R=\Z$ which violates the assumption. Hence $k=\pm p$ so $(k)=(p)$.
}
\propp{
    Let $(D,+,\times)$ be a domain. Then every maximal ideal $I\subseteq D$ is prime.
}{
    Suppose $I$ is maximal but not prime. Then there exists $A,B\subseteq D$ ideals such that  $AB\subseteq I$ but $A\nsubseteq I$ and $B\nsubseteq I$. There exists $a\in A\setminus I$ and $b\in B\setminus I$, but $ab\in I$.\\
    Let $J=\langle I\cup \{a\}\rangle$, $K=\{t+ar:i\in I,r\in D\}$. $I\subset J$, but $I\neq K$ (because $a\neq\in I$). Hence, $K=D$. Hence, $1\in K$. Therefore, there exists $i\in I$, $r\in D$ such that $1=i+ar$. Thus, $b1=b(i+ar)=bi+bar$, $bi\in bI\subseteq DI=I$. Then $bar\in I\Rightarrow b\in I$ and this is a contradiction.
}
\thm{}{
    Let $R$ be a unital, commutative ring. Let $I\subseteq R$ be an ideal. Then $a)\Leftrightarrow b)$, whenever
    \begin{enumerate}
        \item $R\setminus I$ is an integral domain;
        \item $I$ is prime.
    \end{enumerate}
}
\thm{}{
    Let $R$ be a unital, commutative ring. Let $I\subseteq R$ be an ideal. Then $a) \leftrightarrow b)$, whenever
    \begin{enumerate}
        \item $R\setminus I$ is a field
        \item $I$ is \underline{maximal}.
    \end{enumerate}
}
Theorem 1.7 to 1.6 is how we get that all maximal ideals in an itegral domain are prime. All field's are integral domains.
\end{document}