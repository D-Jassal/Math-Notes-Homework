\documentclass[12pt]{article}
\usepackage{color,soul}
% Import preambles and macros for notes
% Essential packages for notes
\usepackage{amsmath, amssymb, amsthm}
\usepackage{mathtools}  % for \coloneqq, etc.
\usepackage{geometry}   % Better page margins
\usepackage{parskip}    % Better paragraph spacing
\usepackage{microtype}  % Better typography
\usepackage{enumitem}   % Customize lists
\usepackage{hyperref}   % Clickable links
\usepackage{booktabs}   % Better tables
\usepackage{tcolorbox}  % For colored boxes/theorems

% Page layout for notes
\geometry{a4paper, margin=1in}
\setlength{\parskip}{0.8em}

% Theorem environments with shared numbering
\newtheorem{theorem}{Theorem}[subsection]  % Number within sections: 2.3.1, 2.3.2, etc.

% All other environments share the same counter as theorem
\newtheorem{lemma}[theorem]{Lemma}
\newtheorem{proposition}[theorem]{Proposition}
\newtheorem{corollary}[theorem]{Corollary}
\newtheorem{definition}[theorem]{Definition}
\newtheorem{example}[theorem]{Example}
\newtheorem{remark}[theorem]{Remark}
\newtheorem{claim}[theorem]{Claim}

% Custom colors for notes
\usepackage{xcolor}
\definecolor{note-blue}{RGB}{220, 230, 255}
\definecolor{theorem-green}{RGB}{220, 255, 220}
% Math notation shortcuts for notes
\newcommand{\R}{\mathbb{R}}
\newcommand{\C}{\mathbb{C}}
\newcommand{\Q}{\mathbb{Q}}
\newcommand{\Z}{\mathbb{Z}}
\newcommand{\N}{\mathbb{N}}

% Calculus
\newcommand{\diff}{\mathop{}\!\mathrm{d}}
\newcommand{\deriv}[2]{\frac{\mathrm{d}#1}{\mathrm{d}#2}}
\newcommand{\pderiv}[2]{\frac{\partial #1}{\partial #2}}

% Linear Algebra
\newcommand{\inner}[2]{\langle #1, #2 \rangle}
\newcommand{\norm}[1]{\| #1 \|}
\newcommand{\tr}{\operatorname{tr}}
\newcommand{\spn}{\operatorname{span}}
\newcommand{\rank}{\operatorname{rank}}
\newcommand{\nullity}{\operatorname{nullity}}

% Logic
\newcommand{\contra}{\Rightarrow\Leftarrow}

% Custom commands for notes
\newcommand{\todo}[1]{\textcolor{red}{[TODO: #1]}}
\newcommand{\important}[1]{\textbf{\textcolor{blue}{#1}}}
\input{../../../../preambles/theorem-system-section.tex}

\title{MATH 420 Lecture 6}
\author{Deepak Jassal}
\date{January 26\textsuperscript{th}, 2026}

\begin{document}
\stepcounter{section}
\maketitle 
\thm{Second Isomorphism Theorem}{
    Let $R$ be a ring and $I,J$ ideals of $R$. Then $I$ is an ideal of $I+J$, and $I\cap J$ is an ideal of $J$ then
    \[
        I+J\setminus J\cong I\setminus I\cap J
    \]
}
\pf[Proof of the Second Isomorphism Theorem]{
    Clearly $I\subseteq I+J$, and also a subring. For $r\in I+J\Rightarrow r\in R$, so $r\cdot i\in I$ for $i,r\in I$ for all $i\in I$, since $I\subseteq R$ is an ideal in $R$. Similar argument shows $I\cap J$ is an ideal in $J$. Now define
    \[
        \tau:I+J\to J\setminus I\cap J,
    \]
    \[
        \tau(i+j)=i+(I\cap J)=[I]_{I\cap J}.
    \]
    It is possible that $i_1+j_1=i_2+j_2$ so we must check that this is well-defined.
    \[
        \underset{\in I}{i_1-i_2}=\underset{\in J}{j_2-j_1}\Rightarrow i_1-i_2,j_2-j_1\in I\cap J.
    \]
    \[
        \tau(i_1+j_2)=i_1+(I\cap J)=i_1-(i_1-i_2)+(I\cap J)=i_2+(I\cap J)=\tau(i_2+j_2).
    \]
    Next, we must show that $\tau$ is a homomorphism.
    \begin{align*}
        \tau(i_1+j_2+i_2+j_2)&=\tau(i_1+i_2+j_1+j_2)\\
        &=i_1+i_2+(I\cap J)\\
        &=i_1+(I\cap J)+i_2+(I\cap J)\\
        &=\tau(i_1+j_1)+\tau(i_2+j_2),
    \end{align*}
    \begin{align*}
        \tau((i_1+j_1)(i_2+j_2))&=\tau(i_1i_2+i_1j_2+j_1i_2+j_1j_2)\\
        &=i_1i_2+(I\cap J)\\
        &=(i_1+(I\cap J))(i_2+(I\cap J))\\
        &=\tau(i_1+j_1)\tau(i_2+j_2).
    \end{align*}
    Since $\tau(i+j)=i+I\cap J$ we have that $\tau$ is surrjective.\\
    Now we need only show that $\ker(\tau)=J$.\\
    Suppose $j\in J$. Then $\tau(0+j)=0+(I\cap J)$, so $j\in\ker(\tau)$ ($J\subseteq\ker(\tau)$).\\
    Now suppose that $j\in\ker(\tau)$. Then $\tau(i+j)=i+I\cap J=0+(I\cap J)$. Hence, $i\in (I\cap J)$, to $i\in J$. Therefore, $i+j\in J$. So $\ker(\tau)\subseteq J$. Thus, $J=\ker(\tau)$. Hense, by the first isomorphism theorem we have our desired result. 
}
\thm{Third Isomorphism Theorem}{
    Let $R$ be a a ring and $I,J$ ideals of $R$ such that $I\subseteq J$. Then,
    \[
        R\setminus J\cong \frac{R\setminus I}{J\setminus I}.
    \]
}
\pf[Proof of the Third Isomorphism Theorem]{
    Left as an excersize.
}
The earliest\footnote{Probably the earliest} rings to be considered are close cousins of $\Z:$ (integral) domains.
\defn{Integral Domains}{
    A ring $(D,+,\times)$ is siad to be an \textit{integral domain} if
    \begin{enumerate}
        \item $D$ is unital (has multiplicative identity $1\neq 0$);
        \item $D$ is commutative;
        \item for $a,b\in D$, if $ab=0$, then $a=0$ or $b=0$.
    \end{enumerate}
}
\end{document}