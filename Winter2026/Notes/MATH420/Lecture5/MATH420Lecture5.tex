\documentclass[12pt]{article}
\usepackage{color,soul}
% Import preambles and macros for notes
% Essential packages for notes
\usepackage{amsmath, amssymb, amsthm}
\usepackage{mathtools}  % for \coloneqq, etc.
\usepackage{geometry}   % Better page margins
\usepackage{parskip}    % Better paragraph spacing
\usepackage{microtype}  % Better typography
\usepackage{enumitem}   % Customize lists
\usepackage{hyperref}   % Clickable links
\usepackage{booktabs}   % Better tables
\usepackage{tcolorbox}  % For colored boxes/theorems

% Page layout for notes
\geometry{a4paper, margin=1in}
\setlength{\parskip}{0.8em}

% Theorem environments with shared numbering
\newtheorem{theorem}{Theorem}[subsection]  % Number within sections: 2.3.1, 2.3.2, etc.

% All other environments share the same counter as theorem
\newtheorem{lemma}[theorem]{Lemma}
\newtheorem{proposition}[theorem]{Proposition}
\newtheorem{corollary}[theorem]{Corollary}
\newtheorem{definition}[theorem]{Definition}
\newtheorem{example}[theorem]{Example}
\newtheorem{remark}[theorem]{Remark}
\newtheorem{claim}[theorem]{Claim}

% Custom colors for notes
\usepackage{xcolor}
\definecolor{note-blue}{RGB}{220, 230, 255}
\definecolor{theorem-green}{RGB}{220, 255, 220}
% Math notation shortcuts for notes
\newcommand{\R}{\mathbb{R}}
\newcommand{\C}{\mathbb{C}}
\newcommand{\Q}{\mathbb{Q}}
\newcommand{\Z}{\mathbb{Z}}
\newcommand{\N}{\mathbb{N}}

% Calculus
\newcommand{\diff}{\mathop{}\!\mathrm{d}}
\newcommand{\deriv}[2]{\frac{\mathrm{d}#1}{\mathrm{d}#2}}
\newcommand{\pderiv}[2]{\frac{\partial #1}{\partial #2}}

% Linear Algebra
\newcommand{\inner}[2]{\langle #1, #2 \rangle}
\newcommand{\norm}[1]{\| #1 \|}
\newcommand{\tr}{\operatorname{tr}}
\newcommand{\spn}{\operatorname{span}}
\newcommand{\rank}{\operatorname{rank}}
\newcommand{\nullity}{\operatorname{nullity}}

% Logic
\newcommand{\contra}{\Rightarrow\Leftarrow}

% Custom commands for notes
\newcommand{\todo}[1]{\textcolor{red}{[TODO: #1]}}
\newcommand{\important}[1]{\textbf{\textcolor{blue}{#1}}}
\input{../../../../preambles/theorem-system-section.tex}

\title{MATH 420 Lecture 5}
\author{Deepak Jassal}
\date{January 21\textsuperscript{th}, 2026}

\begin{document}
\stepcounter{section}
\maketitle  
\textit{Last Time:} Defined a \underline{canonical} isomorphism (candidate) $\tau:R\setminus\ker(f)\to f(R)$.
\[
    \tau(a+\underbrace{\ker(f)}_{+I})=f(a)
\]
is well defined.\\
\[
    \tau((a+I)+(b+I))=\tau((a+b)+I)=f(a+b)=f(a)\oplus f(b)=\tau(a+I)\oplus\tau(b+I),
\]
\[
    \tau((a+I)(b+I))=\tau((ab)+I)=f(a+b)=f(a)\otimes f(b)=\tau(a+I)\otimes\tau(b+I).
\]
Thus, $\tau$ is a homomorphism.\\
Must show that $\tau$ is bijective.\\
If $y\in f(R)$, then there exists $e\in R$ such that $f(a)=y$. Hence, $\tau(a+I)=f(a)=y$. Thus, $\tau$ is surrjective.\\
If $\tau(a+I)=\tau(b+I)$, then 
\begin{align*}
    0_S&=\tau(a+I)\oplus(-\tau(b+I))\\
    &=f(a)\oplus f(b)\\
    &=f(a)\oplus f(-b)\\
    &=\tau((a-b)+I)\Rightarrow a-b\in I,
\end{align*}
thus, $a+I=b+I$. Hence, $\tau$ as a function is bijective. Therefore, $\tau^{-1}$ is well defined.\\
It remains to show that $\tau^{-1}$ is a homomorphism. Given that $y_1=f(a_1)$ and $y_2=f(a_2)$
\begin{align*}
    \tau^{-1}(y_1\oplus y_2)&=\tau^{-1}(f(a_1)\oplus f(a_2))\\
    &=\tau^{-1}(f(a_1+a_2))\\
    &=\tau^{-1}(\tau((a_1+a_2)+I))\\
    &=(a_1+a_2)+I\\
    &=\tau^{-1}(y_1)+\tau^{-1}(y_2)
\end{align*}
Similar process for multiplication. Thus, we have that $\tau^{-1}$ is a homomorphism.\\
Ideals were actually meant to be ``numbers''. In fact, ``Ideal numbers'' are a thing. This compels and \underline{arithmetic} on the set of ideals.
\defn{Arithmetic on Ideal}{
    Let $R$ be a ring, and $I,J$ (2-sided) ideals. Then:
    \[
        I+J:=\{i+j:i\in I,j\in J\}
    \]
    \[
        I\cap J=\{m:m\in I\cap J\}
    \]
    \[
        I\cdot J:=\left\{\sum_{\ell=1}^{q}j_\ell i_\ell:j_\ell\in J, i_\ell\in I \right\}
    \]
}
\propp{
    Let $R$ be a ring and $I,J$ ideals of $R$. Then $I+J$, $I\cap J$ and $I\cdot J$ are ideals.
}{
    \begin{enumerate}
        \item $I+J$ is an ideal. \\
        Suppose $a_1,a_2\in I+J$. Then there exist $i_1,j_1,i_2,j_2$ such that $a_k=i_k+j_k$ for $k=1,2$. Then
        \begin{align*}
            a_1+a_2&=(i_1+j_1)+(i_2+j_2)\\
            &=\underbrace{(i_1+i_2)}_{\in I}+\underbrace{(j_1+j_2)}_{\in J}\\
            &\in I+J.
        \end{align*}
        \begin{align*}
            a_1a_2&=(i_1+j_1)(i_2+j_2)\\
            &=i_1i_2+i_1j_2+i_2j_1+j_1j_2\\
            &\in I+J.
        \end{align*}
        Other details are an exercise.
        \item $I\cap J$ is an ideal.\\
        Suppose $a_1,a_2\in I\cap J$. Then $a_1+a_2\in I$ and $a_1+a_2\in J\Rightarrow a_1+a_2\in I\cap J$. Similarly $a_1a_2\in I$ and $a_1a_2\in J\Rightarrow a_1a_2\in I\cap J$. Other details are an exercise.
        \item $I\cdot J$ is an ideal.
        $a_1,a_2\in I\cdot J$.
        \[
            a_1=\sum_{\ell=1}^{q_1}i_\ell j_\ell,
        \]
        \[
            a_2=\sum_{\ell'=1}^{q_2}i_\ell' j_\ell',
        \]
        then
        \[
            a_1+a_2=\sum_{\ell=1}^{q_1}i_\ell j_\ell+\sum_{\ell'=1}^{q_2}i_\ell' j_\ell'.
        \]
        Define $q=\max\{q_1,q_2\}$, WLOG $q=q_1$, define $i_\ell'=j_\ell'=0$ if $q_2<\ell \leq q_1$. then
        \[
            a_1+a_2=\sum_{\ell=1}^{q}(i_\ell j_\ell+i_\ell'+j_\ell').
        \]
        \textit{Note.} $i_\ell j_\ell+i_\ell'+j_\ell'\in I\cdot J$ for $1\leq \ell\leq q$. Hence, $a_1+a_2\in I\cdot J$.
        \begin{align*}
            a_1a_2&=\left(\sum_{\ell=1}^{q_1}i_\ell j_\ell\right)\left(\sum_{\ell'=1}^{q_2}i_\ell' j_\ell'\right)\\
            &=\sum_{\ell=1}^{q_1}i_\ell j_\ell\left(\sum_{\ell=1}^{q_2}i_\ell'j_\ell'\right)\\
            &=\sum_{k=1}^{q_1}\sum_{\ell=1}^{q_2}i_k j_k i_\ell'j_\ell'.
        \end{align*}
        Other details are an exercise.
    \end{enumerate}
    This is a finite sum of summands of the form $ij$. Hence, $a_1a_2\in I\cdot J$.
}
(If $r\in R$, $ra_2=\sum_{\ell=1}^{q}ri_\ell'j_\ell'\Rightarrow ra_2\in I\cdot J$)\\
Original Motivation: trying to solve diophantine equations (Fermat's last theorem).
\subsection*{Fermat's Equation}
\[
    x^n+y^n=z^n,\quad x,y,z\in\Z,n\geq3
\]
\textit{Claim (FLT).} This has no solutions with $xyz\neq0$.\\
Fermat solved the case for $n=4$. He saw that you can factor the LHS as $(x+\zeta_1y)\cdots(x+\zeta_ny),$ where $\zeta_i$'s are the $n-$th roots of unity. If $x+\zeta_iy$ were all $n-$th powers, then FLT would be easy to prove. Unfortunately this is not case in general. This led to the notion of ``ideal numbers'' to try to recover unique factorization. This ``recovered'' by a theorem of Hilbert. Unfortunately, in practice it does not help.\\
This is because the arithmetic of ideals just introduced does not behave well with with relevant number theoretic properties. The multiplicative aspect of ideal arithmetic gave rise to so-called \underline{composition laws} in number rings.\\
This is realted to the \textit{class number problem}. Gauss had two conjectures:
\begin{enumerate}
    \item For imaginary quadratic fields, the class numbers go to infiity, and only a specified list of such fields have class \#1 (i.e., unique factorization)
    \item For real quadratic fields, infiitely many have unique factorization. 
\end{enumerate}
1. was proven by showing that if the Reimann hypothesis is true then it is true, and if the Reimann hypothesis is false then the statement is also true.
\end{document}