\documentclass[12pt]{article}
\usepackage{color,soul}
% Import preambles and macros for notes
% Essential packages for notes
\usepackage{amsmath, amssymb, amsthm}
\usepackage{mathtools}  % for \coloneqq, etc.
\usepackage{geometry}   % Better page margins
\usepackage{parskip}    % Better paragraph spacing
\usepackage{microtype}  % Better typography
\usepackage{enumitem}   % Customize lists
\usepackage{hyperref}   % Clickable links
\usepackage{booktabs}   % Better tables
\usepackage{tcolorbox}  % For colored boxes/theorems

% Page layout for notes
\geometry{a4paper, margin=1in}
\setlength{\parskip}{0.8em}

% Theorem environments with shared numbering
\newtheorem{theorem}{Theorem}[subsection]  % Number within sections: 2.3.1, 2.3.2, etc.

% All other environments share the same counter as theorem
\newtheorem{lemma}[theorem]{Lemma}
\newtheorem{proposition}[theorem]{Proposition}
\newtheorem{corollary}[theorem]{Corollary}
\newtheorem{definition}[theorem]{Definition}
\newtheorem{example}[theorem]{Example}
\newtheorem{remark}[theorem]{Remark}
\newtheorem{claim}[theorem]{Claim}

% Custom colors for notes
\usepackage{xcolor}
\definecolor{note-blue}{RGB}{220, 230, 255}
\definecolor{theorem-green}{RGB}{220, 255, 220}
% Math notation shortcuts for notes
\newcommand{\R}{\mathbb{R}}
\newcommand{\C}{\mathbb{C}}
\newcommand{\Q}{\mathbb{Q}}
\newcommand{\Z}{\mathbb{Z}}
\newcommand{\N}{\mathbb{N}}

% Calculus
\newcommand{\diff}{\mathop{}\!\mathrm{d}}
\newcommand{\deriv}[2]{\frac{\mathrm{d}#1}{\mathrm{d}#2}}
\newcommand{\pderiv}[2]{\frac{\partial #1}{\partial #2}}

% Linear Algebra
\newcommand{\inner}[2]{\langle #1, #2 \rangle}
\newcommand{\norm}[1]{\| #1 \|}
\newcommand{\tr}{\operatorname{tr}}
\newcommand{\spn}{\operatorname{span}}
\newcommand{\rank}{\operatorname{rank}}
\newcommand{\nullity}{\operatorname{nullity}}

% Logic
\newcommand{\contra}{\Rightarrow\Leftarrow}

% Custom commands for notes
\newcommand{\todo}[1]{\textcolor{red}{[TODO: #1]}}
\newcommand{\important}[1]{\textbf{\textcolor{blue}{#1}}}
\input{../../../../preambles/theorem-system-section.tex}

\title{MATH 420 Lecture 4}
\author{Deepak Jassal}
\date{January 19\textsuperscript{th}, 2026}

\begin{document}
\setcounter{section}{1}
\maketitle  
\textit{Last time:} Ring Homomorphisms
\[
    f: R\to S
\]
\[
    f(a+b)=f(a)+f(b)
\]
\[
    f(ab)=f(a)f(b)
\]
\[
\ker(f)=\{r\in R: f(r)=0\}
\]
What properties for $\ker(f)$ have?\\
In linear algebra, $\ker(f)$ is a \underline{subspace}. In group theory we know that it is a normal \underline{subgroup}.
\lemp{Ring Homomorphism Kernel's}{
    $\ker(f)$ ($f$ is a homomorphism) is a subring of $R$.
}{
    Suppose $a,b\in\ker(f)$.
    \begin{enumerate}
        \item $a+b\in\ker(f)$
        \[
            f(a+b)=f(a)\oplus f(b)=0_S\oplus 0_S=0_S.
        \]
        \[
            f(ab)=f(a)\otimes f(b)=\underbrace{0_S\otimes 0_S=0_S}_{\substack{\text{requires proof} \\ \text{Excersize}}}.
        \]
        \begin{align*}
            f(-a)&=f((-1)a)\\
            &=f(-1)\otimes \underbrace{f(a)}_{\in \ker(f)}\\
            f(-1)\otimes 0_S&=0_S
        \end{align*}
        \begin{align*}
            f(0_R)&=0_S\\
            f(a+(-a))&=f(a)\oplus f(-a)\\
            \Rightarrow f(-a)=-f(a)&=-0_S=0_S.
        \end{align*}
    \end{enumerate}
}
\defn{Subring}{
    Let $(R,+,\times)$ be a ring. A subset $S\subseteq R$ is said to be a subring of $R$ if $(S,+,\times)$ is a ring.
}
Looking to group theory, kernels are not just subgroups by \underline{normal} subgroups.
\propp{
    Let $R,S$ be rings and $f:R\to S$ be a homomorphism. Then $\ker(f)$ is an ideal of $R$. (Note: if $R$ is not commutative, then a (two-sided) ideal $I$ is one where $sr,rs\in I$ for all $s\in I$ and $r\in R$).
}{
    Already proved that it is a subring. Suffice to prove that for all $s\in I=\ker(f)$ and $r\in R$, $rs,sr\in I$.
    \[
        f(sr)=f(s)\otimes f(r)=0_S,
    \]
    \[
        f(rs)=f(r)\otimes f(s)=0_S.
    \]
}
Since $\ker(f)$ is an ideal, $R\setminus\ker(f)$ is a (quotient) ring.\\
In linear alegbra, there is the \textit{rank-nullity} theorem:
\thm{Rank-Nullity Theorem}{
    If $V,W$ are vector spaces over the same scalar field $F$, and $L:V\to W$ a linear map, then $\mathrm{dim}(L(V))+\mathrm{dim}(\ker(L))=\mathrm{dim}(V).$
}
The analogy of this is the \underline{First Isomorphism Theorem}.
\thm{First Isomorphism Theorem}{
    Let $R,S$ be rings and $f:R\to S$ a homomorphism. Then $R\setminus\ker(f)\cong f(R)$. Here $\cong$ means isomorphic.
}
\defn{Isomorphic}{
    A ring homomorphism is an isomorphism if it is invertible and the inverse map is also a homomorphism.
}
\lem{Homorphic Image of Rings}{
    The homorphic image of a ring is a ring.
}
\pf[Proof for Lemma]{
    Suppose $x,y\in f(R)$, by definition there exists $a,b\in R$ such that $f(a)=x$, and $f(b)=y$.
    \[
        x\oplus y=f(a)\oplus f(b)=f(a+b)\Rightarrow x\oplus y\in f(R).
    \]
    \[
        x\otimes y=f(a)\otimes f(b)=f(a\times b)\Rightarrow x\otimes y\in f(R).
    \]
    Prove there exists an inverse map as an excersize. \qedhere
}
\pf[Proof of the First Isomorphism Theorem]{
   Recall $r\setminus\ker(f)$ is a set of cosets (or representatives of equivalence classes). $[a]_{\ker(f)}:=a+\ker(f).$ We want to write down a map, say $\tau$, mapping each coset to an element in $f(R)$.\\
   Define 
   \[
        \tau:r\setminus\ker(f)\to f(R)
   \]
   by $\tau(a+\ker(f))=f(a)$.\\
   Check well-definedness:\\
   If $a-a'\in\ker(f)$, then 
   \begin{align*}
        \tau(a+\ker(f))&=\tau(a+(a-a')+\ker(f))\\
        f(a)&=\tau(a'+\ker(f))\\
        &=f(a').
   \end{align*}
}
\end{document}