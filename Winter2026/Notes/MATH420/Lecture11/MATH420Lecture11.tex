\documentclass[12pt]{article}
\usepackage{color,soul}
% Import preambles and macros for notes
% Essential packages for notes
\usepackage{amsmath, amssymb, amsthm}
\usepackage{mathtools}  % for \coloneqq, etc.
\usepackage{geometry}   % Better page margins
\usepackage{parskip}    % Better paragraph spacing
\usepackage{microtype}  % Better typography
\usepackage{enumitem}   % Customize lists
\usepackage{hyperref}   % Clickable links
\usepackage{booktabs}   % Better tables
\usepackage{tcolorbox}  % For colored boxes/theorems

% Page layout for notes
\geometry{a4paper, margin=1in}
\setlength{\parskip}{0.8em}

% Theorem environments with shared numbering
\newtheorem{theorem}{Theorem}[subsection]  % Number within sections: 2.3.1, 2.3.2, etc.

% All other environments share the same counter as theorem
\newtheorem{lemma}[theorem]{Lemma}
\newtheorem{proposition}[theorem]{Proposition}
\newtheorem{corollary}[theorem]{Corollary}
\newtheorem{definition}[theorem]{Definition}
\newtheorem{example}[theorem]{Example}
\newtheorem{remark}[theorem]{Remark}
\newtheorem{claim}[theorem]{Claim}

% Custom colors for notes
\usepackage{xcolor}
\definecolor{note-blue}{RGB}{220, 230, 255}
\definecolor{theorem-green}{RGB}{220, 255, 220}
% Math notation shortcuts for notes
\newcommand{\R}{\mathbb{R}}
\newcommand{\C}{\mathbb{C}}
\newcommand{\Q}{\mathbb{Q}}
\newcommand{\Z}{\mathbb{Z}}
\newcommand{\N}{\mathbb{N}}

% Calculus
\newcommand{\diff}{\mathop{}\!\mathrm{d}}
\newcommand{\deriv}[2]{\frac{\mathrm{d}#1}{\mathrm{d}#2}}
\newcommand{\pderiv}[2]{\frac{\partial #1}{\partial #2}}

% Linear Algebra
\newcommand{\inner}[2]{\langle #1, #2 \rangle}
\newcommand{\norm}[1]{\| #1 \|}
\newcommand{\tr}{\operatorname{tr}}
\newcommand{\spn}{\operatorname{span}}
\newcommand{\rank}{\operatorname{rank}}
\newcommand{\nullity}{\operatorname{nullity}}

% Logic
\newcommand{\contra}{\Rightarrow\Leftarrow}

% Custom commands for notes
\newcommand{\todo}[1]{\textcolor{red}{[TODO: #1]}}
\newcommand{\important}[1]{\textbf{\textcolor{blue}{#1}}}
\input{../../../../preambles/theorem-system-section.tex}

\title{MATH 420 Lecture 11}
\author{Deepak Jassal}
\date{February 11\textsuperscript{th}, 2026}

\begin{document}
\stepcounter{section}
\maketitle
\thm{Euclidean Domains $\Leftrightarrow$ PID}{
    Euclidean domains are principal ideal domains
}
\pf[Proof of the Theorem]{
    Let $T\subset D$ be an ideal. If $I=\{0\}$, then $I=(0)$. Otherwise, $I$ contains a non-zero element. Let $S_I=\{\mu(i):i\in I\}\subseteq\Z_{\leq0}$. By the well-ordering principal $S_I$ has a least element. Hence there exists $k\in I$ such that $\mu(k)=\min\{S_I\}$.
    \begin{claim}
        $I=(k)$.
    \end{claim} 
    \pf[Proof of the Claim]{
        Suppose $\ell\in I$, $\ell\neq0$. By the division algorithm (part a) there exists $q,r\in D$ such that 
        \[
            \ell=qk+r,\quad \mu(r)<\mu(k),\quad r=0.
        \]
        \[
            \ell\in I\Rightarrow \ell-pk=r\in I.
        \]
        By definition $\mu(k)=\min\{S_I\}$ so $r=0$. Therefore $\ell=pk\in(k)$, so $I\subseteq(k)$. Other direction is trivial.
    }
}
\defn{Unique Factorization Domains}{
    Let $D$ be an integral domain.
    \begin{enumerate}[label=(\alph*)]
        \item An element $q\in D$ is a unit if $q$ has a multiplicative inverse.
        \item An element $q\in D$ is said to be prime if $(q)$ is a prime ideal.
        \item An element $q\in D$ is said to be irreducible if $q=ab$, $a,b\in D$ then one of $a,b$ is a unit.
        \item If $q\in D$ and $u\in D$ is a unit, then $qu$ is an associate of $q$. 
    \end{enumerate}
}
\propp{Let $D$ be a domain. If $q\in D$ is prime, then it is irreducible}{
    Let $q\in D$ be prime. Suppose for the sake of a contradiction, that $q$ is reducible. Then there exists non-units $a,b\in D$ such that 
    \[
        q=ab.
    \]
    Since $q$ is a prime, $(q)$ is a prime ideal. By definition $(a)\subseteq (q)$ or $(b)\subseteq(q)$. WLOG suppose $(a)\subseteq(q)$. Then there exists $c\in D$ such that $a=qc$. Then we have $q=ab=(qc)b=q(bc)\Rightarrow q=(1-bc)=0$. Since $D$ is a domain, it has no zero divisors. Since 0 is not prime we assume that $q\neq0$, so $1-bc=0$ but $b$ is not a unit.
}
\thm{}{
    Let $D$ be a PID. Then $q\in D$ is (a) prime \textit{iff} $q$ is (b) irreducible \textit{iff} $(q)$ is (c) maximal.
}
\pf[Proof of the Theorem]{
    (a)$\Rightarrow$(b) Always true.\\
    (b)$\Rightarrow$(a) Suppose $q$ is irreducible. Suppose $A,B\in D$ are ideals such that $AB\subseteq (q)$. Since $D$ is a PID, $A=(a)$, and $B=(b)$ for some $a\in A$ and $b\in B$. $\Rightarrow$ $ab\in(q)$, so there exists $s\in D$ such that $ab=qs$. Put $J=\langle a,q \rangle$. Since $D$ is a PID $J=\langle j\rangle$ for some $j\in D$. Hence there exists $x,y\in D$ sick that $a=jx$, and $q=jy$. Since $q$ is irreducible either $j$ or $y$ is a unit. 
    \begin{enumerate}
        \item If $j$ is a unit. Then $J=D$. Hence there exists $v,w\in D$ such that $1=av+qw\Rightarrow b=(ab)v+qbw$, $ab=qs$, so $b=(qs)v+(ab)w=q(sv+bw)\in(q)\Rightarrow (b)\subset(q)$.
        \item $y$ is a unit. Then $q,j$ are associates. This implies that $(q)=(j)=J=(a,q)$. Then we have $(a)\subseteq\langle a,q\rangle\subseteq(q)$. Hence, $q$ is prime. 
    \end{enumerate}
    (b)$\Rightarrow$(c) Suppose that $q$ is irreducible. Let $J$ be an ideal, such that $(q)\subseteq J$. Since $D$ is a PID we have $J=(j)$ for some $j\in D$. So there exists $s\in D$ such that $q=js$. Then since $q$ is irreducible $j$ or $s$ is a unit. If $j$ is a unit then $J=D$. Otherwise if $s$ is a unit then $q,j$ are associates and $(q)=(j)$. By definition $q$ is maximal.\\
    (c)$\Rightarrow$(a) We always have maximal implies prime.
}
\cor{
    Let $\mathbb{F}$ be a field. Then and $f(x)\in\mathbb{F}[x]$ is irreducible \textit{iff} $(f(x))$ is maximal
}
\defn{UFD}{
    Let $D$ be a domain. We say that $D$ is a UFD if for all $q\in D$ which is not a unit, we have:
    \begin{enumerate}[label=(\alph*)]
        \item $q=q_1\dots q_r$ where each $q_i$ $1\leq i\leq r$ are irreducible;
        \item if $q=q_1\dots q_r$ and $q=p_1\dots p_\ell$, then there exists a bijection $\sigma:\{1,\dots,r\}\to\{1,\dots,\ell\}$ such that 
        \[
            q_i=p_{\sigma(i)}U_{\sigma(i)}.
        \]
        Where $U_{\sigma(i)}$ are units.  
    \end{enumerate}
}
\defn{Noetherian Rings}{
    A ring $R$ is \underline{Noetherian}, or $R$ satisifies the ASC (Ascending Chain Condition), if whenever 
    \[
        J_1\subseteq\cdots\subseteq J_n\subseteq\cdots,\quad J_i\text{'s are ideals}.
    \]
    Then there exists $N\geq 1$ such that for all $m\geq N$ we have $J_m=J_N$. 
}



\end{document}