\documentclass[12pt]{article}
\usepackage{color,soul}
% Import preambles and macros for notes
% Essential packages for notes
\usepackage{amsmath, amssymb, amsthm}
\usepackage{mathtools}  % for \coloneqq, etc.
\usepackage{geometry}   % Better page margins
\usepackage{parskip}    % Better paragraph spacing
\usepackage{microtype}  % Better typography
\usepackage{enumitem}   % Customize lists
\usepackage{hyperref}   % Clickable links
\usepackage{booktabs}   % Better tables
\usepackage{tcolorbox}  % For colored boxes/theorems

% Page layout for notes
\geometry{a4paper, margin=1in}
\setlength{\parskip}{0.8em}

% Theorem environments with shared numbering
\newtheorem{theorem}{Theorem}[subsection]  % Number within sections: 2.3.1, 2.3.2, etc.

% All other environments share the same counter as theorem
\newtheorem{lemma}[theorem]{Lemma}
\newtheorem{proposition}[theorem]{Proposition}
\newtheorem{corollary}[theorem]{Corollary}
\newtheorem{definition}[theorem]{Definition}
\newtheorem{example}[theorem]{Example}
\newtheorem{remark}[theorem]{Remark}
\newtheorem{claim}[theorem]{Claim}

% Custom colors for notes
\usepackage{xcolor}
\definecolor{note-blue}{RGB}{220, 230, 255}
\definecolor{theorem-green}{RGB}{220, 255, 220}
% Math notation shortcuts for notes
\newcommand{\R}{\mathbb{R}}
\newcommand{\C}{\mathbb{C}}
\newcommand{\Q}{\mathbb{Q}}
\newcommand{\Z}{\mathbb{Z}}
\newcommand{\N}{\mathbb{N}}

% Calculus
\newcommand{\diff}{\mathop{}\!\mathrm{d}}
\newcommand{\deriv}[2]{\frac{\mathrm{d}#1}{\mathrm{d}#2}}
\newcommand{\pderiv}[2]{\frac{\partial #1}{\partial #2}}

% Linear Algebra
\newcommand{\inner}[2]{\langle #1, #2 \rangle}
\newcommand{\norm}[1]{\| #1 \|}
\newcommand{\tr}{\operatorname{tr}}
\newcommand{\spn}{\operatorname{span}}
\newcommand{\rank}{\operatorname{rank}}
\newcommand{\nullity}{\operatorname{nullity}}

% Logic
\newcommand{\contra}{\Rightarrow\Leftarrow}

% Custom commands for notes
\newcommand{\todo}[1]{\textcolor{red}{[TODO: #1]}}
\newcommand{\important}[1]{\textbf{\textcolor{blue}{#1}}}
% Theorem system 
% Theorem System original by https://github.com/kcajc/math-notes-template, this is modified 
% The following boxes are provided:
%   Definition:     \defn 
%   Theorem:        \thm 
%   Lemma:          \lem
%   Corollary:      \cor
%   Proposition:    \prop   
%   Claim:          \clm
%   Fact:           \fact
%   Proof:          \pf
%   Example:        \ex
%   Remark:         \rmk (sentence), \rmkb (block)
% Suffix
%   r:              Allow Theorem/Definition to be referenced, e.g. thmr
%   p:              Add a short proof block for Lemma, Corollary, Proposition or Claim, e.g. lemp
%                   For theorems, use \pf for proof blocks

% Definition - subsection numbering for 1.1.1, 1.1.2
\newtcbtheorem[number within=subsection]{mydefinition}{Definition}
{
    colbacktitle=green!20!white,
    colback=green!10!white,
    coltitle=black,
    fonttitle=\bfseries\large,
}{defn}

\NewDocumentCommand{\defn}{m+m}{
    \begin{mydefinition}{#1}{}
        #2
    \end{mydefinition}
}

\NewDocumentCommand{\defnr}{mm+m}{
    \begin{mydefinition}{#1}{#2}
        #3
    \end{mydefinition}
}

% Theorem - subsection numbering
\newtcbtheorem[use counter from=mydefinition]{mytheorem}{Theorem}
{
    colbacktitle=cyan!20!white,
    colback=cyan!10!white,
    coltitle=black,
    fonttitle=\bfseries\large,
}{thm}

\NewDocumentCommand{\thm}{m+m}{
    \begin{mytheorem}{#1}{}
        #2
    \end{mytheorem}
}

\NewDocumentCommand{\thmr}{mm+m}{
    \begin{mytheorem}{#1}{#2}
        #3
    \end{mytheorem}
}

% Lemma - subsection numbering
\newtcbtheorem[use counter from=mydefinition]{mylemma}{Lemma}
{
    colbacktitle=violet!20!white,
    colback=violet!10!white,
    coltitle=black,
    fonttitle=\bfseries\large,
}{lem}

\NewDocumentCommand{\lem}{m+m}{
    \begin{mylemma}{#1}{}
        #2
    \end{mylemma}
}

% Improved proof environments with consistent QED placement
\newenvironment{lempf}{
    \par\noindent{\it \textbf{Proof for Lemma.}}\par\nopagebreak
    \begin{list}{}{\setlength\leftmargin{1em}\setlength\rightmargin{0em}}
    \item\relax
}{
    \hfill$\qed$\end{list}
}

\NewDocumentCommand{\lemp}{m+m+m}{
    \begin{mylemma}{#1}{}
        #2
    \end{mylemma}
    \begin{lempf}
        #3
    \end{lempf}
}

% Corollary - subsection numbering
\newtcbtheorem[use counter from=mydefinition]{mycorollary}{Corollary}
{
    colbacktitle=orange!20!white,
    colback=orange!10!white,
    coltitle=black,
    fonttitle=\bfseries\large,
}{cor}

\NewDocumentCommand{\cor}{+m}{
    \begin{mycorollary}{}{}
        #1
    \end{mycorollary}
}

\newenvironment{corpf}{
    \par\noindent{\it \textbf{Proof for Corollary.}}\par\nopagebreak
    \begin{list}{}{\setlength\leftmargin{1em}\setlength\rightmargin{0em}}
    \item\relax
}{
    \hfill$\qed$\end{list}
}

\NewDocumentCommand{\corp}{m+m+m}{
    \begin{mycorollary}{}{}
        #1
    \end{mycorollary}
    \begin{corpf}
        #2
    \end{corpf}
}

% Proposition - subsection numbering
\newtcbtheorem[use counter from=mydefinition]{myproposition}{Proposition}
{
    colbacktitle=yellow!30!white,
    colback=yellow!20!white,
    coltitle=black,
    fonttitle=\bfseries\large,
}{prop}

\NewDocumentCommand{\prop}{+m}{
    \begin{myproposition}{}{}
        #1
    \end{myproposition}
}

\newenvironment{proppf}{
    \par\noindent{\it \textbf{Proof for Proposition.}}\par\nopagebreak
    \begin{list}{}{\setlength\leftmargin{1em}\setlength\rightmargin{0em}}
    \item\relax
}{
    \hfill$\qed$\end{list}
}

\NewDocumentCommand{\propp}{+m+m}{
    \begin{myproposition}{}{}
        #1
    \end{myproposition}
    \begin{proppf}
        #2
    \end{proppf}
}

% Claim - subsection numbering
\newtcbtheorem[use counter from=mydefinition]{myclaim}{Claim}
{
    colbacktitle=pink!30!white,
    colback=pink!20!white,
    coltitle=black,
    fonttitle=\bfseries\large,
}{clm}

\NewDocumentCommand{\clm}{m+m}{
    \begin{myclaim}{#1}{}
        #2
    \end{myclaim}
}

\newenvironment{clmpf}{
    \par\noindent{\it \textbf{Proof for Claim.}}\par\nopagebreak
    \begin{list}{}{\setlength\leftmargin{1em}\setlength\rightmargin{0em}}
    \item\relax
}{
    \hfill$\qed$\end{list}
}

\NewDocumentCommand{\clmp}{m+m+m}{
    \begin{myclaim}{#1}{}
        #2
    \end{myclaim}
    \begin{clmpf}
        #3
    \end{clmpf}
}

% Fact - subsection numbering
\newtcbtheorem[use counter from=mydefinition]{myfact}{Fact}
{
    colbacktitle=purple!20!white,
    colback=purple!10!white,
    coltitle=black,
    fonttitle=\bfseries\large,
}{fact}

\NewDocumentCommand{\fact}{+m}{
    \begin{myfact}{}{}
        #1
    \end{myfact}
}

% Proof - customizable name
\NewDocumentEnvironment{custompf}{m}
{
    \par\noindent{\it \textbf{#1}}\par\nopagebreak
    \begin{list}{}{\setlength\leftmargin{1em}\setlength\rightmargin{0em}}
    \item\relax
}
{
    \hfill$\qed$\end{list}
}

\NewDocumentCommand{\pf}{O{Proof}+m}{
    \begin{custompf}{#1.}
        #2
    \end{custompf}
}

% Example - improved environment
\newenvironment{example}{
    \par\vspace{5pt}
    \noindent\textbf{Example.}\par\nopagebreak
    \begin{list}{}{\setlength\leftmargin{1em}\setlength\rightmargin{0em}}
    \item\relax
}{
    \end{list}\vspace{5pt}
}

\NewDocumentCommand{\ex}{+m}{
    \begin{example}
        #1
    \end{example}
}

% Remark
\NewDocumentCommand{\rmk}{+m}{
    {\it \color{blue!50!white}#1}
}

% Remark block - improved environment
\newenvironment{remark}{
    \par\vspace{5pt}
    \noindent\textbf{Remark.}\par\nopagebreak
    \begin{list}{}{\setlength\leftmargin{1em}\setlength\rightmargin{0em}}
    \item\relax
}{
    \end{list}\vspace{5pt}
}

\NewDocumentCommand{\rmkb}{+m}{
    \begin{remark}
        #1
    \end{remark}
}

\title{MATH 204 Lecture 1}
\author{Deepak Jassal}
\date{January 12\textsuperscript{th}, 2026}

\begin{document}
\maketitle  
\setcounter{section}{1}
\setcounter{subsection}{1}
\section*{\textsection 1.1 Parametric Equations}


\begin{figure}[h]
    \centering
    
    \begin{tikzpicture}

    \begin{axis}[
        trig format plots=rad,
        axis equal, axis x line=middle, axis y line=
        middle, xlabel={$x$}, ylabel={$y$},xmin=-3,xmax=3,ymin=-4,ymax=4] 

    \addplot [domain=-1:180, smooth,samples=2000, red,very thick]({(3*x)/(1+x^3)},{(3*x^2)/(1+x^3)}); 

    \addplot [domain=-180:-1, smooth,samples=2000, red,very thick]({(3*x)/(1+x^3)},{(3*x^2)/(1+x^3)}); 

    \end{axis} 


    \end{tikzpicture}
    
    \label{fig:simple_curve}
    
\end{figure}
Some curves such as this one cannot be expressed as $y=y(x)$. However, we can express this in terms of a third variable, say $t$.
\ex{A particle moving in the plane having a trajectory that could possibly ``self intersect''.}
The variable (in this case $t$) is called the parameter. In such a case we express $x$ and $y$ in terms of $t$ as $x=x(t)$, $y=y(t)$.
\defn{Parametric Curve}{A parametric curve is a path in the $xy$-plane that consists of points of the form
\[
    (x(t),y(t))\quad a\leq t\leq b
\]
where $t$ is called the parameter.
}
\newpage
\ex{
\[
\begin{cases}
    x(t)=3t+2\\
    y(t)=-t+5;\quad 0\leq t \leq 2
\end{cases}
\]
We notice, here that $t=-y+5$. Hence, 
\[
    x=-3y+17 \Leftrightarrow y=-\frac{x}{3}+\frac{17}{3}.
\]
}
This is an equation of a straight line. However, as $0\leq t\leq 2$, we see that the ``initial'' location of the particle is (2,5), and the ``terminal'' location is (8,3).
\defn{Initial and Terminal Points}{
    In \textit{definition 1.1.1} the point $(x(a),y(a))$ is called the ``initial'' point, and the point $(x(b),y(b))$ is called the ``termial'' point.
}
\ex{Analyze the parametic equations
\[
\begin{cases}
    x(t)=\cos(5\theta)\\
    y(t)=\sin(5\theta);\quad 0\leq t\leq \frac{\pi}{10}
\end{cases}
\]
Since $x^2+y^2=1$, the curve starts at (1,0) and ends at (0,1).\\
So, the parametric equation given is the first quarter of a circle with centre (0,0) and radius 1.
\begin{figure}[h]
    \centering
    \begin{tikzpicture}
        % Draw axes
        \draw[->] (-0.5,0) -- (2.5,0) node[right] {$x$};
        \draw[->] (0,-0.5) -- (0,2.5) node[above] {$y$};
        
        % Draw the quarter circle arc with counterclockwise arrow
        \draw[thick, blue, ->] (2,0) arc[start angle=0, end angle=90, radius=2];
        
        % Mark points
        \filldraw (2,0) circle (1.5pt) node[below right] {$(1,0)$};
        \filldraw (0,2) circle (1.5pt) node[above left] {$(0,1)$};
        \node at (0,0) [below left] {$O$};
    \end{tikzpicture}
\end{figure}
}
\rec{
\begin{enumerate}
    \item The Euclidean distance between points $A=(x_1,y_1)$, $B=(x_2,y_2)$ is
            \[
                ||\overrightarrow{AB}||=\sqrt{(x_2-x_1)^2+(y_2-y_1)^2}.
            \]
    \item Let $A=(x_0,y_0)$. Then we can make a circle with centre $A$ in the $xy-$plane with radius $R$. Then a point is on the circle $P$ \textit{iff} $||\overrightarrow{AP}||=R$.
    \begin{figure}[h]
        \centering
        \begin{tikzpicture}
            % Center and radius
            \def\x0{3}
            \def\y0{2}
            \def\R{1.5}
            
            % Draw axes
            \draw[->] (-0.5,0) -- (5,0) node[right] {$x$};
            \draw[->] (0,-0.5) -- (0,4) node[above] {$y$};
            
            % Draw circle
            \draw[thick] (\x0,\y0) circle (\R);
            
            % Mark center
            \filldraw[red] (\x0,\y0) circle (2pt) 
                node[below] {$A=(x_0,y_0)$};
            
            % Draw an arbitrary point on the circle
            \coordinate (P) at ($(\x0,\y0) + (30:\R)$);
            \filldraw[blue] (P) circle (2pt) node[above right] {$P=(x,y)$};
            
            % Draw radius
            \draw[thick, dashed] (\x0,\y0) -- (P);
            \node at ($(\x0,\y0)!0.5!(P) + (0,0.3)$) {$R$};
            
            % Label equation
            \node at (4,4) {$(x-x_0)^2 + (y-y_0)^2 = R^2$};
        \end{tikzpicture}
    \end{figure}
\end{enumerate}    

That is 
\begin{align*}
    R^2&=||\overrightarrow{AP}||\\
    &=(x_0-x)^2+(y_0-y)^2.
\end{align*}
}
\ex{
\[
\begin{cases}
    x(t)=\cos(5\theta)+5\\
    y(t)=\sin(5\theta)-2;\quad -\frac{\pi}{40}\leq t\leq \frac{\pi}{20}. 
\end{cases}
\]
This equation can be solved by moving the constants to the other side and applying the identity used in the first exmaple.
}
\end{document}