\documentclass[12pt]{article}
\usepackage{color,soul}
% Import preambles and macros for notes
% Essential packages for notes
\usepackage{amsmath, amssymb, amsthm}
\usepackage{mathtools}  % for \coloneqq, etc.
\usepackage{geometry}   % Better page margins
\usepackage{parskip}    % Better paragraph spacing
\usepackage{microtype}  % Better typography
\usepackage{enumitem}   % Customize lists
\usepackage{hyperref}   % Clickable links
\usepackage{booktabs}   % Better tables
\usepackage{tcolorbox}  % For colored boxes/theorems

% Page layout for notes
\geometry{a4paper, margin=1in}
\setlength{\parskip}{0.8em}

% Theorem environments with shared numbering
\newtheorem{theorem}{Theorem}[subsection]  % Number within sections: 2.3.1, 2.3.2, etc.

% All other environments share the same counter as theorem
\newtheorem{lemma}[theorem]{Lemma}
\newtheorem{proposition}[theorem]{Proposition}
\newtheorem{corollary}[theorem]{Corollary}
\newtheorem{definition}[theorem]{Definition}
\newtheorem{example}[theorem]{Example}
\newtheorem{remark}[theorem]{Remark}
\newtheorem{claim}[theorem]{Claim}

% Custom colors for notes
\usepackage{xcolor}
\definecolor{note-blue}{RGB}{220, 230, 255}
\definecolor{theorem-green}{RGB}{220, 255, 220}
% Math notation shortcuts for notes
\newcommand{\R}{\mathbb{R}}
\newcommand{\C}{\mathbb{C}}
\newcommand{\Q}{\mathbb{Q}}
\newcommand{\Z}{\mathbb{Z}}
\newcommand{\N}{\mathbb{N}}

% Calculus
\newcommand{\diff}{\mathop{}\!\mathrm{d}}
\newcommand{\deriv}[2]{\frac{\mathrm{d}#1}{\mathrm{d}#2}}
\newcommand{\pderiv}[2]{\frac{\partial #1}{\partial #2}}

% Linear Algebra
\newcommand{\inner}[2]{\langle #1, #2 \rangle}
\newcommand{\norm}[1]{\| #1 \|}
\newcommand{\tr}{\operatorname{tr}}
\newcommand{\spn}{\operatorname{span}}
\newcommand{\rank}{\operatorname{rank}}
\newcommand{\nullity}{\operatorname{nullity}}

% Logic
\newcommand{\contra}{\Rightarrow\Leftarrow}

% Custom commands for notes
\newcommand{\todo}[1]{\textcolor{red}{[TODO: #1]}}
\newcommand{\important}[1]{\textbf{\textcolor{blue}{#1}}}
% Theorem system 
% Theorem System original by https://github.com/kcajc/math-notes-template, this is modified 
% The following boxes are provided:
%   Definition:     \defn 
%   Theorem:        \thm 
%   Lemma:          \lem
%   Corollary:      \cor
%   Proposition:    \prop   
%   Claim:          \clm
%   Fact:           \fact
%   Proof:          \pf
%   Example:        \ex
%   Remark:         \rmk (sentence), \rmkb (block)
% Suffix
%   r:              Allow Theorem/Definition to be referenced, e.g. thmr
%   p:              Add a short proof block for Lemma, Corollary, Proposition or Claim, e.g. lemp
%                   For theorems, use \pf for proof blocks

% Definition - subsection numbering for 1.1.1, 1.1.2
\newtcbtheorem[number within=subsection]{mydefinition}{Definition}
{
    colbacktitle=green!20!white,
    colback=green!10!white,
    coltitle=black,
    fonttitle=\bfseries\large,
}{defn}

\NewDocumentCommand{\defn}{m+m}{
    \begin{mydefinition}{#1}{}
        #2
    \end{mydefinition}
}

\NewDocumentCommand{\defnr}{mm+m}{
    \begin{mydefinition}{#1}{#2}
        #3
    \end{mydefinition}
}

% Theorem - subsection numbering
\newtcbtheorem[use counter from=mydefinition]{mytheorem}{Theorem}
{
    colbacktitle=cyan!20!white,
    colback=cyan!10!white,
    coltitle=black,
    fonttitle=\bfseries\large,
}{thm}

\NewDocumentCommand{\thm}{m+m}{
    \begin{mytheorem}{#1}{}
        #2
    \end{mytheorem}
}

\NewDocumentCommand{\thmr}{mm+m}{
    \begin{mytheorem}{#1}{#2}
        #3
    \end{mytheorem}
}

% Lemma - subsection numbering
\newtcbtheorem[use counter from=mydefinition]{mylemma}{Lemma}
{
    colbacktitle=violet!20!white,
    colback=violet!10!white,
    coltitle=black,
    fonttitle=\bfseries\large,
}{lem}

\NewDocumentCommand{\lem}{m+m}{
    \begin{mylemma}{#1}{}
        #2
    \end{mylemma}
}

% Improved proof environments with consistent QED placement
\newenvironment{lempf}{
    \par\noindent{\it \textbf{Proof for Lemma.}}\par\nopagebreak
    \begin{list}{}{\setlength\leftmargin{1em}\setlength\rightmargin{0em}}
    \item\relax
}{
    \hfill$\qed$\end{list}
}

\NewDocumentCommand{\lemp}{m+m+m}{
    \begin{mylemma}{#1}{}
        #2
    \end{mylemma}
    \begin{lempf}
        #3
    \end{lempf}
}

% Corollary - subsection numbering
\newtcbtheorem[use counter from=mydefinition]{mycorollary}{Corollary}
{
    colbacktitle=orange!20!white,
    colback=orange!10!white,
    coltitle=black,
    fonttitle=\bfseries\large,
}{cor}

\NewDocumentCommand{\cor}{+m}{
    \begin{mycorollary}{}{}
        #1
    \end{mycorollary}
}

\newenvironment{corpf}{
    \par\noindent{\it \textbf{Proof for Corollary.}}\par\nopagebreak
    \begin{list}{}{\setlength\leftmargin{1em}\setlength\rightmargin{0em}}
    \item\relax
}{
    \hfill$\qed$\end{list}
}

\NewDocumentCommand{\corp}{m+m+m}{
    \begin{mycorollary}{}{}
        #1
    \end{mycorollary}
    \begin{corpf}
        #2
    \end{corpf}
}

% Proposition - subsection numbering
\newtcbtheorem[use counter from=mydefinition]{myproposition}{Proposition}
{
    colbacktitle=yellow!30!white,
    colback=yellow!20!white,
    coltitle=black,
    fonttitle=\bfseries\large,
}{prop}

\NewDocumentCommand{\prop}{+m}{
    \begin{myproposition}{}{}
        #1
    \end{myproposition}
}

\newenvironment{proppf}{
    \par\noindent{\it \textbf{Proof for Proposition.}}\par\nopagebreak
    \begin{list}{}{\setlength\leftmargin{1em}\setlength\rightmargin{0em}}
    \item\relax
}{
    \hfill$\qed$\end{list}
}

\NewDocumentCommand{\propp}{+m+m}{
    \begin{myproposition}{}{}
        #1
    \end{myproposition}
    \begin{proppf}
        #2
    \end{proppf}
}

% Claim - subsection numbering
\newtcbtheorem[use counter from=mydefinition]{myclaim}{Claim}
{
    colbacktitle=pink!30!white,
    colback=pink!20!white,
    coltitle=black,
    fonttitle=\bfseries\large,
}{clm}

\NewDocumentCommand{\clm}{m+m}{
    \begin{myclaim}{#1}{}
        #2
    \end{myclaim}
}

\newenvironment{clmpf}{
    \par\noindent{\it \textbf{Proof for Claim.}}\par\nopagebreak
    \begin{list}{}{\setlength\leftmargin{1em}\setlength\rightmargin{0em}}
    \item\relax
}{
    \hfill$\qed$\end{list}
}

\NewDocumentCommand{\clmp}{m+m+m}{
    \begin{myclaim}{#1}{}
        #2
    \end{myclaim}
    \begin{clmpf}
        #3
    \end{clmpf}
}

% Fact - subsection numbering
\newtcbtheorem[use counter from=mydefinition]{myfact}{Fact}
{
    colbacktitle=purple!20!white,
    colback=purple!10!white,
    coltitle=black,
    fonttitle=\bfseries\large,
}{fact}

\NewDocumentCommand{\fact}{+m}{
    \begin{myfact}{}{}
        #1
    \end{myfact}
}

% Proof - customizable name
\NewDocumentEnvironment{custompf}{m}
{
    \par\noindent{\it \textbf{#1}}\par\nopagebreak
    \begin{list}{}{\setlength\leftmargin{1em}\setlength\rightmargin{0em}}
    \item\relax
}
{
    \hfill$\qed$\end{list}
}

\NewDocumentCommand{\pf}{O{Proof}+m}{
    \begin{custompf}{#1.}
        #2
    \end{custompf}
}

% Example - improved environment
\newenvironment{example}{
    \par\vspace{5pt}
    \noindent\textbf{Example.}\par\nopagebreak
    \begin{list}{}{\setlength\leftmargin{1em}\setlength\rightmargin{0em}}
    \item\relax
}{
    \end{list}\vspace{5pt}
}

\NewDocumentCommand{\ex}{+m}{
    \begin{example}
        #1
    \end{example}
}

% Remark
\NewDocumentCommand{\rmk}{+m}{
    {\it \color{blue!50!white}#1}
}

% Remark block - improved environment
\newenvironment{remark}{
    \par\vspace{5pt}
    \noindent\textbf{Remark.}\par\nopagebreak
    \begin{list}{}{\setlength\leftmargin{1em}\setlength\rightmargin{0em}}
    \item\relax
}{
    \end{list}\vspace{5pt}
}

\NewDocumentCommand{\rmkb}{+m}{
    \begin{remark}
        #1
    \end{remark}
}

\title{MATH 326 Lecture 1}
\author{Deepak Jassal}
\date{January 9\textsuperscript{th}, 2026}

\begin{document}
\maketitle  
\section{Vector Spaces}
\defn{Vector Space}{
    A vector space overa field $F$ usually $f=\R,\C$ is a sest of objects called ``vectors'' $V$ with two operations:
    \begin{enumerate}
        \item Vector Addition\[u+v,\quad\forall u,v\in V\]
        \item Multiplication by Scalars\[cv\in V,\quad \forall c\in F, \forall v\in V\]
    \end{enumerate}
    satisfying the following axioms.
    \begin{enumerate}
        \item $u+v=v+u,\quad\forall u,v\in V$
        \item $(u+v)+w=u+(v+w),\quad \forall u,v,w\in V$
        \item There exists a zero vector in $v$, such that $0+v=v+0=v,\quad \forall v\in V$.
        \item $\forall v\in V,\exists -v\in V$ such that $v+(-v)=0$.
        \item $c(u+v)=cu+cv,\quad \forall c\in F,\;\forall u,v\in V$.
        \item $(a+b)v=av+bv,\quad \forall a,b\in F,\; \forall v\in V$.
        \item $(ab)v=a(bv),\quad\forall a,b\in F,\; \forall v \in V$
        \item $1v=v,\quad \forall v\in V$.
    \end{enumerate}
}
\ex{
    \begin{enumerate}
        \item $V=\R^n,$ $F=\R$
        \item $V=\C^n,$ $F=\C$
        \item $V=\C^n,$ $F=\R$
        \item $V=M_{n\times n}(F),$ $F=F$
        \item $V=\mathcal{C}[a,b],$ $F=F$
    \end{enumerate}
}
\ex{
    $V=\R$ is a vector space on $\Q$.
}
\rmkb{We have only said that 0 exists, not that is it unique.}
\rmkb{0 is unique.}
\begin{proof}
    Suppose for contradiction that there exists two zero elements, $0_1,0_2$. Then,
    \begin{align*}
        0_1+0_2&=0_2\\
        0_1+0_2&=0_1=0_2.\qedhere
    \end{align*}
\end{proof}
\rmkb{$-v$ is unique.}
\begin{proof}
    Suppose for contradiction that there exists two inverse elements, $-v_1,-v_2$. Then,
    \begin{align*}
        v+(-v_1)&=0\\
        -v_2+(v+(-v_1))&=0+(-v_2)=-v_2\\
        (-v_2+v)+(-v_1)&=-v_2\\
        0-v_1&=-v_2\\
        -v_1&=-v_2.\qedhere
    \end{align*}
\end{proof}
\subsection{``Exotic Example'' of Vector Spaces}
Let $V=\R^2$ over $\R$ with 
\begin{enumerate}
    \item 
    $\begin{bmatrix}
        x_1\\x_2
    \end{bmatrix}\oplus
    \begin{bmatrix}
        y_1\\y_2
    \end{bmatrix}\overset{\text{def}}{=}
    \begin{bmatrix}
        x_1+y_1+1\\x_2+y_2+3
    \end{bmatrix}$
    \item     
    $c\begin{bmatrix}
        x_1\\x_2
    \end{bmatrix}\overset{\text{def}}{=}
    \begin{bmatrix}
        cx_1+c-1\\cx_2+3c-3
    \end{bmatrix}$.
\end{enumerate}
Show that $V$ is a vector space with these operations.
\begin{enumerate}
    \item What is zero in $V$?\\
    \[
        0_V=\begin{bmatrix}
            -1\\-3
        \end{bmatrix}
    \]
    \item Check that $(ab)v=a(bv)$
    \[
        \mathrm{LHS}:\quad (ab)v=
        \begin{bmatrix}
            abx+ab-1\\
            aby+3ab-3    
        \end{bmatrix}
    \]
    \begin{align*}
        \mathrm{RHS:}\quad a(bv)&=a
        \begin{bmatrix}
            bx+b-1\\
            by+3b-3
        \end{bmatrix}\\
        &=
        \begin{bmatrix}
            a(bx+b-1)+a-1\\
            a(by+3b-3)+3b-3
        \end{bmatrix}\\
        &=
        \begin{bmatrix}
            abx+ab-a+a-1\\
            aby+3ab-3a+3a-3
        \end{bmatrix}\\
        &=\begin{bmatrix}
            abx+ab-1\\
            aby+3ab-3 
        \end{bmatrix}\\
        &=(ab)v.
    \end{align*}
\end{enumerate}
\subsection{Subspaces}
\defn{Subsapces}{
    Let $V$ be a vector space over a field $F$, and $U\subseteq V$. Then $U$ is called a subspace of $V$if $U$ is a vector space with the operations from $V$.
}
\end{document}