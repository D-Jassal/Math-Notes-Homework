\documentclass[12pt]{article}
\usepackage{color,soul}
% Import preambles and macros for notes
% Essential packages for notes
\usepackage{amsmath, amssymb, amsthm}
\usepackage{mathtools}  % for \coloneqq, etc.
\usepackage{geometry}   % Better page margins
\usepackage{parskip}    % Better paragraph spacing
\usepackage{microtype}  % Better typography
\usepackage{enumitem}   % Customize lists
\usepackage{hyperref}   % Clickable links
\usepackage{booktabs}   % Better tables
\usepackage{tcolorbox}  % For colored boxes/theorems

% Page layout for notes
\geometry{a4paper, margin=1in}
\setlength{\parskip}{0.8em}

% Theorem environments with shared numbering
\newtheorem{theorem}{Theorem}[subsection]  % Number within sections: 2.3.1, 2.3.2, etc.

% All other environments share the same counter as theorem
\newtheorem{lemma}[theorem]{Lemma}
\newtheorem{proposition}[theorem]{Proposition}
\newtheorem{corollary}[theorem]{Corollary}
\newtheorem{definition}[theorem]{Definition}
\newtheorem{example}[theorem]{Example}
\newtheorem{remark}[theorem]{Remark}
\newtheorem{claim}[theorem]{Claim}

% Custom colors for notes
\usepackage{xcolor}
\definecolor{note-blue}{RGB}{220, 230, 255}
\definecolor{theorem-green}{RGB}{220, 255, 220}
% Math notation shortcuts for notes
\newcommand{\R}{\mathbb{R}}
\newcommand{\C}{\mathbb{C}}
\newcommand{\Q}{\mathbb{Q}}
\newcommand{\Z}{\mathbb{Z}}
\newcommand{\N}{\mathbb{N}}

% Calculus
\newcommand{\diff}{\mathop{}\!\mathrm{d}}
\newcommand{\deriv}[2]{\frac{\mathrm{d}#1}{\mathrm{d}#2}}
\newcommand{\pderiv}[2]{\frac{\partial #1}{\partial #2}}

% Linear Algebra
\newcommand{\inner}[2]{\langle #1, #2 \rangle}
\newcommand{\norm}[1]{\| #1 \|}
\newcommand{\tr}{\operatorname{tr}}
\newcommand{\spn}{\operatorname{span}}
\newcommand{\rank}{\operatorname{rank}}
\newcommand{\nullity}{\operatorname{nullity}}

% Logic
\newcommand{\contra}{\Rightarrow\Leftarrow}

% Custom commands for notes
\newcommand{\todo}[1]{\textcolor{red}{[TODO: #1]}}
\newcommand{\important}[1]{\textbf{\textcolor{blue}{#1}}}
% Theorem system 
% Theorem System original by https://github.com/kcajc/math-notes-template, this is modified 
% The following boxes are provided:
%   Definition:     \defn 
%   Theorem:        \thm 
%   Lemma:          \lem
%   Corollary:      \cor
%   Proposition:    \prop   
%   Claim:          \clm
%   Fact:           \fact
%   Proof:          \pf
%   Example:        \ex
%   Remark:         \rmk (sentence), \rmkb (block)
% Suffix
%   r:              Allow Theorem/Definition to be referenced, e.g. thmr
%   p:              Add a short proof block for Lemma, Corollary, Proposition or Claim, e.g. lemp
%                   For theorems, use \pf for proof blocks

% Definition - subsection numbering for 1.1.1, 1.1.2
\newtcbtheorem[number within=subsection]{mydefinition}{Definition}
{
    colbacktitle=green!20!white,
    colback=green!10!white,
    coltitle=black,
    fonttitle=\bfseries\large,
}{defn}

\NewDocumentCommand{\defn}{m+m}{
    \begin{mydefinition}{#1}{}
        #2
    \end{mydefinition}
}

\NewDocumentCommand{\defnr}{mm+m}{
    \begin{mydefinition}{#1}{#2}
        #3
    \end{mydefinition}
}

% Theorem - subsection numbering
\newtcbtheorem[use counter from=mydefinition]{mytheorem}{Theorem}
{
    colbacktitle=cyan!20!white,
    colback=cyan!10!white,
    coltitle=black,
    fonttitle=\bfseries\large,
}{thm}

\NewDocumentCommand{\thm}{m+m}{
    \begin{mytheorem}{#1}{}
        #2
    \end{mytheorem}
}

\NewDocumentCommand{\thmr}{mm+m}{
    \begin{mytheorem}{#1}{#2}
        #3
    \end{mytheorem}
}

% Lemma - subsection numbering
\newtcbtheorem[use counter from=mydefinition]{mylemma}{Lemma}
{
    colbacktitle=violet!20!white,
    colback=violet!10!white,
    coltitle=black,
    fonttitle=\bfseries\large,
}{lem}

\NewDocumentCommand{\lem}{m+m}{
    \begin{mylemma}{#1}{}
        #2
    \end{mylemma}
}

% Improved proof environments with consistent QED placement
\newenvironment{lempf}{
    \par\noindent{\it \textbf{Proof for Lemma.}}\par\nopagebreak
    \begin{list}{}{\setlength\leftmargin{1em}\setlength\rightmargin{0em}}
    \item\relax
}{
    \hfill$\qed$\end{list}
}

\NewDocumentCommand{\lemp}{m+m+m}{
    \begin{mylemma}{#1}{}
        #2
    \end{mylemma}
    \begin{lempf}
        #3
    \end{lempf}
}

% Corollary - subsection numbering
\newtcbtheorem[use counter from=mydefinition]{mycorollary}{Corollary}
{
    colbacktitle=orange!20!white,
    colback=orange!10!white,
    coltitle=black,
    fonttitle=\bfseries\large,
}{cor}

\NewDocumentCommand{\cor}{+m}{
    \begin{mycorollary}{}{}
        #1
    \end{mycorollary}
}

\newenvironment{corpf}{
    \par\noindent{\it \textbf{Proof for Corollary.}}\par\nopagebreak
    \begin{list}{}{\setlength\leftmargin{1em}\setlength\rightmargin{0em}}
    \item\relax
}{
    \hfill$\qed$\end{list}
}

\NewDocumentCommand{\corp}{m+m+m}{
    \begin{mycorollary}{}{}
        #1
    \end{mycorollary}
    \begin{corpf}
        #2
    \end{corpf}
}

% Proposition - subsection numbering
\newtcbtheorem[use counter from=mydefinition]{myproposition}{Proposition}
{
    colbacktitle=yellow!30!white,
    colback=yellow!20!white,
    coltitle=black,
    fonttitle=\bfseries\large,
}{prop}

\NewDocumentCommand{\prop}{+m}{
    \begin{myproposition}{}{}
        #1
    \end{myproposition}
}

\newenvironment{proppf}{
    \par\noindent{\it \textbf{Proof for Proposition.}}\par\nopagebreak
    \begin{list}{}{\setlength\leftmargin{1em}\setlength\rightmargin{0em}}
    \item\relax
}{
    \hfill$\qed$\end{list}
}

\NewDocumentCommand{\propp}{+m+m}{
    \begin{myproposition}{}{}
        #1
    \end{myproposition}
    \begin{proppf}
        #2
    \end{proppf}
}

% Claim - subsection numbering
\newtcbtheorem[use counter from=mydefinition]{myclaim}{Claim}
{
    colbacktitle=pink!30!white,
    colback=pink!20!white,
    coltitle=black,
    fonttitle=\bfseries\large,
}{clm}

\NewDocumentCommand{\clm}{m+m}{
    \begin{myclaim}{#1}{}
        #2
    \end{myclaim}
}

\newenvironment{clmpf}{
    \par\noindent{\it \textbf{Proof for Claim.}}\par\nopagebreak
    \begin{list}{}{\setlength\leftmargin{1em}\setlength\rightmargin{0em}}
    \item\relax
}{
    \hfill$\qed$\end{list}
}

\NewDocumentCommand{\clmp}{m+m+m}{
    \begin{myclaim}{#1}{}
        #2
    \end{myclaim}
    \begin{clmpf}
        #3
    \end{clmpf}
}

% Fact - subsection numbering
\newtcbtheorem[use counter from=mydefinition]{myfact}{Fact}
{
    colbacktitle=purple!20!white,
    colback=purple!10!white,
    coltitle=black,
    fonttitle=\bfseries\large,
}{fact}

\NewDocumentCommand{\fact}{+m}{
    \begin{myfact}{}{}
        #1
    \end{myfact}
}

% Proof - customizable name
\NewDocumentEnvironment{custompf}{m}
{
    \par\noindent{\it \textbf{#1}}\par\nopagebreak
    \begin{list}{}{\setlength\leftmargin{1em}\setlength\rightmargin{0em}}
    \item\relax
}
{
    \hfill$\qed$\end{list}
}

\NewDocumentCommand{\pf}{O{Proof}+m}{
    \begin{custompf}{#1.}
        #2
    \end{custompf}
}

% Example - improved environment
\newenvironment{example}{
    \par\vspace{5pt}
    \noindent\textbf{Example.}\par\nopagebreak
    \begin{list}{}{\setlength\leftmargin{1em}\setlength\rightmargin{0em}}
    \item\relax
}{
    \end{list}\vspace{5pt}
}

\NewDocumentCommand{\ex}{+m}{
    \begin{example}
        #1
    \end{example}
}

% Remark
\NewDocumentCommand{\rmk}{+m}{
    {\it \color{blue!50!white}#1}
}

% Remark block - improved environment
\newenvironment{remark}{
    \par\vspace{5pt}
    \noindent\textbf{Remark.}\par\nopagebreak
    \begin{list}{}{\setlength\leftmargin{1em}\setlength\rightmargin{0em}}
    \item\relax
}{
    \end{list}\vspace{5pt}
}

\NewDocumentCommand{\rmkb}{+m}{
    \begin{remark}
        #1
    \end{remark}
}
\usepackage{blkarray}
\title{MATH 326 Lecture 2}
\author{Deepak Jassal}
\date{January 12\textsuperscript{th}, 2026}

\begin{document}
\maketitle
\section{Vector Spaces}
\setcounter{subsection}{1}
\setcounter{subsubsection}{1}  
\subsection{Subspaces}
\prop{\label{prop1}
Subspace Test\\
Let $V$ be a vector space over $F$, and $U\subseteq V$. Then $U$ is a subspace of $V$ if 
\begin{enumerate}
    \item $0\in U$;
    \item If $u,v\in U$ hten $u+v\in U$;
    \item If $u\in U$ and $c\in F$, then $cu\in U$.
\end{enumerate}
}
The last two conditions in \textit{prop 1.2.1} is equivalent to the condition 
\begin{itemize}
    \item For all $a,b\in F$ and for all $u,v\in U$, $au+bv\in U$. That is, $U$ is closed under linear combinations.
\end{itemize}
\ex{
Let $A=M_{n\times n}(F)$
\begin{enumerate}
    \item $\mathrm{null}(A)=\{x\in F^n: AX=0\}$
    Check: If $x,y\in \mathrm{null}(A),$ $a,b\in F$, then 
    \[
        Ax=0,Ay=0\Rightarrow A(ax+by)=a(Ax)+b(Ay)=a0+b0=0\Rightarrow aX+bY\in \mathrm{null}(A).
    \]
    \item $\mathrm{col}(A)=\{x_1a_1+x_2a_2+\dots+x_na_n:x_1,\dots,x_n\in F\}$.
\end{enumerate}
}
\prop{
    Span of $U$ is always a subspace of $V$.
}
\thm{Properties of Column Spaces}{
    Let $Y=[y_1,y_2,\dots,y_p]$ be and $m\times p$ matrix with entries in $F$, and let $A\in M_{m\times n}(F)$. Then 
    \[
        \mathrm{col}(Y)\subseteq \mathrm{col}(A)
    \] 
    if and only if there exists $X\in M_{n\times p}(F)$ such that $Y=AX$.
}
\pf[Proof of Theorem 1.2.3]{
    $(\Rightarrow)$ Suppose $\mathrm{col}(Y)\subseteq \mathrm{col}(A)$ then each column of $Y$, say $y_j\in \mathrm{col}(A)$ implies 
    \begin{align*}
        y_j&=x_{j_1}a_1+\cdots+x_{j_n}a_n\\
        &=A\begin{bmatrix}
            x_{j_1}\\
            \vdots\\
            x_{j_n}
        \end{bmatrix}=AX
    \end{align*}
    which in turn implies
    \begin{align*}
        [y_1\;y_2\;\dots\;y_n]=[Ax_1\;Ax_2\;\dots\;Ax_n]=AX
    \end{align*}
    $(\Leftarrow)$ Suppose $Y=AX$, this implies that for all $a\leq j\leq n$ $y_j=Ax_j$. This implies that $y_j=x_{j_1}a_1+\cdots+x_{j_n}a_n$. From here we see that $y_j\in\mathrm{col}(A)$ for all $1\leq j \leq n$. But $\mathrm{col}(A)\leq F^m$ meaning it is closed under linear combinations of $y_1,\cdots,y_n$. Thus, $\mathrm{col}(Y)\leq\mathrm{col}(A)$.
}
\subsection*{Properties of Span}
Let $V$ be a vector space over $F$ and $U\subseteq V$. Then 
\begin{enumerate}
    \item $\mathrm{span}(U)\leq V$;
    \item $U\subset \mathrm{span}(U)$;
    \item $U=\mathrm{span}(U)\Leftrightarrow U\leq V$;
    \item $U_1\subseteq U_2\Rightarrow \mathrm{span}(U_1)\leq \mathrm{span}(U_2)$;
    \item $\mathrm{span}(\mathrm{span}(U))=\mathrm{span}(U)$
\end{enumerate}
\subsection*{Operations on Subspaces}
\begin{enumerate}
    \item If $U,W\leq V$ then $U\cap W\leq V$
    \item $U+W\overset{\text{def}}{=}\{u+w:u\in U,w\in W\}$.
    If $U,W\leq V$ then 
    \begin{align*}
        U+W&=\{au+bw:a,b\in F,u\in U,w\in W\}\\
        &=\mathrm{span}(U\cup W)\leq W.
    \end{align*} 
\end{enumerate}
\defn{Direct Sums of Subspaces}{
    Let $V$ be a vectore space over $F$, and $U,W\leq V$ and $U\cap W=\{0\}$. Then we write $U+W=U\oplus W$, and say $V$ is a direct sum of $U$ and $W$.
}
\ex{
    Let $V=\R^2$, $U=\left\{\begin{bmatrix}
        x\\0
    \end{bmatrix}:x\in\R\right\}$,
    $W=\left\{\begin{bmatrix}
        0\\x
    \end{bmatrix}:x\in\R\right\}$, then $U,W\subseteq \R^2$. Since $U\cap W=\begin{bmatrix}
        0\\0
    \end{bmatrix}=\{0\}$. So, $\R^2=V\oplus W$.
}
\propp{
    Let $V$ be a vector space over $F$, Is $V=U\oplus W$ then for every vector $v\in V$, there are unique vectors $u\in U$ and $w\in W$ such that $v=u+w$.
}{
    Clearly $v=u+w$. Suppose that $v=u_1+w_1=u_2+w_2$ such that $u_1,u_2\in U$ and $w_1,w_2\in W$. Then 
    \[
        \underbrace{u_1-u_2}_{\in U}=\underbrace{w_2-w_1}_{\in W}=z
    \]
    this implies that   
    \[
        z\in U\cap W\Rightarrow z=0.
    \]
    Hence, $u_1-u_2=0\Rightarrow u_1=u_2$ and $w_2-w_1=0\Rightarrow w_1=w_2$.
}
\subsection*{Finding a Spanning Set of $\mathrm{col}(A)$}
\end{document}