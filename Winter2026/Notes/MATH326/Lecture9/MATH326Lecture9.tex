\documentclass[12pt]{article}
\usepackage{color,soul}
\usepackage{bookmark}
\usepackage{tensor}
% Import preambles and macros for notes
% Essential packages for notes
\usepackage{amsmath, amssymb, amsthm}
\usepackage{mathtools}  % for \coloneqq, etc.
\usepackage{geometry}   % Better page margins
\usepackage{parskip}    % Better paragraph spacing
\usepackage{microtype}  % Better typography
\usepackage{enumitem}   % Customize lists
\usepackage{hyperref}   % Clickable links
\usepackage{booktabs}   % Better tables
\usepackage{tcolorbox}  % For colored boxes/theorems

% Page layout for notes
\geometry{a4paper, margin=1in}
\setlength{\parskip}{0.8em}

% Theorem environments with shared numbering
\newtheorem{theorem}{Theorem}[subsection]  % Number within sections: 2.3.1, 2.3.2, etc.

% All other environments share the same counter as theorem
\newtheorem{lemma}[theorem]{Lemma}
\newtheorem{proposition}[theorem]{Proposition}
\newtheorem{corollary}[theorem]{Corollary}
\newtheorem{definition}[theorem]{Definition}
\newtheorem{example}[theorem]{Example}
\newtheorem{remark}[theorem]{Remark}
\newtheorem{claim}[theorem]{Claim}

% Custom colors for notes
\usepackage{xcolor}
\definecolor{note-blue}{RGB}{220, 230, 255}
\definecolor{theorem-green}{RGB}{220, 255, 220}
% Math notation shortcuts for notes
\newcommand{\R}{\mathbb{R}}
\newcommand{\C}{\mathbb{C}}
\newcommand{\Q}{\mathbb{Q}}
\newcommand{\Z}{\mathbb{Z}}
\newcommand{\N}{\mathbb{N}}

% Calculus
\newcommand{\diff}{\mathop{}\!\mathrm{d}}
\newcommand{\deriv}[2]{\frac{\mathrm{d}#1}{\mathrm{d}#2}}
\newcommand{\pderiv}[2]{\frac{\partial #1}{\partial #2}}

% Linear Algebra
\newcommand{\inner}[2]{\langle #1, #2 \rangle}
\newcommand{\norm}[1]{\| #1 \|}
\newcommand{\tr}{\operatorname{tr}}
\newcommand{\spn}{\operatorname{span}}
\newcommand{\rank}{\operatorname{rank}}
\newcommand{\nullity}{\operatorname{nullity}}

% Logic
\newcommand{\contra}{\Rightarrow\Leftarrow}

% Custom commands for notes
\newcommand{\todo}[1]{\textcolor{red}{[TODO: #1]}}
\newcommand{\important}[1]{\textbf{\textcolor{blue}{#1}}}
% Theorem system 
% Theorem System original by https://github.com/kcajc/math-notes-template, this is modified 
% The following boxes are provided:
%   Definition:     \defn 
%   Theorem:        \thm 
%   Lemma:          \lem
%   Corollary:      \cor
%   Proposition:    \prop   
%   Claim:          \clm
%   Fact:           \fact
%   Proof:          \pf
%   Example:        \ex
%   Remark:         \rmk (sentence), \rmkb (block)
% Suffix
%   r:              Allow Theorem/Definition to be referenced, e.g. thmr
%   p:              Add a short proof block for Lemma, Corollary, Proposition or Claim, e.g. lemp
%                   For theorems, use \pf for proof blocks

% Definition - subsection numbering for 1.1.1, 1.1.2
\newtcbtheorem[number within=subsection]{mydefinition}{Definition}
{
    colbacktitle=green!20!white,
    colback=green!10!white,
    coltitle=black,
    fonttitle=\bfseries\large,
}{defn}

\NewDocumentCommand{\defn}{m+m}{
    \begin{mydefinition}{#1}{}
        #2
    \end{mydefinition}
}

\NewDocumentCommand{\defnr}{mm+m}{
    \begin{mydefinition}{#1}{#2}
        #3
    \end{mydefinition}
}

% Theorem - subsection numbering
\newtcbtheorem[use counter from=mydefinition]{mytheorem}{Theorem}
{
    colbacktitle=cyan!20!white,
    colback=cyan!10!white,
    coltitle=black,
    fonttitle=\bfseries\large,
}{thm}

\NewDocumentCommand{\thm}{m+m}{
    \begin{mytheorem}{#1}{}
        #2
    \end{mytheorem}
}

\NewDocumentCommand{\thmr}{mm+m}{
    \begin{mytheorem}{#1}{#2}
        #3
    \end{mytheorem}
}

% Lemma - subsection numbering
\newtcbtheorem[use counter from=mydefinition]{mylemma}{Lemma}
{
    colbacktitle=violet!20!white,
    colback=violet!10!white,
    coltitle=black,
    fonttitle=\bfseries\large,
}{lem}

\NewDocumentCommand{\lem}{m+m}{
    \begin{mylemma}{#1}{}
        #2
    \end{mylemma}
}

% Improved proof environments with consistent QED placement
\newenvironment{lempf}{
    \par\noindent{\it \textbf{Proof for Lemma.}}\par\nopagebreak
    \begin{list}{}{\setlength\leftmargin{1em}\setlength\rightmargin{0em}}
    \item\relax
}{
    \hfill$\qed$\end{list}
}

\NewDocumentCommand{\lemp}{m+m+m}{
    \begin{mylemma}{#1}{}
        #2
    \end{mylemma}
    \begin{lempf}
        #3
    \end{lempf}
}

% Corollary - subsection numbering
\newtcbtheorem[use counter from=mydefinition]{mycorollary}{Corollary}
{
    colbacktitle=orange!20!white,
    colback=orange!10!white,
    coltitle=black,
    fonttitle=\bfseries\large,
}{cor}

\NewDocumentCommand{\cor}{+m}{
    \begin{mycorollary}{}{}
        #1
    \end{mycorollary}
}

\newenvironment{corpf}{
    \par\noindent{\it \textbf{Proof for Corollary.}}\par\nopagebreak
    \begin{list}{}{\setlength\leftmargin{1em}\setlength\rightmargin{0em}}
    \item\relax
}{
    \hfill$\qed$\end{list}
}

\NewDocumentCommand{\corp}{m+m+m}{
    \begin{mycorollary}{}{}
        #1
    \end{mycorollary}
    \begin{corpf}
        #2
    \end{corpf}
}

% Proposition - subsection numbering
\newtcbtheorem[use counter from=mydefinition]{myproposition}{Proposition}
{
    colbacktitle=yellow!30!white,
    colback=yellow!20!white,
    coltitle=black,
    fonttitle=\bfseries\large,
}{prop}

\NewDocumentCommand{\prop}{+m}{
    \begin{myproposition}{}{}
        #1
    \end{myproposition}
}

\newenvironment{proppf}{
    \par\noindent{\it \textbf{Proof for Proposition.}}\par\nopagebreak
    \begin{list}{}{\setlength\leftmargin{1em}\setlength\rightmargin{0em}}
    \item\relax
}{
    \hfill$\qed$\end{list}
}

\NewDocumentCommand{\propp}{+m+m}{
    \begin{myproposition}{}{}
        #1
    \end{myproposition}
    \begin{proppf}
        #2
    \end{proppf}
}

% Claim - subsection numbering
\newtcbtheorem[use counter from=mydefinition]{myclaim}{Claim}
{
    colbacktitle=pink!30!white,
    colback=pink!20!white,
    coltitle=black,
    fonttitle=\bfseries\large,
}{clm}

\NewDocumentCommand{\clm}{m+m}{
    \begin{myclaim}{#1}{}
        #2
    \end{myclaim}
}

\newenvironment{clmpf}{
    \par\noindent{\it \textbf{Proof for Claim.}}\par\nopagebreak
    \begin{list}{}{\setlength\leftmargin{1em}\setlength\rightmargin{0em}}
    \item\relax
}{
    \hfill$\qed$\end{list}
}

\NewDocumentCommand{\clmp}{m+m+m}{
    \begin{myclaim}{#1}{}
        #2
    \end{myclaim}
    \begin{clmpf}
        #3
    \end{clmpf}
}

% Fact - subsection numbering
\newtcbtheorem[use counter from=mydefinition]{myfact}{Fact}
{
    colbacktitle=purple!20!white,
    colback=purple!10!white,
    coltitle=black,
    fonttitle=\bfseries\large,
}{fact}

\NewDocumentCommand{\fact}{+m}{
    \begin{myfact}{}{}
        #1
    \end{myfact}
}

% Proof - customizable name
\NewDocumentEnvironment{custompf}{m}
{
    \par\noindent{\it \textbf{#1}}\par\nopagebreak
    \begin{list}{}{\setlength\leftmargin{1em}\setlength\rightmargin{0em}}
    \item\relax
}
{
    \hfill$\qed$\end{list}
}

\NewDocumentCommand{\pf}{O{Proof}+m}{
    \begin{custompf}{#1.}
        #2
    \end{custompf}
}

% Example - improved environment
\newenvironment{example}{
    \par\vspace{5pt}
    \noindent\textbf{Example.}\par\nopagebreak
    \begin{list}{}{\setlength\leftmargin{1em}\setlength\rightmargin{0em}}
    \item\relax
}{
    \end{list}\vspace{5pt}
}

\NewDocumentCommand{\ex}{+m}{
    \begin{example}
        #1
    \end{example}
}

% Remark
\NewDocumentCommand{\rmk}{+m}{
    {\it \color{blue!50!white}#1}
}

% Remark block - improved environment
\newenvironment{remark}{
    \par\vspace{5pt}
    \noindent\textbf{Remark.}\par\nopagebreak
    \begin{list}{}{\setlength\leftmargin{1em}\setlength\rightmargin{0em}}
    \item\relax
}{
    \end{list}\vspace{5pt}
}

\NewDocumentCommand{\rmkb}{+m}{
    \begin{remark}
        #1
    \end{remark}
}
\usepackage{blkarray}
\usepackage{tensor}
\title{MATH 326 Lecture 9}
\author{Deepak Jassal}
\date{January 28\textsuperscript{th}, 2026}

\begin{document}
\maketitle
\section{Rank}
\begin{enumerate}
    \item $\rank(AB)\leq\rank(A)$, because $\mathrm{col}(AB)\subseteq \mathrm{col}(A)$, and $\rank(A)=\mathrm{dim(col)}(A)=\mathrm{dim}(A)$. col$(AB)\subseteq$col$(A)$.
    \[
        \mathrm{col}(AB)=\mathrm{col}([Ab_1\dots Ab_n])
    \]
    is a linear combination of the columns of $A$.
    \[
        Ab=b_1a_1+b_2a_2+\cdots+b_na_n,
    \]
    So col$(AB)\leq$col$(A)$, dim(col$(AB))\leq$dim(col$(A))$.
    \item rank$(AB)\leq$rank$(A)$ because row$(AB)\subseteq$row$(A)$
    \[
        AB=\begin{bmatrix}
            \mathrm{row}_1A\\\vdots\\\mathrm{row}_mA
        \end{bmatrix}B
    \]
    \[
        \mathrm{row}_i(A)B=[a_{i_1}\dots a_{i_n}]\begin{bmatrix}\mathrm{row}_1B\\\vdots\\\mathrm{row}_mB\end{bmatrix}=a_{i_1}{row}_1B+\cdots+a_{i_n}{row}_nB.
    \]
    So each row of $AB$ is a linear combination of the rows of $B$.\\
    So row($AB$)$\subseteq$row($B$), then rank$(AB)\leq$rank$(B)$.\\
    (1) and (2) give that
    \[
        \mathrm{rank}(AB)\leq\min\{\mathrm{rank}(A),\rank(B)\}
    \]
    \item If $A^{-1}$ exists then $\rank(AB)=\rank(B)$. 
    \pf[Proof of the above]{
        \begin{align*}
            \rank(AB)&\leq\rank(B)\\
            \rank(A^{-1}(AB))&\leq\rank(AB).
        \end{align*}
        \[
            \rank(B)\leq\rank(AB)\leq\rank(B).
        \]
    }
    \item If $B^{-1}$ exists then
    \[
        \rank(AB)=\rank(A)
    \]
\end{enumerate}
    \corp{If $A$ and $B$ are similar then $\rank(A)=\rank(B)$.}{
        $B=P^{-1}AP$ so $\rank(B)=\rank(AP).$
    }{}
\textit{Conclusion.} If $T\in\mathcal{L}(V)=\mathcal{L}(V,V)$ then we can make a linear operator.\\
Define $\rank(T)=\rank(\tensor*[_\beta]{[T]}{_\beta})$,
\begin{enumerate}
    \item where $\beta$ is any basis of $T$
    \item We can also define $\tr(T)=\tr(\tensor[_\beta]{[T]}{_\beta})$
    \item $\chi_T(x)=\chi_{\tensor[_\beta]{[T]}{_\beta}}(x)$
\end{enumerate}
\subsection{Rank-Nullity Theorem}
\defn{Nullity}{
    Let $A\in M_{m\times n}(F)$. Then nullity$(A)=$dim$\{\null(A)\}$.
}
\thm{Rank-Nullity Theorem}{
    Let $A\in M_{m\times n}$. Then
    \[
        n=\rank(A)+\mathrm{nullity}(A)
    \]
}
\pf[Proof of the Theorem from MATH 220]{
    rank$(A)$ is the number of leading 1's in $r.r.e.f(A)$. The solution of $AX=0$ will have $n-r$ free parameters $X=c_1x_1+\cdots+c_{n-r}x_{n-r}$, where $x_i$ $1\leq i\leq n-r$ are called basic solutions, and they are linearaly independant. So $x_1,x_2,\dots,x_{n-r}$ is a basis of null$(A)$. So $n=r+n-r'$, $r=\rank(A)$ and $r'=$nullity$(A)$.
}
\pf[Proof of Theorem 1.1.2]{
    Let $d=$nullity$(A)=$dim$\null(A)$
    \begin{enumerate}
    \item[Case 1.] $d=0$ so $AX=0\Rightarrow X=0$. $AX$ is a linear combination of the columns of $A$. $(AX=0\Rightarrow X=0)\Rightarrow$ the columns of $A$ are L.I. So $\rank(A)=n$, rank+nullity=$n+0$.
    \end{enumerate}
    \item[Case 2.] $d=n$ $AX=0,$ $(\forall x\in F^c,\, AX=0)\Rightarrow A=0$, so rank$(A)=0$. 
    \item[Case 3.] $1\leq d\leq n-1$. Let $v_1,\dots,v_d$ be a basis of $\null(A)$. Let $V=[v_1\dots v_d]$ so $V\in M_{n\times d}(F)$. Complete $v_1,\dots,v_d$ to a basis of $F^n$ by adding $n-d$ vectors $w_1,\dots,w_{n-d}$ and let $W=[w_1,\dots,w_{n-d}]$. Now let $U=[VW]\in M_{n\times n}(F)$, and $U$ is invertible.
        
        \begin{align*}
            AU&=A[VW]\\
            &=[AV,AW]\\
            &=[0,AW].
        \end{align*}
        
        \[
            \mathrm{col}(AU)=\mathrm{col}(A)
        \]
        but $AU=[0,AW]$, $\mathrm{col}(A)=\mathrm{col}(AU)=\mathrm{col}(AW)$. 
        
        \[
            \null(AW)=\{y\in F^{n-d}:AWy=0\}
        \]
        
        Note that $A_{m\times n}$, $W_{n\times n-d}$ implies $Wy\in\null(A)$.
        But the basis of $\null(A)$ is $v_1,\dots,v_d$. Both of these imply that $Wy=Vx$, where $y\in F^{n-d}$ and $x\in F^d$. This implies that 
        \[
            Wy-Vx=0 \quad\Leftrightarrow\quad \begin{bmatrix}V & W\end{bmatrix}\begin{bmatrix}-x\\y\end{bmatrix}=0
        \]
        and $[V\ W]=U$ is invertible $\Rightarrow \begin{bmatrix}-x\\y\end{bmatrix}=0\Rightarrow x=y=0$. 
        
        Thus $AWy=0\Rightarrow y=0$, so $\null(AW)=\{0\}$. Therefore $\mathrm{col}[AW]_{m\times (n-d)}$ has dimension $n-d$, and we conclude that $\rank(A)=n-d$.
}
\end{document}