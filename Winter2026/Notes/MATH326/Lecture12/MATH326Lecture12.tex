\documentclass[12pt]{article}
\usepackage{color,soul}
\usepackage{bookmark}
\usepackage{tensor}
% Import preambles and macros for notes
% Essential packages for notes
\usepackage{amsmath, amssymb, amsthm}
\usepackage{mathtools}  % for \coloneqq, etc.
\usepackage{geometry}   % Better page margins
\usepackage{parskip}    % Better paragraph spacing
\usepackage{microtype}  % Better typography
\usepackage{enumitem}   % Customize lists
\usepackage{hyperref}   % Clickable links
\usepackage{booktabs}   % Better tables
\usepackage{tcolorbox}  % For colored boxes/theorems

% Page layout for notes
\geometry{a4paper, margin=1in}
\setlength{\parskip}{0.8em}

% Theorem environments with shared numbering
\newtheorem{theorem}{Theorem}[subsection]  % Number within sections: 2.3.1, 2.3.2, etc.

% All other environments share the same counter as theorem
\newtheorem{lemma}[theorem]{Lemma}
\newtheorem{proposition}[theorem]{Proposition}
\newtheorem{corollary}[theorem]{Corollary}
\newtheorem{definition}[theorem]{Definition}
\newtheorem{example}[theorem]{Example}
\newtheorem{remark}[theorem]{Remark}
\newtheorem{claim}[theorem]{Claim}

% Custom colors for notes
\usepackage{xcolor}
\definecolor{note-blue}{RGB}{220, 230, 255}
\definecolor{theorem-green}{RGB}{220, 255, 220}
% Math notation shortcuts for notes
\newcommand{\R}{\mathbb{R}}
\newcommand{\C}{\mathbb{C}}
\newcommand{\Q}{\mathbb{Q}}
\newcommand{\Z}{\mathbb{Z}}
\newcommand{\N}{\mathbb{N}}

% Calculus
\newcommand{\diff}{\mathop{}\!\mathrm{d}}
\newcommand{\deriv}[2]{\frac{\mathrm{d}#1}{\mathrm{d}#2}}
\newcommand{\pderiv}[2]{\frac{\partial #1}{\partial #2}}

% Linear Algebra
\newcommand{\inner}[2]{\langle #1, #2 \rangle}
\newcommand{\norm}[1]{\| #1 \|}
\newcommand{\tr}{\operatorname{tr}}
\newcommand{\spn}{\operatorname{span}}
\newcommand{\rank}{\operatorname{rank}}
\newcommand{\nullity}{\operatorname{nullity}}

% Logic
\newcommand{\contra}{\Rightarrow\Leftarrow}

% Custom commands for notes
\newcommand{\todo}[1]{\textcolor{red}{[TODO: #1]}}
\newcommand{\important}[1]{\textbf{\textcolor{blue}{#1}}}
% Theorem system 
% Theorem System original by https://github.com/kcajc/math-notes-template, this is modified 
% The following boxes are provided:
%   Definition:     \defn 
%   Theorem:        \thm 
%   Lemma:          \lem
%   Corollary:      \cor
%   Proposition:    \prop   
%   Claim:          \clm
%   Fact:           \fact
%   Proof:          \pf
%   Example:        \ex
%   Remark:         \rmk (sentence), \rmkb (block)
% Suffix
%   r:              Allow Theorem/Definition to be referenced, e.g. thmr
%   p:              Add a short proof block for Lemma, Corollary, Proposition or Claim, e.g. lemp
%                   For theorems, use \pf for proof blocks

% Definition - subsection numbering for 1.1.1, 1.1.2
\newtcbtheorem[number within=subsection]{mydefinition}{Definition}
{
    colbacktitle=green!20!white,
    colback=green!10!white,
    coltitle=black,
    fonttitle=\bfseries\large,
}{defn}

\NewDocumentCommand{\defn}{m+m}{
    \begin{mydefinition}{#1}{}
        #2
    \end{mydefinition}
}

\NewDocumentCommand{\defnr}{mm+m}{
    \begin{mydefinition}{#1}{#2}
        #3
    \end{mydefinition}
}

% Theorem - subsection numbering
\newtcbtheorem[use counter from=mydefinition]{mytheorem}{Theorem}
{
    colbacktitle=cyan!20!white,
    colback=cyan!10!white,
    coltitle=black,
    fonttitle=\bfseries\large,
}{thm}

\NewDocumentCommand{\thm}{m+m}{
    \begin{mytheorem}{#1}{}
        #2
    \end{mytheorem}
}

\NewDocumentCommand{\thmr}{mm+m}{
    \begin{mytheorem}{#1}{#2}
        #3
    \end{mytheorem}
}

% Lemma - subsection numbering
\newtcbtheorem[use counter from=mydefinition]{mylemma}{Lemma}
{
    colbacktitle=violet!20!white,
    colback=violet!10!white,
    coltitle=black,
    fonttitle=\bfseries\large,
}{lem}

\NewDocumentCommand{\lem}{m+m}{
    \begin{mylemma}{#1}{}
        #2
    \end{mylemma}
}

% Improved proof environments with consistent QED placement
\newenvironment{lempf}{
    \par\noindent{\it \textbf{Proof for Lemma.}}\par\nopagebreak
    \begin{list}{}{\setlength\leftmargin{1em}\setlength\rightmargin{0em}}
    \item\relax
}{
    \hfill$\qed$\end{list}
}

\NewDocumentCommand{\lemp}{m+m+m}{
    \begin{mylemma}{#1}{}
        #2
    \end{mylemma}
    \begin{lempf}
        #3
    \end{lempf}
}

% Corollary - subsection numbering
\newtcbtheorem[use counter from=mydefinition]{mycorollary}{Corollary}
{
    colbacktitle=orange!20!white,
    colback=orange!10!white,
    coltitle=black,
    fonttitle=\bfseries\large,
}{cor}

\NewDocumentCommand{\cor}{+m}{
    \begin{mycorollary}{}{}
        #1
    \end{mycorollary}
}

\newenvironment{corpf}{
    \par\noindent{\it \textbf{Proof for Corollary.}}\par\nopagebreak
    \begin{list}{}{\setlength\leftmargin{1em}\setlength\rightmargin{0em}}
    \item\relax
}{
    \hfill$\qed$\end{list}
}

\NewDocumentCommand{\corp}{m+m+m}{
    \begin{mycorollary}{}{}
        #1
    \end{mycorollary}
    \begin{corpf}
        #2
    \end{corpf}
}

% Proposition - subsection numbering
\newtcbtheorem[use counter from=mydefinition]{myproposition}{Proposition}
{
    colbacktitle=yellow!30!white,
    colback=yellow!20!white,
    coltitle=black,
    fonttitle=\bfseries\large,
}{prop}

\NewDocumentCommand{\prop}{+m}{
    \begin{myproposition}{}{}
        #1
    \end{myproposition}
}

\newenvironment{proppf}{
    \par\noindent{\it \textbf{Proof for Proposition.}}\par\nopagebreak
    \begin{list}{}{\setlength\leftmargin{1em}\setlength\rightmargin{0em}}
    \item\relax
}{
    \hfill$\qed$\end{list}
}

\NewDocumentCommand{\propp}{+m+m}{
    \begin{myproposition}{}{}
        #1
    \end{myproposition}
    \begin{proppf}
        #2
    \end{proppf}
}

% Claim - subsection numbering
\newtcbtheorem[use counter from=mydefinition]{myclaim}{Claim}
{
    colbacktitle=pink!30!white,
    colback=pink!20!white,
    coltitle=black,
    fonttitle=\bfseries\large,
}{clm}

\NewDocumentCommand{\clm}{m+m}{
    \begin{myclaim}{#1}{}
        #2
    \end{myclaim}
}

\newenvironment{clmpf}{
    \par\noindent{\it \textbf{Proof for Claim.}}\par\nopagebreak
    \begin{list}{}{\setlength\leftmargin{1em}\setlength\rightmargin{0em}}
    \item\relax
}{
    \hfill$\qed$\end{list}
}

\NewDocumentCommand{\clmp}{m+m+m}{
    \begin{myclaim}{#1}{}
        #2
    \end{myclaim}
    \begin{clmpf}
        #3
    \end{clmpf}
}

% Fact - subsection numbering
\newtcbtheorem[use counter from=mydefinition]{myfact}{Fact}
{
    colbacktitle=purple!20!white,
    colback=purple!10!white,
    coltitle=black,
    fonttitle=\bfseries\large,
}{fact}

\NewDocumentCommand{\fact}{+m}{
    \begin{myfact}{}{}
        #1
    \end{myfact}
}

% Proof - customizable name
\NewDocumentEnvironment{custompf}{m}
{
    \par\noindent{\it \textbf{#1}}\par\nopagebreak
    \begin{list}{}{\setlength\leftmargin{1em}\setlength\rightmargin{0em}}
    \item\relax
}
{
    \hfill$\qed$\end{list}
}

\NewDocumentCommand{\pf}{O{Proof}+m}{
    \begin{custompf}{#1.}
        #2
    \end{custompf}
}

% Example - improved environment
\newenvironment{example}{
    \par\vspace{5pt}
    \noindent\textbf{Example.}\par\nopagebreak
    \begin{list}{}{\setlength\leftmargin{1em}\setlength\rightmargin{0em}}
    \item\relax
}{
    \end{list}\vspace{5pt}
}

\NewDocumentCommand{\ex}{+m}{
    \begin{example}
        #1
    \end{example}
}

% Remark
\NewDocumentCommand{\rmk}{+m}{
    {\it \color{blue!50!white}#1}
}

% Remark block - improved environment
\newenvironment{remark}{
    \par\vspace{5pt}
    \noindent\textbf{Remark.}\par\nopagebreak
    \begin{list}{}{\setlength\leftmargin{1em}\setlength\rightmargin{0em}}
    \item\relax
}{
    \end{list}\vspace{5pt}
}

\NewDocumentCommand{\rmkb}{+m}{
    \begin{remark}
        #1
    \end{remark}
}
\usepackage{blkarray}
\usepackage{tensor}
\title{MATH 326 Lecture 12}
\author{Deepak Jassal}
\date{February 4\textsuperscript{th}, 2026}

\begin{document}
\maketitle
\thm{}{
    Let $A\in M_{m\times k}(F)$ $B\in M_{k\times n}$ and $C\in M_{M\times p}$. Then,
    \begin{enumerate}
        \item $\rank(AB)\leq\min\{\rank(A),\rank(B)\}$
        \item $\rank(AB)=\rank(A)$ \textit{iff} $\mathrm{col}(AB)=\mathrm{col}(A)$.
        \item $\max\{\rank(A),\rank(C)\}\leq \rank([AC])$
        \item $\rank(A)=\rank([AC])\Leftrightarrow \mathrm{col}(C)\subseteq\mathrm{col}(A)$.
    \end{enumerate}
}
\pf[Proof of the Theorem]{
    \begin{enumerate}
        \item Given in last class.
        \item $\rank(AB)=\rank(A)\Leftrightarrow$col$(AB)=$col$(A)$. We have col$(AB)\subseteq$col$(A)$. 
        \[
            AB=[Ab_1\,\dots\,Ab_n].
        \]
        \[
            Ab_1=\sum_{j=1}^{n}c_ja_j\subseteq\text{col}(AB)
        \]
        The same for all $Ab_j$ for all $j$. This gives the forward direction.\\
        $(\Leftarrow)$ suppose col$(AB)=$col$(A)\Rightarrow \rank(AB)=\rank(A)$ because $\rank(C)=$dim(col($C$)).
        \item $\max\{\rank(A)m\rank(C)\}\leq\rank(A\,\,C)$. Then $\rank(A)\leq\rank(A\,\,C)$ is trivial, col$(A)\subseteq$col($A\,\,C$). Thus we have the desired result.
        \item $(\Leftarrow)$ Suppose that col$(C)\subseteq$ col$(A)\Rightarrow$col$(A\,\,C)=$col$(A)$. In this case col$(A\,\,C)=$col$(A\,\,$cancel $C$).\\
        $(\Rightarrow)$ Suppose that $\rank(A)=\rank(A\,\,C)$ the same for columns then all columns of $A$ and $C$ $\subseteq$col$(A)$. Which gives the desired result.
    \end{enumerate}
}
\thm{}{
    Let $A\in M_{m\times k}$ and $B\in M_{k\times m}$. Then 
    \[
        \rank(A)+\rank(B)\leq\rank(AB)+k\tag{$\ast$}
    \]
}
\pf[Proof of the Theorem]{
    \begin{enumerate}
        \item[Case 1.] $AB=0$
        \begin{enumerate}
            \item[Subcase 1.] $A=0$ then $\rank(A)+\rank(B)\leq \rank(AB)+k$ then $\rank(B)\leq k$ which is true. 
            \item[Subcase 2.] $A\neq0$. $AB=0$ implies that all columns of $B$ are in the null space of $A$, that is $\rank(B)\leq\nullity(A)$. Apply rank-nullity theorem to $A$.
            \begin{align*}
                \rank(A)+\nullity(A)&=k\\
                \rank(A)+\rank(B)\leq k.
            \end{align*}
        \end{enumerate}  
        \item[Case 2.] $AB\neq0$, suppose $\rank(AB)=r$ with $r>0.$\\
        Let $XY=AB$ be the full rank decomposition of $AB$. Then $X$ is a basis of col$(AB)$, that is $X$ has $r$ L.I columns. So $X$ is an $m\times r$ matrix and col$(X)=$col$AB$. By similar reasoning we have row$(Y)=$row($AB$). Now let $C=[A\,\,X]_{m\times(k+r)}$ and let $D=\begin{bmatrix}B\\-Y\end{bmatrix}$. Now
        \[
            CD=AB-XY=0.
        \]
        Applying the result of case 1. We get
        \[
            \rank(A)+\rank(B)\leq\rank(C)+\rank(D)\leq\rank(CD)+k+r.
        \]
        Which is the desired result.
    \end{enumerate}
}
\thm{}{
    Let both $A,B\in M_{m\times n}$. Then $|\rank(A)-\rank(B)|\leq \rank(A+B)\leq\rank(A)+\rank(B)$.
}
\pf[Proof of the Theorem]{
    For the second inequality. $\rank(A+B)\leq$dim(col($A\,\,B$)). Suppose that $X$ is formed by the columns of $A$ and forms its basis. Same for $Y$ with $B$. Then col$(A\,\,B)=$col$(X\,\,Y)\leq\rank(X)+\rank(Y)=\rank(A)+\rank(B)$.\\
    For the second inequality. The inequality is equivalent to $\rank(A)-\rank(B)\leq\rank(A-B) (**)$ and $\rank(B)-\rank(A)\leq \rank(A+B)(***)$. But $(**)\Leftrightarrow\rank(A)\leq\rank(B)+\rank(A+B)=\rank(-B)+\rank(A+B)$
}


\end{document}