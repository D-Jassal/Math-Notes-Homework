\documentclass[12pt]{article}
\usepackage{color,soul}
% Import preambles and macros for notes
% Essential packages for notes
\usepackage{amsmath, amssymb, amsthm}
\usepackage{mathtools}  % for \coloneqq, etc.
\usepackage{geometry}   % Better page margins
\usepackage{parskip}    % Better paragraph spacing
\usepackage{microtype}  % Better typography
\usepackage{enumitem}   % Customize lists
\usepackage{hyperref}   % Clickable links
\usepackage{booktabs}   % Better tables
\usepackage{tcolorbox}  % For colored boxes/theorems

% Page layout for notes
\geometry{a4paper, margin=1in}
\setlength{\parskip}{0.8em}

% Theorem environments with shared numbering
\newtheorem{theorem}{Theorem}[subsection]  % Number within sections: 2.3.1, 2.3.2, etc.

% All other environments share the same counter as theorem
\newtheorem{lemma}[theorem]{Lemma}
\newtheorem{proposition}[theorem]{Proposition}
\newtheorem{corollary}[theorem]{Corollary}
\newtheorem{definition}[theorem]{Definition}
\newtheorem{example}[theorem]{Example}
\newtheorem{remark}[theorem]{Remark}
\newtheorem{claim}[theorem]{Claim}

% Custom colors for notes
\usepackage{xcolor}
\definecolor{note-blue}{RGB}{220, 230, 255}
\definecolor{theorem-green}{RGB}{220, 255, 220}
% Math notation shortcuts for notes
\newcommand{\R}{\mathbb{R}}
\newcommand{\C}{\mathbb{C}}
\newcommand{\Q}{\mathbb{Q}}
\newcommand{\Z}{\mathbb{Z}}
\newcommand{\N}{\mathbb{N}}

% Calculus
\newcommand{\diff}{\mathop{}\!\mathrm{d}}
\newcommand{\deriv}[2]{\frac{\mathrm{d}#1}{\mathrm{d}#2}}
\newcommand{\pderiv}[2]{\frac{\partial #1}{\partial #2}}

% Linear Algebra
\newcommand{\inner}[2]{\langle #1, #2 \rangle}
\newcommand{\norm}[1]{\| #1 \|}
\newcommand{\tr}{\operatorname{tr}}
\newcommand{\spn}{\operatorname{span}}
\newcommand{\rank}{\operatorname{rank}}
\newcommand{\nullity}{\operatorname{nullity}}

% Logic
\newcommand{\contra}{\Rightarrow\Leftarrow}

% Custom commands for notes
\newcommand{\todo}[1]{\textcolor{red}{[TODO: #1]}}
\newcommand{\important}[1]{\textbf{\textcolor{blue}{#1}}}
% Theorem system 
% Theorem System original by https://github.com/kcajc/math-notes-template, this is modified 
% The following boxes are provided:
%   Definition:     \defn 
%   Theorem:        \thm 
%   Lemma:          \lem
%   Corollary:      \cor
%   Proposition:    \prop   
%   Claim:          \clm
%   Fact:           \fact
%   Proof:          \pf
%   Example:        \ex
%   Remark:         \rmk (sentence), \rmkb (block)
% Suffix
%   r:              Allow Theorem/Definition to be referenced, e.g. thmr
%   p:              Add a short proof block for Lemma, Corollary, Proposition or Claim, e.g. lemp
%                   For theorems, use \pf for proof blocks

% Definition - subsection numbering for 1.1.1, 1.1.2
\newtcbtheorem[number within=subsection]{mydefinition}{Definition}
{
    colbacktitle=green!20!white,
    colback=green!10!white,
    coltitle=black,
    fonttitle=\bfseries\large,
}{defn}

\NewDocumentCommand{\defn}{m+m}{
    \begin{mydefinition}{#1}{}
        #2
    \end{mydefinition}
}

\NewDocumentCommand{\defnr}{mm+m}{
    \begin{mydefinition}{#1}{#2}
        #3
    \end{mydefinition}
}

% Theorem - subsection numbering
\newtcbtheorem[use counter from=mydefinition]{mytheorem}{Theorem}
{
    colbacktitle=cyan!20!white,
    colback=cyan!10!white,
    coltitle=black,
    fonttitle=\bfseries\large,
}{thm}

\NewDocumentCommand{\thm}{m+m}{
    \begin{mytheorem}{#1}{}
        #2
    \end{mytheorem}
}

\NewDocumentCommand{\thmr}{mm+m}{
    \begin{mytheorem}{#1}{#2}
        #3
    \end{mytheorem}
}

% Lemma - subsection numbering
\newtcbtheorem[use counter from=mydefinition]{mylemma}{Lemma}
{
    colbacktitle=violet!20!white,
    colback=violet!10!white,
    coltitle=black,
    fonttitle=\bfseries\large,
}{lem}

\NewDocumentCommand{\lem}{m+m}{
    \begin{mylemma}{#1}{}
        #2
    \end{mylemma}
}

% Improved proof environments with consistent QED placement
\newenvironment{lempf}{
    \par\noindent{\it \textbf{Proof for Lemma.}}\par\nopagebreak
    \begin{list}{}{\setlength\leftmargin{1em}\setlength\rightmargin{0em}}
    \item\relax
}{
    \hfill$\qed$\end{list}
}

\NewDocumentCommand{\lemp}{m+m+m}{
    \begin{mylemma}{#1}{}
        #2
    \end{mylemma}
    \begin{lempf}
        #3
    \end{lempf}
}

% Corollary - subsection numbering
\newtcbtheorem[use counter from=mydefinition]{mycorollary}{Corollary}
{
    colbacktitle=orange!20!white,
    colback=orange!10!white,
    coltitle=black,
    fonttitle=\bfseries\large,
}{cor}

\NewDocumentCommand{\cor}{+m}{
    \begin{mycorollary}{}{}
        #1
    \end{mycorollary}
}

\newenvironment{corpf}{
    \par\noindent{\it \textbf{Proof for Corollary.}}\par\nopagebreak
    \begin{list}{}{\setlength\leftmargin{1em}\setlength\rightmargin{0em}}
    \item\relax
}{
    \hfill$\qed$\end{list}
}

\NewDocumentCommand{\corp}{m+m+m}{
    \begin{mycorollary}{}{}
        #1
    \end{mycorollary}
    \begin{corpf}
        #2
    \end{corpf}
}

% Proposition - subsection numbering
\newtcbtheorem[use counter from=mydefinition]{myproposition}{Proposition}
{
    colbacktitle=yellow!30!white,
    colback=yellow!20!white,
    coltitle=black,
    fonttitle=\bfseries\large,
}{prop}

\NewDocumentCommand{\prop}{+m}{
    \begin{myproposition}{}{}
        #1
    \end{myproposition}
}

\newenvironment{proppf}{
    \par\noindent{\it \textbf{Proof for Proposition.}}\par\nopagebreak
    \begin{list}{}{\setlength\leftmargin{1em}\setlength\rightmargin{0em}}
    \item\relax
}{
    \hfill$\qed$\end{list}
}

\NewDocumentCommand{\propp}{+m+m}{
    \begin{myproposition}{}{}
        #1
    \end{myproposition}
    \begin{proppf}
        #2
    \end{proppf}
}

% Claim - subsection numbering
\newtcbtheorem[use counter from=mydefinition]{myclaim}{Claim}
{
    colbacktitle=pink!30!white,
    colback=pink!20!white,
    coltitle=black,
    fonttitle=\bfseries\large,
}{clm}

\NewDocumentCommand{\clm}{m+m}{
    \begin{myclaim}{#1}{}
        #2
    \end{myclaim}
}

\newenvironment{clmpf}{
    \par\noindent{\it \textbf{Proof for Claim.}}\par\nopagebreak
    \begin{list}{}{\setlength\leftmargin{1em}\setlength\rightmargin{0em}}
    \item\relax
}{
    \hfill$\qed$\end{list}
}

\NewDocumentCommand{\clmp}{m+m+m}{
    \begin{myclaim}{#1}{}
        #2
    \end{myclaim}
    \begin{clmpf}
        #3
    \end{clmpf}
}

% Fact - subsection numbering
\newtcbtheorem[use counter from=mydefinition]{myfact}{Fact}
{
    colbacktitle=purple!20!white,
    colback=purple!10!white,
    coltitle=black,
    fonttitle=\bfseries\large,
}{fact}

\NewDocumentCommand{\fact}{+m}{
    \begin{myfact}{}{}
        #1
    \end{myfact}
}

% Proof - customizable name
\NewDocumentEnvironment{custompf}{m}
{
    \par\noindent{\it \textbf{#1}}\par\nopagebreak
    \begin{list}{}{\setlength\leftmargin{1em}\setlength\rightmargin{0em}}
    \item\relax
}
{
    \hfill$\qed$\end{list}
}

\NewDocumentCommand{\pf}{O{Proof}+m}{
    \begin{custompf}{#1.}
        #2
    \end{custompf}
}

% Example - improved environment
\newenvironment{example}{
    \par\vspace{5pt}
    \noindent\textbf{Example.}\par\nopagebreak
    \begin{list}{}{\setlength\leftmargin{1em}\setlength\rightmargin{0em}}
    \item\relax
}{
    \end{list}\vspace{5pt}
}

\NewDocumentCommand{\ex}{+m}{
    \begin{example}
        #1
    \end{example}
}

% Remark
\NewDocumentCommand{\rmk}{+m}{
    {\it \color{blue!50!white}#1}
}

% Remark block - improved environment
\newenvironment{remark}{
    \par\vspace{5pt}
    \noindent\textbf{Remark.}\par\nopagebreak
    \begin{list}{}{\setlength\leftmargin{1em}\setlength\rightmargin{0em}}
    \item\relax
}{
    \end{list}\vspace{5pt}
}

\NewDocumentCommand{\rmkb}{+m}{
    \begin{remark}
        #1
    \end{remark}
}
\usepackage{blkarray}
\title{MATH 326 Lecture 3}
\author{Deepak Jassal}
\date{January 14\textsuperscript{th}, 2026}

\begin{document}
\maketitle
\setcounter{section}{1}
\setcounter{subsection}{2}
\setcounter{subsubsection}{5}
\defn{Basis of a Vector Space}{
    Let $\mathcal{V}$ be a vector space, $\mathcal{B}\subseteq \mathcal{V}$, then $\mathcal{B}$ is a basis of $\mathcal{V}$ if
    \begin{enumerate}
        \item $\mathrm{span}(\mathcal{B})=\mathcal{V}$;
        \item $\mathcal{B}$ is linearily independant (LI).
    \end{enumerate}
}
\ex{
    Let $\mathcal{V}=P_\R$ (vector space of all polynomial equations with real coefficients). Then 
    \[
        \mathcal{B}=1,x,x^2,x^3,\dots
    \]
}
\prop{
    If $\beta=v_1,v_2,\cdots,v_n$ is a basis of $\mathcal{V}$, then every vector $u\in\mathcal{V}$ can be expressed as a unique linear combination of vectors in $\beta$.
}
\defn{Finite Dimensional Vector Spaces}{
    A vector space is called finite dimensional if it has a finite basis.
}
\propp{
    Every vector space has a basis.
}
{
    Involves Zorn's lemma.
}
\thm{Equivalence of Basis Vectors}{
    Let $\mathcal{V}$ be a vector space and $\mathcal{B}$ a basis of $\mathcal{V}$. Then the following statements are equivalent
    \begin{enumerate}
        \item $\mathcal{B}$ is a basis of $\mathcal{V}$;
        \item $\mathcal{B}$ is a maximal LI subset of $\mathcal{V}$;
        \item $\mathcal{B}$ is a minimal span set of $\mathcal{V}$.
    \end{enumerate}
}
\pf[Proof of Theorem 1.2.5]{
    $(1\Rightarrow 2)$ Suppose that $\mathcal{B}$ is a basis of $\mathcal{V}$. Then $\mathcal{B}$ is L.I. Further suppose $v\not\in \mathcal{B}$.
    \textit{Claim.} $\mathcal{B}\cup v$ is not L.I.
    \pf[Proof of the Claim]{
        $\mathrm{span}(\mathcal{B})=V$ so $v=c_1v_1+\cdots+c_n+v_n\Rightarrow \mathcal{B}\cup v$ is not L.I.
    }
    $(2\Rightarrow 1)$ Suppose that $S\subseteq \mathcal{V}$ is maximal L.I. Then for any $v\in\mathcal{V}$ $v=c_1v_1+\cdots+c_n+v_n$. Since this is true for any arbitrary $v\in\mathcal{V}$ we have $\mathrm{span}(\mathcal{S})=\mathcal{V}$.
}
\thm{}{Every spannign set of a vector space $\mathcal{V}$ can be made into a basis}
\thm{Cardinalioty of Basis}{Let $\mathcal{V}$ be a vector space, and $\mathcal{B}_1$, $\mathcal{B}_2$ a basis of $\mathcal{V}$. Then $|\mathcal{B}_1|=|\mathcal{B}_2|$.}
The proof of this theorem uses the following lemma
\lem{Steinitz's Exchange Lemma}{
    Let $\mathcal{V}$ be a vector space over $\mathbb{F}$, $\mathcal{I}=\{v_1,v_2,\dots v_m\}$ be a L.I subset of $\mathcal{V}$, $\mathcal{S}=\{w_1,w_1,\dots,w_n\}$ be such that $\mathrm{span}(\mathcal{S})=V$. then
    \begin{enumerate}
        \item $m\leq n$;
        \item we can replace $m$ vectors in $\mathcal{S}$ with the vectors in $\mathcal{I}$, and the new set is a spanning set.
    \end{enumerate}
}
\pf[Proof of Lemma 1.28]{
    \begin{enumerate}
        \item This is trivial from the properties of L.I and spanning sets.
        \item Exchange some $w_i$ with $v_1$.\\
        Since $\mathrm{span}(\mathcal{S})=\mathcal{V}$ then $v_1\in\mathrm{span}(\mathcal{V})\Rightarrow v_1=c_1w_1+\dots+c_nw_n$ and $v_1\neq0$ (due to L.I). This tells us that not all of the $c_1,\dots,c_n$ are equal to zero. Without loss of generality let $i=1$, then replace $w_1$ with $v_1$. From here we obtain
        \[
            w_1=\frac{1}{c_1}v_1-\frac{1}{c_2}w_2-\cdots-\frac{1}{c_n}w_n.
        \]
        We claim that $\mathrm{span}(\{v_1,w_2,w_3,\dots,w_n\})=\mathcal{V}$.\\
        Assume that we have exchanged $k$ vectors in $\mathcal{S}$ with vectors from $\mathcal{I}$. Without loss of generality set
        \[
            \widehat{\mathcal{S}}=\{v_1,v_2\dots,v_k,w_{k+1},w_{k+2},\dots,w_{n}\}. 
        \]
        So we have $v_{k+1}\in\mathrm{span}(\mathcal{\widehat{S}})$, and since $v_{k+1}$ is L.I with $\{v_1,v_2\dots,v_k\}$ we know that $v_{k+1}=c_{k+1}w_{k+1}+c_{k+2}w_{k+2}+\cdots+c_{n}w_n$ where not all $c_{k+1}\dots c_n$ are equal to zero. This means that we can replace a vector from $\{w_{k+1},\dots,w_n\}$ with $v_{k+1}$. Repeat until we obtain the desired result. $\qedhere$
    \end{enumerate}
}
\cor{$|\mathcal{B}_1|=|\mathcal{B}_2|$ if both are basis sets for $\mathcal{V}$.}
\pf[Proof of Corollary 1.2.9]{
    Suppose $|\mathcal{B}_1|=m$ and $|\mathcal{B}_1|=n$. Then applying Steinitz'x lemme with $I=\mathcal{B}_1$ and $S=\mathcal{B}_2$ we get $m\leq n$. Applying the lemma agin with $I=\mathcal{B}_2$ and $S=\mathcal{B}_1$ we get $n\leq m$. Thus, $m=n\Rightarrow |\mathcal{B}_1|=|\mathcal{B}_2|$.
}
\end{document}