\documentclass[12pt]{article}

% Import preambles and macros for homework
% Essential packages
\usepackage{amsmath, amsfonts, amssymb, amsthm}
\usepackage{mathtools}
\usepackage{enumitem}
\usepackage{graphicx}
\usepackage{wrapfig}
\usepackage{systeme}
\usepackage{caption}
\usepackage{soul}
\usepackage[dvipsnames]{xcolor}
\usepackage{fancyhdr}
\allowdisplaybreaks

% Page layout
\usepackage[
  top=2cm,
  bottom=2cm,
  left=2cm,
  right=2cm,
  headheight=17pt,
  includehead,includefoot,
  heightrounded,
]{geometry}


% pgfornament for title page decorations
\usepackage[object=vectorian]{pgfornament}

% Fancy header/footer setup
\pagestyle{fancy}
\setlength{\headheight}{14.49998pt}
\addtolength{\topmargin}{-2.49998pt}
\renewcommand{\footrulewidth}{0.4pt}
\setlength\parindent{15pt}
% Math notation shortcuts
\newcommand{\R}{\mathbb{R}}
\newcommand{\Q}{\mathbb{Q}}
\newcommand{\Z}{\mathbb{Z}}
\newcommand{\N}{\mathbb{N}}
\newcommand{\C}{\mathbb{C}}
\newcommand{\X}{\mathcal{X}}

% Theorem environments
\newtheorem{mainthm}{Theorem}[section]
\newtheorem{theorem}{Theorem}[section]  
\newtheorem{lemma}[theorem]{Lemma}
\newtheorem{proposition}[theorem]{Proposition}
\newtheorem{corollary}[theorem]{Corollary}
\newtheorem{definition}[theorem]{Definition}
\newtheorem{claim}[theorem]{Claim}

% Calculus
\newcommand{\diff}{\mathop{}\!\mathrm{d}}
\newcommand{\deriv}[2]{\frac{\mathrm{d}#1}{\mathrm{d}#2}}
\newcommand{\pderiv}[2]{\frac{\partial #1}{\partial #2}}

% Linear Algebra
\newcommand{\inner}[2]{\langle #1, #2 \rangle}
\newcommand{\norm}[1]{\| #1 \|}
\newcommand{\tr}{\operatorname{tr}}
\newcommand{\spn}{\operatorname{span}}
\newcommand{\rank}{\operatorname{rank}}
\newcommand{\nullity}{\operatorname{nullity}}

% Logic
\newcommand{\contra}{\Rightarrow\Leftarrow}

% Custom commands for notes
\newcommand{\todo}[1]{\textcolor{red}{[TODO: #1]}}
\newcommand{\important}[1]{\textbf{\textcolor{blue}{#1}}}

%Number Theory
\DeclareMathOperator{\Li}{Li}
\newcommand{\floor}[1]{\left\lfloor #1 \right\rfloor}
\newcommand{\fract}[1]{\left\{ #1 \right\}}




\newcommand{\maketitlepage}{
    \begin{titlepage}
        \centering
        \vspace*{2.0cm}
        \pgfornament{84}\\
        {\LARGE \textsc{\coursename}\par}
        \vspace{0.5cm}
        {\large\coursecode\par}
        \vspace{0.5cm}
        {\large\instructor\par}
        \vspace{1.5cm}
        {\huge\bfseries\assignment\par}
        \vspace{1cm}
        {\LARGE\itshape\author\par}
        \vspace{2cm}
        {\large\bfseries Due Date:\par}
        \vspace{0.5cm}
        {\Large \duedate}\\
        \pgfornament{84}
    \end{titlepage}
}
% =============================================
% HOMEWORK CONFIGURATION - EDIT THESE VALUES!
% =============================================

% Your personal info
\renewcommand{\author}{Deepak Jassal}
\newcommand{\authorlast}{Jassal}

% Course info
\newcommand{\coursename}{Course Name}
\newcommand{\coursecode}{Course code}
\newcommand{\instructor}{Instructor}

% Assignment-specific info (CHANGE THESE FOR EACH HOMEWORK)
\newcommand{\assignment}{Assignment }
\newcommand{\duedate}{Month Day\textsuperscript{th}, 20XX}

% Header configuration
\fancyhead[l]{\assignment}
\fancyhead[c]{\coursecode}
\fancyhead[r]{\monthyear}
\fancyfoot[c]{\authorlast{ }\thepage}

\renewcommand{\author}{Deepak Jassal}
\renewcommand{\authorlast}{Jassal}
\renewcommand{\coursename}{Advanced Linear Algebra}
\renewcommand{\coursecode}{MATH 326}
\renewcommand{\assignment}{Assignment 4}
\renewcommand{\instructor}{Dr. Edward Dobrowolski}
\renewcommand{\duedate}{February 25\textsuperscript{th}, 2026}
\usepackage{tensor}

\begin{document}
\begin{titlepage}
	\centering
	\vspace*{2.0cm}	
	\pgfornament{84}\\
	{\LARGE \textsc{\coursename}\par}
	\vspace{0.5cm}
	{\large\coursecode\par}
    \vspace{0.5cm}
    {\large\instructor\par}
	\vspace{1.5cm}
	{\huge\bfseries\assignment\par}
	\vspace{1cm}  
	{\LARGE\itshape\author\par}
    \vspace{2cm}
	{\large\bfseries Due Date:\par}
	\vspace{0.5cm}
	{\Large \duedate}\\
	\pgfornament{84}
\end{titlepage}
\stepcounter{section}
\section*{Problem 1}
Let
\[
    A=\begin{bmatrix}
        1& 3& 0& 2& 5\\
        2& 6& 1& 1& 7\\
        3& 9& 2& 0& 9\\
        4& 12& 1& 5& 17\\
    \end{bmatrix}
\]
and
\[
    B=\begin{bmatrix}
        10& 30& 4& 8& 38\\
        9& 27& 4& 6& 33\\
        4& 12& 1& 5& 17\\
        5& 15& 3& 1& 16&
    \end{bmatrix}.
\]
\begin{enumerate}[label=(\alph*)]
    \item Find a basis of the row space of $A$ and a basis of the row space of $B$. This can be done by a single Maple command for each matrix. Are $A$ and $B$ row equivalent.
    \item Find the pivot matrix $P$ for the matrix $A$, and the pivot matrix $Q$ for the matrix $B$. For creating a matrix with selected columns of $A$ you can use Maple command [> P := SubMatrix(A, [1..5], [j1, j2]) The brackets [1..5] means that you want all rows from 1 to 5, in the second brackets you indicate the columns you want to take from matrix $A$.
    \item Find the matrix $R$ such that $A = P R$ and $B = QR$ and check that these products work.
    \item Find an invertible matrix $C$ such that $B = CA$. Check the lecture where the Lemma is proved. For this you will need to concatenate something to $P$ and something else (possibly else it not necessarily) so that the new matrices are invertible.
    \item Find a basis of $\null(A)$ and of $\null(B)$. For this recall how to we solve equations $Ax = 0$ and $Bx = 0$. If you have completed (c) you will know that we do not need to use $A$ or $B$ anymore.
\end{enumerate}
\stepcounter{section}
\section*{Problem 2}
Suppose that $A$ and $B$ are in $M_n(\mathbb{F})$. Prove that if $A$ is invertible then $\rank(AB) = \rank(B)$ and $\rank(BA) = \rank(B)$ by considering dimensions of the row space or column space on both sides, whichever is appropriate in a given case.
\stepcounter{section}
\section*{Problem 3}
Let
\[
    T:\mathbb{F}^5\to\mathbb{F}^4
\]
be defined by $Tx=Ax$, where $A$ is a matrix from Problem 1.\\
Find $\dim(\ker(T))$ and $\dim(\mathrm{image}(T))$ and verify that 
\[
    4=\dim(\ker(T))+\dim(\mathrm{image}(T)).
\]
\end{document}