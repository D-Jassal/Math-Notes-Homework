\documentclass[12pt]{article}

% Import preambles and macros for homework
% Essential packages
\usepackage{amsmath, amsfonts, amssymb, amsthm}
\usepackage{mathtools}
\usepackage{enumitem}
\usepackage{graphicx}
\usepackage{wrapfig}
\usepackage{systeme}
\usepackage{caption}
\usepackage{soul}
\usepackage[dvipsnames]{xcolor}
\usepackage{fancyhdr}
\allowdisplaybreaks

% Page layout
\usepackage[
  top=2cm,
  bottom=2cm,
  left=2cm,
  right=2cm,
  headheight=17pt,
  includehead,includefoot,
  heightrounded,
]{geometry}


% pgfornament for title page decorations
\usepackage[object=vectorian]{pgfornament}

% Fancy header/footer setup
\pagestyle{fancy}
\setlength{\headheight}{14.49998pt}
\addtolength{\topmargin}{-2.49998pt}
\renewcommand{\footrulewidth}{0.4pt}
\setlength\parindent{15pt}
% Math notation shortcuts
\newcommand{\R}{\mathbb{R}}
\newcommand{\Q}{\mathbb{Q}}
\newcommand{\Z}{\mathbb{Z}}
\newcommand{\N}{\mathbb{N}}
\newcommand{\C}{\mathbb{C}}
\newcommand{\X}{\mathcal{X}}

% Theorem environments
\newtheorem{mainthm}{Theorem}[section]
\newtheorem{theorem}{Theorem}[section]  
\newtheorem{lemma}[theorem]{Lemma}
\newtheorem{proposition}[theorem]{Proposition}
\newtheorem{corollary}[theorem]{Corollary}
\newtheorem{definition}[theorem]{Definition}
\newtheorem{claim}[theorem]{Claim}

% Calculus
\newcommand{\diff}{\mathop{}\!\mathrm{d}}
\newcommand{\deriv}[2]{\frac{\mathrm{d}#1}{\mathrm{d}#2}}
\newcommand{\pderiv}[2]{\frac{\partial #1}{\partial #2}}

% Linear Algebra
\newcommand{\inner}[2]{\langle #1, #2 \rangle}
\newcommand{\norm}[1]{\| #1 \|}
\newcommand{\tr}{\operatorname{tr}}
\newcommand{\spn}{\operatorname{span}}
\newcommand{\rank}{\operatorname{rank}}
\newcommand{\nullity}{\operatorname{nullity}}

% Logic
\newcommand{\contra}{\Rightarrow\Leftarrow}

% Custom commands for notes
\newcommand{\todo}[1]{\textcolor{red}{[TODO: #1]}}
\newcommand{\important}[1]{\textbf{\textcolor{blue}{#1}}}

%Number Theory
\DeclareMathOperator{\Li}{Li}
\newcommand{\floor}[1]{\left\lfloor #1 \right\rfloor}
\newcommand{\fract}[1]{\left\{ #1 \right\}}




\newcommand{\maketitlepage}{
    \begin{titlepage}
        \centering
        \vspace*{2.0cm}
        \pgfornament{84}\\
        {\LARGE \textsc{\coursename}\par}
        \vspace{0.5cm}
        {\large\coursecode\par}
        \vspace{0.5cm}
        {\large\instructor\par}
        \vspace{1.5cm}
        {\huge\bfseries\assignment\par}
        \vspace{1cm}
        {\LARGE\itshape\author\par}
        \vspace{2cm}
        {\large\bfseries Due Date:\par}
        \vspace{0.5cm}
        {\Large \duedate}\\
        \pgfornament{84}
    \end{titlepage}
}
% =============================================
% HOMEWORK CONFIGURATION - EDIT THESE VALUES!
% =============================================

% Your personal info
\renewcommand{\author}{Deepak Jassal}
\newcommand{\authorlast}{Jassal}

% Course info
\newcommand{\coursename}{Course Name}
\newcommand{\coursecode}{Course code}
\newcommand{\instructor}{Instructor}

% Assignment-specific info (CHANGE THESE FOR EACH HOMEWORK)
\newcommand{\assignment}{Assignment }
\newcommand{\duedate}{Month Day\textsuperscript{th}, 20XX}

% Header configuration
\fancyhead[l]{\assignment}
\fancyhead[c]{\coursecode}
\fancyhead[r]{\monthyear}
\fancyfoot[c]{\authorlast{ }\thepage}

\renewcommand{\author}{Deepak Jassal}
\renewcommand{\authorlast}{Jassal}
\renewcommand{\coursename}{Advanced Linear Algebra}
\renewcommand{\coursecode}{MATH 326}
\renewcommand{\assignment}{Assignment 4}
\renewcommand{\instructor}{Dr. Edward Dobrowolski}
\renewcommand{\duedate}{February 25\textsuperscript{th}, 2026}
\usepackage{tensor}

\begin{document}
\begin{titlepage}
	\centering
	\vspace*{2.0cm}	
	\pgfornament{84}\\
	{\LARGE \textsc{\coursename}\par}
	\vspace{0.5cm}
	{\large\coursecode\par}
    \vspace{0.5cm}
    {\large\instructor\par}
	\vspace{1.5cm}
	{\huge\bfseries\assignment\par}
	\vspace{1cm}  
	{\LARGE\itshape\author\par}
    \vspace{2cm}
	{\large\bfseries Due Date:\par}
	\vspace{0.5cm}
	{\Large \duedate}\\
	\pgfornament{84}
\end{titlepage}
\stepcounter{section}
\section*{Problem 1}
Let
\[
    A=\begin{bmatrix}
        1& 3& 0& 2& 5\\
        2& 6& 1& 1& 7\\
        3& 9& 2& 0& 9\\
        4& 12& 1& 5& 17\\
    \end{bmatrix}
\]
and
\[
    B=\begin{bmatrix}
        10& 30& 4& 8& 38\\
        9& 27& 4& 6& 33\\
        4& 12& 1& 5& 17\\
        5& 15& 3& 1& 16&
    \end{bmatrix}.
\]
\begin{enumerate}[label=(\alph*)]
    \item Find a basis of the row space of $A$ and a basis of the row space of $B$. This can be done by a single Maple command for each matrix. Are $A$ and $B$ row equivalent.\\
    \textit{Solution.}
    \[
        r.r.e.f(A)=
        \begin{bmatrix}
        1& 3& 0& 2& 5\\
        0& 0& 1& -3& -3\\
        0& 0& 0& 0& 0\\
        0& 0& 0& 0& 0\\                
        \end{bmatrix},\quad
        r.r.e.f(B)=
        \begin{bmatrix}
        1& 3& 0& 0.8& 3.8\\
        0& 0& 1& -3& -3\\
        0& 0& 0& 0& 0\\
        0& 0& 0& 0& 0\\                
        \end{bmatrix}.
    \]
    Basis row space of $A=\{(1,3,0,2,5),(0,0,1,-3,-3)\}$ and basis of the row space of $B=\{(1,3,0,0.8,3.8),(0,0,1,-3,-3)\}$ 
    \item Find the pivot matrix $P$ for the matrix $A$, and the pivot matrix $Q$ for the matrix $B$. For creating a matrix with selected columns of $A$ you can use Maple command [> P := SubMatrix(A, [1..5], [j1, j2]) The brackets [1..5] means that you want all rows from 1 to 5, in the second brackets you indicate the columns you want to take from matrix $A$.\\
    \textit{Solution.}
    \[
        P=
        \begin{bmatrix}
            1&0\\
            2&1\\
            3&2\\
            4&1
        \end{bmatrix},\quad
        Q=
        \begin{bmatrix}
            10&4\\
            9&4\\
            4&1\\
            5&3
        \end{bmatrix}    
    \]
    \item Find the matrix $R$ such that $A = P R$ and $B = QR$ and check that these products work.\\
    \textit{Solution.} For $A = PR_A$, $R_A$ is the matrix of non-pivot columns expressed in terms of pivot columns, the same is for $Q$
    \[
        R_A=
        \begin{bmatrix}
            3&2&5\\
            0&-3&-3
        \end{bmatrix},\quad
    \]
    \[
        PR_A=
        \begin{bmatrix}
            1&0\\
            2&1\\
            3&2\\
            4&1
        \end{bmatrix}
        \begin{bmatrix}
            3&2&5\\
            0&-3&-3
        \end{bmatrix}=
        A=\begin{bmatrix}
            1& 3& 0& 2& 5\\
            2& 6& 1& 1& 7\\
            3& 9& 2& 0& 9\\
            4& 12& 1& 5& 17\\
        \end{bmatrix}.
    \]
    The same goes for $B$ giving
    \[
        R_B=
        \begin{bmatrix}
            3&0.8&3.8\\
            0&-3&-3
        \end{bmatrix}
    \]
    \[
        QR_B=
        \begin{bmatrix}
            10&4\\
            9&4\\
            4&1\\
            5&3
        \end{bmatrix}  
        \begin{bmatrix}
            3&0.8&3.8\\
            0&-3&-3
        \end{bmatrix}=
        \begin{bmatrix}
            10& 30& 4& 8& 38\\
            9& 27& 4& 6& 33\\
            4& 12& 1& 5& 17\\
            5& 15& 3& 1& 16&
        \end{bmatrix}.
    \]
    \item Find an invertible matrix $C$ such that $B = CA$. Check the lecture where the Lemma is proved. For this you will need to concatenate something to $P$ and something else (possibly else it not necessarily) so that the new matrices are invertible.\\
    \textit{Solution.} We can concatenate something to the matrices $P,Q$ to make them invertible
    \[
        P_e=
        \begin{bmatrix}
            1&0&0&0\\
            2&1&0&0\\
            3&2&1&0\\
            4&1&0&1
        \end{bmatrix},\quad
        Q_e=
        \begin{bmatrix}
            10&4&0&0\\
            9&4&0&0\\
            4&1&1&0\\
            5&3&0&1
        \end{bmatrix}.
    \]
    Now
    \[
        C=Q_eP_e^{-1}=
        \begin{bmatrix}
            10&4&0&0\\
            9&4&0&0\\
            4&1&1&0\\
            5&3&0&1
        \end{bmatrix}
        \begin{bmatrix}
            1&0&0&0\\
            -2&1&0&0\\
            1&-2&1&0\\
            -2&-1&0&1
        \end{bmatrix}=
        \begin{bmatrix}
            2&4&0&0\\
            1&5&0&0\\
            1&2&1&0\\
            0&2&0&1           
        \end{bmatrix}.
    \]
    \item Find a basis of $\null(A)$ and of $\null(B)$. For this recall how to we solve equations $Ax = 0$ and $Bx = 0$. If you have completed (c) you will know that we do not need to use $A$ or $B$ anymore.\\
    \textit{Solution.} Since $A=PR$ and $B=QR$, we need to solve $PR_Ax=0$, and $QR_Ax=0$, since $P,Q$ have full column rank we need to solve $R_Ax=0$, $R_Ax=0$.
    \[
        Rx=0\Rightarrow 3x_1+2x_3+x_5=0,\quad -3x_3-3x_5=0
    \] 
    this gives $x_3=-x_5$ and $x_1=-x_5$. A basis for the null space is $\{(0,1,0,0,0),(0,0,0,1,0),(-1,0,-1,0,1)\}$. Similarrily for $Bx=0$ we have the basis we have $3x_1 + 0.8x_3 + 3.8x_5 = 0 -3x_3 - 3x_5 = 0$, giving $x_3=-x_5$ and $x_1=-x_5$. This means they both have the same basis vectors for the null space.
\end{enumerate}
\stepcounter{section}
\newpage
\section*{Problem 2}
Suppose that $A$ and $B$ are in $M_n(\mathbb{F})$. Prove that if $A$ is invertible then $\rank(AB) = \rank(B)$ and $\rank(BA) = \rank(B)$ by considering dimensions of the row space or column space on both sides, whichever is appropriate in a given case.
\begin{proof}
    \[
        \col(AB)=\{(AB)x\mid x\in\mathbb{F}^n\}=\{A(B)x\mid x\in\mathbb{F}^n\}=\{Ay\mid y\in\col(B)\}.
    \]
    Due to the invertibility of $A$ we have that it is injective so the image of the column space of $B$ under $A$ has the same dimension as the column space of $B$ giving
    \begin{align*}
        \dim\col(AB)&=\dim\col(B)\\
        \rank(AB)&=\rank(B).
    \end{align*} 
    \[
        \row(BA)=\{v^T(BA)\mid v^T\in\mathbb{F}^n\}=\{wA\mid w\in\row(B)\}.
    \]
    Right multiplication of an invertible matrix is injective on the row vectors, so dimensionality is preserved giving
    \begin{align*}
        \dim\row(BA)&=\dim\row(B)\\
        \rank(BA)&=\rank(B).
    \end{align*} 
\end{proof}
\stepcounter{section}
\section*{Problem 3}
Let
\[
    T:\mathbb{F}^5\to\mathbb{F}^4
\]
be defined by $Tx=Ax$, where $A$ is a matrix from Problem 1.\\
Find $\dim(\ker(T))$ and $\dim(\mathrm{image}(T))$ and verify that 
\[
    5=\dim(\ker(T))+\dim(\mathrm{image}(T)).
\]
\textit{Solution.} We can use $rref(A)$ instead of $A$ because $rref(A)$ preserves the essential components of the matrix. The image of $T$ is the column space of $A$ due the vector multiplication is being multiplied on the left of the matrix. This gives
\[
    \dim(\image(T))=\dim(\col(A))=2.
\]
The dimension of the kernel of $T$ is going to be the dimension of the null space of $A$. In (1e) we found that 
\[
    \dim(\ker(T))=\dim(\null(A))=3.
\]
This gives 
\[
    5=\dim(\ker(T))+\dim(\image(T)).
\]
\end{document}