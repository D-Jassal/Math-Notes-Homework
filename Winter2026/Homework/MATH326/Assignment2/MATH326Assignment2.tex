\documentclass[12pt]{article}

% Import preambles and macros for homework
% Essential packages
\usepackage{amsmath, amsfonts, amssymb, amsthm}
\usepackage{mathtools}
\usepackage{enumitem}
\usepackage{graphicx}
\usepackage{wrapfig}
\usepackage{systeme}
\usepackage{caption}
\usepackage{soul}
\usepackage[dvipsnames]{xcolor}
\usepackage{fancyhdr}
\allowdisplaybreaks

% Page layout
\usepackage[
  top=2cm,
  bottom=2cm,
  left=2cm,
  right=2cm,
  headheight=17pt,
  includehead,includefoot,
  heightrounded,
]{geometry}


% pgfornament for title page decorations
\usepackage[object=vectorian]{pgfornament}

% Fancy header/footer setup
\pagestyle{fancy}
\setlength{\headheight}{14.49998pt}
\addtolength{\topmargin}{-2.49998pt}
\renewcommand{\footrulewidth}{0.4pt}
\setlength\parindent{15pt}
% Math notation shortcuts
\newcommand{\R}{\mathbb{R}}
\newcommand{\Q}{\mathbb{Q}}
\newcommand{\Z}{\mathbb{Z}}
\newcommand{\N}{\mathbb{N}}
\newcommand{\C}{\mathbb{C}}
\newcommand{\X}{\mathcal{X}}

% Theorem environments
\newtheorem{mainthm}{Theorem}[section]
\newtheorem{theorem}{Theorem}[section]  
\newtheorem{lemma}[theorem]{Lemma}
\newtheorem{proposition}[theorem]{Proposition}
\newtheorem{corollary}[theorem]{Corollary}
\newtheorem{definition}[theorem]{Definition}
\newtheorem{claim}[theorem]{Claim}

% Calculus
\newcommand{\diff}{\mathop{}\!\mathrm{d}}
\newcommand{\deriv}[2]{\frac{\mathrm{d}#1}{\mathrm{d}#2}}
\newcommand{\pderiv}[2]{\frac{\partial #1}{\partial #2}}

% Linear Algebra
\newcommand{\inner}[2]{\langle #1, #2 \rangle}
\newcommand{\norm}[1]{\| #1 \|}
\newcommand{\tr}{\operatorname{tr}}
\newcommand{\spn}{\operatorname{span}}
\newcommand{\rank}{\operatorname{rank}}
\newcommand{\nullity}{\operatorname{nullity}}

% Logic
\newcommand{\contra}{\Rightarrow\Leftarrow}

% Custom commands for notes
\newcommand{\todo}[1]{\textcolor{red}{[TODO: #1]}}
\newcommand{\important}[1]{\textbf{\textcolor{blue}{#1}}}

%Number Theory
\DeclareMathOperator{\Li}{Li}
\newcommand{\floor}[1]{\left\lfloor #1 \right\rfloor}
\newcommand{\fract}[1]{\left\{ #1 \right\}}




\newcommand{\maketitlepage}{
    \begin{titlepage}
        \centering
        \vspace*{2.0cm}
        \pgfornament{84}\\
        {\LARGE \textsc{\coursename}\par}
        \vspace{0.5cm}
        {\large\coursecode\par}
        \vspace{0.5cm}
        {\large\instructor\par}
        \vspace{1.5cm}
        {\huge\bfseries\assignment\par}
        \vspace{1cm}
        {\LARGE\itshape\author\par}
        \vspace{2cm}
        {\large\bfseries Due Date:\par}
        \vspace{0.5cm}
        {\Large \duedate}\\
        \pgfornament{84}
    \end{titlepage}
}
% =============================================
% HOMEWORK CONFIGURATION - EDIT THESE VALUES!
% =============================================

% Your personal info
\renewcommand{\author}{Deepak Jassal}
\newcommand{\authorlast}{Jassal}

% Course info
\newcommand{\coursename}{Course Name}
\newcommand{\coursecode}{Course code}
\newcommand{\instructor}{Instructor}

% Assignment-specific info (CHANGE THESE FOR EACH HOMEWORK)
\newcommand{\assignment}{Assignment }
\newcommand{\duedate}{Month Day\textsuperscript{th}, 20XX}

% Header configuration
\fancyhead[l]{\assignment}
\fancyhead[c]{\coursecode}
\fancyhead[r]{\monthyear}
\fancyfoot[c]{\authorlast{ }\thepage}

\renewcommand{\author}{Deepak Jassal}
\renewcommand{\authorlast}{Jassal}
\renewcommand{\coursename}{Advanced Linear Algebra}
\renewcommand{\coursecode}{MATH 326}
\renewcommand{\assignment}{Assignment 2}
\renewcommand{\instructor}{Dr. Edward Dobrowolski}
\renewcommand{\duedate}{January 30\textsuperscript{th}, 2026}
\usepackage{tensor}

\begin{document}
\begin{titlepage}
	\centering
	\vspace*{2.0cm}	
	\pgfornament{84}\\
	{\LARGE \textsc{\coursename}\par}
	\vspace{0.5cm}
	{\large\coursecode\par}
    \vspace{0.5cm}
    {\large\instructor\par}
	\vspace{1.5cm}
	{\huge\bfseries\assignment\par}
	\vspace{1cm}  
	{\LARGE\itshape\author\par}
    \vspace{2cm}
	{\large\bfseries Due Date:\par}
	\vspace{0.5cm}
	{\Large \duedate}\\
	\pgfornament{84}
\end{titlepage}
\stepcounter{section}
\section*{Problem 1}
Show that the set $\{t^k : k = 0, 1, 2, 3, . . . \}$ is linearly independent in $\mathcal{C}_\R[0,1]$.
\begin{proof}
    Let $n \in \mathbb{N}$ and suppose 
    \[
    \sum_{k=0}^n a_k t^k = 0 \quad \forall t \in [0,1],
    \]
    where $a_k \in \mathbb{R}$.  
    The left-hand side is a polynomial $p(t)$ in $t$.  
    Since $p(t) = 0$ on an infinite set (indeed, on the interval $[0,1]$), the polynomial $p$ has infinitely many roots.  

    A nonzero polynomial of degree at most $n$ has at most $n$ real roots, so $p$ must be the zero polynomial; hence $a_0 = a_1 = \dots = a_n = 0$.

    Thus every finite subset of $\{ t^k : k = 0, 1, 2, \dots \}$ is linearly independent in $\mathcal{C}_\mathbb{R}[0,1]$. This gives the desired result.
\end{proof}

\stepcounter{section}
\section*{Problem 2}
Determine which of the following are subspaces of $\mathcal{C}_\R[0,1]$
\begin{enumerate}[label=(\alph*)]
    \item $\{f\in\mathcal{C}_\R[0,1]:\int_{0}^{1}x^2f(x)\,dx=0 \}$\\
    We need only show that the above mentioned space is non-empty, closed under addition, and is closed under multiplication by scalars. This is because the other axioms which need to be proven are inherited from the vector space $\mathcal{C}_\R[0,1]$.\\
    \begin{claim}
        $\{f\in\mathcal{C}_\R[0,1]:\int_{0}^{1}x^2f(x)\,dx=0 \}$ is a subspace of $\mathcal{C}_\R[0,1]$.
    \end{claim}
    \begin{proof}
        Let $V=\{f\in\mathcal{C}_\R[0,1]:\int_{0}^{1}x^2f(x)\,dx=0 \}$. Since $f(x)=0$ is in $\mathcal{C}_\R[0,1]$ we have that $V$ is non-empty. For any two functions $f_1$ and $f_2$ in $V$ we have 
        \[
            \int_{0}^{1}x^2(f_1(x)+f_2(x))\,dx=\int_{0}^{1}x^2f_1(x)\,dx+\int_{0}^{1}x^2f_2(x)\,dx=0.
        \]
        We also have for any $c\in\R$
        \[
            \int_{0}^{1}x^2(cf_1(x))\,dx=c\int_{0}^{1}x^2f_1(x)\,dx=c(0)=0.
        \]
        Thus, $V$ is a subspace of $\mathcal{C}_\R[0,1]$.
    \end{proof}
    \newpage
    \item $\{f\in\mathcal{C}_\R[0,1]:\int_{0}^{1}f^2(x)\,dx=0 \}$
    We need only show that the above mentioned space is non-empty, closed under addition, and is closed under multiplication by scalars. This is because the other axioms which need to be proven are inherited from the vector space $\mathcal{C}_\R[0,1]$.\\
    \begin{claim}
        $\{f\in\mathcal{C}_\R[0,1]:\int_{0}^{1}f^2(x)\,dx=0 \}$ is a subspace of $\mathcal{C}_\R[0,1]$.
    \end{claim}
    \begin{proof}
        Let $V=\{f\in\mathcal{C}_\R[0,1]:\int_{0}^{1}x^2f(x)\,dx=0 \}$. Since $f(x)=0$ is the only function in $\mathcal{C}_\R[0,1]$ with the property that $\int_{0}^{1}f^2(x)\,dx=0$ we have that $V$ is non-empty. Due to the only element in this being the zero element we have that the set is a subspace.
    \end{proof}
\end{enumerate}


\stepcounter{section}
\section*{Problem 3}
Let $x\in\mathbb{F}^n$ and $y\in\mathbb{F}^m$ be (column) vectors. Compute the rank of $xy^T$.\\
\textit{Solution.} We have
\[
    x=\begin{pmatrix}x_1\\x_2\\\vdots\\x_n\end{pmatrix},\quad y=\begin{pmatrix}y_1\\y_2\\\vdots\\y_m\end{pmatrix},
\]
\[
    \begin{pmatrix}x_1\\x_2\\\vdots\\x_n\end{pmatrix}\begin{pmatrix}y_1\\y_2\\\vdots\\y_m\end{pmatrix}^T=
    \begin{bmatrix}
        x_1y_1&x_1y_2&\cdots&x_1y_m\\
        x_2y_1&x_2y_2&\cdots&x_2y_m\\
        \vdots&\vdots&\ddots&\vdots\\
        x_ny_1&x_ny_2&\cdots&x_ny_m
    \end{bmatrix}.
\]
From here see that each column of $xy^T$ is the vector $x$ and each row of $xy^T$ is the vector $Y$. From here we have two situations, one in which either $x$ or $y$ is the zero vector, in this case $\rank(xy^T)=0$. In the cases with neither $x$ or $y$ are the zero vector we have $\rank(xy^T)=1$.
\newpage
\stepcounter{section}
\section*{Problem 4}
Let $A \in M_n$ be defined by $a_{ij} = 1$ if $i + j$ is even and 0 if $i + j$ is odd. Find a basis of $\mathrm{col}(A)$ and state the rank of (A).\\
\textit{Solution.} The matrix $A$ depends on if $n$ is odd or even. For odd $n$ we have
\[
    A=
    \begin{bmatrix}
        1&0&1&0\cdots&1\\
        0&1&0&1\cdots&0\\
        1&0&1&0\cdots&1\\
        0&1&0&1\cdots&0\\
        \vdots&\vdots&\vdots&\,\,\,\ddots&\vdots\\
        1&0&1&0\cdots&1\\
    \end{bmatrix}.
\]
For even $n$ we have
\[
    A=
    \begin{bmatrix}
        1&0&1&0\cdots&0\\
        0&1&0&1\cdots&1\\
        1&0&1&0\cdots&0\\
        0&1&0&1\cdots&1\\
        \vdots&\vdots&\vdots&\,\,\,\ddots&\vdots\\
        0&1&0&1\cdots&1\\
    \end{bmatrix}.
\]
From here it is clear that for both situations a basis for $\mathrm{col}(A)$ are the vectors
\[
    \begin{pmatrix}1\\0\\1\\0\\\vdots\\1\end{pmatrix},\quad  \begin{pmatrix}0\\1\\0\\1\\\vdots\\0\end{pmatrix}.
\]
It is also clear to see that $\rank(A)=2$ for both situations.

\stepcounter{section}
\section*{Problem 5}
If $A \in M_{n\times n}(F)$ is a nonzero matrix and $\rank(A) = r$. Show that
\begin{enumerate}[label=(\alph*)]
    \item A has an invertible $r \times r$ submatrix.\\
    \textit{Solution.} This can be seen by noting that $r.r.e.f(A)$ has an $r\times r$ submatrix with no zero columns or rows. Since this submatrix is in $r.r.e.f(A)$ it is invertible because all rows are have been made linearly independent. This submatrix is associated with the $r\times r$ submatrix of $A$ that is invertible. 
    \item No square submatrix of size greater than $r$ is invertible.\\
    \textit{Solution.} Look back to $r.r.e.f(A)$ an note that any $n\times n$ submatrix with $n>r$ this submatrix will have a zero row and a zero column and will therefore not be invertible. This is exactly why no $n\times n$ submatrix of $A$ will be invertible as there will be a row or column that is linearly dependant on the others and will therefore not be invertible.  
\end{enumerate}

\stepcounter{section}
\section*{Problem 6}
Let $V=M_2(F)$ with basis $\beta=\{E_1,E_2,E_3,E_4\}$ where $E_1=\begin{bmatrix}1&0\\0&0\end{bmatrix}$, $E_2=\begin{bmatrix}0&1\\0&0\end{bmatrix}$, $E_3=\begin{bmatrix}0&0\\1&0\end{bmatrix}$, $E_4=\begin{bmatrix}0&0\\0&1\end{bmatrix}$. Define $T:V\to V$ by $T(A)=BAB^{-1}$, where $B=\begin{bmatrix}5&1\\9&2\end{bmatrix}$. 
\begin{enumerate}[label=(\alph*)]
    \item Show that $T$ is linear.
    \begin{proof}
        We first have to calculate $B^{-1}$. Det$B=5(2)-9(1)=1$. So 
        \[
            B^{-1}=\begin{bmatrix}2&-1\\-9&5\end{bmatrix}.
        \]
        Let $A_1=\begin{bmatrix}a_1&b_1\\c_1&d_1\end{bmatrix}$ and $A_2=\begin{bmatrix}a_2&b_2\\c_2&d_2\end{bmatrix}$, with $a_1,a_2,b_1,b_2,c_1,c_2,d_1,d_2\in F$.
        \begin{align*}
            T(A_1+A_2)&=B(A_1+A_2)B^{-1}\\
            &=\begin{bmatrix}5&1\\9&2\end{bmatrix}\left(\begin{bmatrix}a_1&b_1\\c_1&d_1\end{bmatrix}+\begin{bmatrix}a_2&b_2\\c_2&d_2\end{bmatrix}\right)\begin{bmatrix}2&-1\\-9&5\end{bmatrix}\\
            &=\begin{bmatrix}5&1\\9&2\end{bmatrix}\begin{bmatrix}a_1+a_2&b_1+b_2\\c_1+c_2&d_1+d_2\end{bmatrix}\begin{bmatrix}2&-1\\-9&5\end{bmatrix}\\
            &=
            \begin{bmatrix}
                E_{11} & E_{12} \\
                E_{21} & E_{22}
            \end{bmatrix},
        \end{align*}
        where
        \[
        \begin{aligned}
            E_{11} &= 10a_1 + 10a_2 + 2c_1 + 2c_2 - 45b_1 - 45b_2 - 9d_1 - 9d_2, \\
            E_{12} &= -5a_1 - 5a_2 - c_1 - c_2 + 25b_1 + 25b_2 + 5d_1 + 5d_2, \\
            E_{21} &= 18a_1 + 18a_2 + 4c_1 + 4c_2 - 81b_1 - 81b_2 - 18d_1 - 18d_2, \\
            E_{22} &= -9a_1 - 9a_2 - 2c_1 - 2c_2 + 45b_1 + 45b_2 + 10d_1 + 10d_2.
        \end{aligned}
        \]
        \begin{align*}
            T(A_1)+T(A_2)&=BA_1B^{-1}+BAB^{-1}\\
            =&\begin{bmatrix}5&1\\9&2\end{bmatrix}\begin{bmatrix}a_1&b_1\\c_1&d_1\end{bmatrix}\begin{bmatrix}2&-1\\-9&5\end{bmatrix}+\begin{bmatrix}5&1\\9&2\end{bmatrix}\begin{bmatrix}a_2&b_2\\c_2&d_2\end{bmatrix}\begin{bmatrix}2&-1\\-9&5\end{bmatrix}\\
            =&
            \begin{bmatrix}
                10a_1 + 2c_1 - 45b_1 - 9d_1 & -5a_1 - c_1 + 25b_1 + 5d_1 \\
                18a_1 + 4c_1 - 81b_1 - 18d_1 & -9a_1 - 2c_1 + 45b_1 + 10d_1
            \end{bmatrix}\\&+
            \begin{bmatrix}
                10a_2 + 2c_2 - 45b_2 - 9d_2 & -5a_2 - c_2 + 25b_2 + 5d_2 \\
                18a_2 + 4c_2 - 81b_2 - 18d_2 & -9a_2 - 2c_2 + 45b_2 + 10d_2
            \end{bmatrix}\\
            =&
            \begin{bmatrix}
                E_{11} & E_{12} \\
                E_{21} & E_{22}
            \end{bmatrix}
        \end{align*}
        where
        \[
        \begin{aligned}
            E_{11} &= 10a_1 + 10a_2 + 2c_1 + 2c_2 - 45b_1 - 45b_2 - 9d_1 - 9d_2, \\
            E_{12} &= -5a_1 - 5a_2 - c_1 - c_2 + 25b_1 + 25b_2 + 5d_1 + 5d_2, \\
            E_{21} &= 18a_1 + 18a_2 + 4c_1 + 4c_2 - 81b_1 - 81b_2 - 18d_1 - 18d_2, \\
            E_{22} &= -9a_1 - 9a_2 - 2c_1 - 2c_2 + 45b_1 + 45b_2 + 10d_1 + 10d_2.
        \end{aligned}
        \]
        Thus we have that 
        \[
            T(A_1+A_2)=T(A_1)+T(A_2)
        \]
        for all $A\in M_2(F)$.
    \end{proof}

    \item Find the matrix of $T$. The bases at both ends are the same.\\
    \textit{Solution. }
    \begin{align*}
        \tensor[_\beta]{{[T]}}{_\beta}=&[T(E_1)T(E_2)T(E_3)T(E_4)]\\
        =&\bigg[\begin{bmatrix}5&1\\9&2\end{bmatrix}\begin{bmatrix}1&0\\0&0\end{bmatrix}\begin{bmatrix}2&-1\\-9&5\end{bmatrix}\begin{bmatrix}5&1\\9&2\end{bmatrix}\begin{bmatrix}0&1\\0&0\end{bmatrix}\begin{bmatrix}2&-1\\-9&5\end{bmatrix}\\
        &\begin{bmatrix}5&1\\9&2\end{bmatrix}\begin{bmatrix}0&0\\1&0\end{bmatrix}\begin{bmatrix}2&-1\\-9&5\end{bmatrix}\begin{bmatrix}5&1\\9&2\end{bmatrix}\begin{bmatrix}0&0\\0&1\end{bmatrix}\begin{bmatrix}2&-1\\-9&5\end{bmatrix}\bigg]\\
        &=\left[\begin{bmatrix} 10 & -5 \\ 18 & -9 \end{bmatrix}\begin{bmatrix} -45 & 25 \\ -81 & 45 \end{bmatrix}\begin{bmatrix} 2 & -1 \\ 4 & -2 \end{bmatrix}\begin{bmatrix} -9 & 5 \\ -18 & 10 \end{bmatrix}\right]\\
        &=\begin{bmatrix}
            10&-45&2&-9\\
            -5&25&-1&5\\
            18&-81&4&-18\\
            -9&45&-2&10
        \end{bmatrix}.
    \end{align*}
\end{enumerate}

\end{document}