\documentclass[12pt]{article}

% Import preambles and macros for homework
% Essential packages
\usepackage{amsmath, amsfonts, amssymb, amsthm}
\usepackage{mathtools}
\usepackage{enumitem}
\usepackage{graphicx}
\usepackage{wrapfig}
\usepackage{systeme}
\usepackage{caption}
\usepackage{soul}
\usepackage[dvipsnames]{xcolor}
\usepackage{fancyhdr}
\allowdisplaybreaks

% Page layout
\usepackage[
  top=2cm,
  bottom=2cm,
  left=2cm,
  right=2cm,
  headheight=17pt,
  includehead,includefoot,
  heightrounded,
]{geometry}


% pgfornament for title page decorations
\usepackage[object=vectorian]{pgfornament}

% Fancy header/footer setup
\pagestyle{fancy}
\setlength{\headheight}{14.49998pt}
\addtolength{\topmargin}{-2.49998pt}
\renewcommand{\footrulewidth}{0.4pt}
\setlength\parindent{15pt}
% Math notation shortcuts
\newcommand{\R}{\mathbb{R}}
\newcommand{\Q}{\mathbb{Q}}
\newcommand{\Z}{\mathbb{Z}}
\newcommand{\N}{\mathbb{N}}
\newcommand{\C}{\mathbb{C}}
\newcommand{\X}{\mathcal{X}}

% Theorem environments
\newtheorem{mainthm}{Theorem}[section]
\newtheorem{theorem}{Theorem}[section]  
\newtheorem{lemma}[theorem]{Lemma}
\newtheorem{proposition}[theorem]{Proposition}
\newtheorem{corollary}[theorem]{Corollary}
\newtheorem{definition}[theorem]{Definition}
\newtheorem{claim}[theorem]{Claim}

% Calculus
\newcommand{\diff}{\mathop{}\!\mathrm{d}}
\newcommand{\deriv}[2]{\frac{\mathrm{d}#1}{\mathrm{d}#2}}
\newcommand{\pderiv}[2]{\frac{\partial #1}{\partial #2}}

% Linear Algebra
\newcommand{\inner}[2]{\langle #1, #2 \rangle}
\newcommand{\norm}[1]{\| #1 \|}
\newcommand{\tr}{\operatorname{tr}}
\newcommand{\spn}{\operatorname{span}}
\newcommand{\rank}{\operatorname{rank}}
\newcommand{\nullity}{\operatorname{nullity}}

% Logic
\newcommand{\contra}{\Rightarrow\Leftarrow}

% Custom commands for notes
\newcommand{\todo}[1]{\textcolor{red}{[TODO: #1]}}
\newcommand{\important}[1]{\textbf{\textcolor{blue}{#1}}}

%Number Theory
\DeclareMathOperator{\Li}{Li}
\newcommand{\floor}[1]{\left\lfloor #1 \right\rfloor}
\newcommand{\fract}[1]{\left\{ #1 \right\}}




\newcommand{\maketitlepage}{
    \begin{titlepage}
        \centering
        \vspace*{2.0cm}
        \pgfornament{84}\\
        {\LARGE \textsc{\coursename}\par}
        \vspace{0.5cm}
        {\large\coursecode\par}
        \vspace{0.5cm}
        {\large\instructor\par}
        \vspace{1.5cm}
        {\huge\bfseries\assignment\par}
        \vspace{1cm}
        {\LARGE\itshape\author\par}
        \vspace{2cm}
        {\large\bfseries Due Date:\par}
        \vspace{0.5cm}
        {\Large \duedate}\\
        \pgfornament{84}
    \end{titlepage}
}
% =============================================
% HOMEWORK CONFIGURATION - EDIT THESE VALUES!
% =============================================

% Your personal info
\renewcommand{\author}{Deepak Jassal}
\newcommand{\authorlast}{Jassal}

% Course info
\newcommand{\coursename}{Course Name}
\newcommand{\coursecode}{Course code}
\newcommand{\instructor}{Instructor}

% Assignment-specific info (CHANGE THESE FOR EACH HOMEWORK)
\newcommand{\assignment}{Assignment }
\newcommand{\duedate}{Month Day\textsuperscript{th}, 20XX}

% Header configuration
\fancyhead[l]{\assignment}
\fancyhead[c]{\coursecode}
\fancyhead[r]{\monthyear}
\fancyfoot[c]{\authorlast{ }\thepage}

\renewcommand{\author}{Deepak Jassal}
\renewcommand{\authorlast}{Jassal}
\renewcommand{\coursename}{Advanced Linear Algebra}
\renewcommand{\coursecode}{MATH 326}
\renewcommand{\assignment}{Assignment 3}
\renewcommand{\instructor}{Dr. Edward Dobrowolski}
\renewcommand{\duedate}{February 6\textsuperscript{th}, 2026}
\usepackage{tensor}

\begin{document}
\begin{titlepage}
	\centering
	\vspace*{2.0cm}	
	\pgfornament{84}\\
	{\LARGE \textsc{\coursename}\par}
	\vspace{0.5cm}
	{\large\coursecode\par}
    \vspace{0.5cm}
    {\large\instructor\par}
	\vspace{1.5cm}
	{\huge\bfseries\assignment\par}
	\vspace{1cm}  
	{\LARGE\itshape\author\par}
    \vspace{2cm}
	{\large\bfseries Due Date:\par}
	\vspace{0.5cm}
	{\Large \duedate}\\
	\pgfornament{84}
\end{titlepage}
\stepcounter{section}
\section*{Problem 1}
Let $V=P_4$ the space of all polynomials with degree $\leq3$ and $W=\mathbb{F}^4$. Let
\[
	T:V\to W,
\]
where
\[
	T(f)=\begin{bmatrix}
		f(0)\\f(1)\\f(2)\\f(3)
	\end{bmatrix}.
\]
\begin{enumerate}[label=(\alph*)]
	\item Let $\alpha = 1, x, x^2, x^3$ be a basis of $V$ and $\beta = e_1, e_2, e_3, e_4$ be the standard basis of $W$. Find $\tensor[_\alpha]{[T]}{_\beta}$.\\
	\textit{Solution.} 
	\[
		\tensor[_\alpha]{[T]}{_\beta}=[T[1],Tx[x],T[x^2],T[x^3]]
	\]
	\[
	T(1)=\begin{bmatrix}
		f(0)\\f(1)\\f(2)\\f(3)
	\end{bmatrix}
	=
	\begin{bmatrix}
		1\\1\\1\\1
	\end{bmatrix},\quad
	T(x)=\begin{bmatrix}
		0\\1\\2\\3
	\end{bmatrix}
	=
	\begin{bmatrix}
		0\\1\\2\\3
	\end{bmatrix},\quad
	T(x^2)=\begin{bmatrix}
		0^2\\1^2\\2^2\\3^2
	\end{bmatrix}
	=
	\begin{bmatrix}
		0\\1\\4\\9
	\end{bmatrix},\quad
	T(x^3)=\begin{bmatrix}
		0^3\\1^3\\2^3\\3^3
	\end{bmatrix}
	=
	\begin{bmatrix}
		0\\1\\8\\27
	\end{bmatrix}.
	\]
	\[
		\tensor[_\alpha]{[T]}{_\beta}=[T[1],Tx[x],T[x^2],T[x^3]]=
		\begin{bmatrix}
			1&0&0&0\\
			1&1&1&1\\
			1&2&4&8\\
			1&3&9&27
		\end{bmatrix}
	\]
	\item Now let $\gamma = 1, 1 + x, 1 + x + x^2, 1 + x + x^2 + x^3$ be another basis of $V$, and $\delta = e_1 + e_2, e_1 - e_2, e_3 + e_4, e_3 - e_4$ be another basis of W. Find $\tensor[_\gamma]{[T]}{_\delta}$.\\
	\textit{Solution.} 
	\[
		\tensor[_\gamma]{[T]}{_\delta}=\tensor[_\delta]{[I_W]}{_\beta}\tensor[_\beta]{[T]}{_\alpha}\tensor[_\alpha]{[I_V]}{_\gamma}
	\]
	\[
		\tensor[_\alpha]{[I_V]}{_\gamma}=
		\begin{bmatrix}
			1&1&1&1\\
			0&1&1&1\\
			0&0&1&1\\
			0&0&0&1
		\end{bmatrix}.
	\]
	For $\beta\to\delta$ we have
	\[
		e_1=a(e_1+e_2)+b(e_1-e_2)+c(e_3+e_4)d(e_3-e_4)
	\]
	\[
		e_2=a(e_1+e_2)+b(e_1-e_2)+c(e_3+e_4)d(e_3-e_4)
	\]
	\[
		e_3=a(e_1+e_2)+b(e_1-e_2)+c(e_3+e_4)d(e_3-e_4)
	\]
	\[
		e_4=a(e_1+e_2)+b(e_1-e_2)+c(e_3+e_4)d(e_3-e_4)
	\]	
	Solving each of these we obtain 
	\[
		e_1=\begin{bmatrix}\frac{1}{2}\\\frac{1}{2}\\0\\0\end{bmatrix},\quad e_2=\begin{bmatrix}\frac{1}{2}\\-\frac{1}{2}\\0\\0\end{bmatrix},\quad e_3=\begin{bmatrix}0\\0\\\frac{1}{2}\\\frac{1}{2}\end{bmatrix},\quad e_4=\begin{bmatrix}0\\0\\\frac{1}{2}\\-\frac{1}{2}\end{bmatrix}.
	\]
	\[
		\tensor[_\delta]{[I_V]}{_\beta}=
		\begin{bmatrix}
			\frac{1}{2}&\frac{1}{2}&0&0\\
			\frac{1}{2}&-\frac{1}{2}&0&0\\
			0&0&\frac{1}{2}&\frac{1}{2}\\
			0&0&\frac{1}{2}&-\frac{1}{2}
		\end{bmatrix}.
	\]
	\begin{align*}
		\tensor[_\gamma]{[T]}{_\delta}&=\tensor[_\delta]{[I_W]}{_\beta}\tensor[_\beta]{[T]}{_\alpha}\tensor[_\alpha]{[I_V]}{_\gamma}\\
		&=
		\begin{bmatrix}
			\frac{1}{2}&\frac{1}{2}&0&0\\
			\frac{1}{2}&-\frac{1}{2}&0&0\\
			0&0&\frac{1}{2}&\frac{1}{2}\\
			0&0&\frac{1}{2}&-\frac{1}{2}
		\end{bmatrix}
		\begin{bmatrix}
			1&0&0&0\\
			1&1&1&1\\
			1&2&4&8\\
			1&1&8&27
		\end{bmatrix}
		\begin{bmatrix}
			1&0&0&0\\
			1&1&1&1\\
			1&2&4&8\\
			1&3&9&27
		\end{bmatrix}\\
		&=
		\begin{bmatrix}
			1&\frac{3}{2}&2&\frac{5}{2}\\
			0&-\frac{1}{2}&-1&-\frac{3}{2}\\
			1&\frac{7}{2}&10&\frac{55}{2}\\
			0&-\frac{1}{2}&-3&-\frac{25}{2}
		\end{bmatrix}
	\end{align*}
	\item Determine if $T$ is invertible.
	\textit{Note:} You can use any software you like.\\
	\textit{Solution.} Using an online software I computed the determinant of $T$ to be 12. Thus $T$ is invertible.
\end{enumerate}

\stepcounter{section}
\section*{Problem 2}
Let $T:P_3\leftrightarrow P_3$ be defined by $T(f)=xf'(x)$.
\begin{enumerate}[label=(\alph*)]
	\item Verify that T is linear.\\
	\textit{Solution.}
	\begin{align*}
		T(af+bg)&=x(af+bg)'(x)\\
		&=x(af'+bg')(x)\\
		&=xaf'(x)+xbg'(x)\\
		&=aT(f)+bT(g).
	\end{align*}
	\item Find $\tr(T)$.\\
	\textit{Solution.} Let $\beta=1,x,x^2,x^3$ be a basis for $P_3$. Then 
	\[
		t(0)=x\cdot0=0,\quad t(x)=x\cdot1=x,\quad t(x^2)=2x\cdot x=2x,\quad t(x^3)=3x\cdot x^2=3x^3.
	\]
	Then we have
	\[
		T=
		\begin{bmatrix}
			0&0&0&0\\
			0&1&0&0\\
			0&0&2&0\\
			0&0&0&3
		\end{bmatrix}.
	\]
	Thus, $\tr(T)=6$.
	\item Find the rank of $T$.\\
	\textit{Solution.} It is easy to see from the above matrix for $T$ that $\rank(T)=3$.
\end{enumerate}

\stepcounter{section}
\section*{Problem 3}
Let
\[
	A=
	\begin{pmatrix}
		1 & 2 & 3 & 6 & 5 & 6\\
		2 & 1 & 3 & 6 & 0 & 1\\
		3 & 3 & 6 & 12 & 1 & 2\\
		6 & 1 & 7 & 14 & 3 & 1\\
	\end{pmatrix}
\]
\begin{enumerate}[label=(\alph*)]
	\item Find the rank and nullity of $A$ and verify the rank - nullity theorem\\
	\textit{Solution.}
	\begin{align*}
		A &\xrightarrow[R_3 - 3R_1]{R_2 - 2R_1} 
		\begin{pmatrix}
			1 & 2 & 3 & 6 & 5 & 6 \\
			0 & -3 & -3 & -6 & -10 & -11 \\
			0 & -3 & -3 & -6 & -14 & -16 \\
			0 & -11 & -11 & -22 & -27 & -35
		\end{pmatrix} \\
		&\xrightarrow{R_3 - R_2}
		\begin{pmatrix}
			1 & 2 & 3 & 6 & 5 & 6 \\
			0 & -3 & -3 & -6 & -10 & -11 \\
			0 & 0 & 0 & 0 & -4 & -5 \\
			0 & -11 & -11 & -22 & -27 & -35
		\end{pmatrix} \\
		&\xrightarrow{R_4 - \frac{11}{3}R_2}
		\begin{pmatrix}
			1 & 2 & 3 & 6 & 5 & 6 \\
			0 & -3 & -3 & -6 & -10 & -11 \\
			0 & 0 & 0 & 0 & -4 & -5 \\
			0 & 0 & 0 & 0 & \frac{29}{3} & \frac{16}{3}
		\end{pmatrix} \\
		&\xrightarrow{R_4 + \frac{29}{12}R_3}
		\begin{pmatrix}
			1 & 2 & 3 & 6 & 5 & 6 \\
			0 & -3 & -3 & -6 & -10 & -11 \\
			0 & 0 & 0 & 0 & -4 & -5 \\
			0 & 0 & 0 & 0 & 0 & -\frac{27}{4}
		\end{pmatrix}\\
		&=
		\begin{pmatrix}
			1&0&1&2&0&0\\
			0&1&1&2&0&0\\
			0&0&0&0&1&0\\
			0&0&0&0&0&1
		\end{pmatrix}
	\end{align*}
	Thus, $\rank(A)=4$, and $\nullity(A)=$ columns without leading numbers =2. Then $\rank(A)+\nullity(A)=4+6=n=6.$
	\item Find full rank factorization of $A$.\\
	\textit{Solution.} We need to find a $4\times 4$ and $4\times6$ matrix. These are obtained from the above reduced form of $A$. The first matrix is composed of the independant columns of $A$. The second is the nonzero rows of 
	\[
		A=
		\begin{pmatrix}
			1&2&5&6\\
			2&1&0&1\\
			3&3&1&2\\
			6&1&3&1
		\end{pmatrix}
		\begin{pmatrix}
			1&0&1&2&0&0\\
			0&1&1&2&0&0\\
			0&0&0&0&1&0\\
			0&0&0&0&0&1
		\end{pmatrix}.
	\]
\end{enumerate}

\stepcounter{section}
\section*{Problem 4}
Find a non-diagonal matrix $A \in M_5(\mathbb{F})$, such that both $\rank(A - I)$ and $\rank(A + I)$ are smaller than 5.
\textit{Hint:} Consider similarity.\\
\textit{Solution.} From the conditions we see that $\rank(A-I)>5\Rightarrow \det(A-I)=0\Rightarrow$ 1 is an eigenvalue of $A$ and $\rank(A+I)>5\Rightarrow \det(A+I)=0\Rightarrow$ -1 is an eigenvalue of $A$. One such $A$ is
\[
	A=
	\begin{bmatrix}
		1&1&0&0&0\\
		0&1&0&0&0\\
		0&0&1&0&0\\
		0&0&0&-1&0\\
		0&0&0&0&-1
	\end{bmatrix}.
\]
Then
\[
	A+I=
	\begin{bmatrix}
		2&1&0&0&0\\
		0&2&0&0&0\\
		0&0&2&0&0\\
		0&0&0&0&0\\
		0&0&0&0&0
	\end{bmatrix},
\]
and
\[
	A-I=
	\begin{bmatrix}
		0&1&0&0&0\\
		0&0&0&0&0\\
		0&0&0&0&0\\
		0&0&0&-2&0\\
		0&0&0&0&-2
	\end{bmatrix}.
\]
$\rank(A+I)=3>5$ and $\rank(A-I)=3>5$.
\newpage
\stepcounter{section}
\section*{Problem 5}
Let $p(x) = x + 1$. Let $S = \spn\{x^4p(x), x^3p^2(x), x^2p^3(x), xp^4(x), p^5(x)\}$. Find dimension of $S$ in the space of all polynomials.\\
\textit{Solution.} 
\[
	x^4p(x)=x^5+x^4
\]
\[
	x^3p^2(x)=x^5+2x^4+x^3
\]
\[
	x^2p^3(x)=x^5+3x^4+3x^3+x^2
\]
\[
	xp^4(x)=x^5+4x^4+6x^3+4x^2+x
\]
\[
	p^5(x)=x^5+5x^4+10x^3+10x^2+5x+1
\]
Using the basis $\beta=1,x,x^2,x^3,x^4,x^5$ we get the following matrix with rows made up of the polynomials
\[
	\begin{bmatrix}
		0&0&0&0&1&1\\
		0&0&0&1&2&1\\
		0&0&1&3&3&1\\
		0&1&4&6&4&1\\
		1&5&10&10&5&1
	\end{bmatrix}.
\]
Since all five of these rows are linearly independant we have \[\dim(\spn\{x^4p(x), x^3p^2(x), x^2p^3(x), xp^4(x), p^5(x)\})=5.\]
\end{document}