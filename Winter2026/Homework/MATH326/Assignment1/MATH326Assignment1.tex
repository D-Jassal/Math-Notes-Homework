\documentclass[12pt]{article}

% Import preambles and macros for homework
% Essential packages
\usepackage{amsmath, amsfonts, amssymb, amsthm}
\usepackage{mathtools}
\usepackage{enumitem}
\usepackage{graphicx}
\usepackage{wrapfig}
\usepackage{systeme}
\usepackage{caption}
\usepackage{soul}
\usepackage[dvipsnames]{xcolor}
\usepackage{fancyhdr}
\allowdisplaybreaks

% Page layout
\usepackage[
  top=2cm,
  bottom=2cm,
  left=2cm,
  right=2cm,
  headheight=17pt,
  includehead,includefoot,
  heightrounded,
]{geometry}


% pgfornament for title page decorations
\usepackage[object=vectorian]{pgfornament}

% Fancy header/footer setup
\pagestyle{fancy}
\setlength{\headheight}{14.49998pt}
\addtolength{\topmargin}{-2.49998pt}
\renewcommand{\footrulewidth}{0.4pt}
\setlength\parindent{15pt}
% Math notation shortcuts
\newcommand{\R}{\mathbb{R}}
\newcommand{\Q}{\mathbb{Q}}
\newcommand{\Z}{\mathbb{Z}}
\newcommand{\N}{\mathbb{N}}
\newcommand{\C}{\mathbb{C}}
\newcommand{\X}{\mathcal{X}}

% Theorem environments
\newtheorem{mainthm}{Theorem}[section]
\newtheorem{theorem}{Theorem}[section]  
\newtheorem{lemma}[theorem]{Lemma}
\newtheorem{proposition}[theorem]{Proposition}
\newtheorem{corollary}[theorem]{Corollary}
\newtheorem{definition}[theorem]{Definition}
\newtheorem{claim}[theorem]{Claim}

% Calculus
\newcommand{\diff}{\mathop{}\!\mathrm{d}}
\newcommand{\deriv}[2]{\frac{\mathrm{d}#1}{\mathrm{d}#2}}
\newcommand{\pderiv}[2]{\frac{\partial #1}{\partial #2}}

% Linear Algebra
\newcommand{\inner}[2]{\langle #1, #2 \rangle}
\newcommand{\norm}[1]{\| #1 \|}
\newcommand{\tr}{\operatorname{tr}}
\newcommand{\spn}{\operatorname{span}}
\newcommand{\rank}{\operatorname{rank}}
\newcommand{\nullity}{\operatorname{nullity}}

% Logic
\newcommand{\contra}{\Rightarrow\Leftarrow}

% Custom commands for notes
\newcommand{\todo}[1]{\textcolor{red}{[TODO: #1]}}
\newcommand{\important}[1]{\textbf{\textcolor{blue}{#1}}}

%Number Theory
\DeclareMathOperator{\Li}{Li}
\newcommand{\floor}[1]{\left\lfloor #1 \right\rfloor}
\newcommand{\fract}[1]{\left\{ #1 \right\}}




\newcommand{\maketitlepage}{
    \begin{titlepage}
        \centering
        \vspace*{2.0cm}
        \pgfornament{84}\\
        {\LARGE \textsc{\coursename}\par}
        \vspace{0.5cm}
        {\large\coursecode\par}
        \vspace{0.5cm}
        {\large\instructor\par}
        \vspace{1.5cm}
        {\huge\bfseries\assignment\par}
        \vspace{1cm}
        {\LARGE\itshape\author\par}
        \vspace{2cm}
        {\large\bfseries Due Date:\par}
        \vspace{0.5cm}
        {\Large \duedate}\\
        \pgfornament{84}
    \end{titlepage}
}
% =============================================
% HOMEWORK CONFIGURATION - EDIT THESE VALUES!
% =============================================

% Your personal info
\renewcommand{\author}{Deepak Jassal}
\newcommand{\authorlast}{Jassal}

% Course info
\newcommand{\coursename}{Course Name}
\newcommand{\coursecode}{Course code}
\newcommand{\instructor}{Instructor}

% Assignment-specific info (CHANGE THESE FOR EACH HOMEWORK)
\newcommand{\assignment}{Assignment }
\newcommand{\duedate}{Month Day\textsuperscript{th}, 20XX}

% Header configuration
\fancyhead[l]{\assignment}
\fancyhead[c]{\coursecode}
\fancyhead[r]{\monthyear}
\fancyfoot[c]{\authorlast{ }\thepage}

\renewcommand{\author}{Deepak Jassal}
\renewcommand{\authorlast}{Jassal}
\renewcommand{\coursename}{Advanced Linear Algebra}
\renewcommand{\coursecode}{MATH 326}
\renewcommand{\assignment}{Assignment 1}
\renewcommand{\instructor}{Dr. Edward Dobrowolski}
\renewcommand{\duedate}{January 23\textsuperscript{th}, 2026}


\begin{document}
\begin{titlepage}
	\centering
	\vspace*{2.0cm}	
	\pgfornament{84}\\
	{\LARGE \textsc{\coursename}\par}
	\vspace{0.5cm}
	{\large\coursecode\par}
    \vspace{0.5cm}
    {\large\instructor\par}
	\vspace{1.5cm}
	{\huge\bfseries\assignment\par}
	\vspace{1cm}  
	{\LARGE\itshape\author\par}
    \vspace{2cm}
	{\large\bfseries Due Date:\par}
	\vspace{0.5cm}
	{\Large \duedate}\\
	\pgfornament{84}
\end{titlepage}
\stepcounter{section}
\section*{Problem 1}
Let $V = \R^2$. Define on $V$ addition of vectors by
\[
(x_1, x_2) \oplus (y_1, y_2) = \left( \sqrt[3]{x_1^3 + y_1^3 + 1},\ \sqrt[3]{x_2^3 + y_2^3 + 2} \right),
\]
and multiplication by scalar by
\[
c \odot (x_1, x_2) = \left( \sqrt[3]{cx_1^3 + c - 1},\ \sqrt[3]{cx_2^3 + 2c - 2} \right).
\]
Show that $V$ with addition of vectors $\oplus$ and multiplication by scalars $\odot$ is a vector space over $\R$.\\
For the following assume that $(x_1, x_2), (y_1, y_2), (w_1,w_2)\in V$ are arbitrary.
\begin{enumerate}
	\item Addition over reals is commutative so we have $(x_1, x_2) \oplus (y_1, y_2)=(y_1, y_2) \oplus (x_1, x_2)$;
	\item Addition over reals is associative so we have $((x_1, x_2) \oplus (y_1, y_2))\oplus (w_1,w_2)=(x_1, x_2) \oplus ((y_1, y_2)\oplus (w_1,w_2))$;
	\item We need $0_V=(a,b)$ with $(x_1,x_2)\oplus(a,b)=(x_1,x_2),\quad \forall(x_1,x_2)$.
	\[
		\sqrt[3]{x_1^3+a^3+1}=x_1\Rightarrow a^3+1=0\Rightarrow a=-1,
	\]
	\[
		\sqrt[3]{x_2^3+b^3+2}=x_2\Rightarrow b^3+2=0\Rightarrow b=-\sqrt[3]{2}.
	\]	
	Thus, $0_V=(-1,-\sqrt[3]{2})$.
	\item For any $v\in V$ with $v=(x_1,x_2)$ define $-v$ by $-v=(\sqrt[3]{-x_1^3-2},\sqrt[3]{-x_2^3-4})$. Then,
	\begin{align*}
		v\oplus(-v)&=\left( \sqrt[3]{x_1^3 + \left(\sqrt[3]{-x_1^3-2}\right)^3 + 1}, \sqrt[3]{x_2^3 + \left(\sqrt[3]{-x_2^3-4}\right)^3 + 2} \right)\\
		&=\left( \sqrt[3]{x_1^3 -x_1^3-2 + 1}, \sqrt[3]{x_2^3 + -x_2^3-4 + 2} \right)=(-1,-\sqrt[3]{2})=0_V.
	\end{align*}
	\item Let $u=(x_1, x_2)$, $v=(y_1, y_2)$ and $c\in\R$. Then 
	\begin{align*}
		c\odot(u\oplus v)&=c\left(\sqrt[3]{x_1^3 + y_1^3 + 1},\ \sqrt[3]{x_2^3 + y_2^3 + 2}\right)\\
		&=\left( \sqrt[3]{c\left(\sqrt[3]{x_1^3 + y_1^3 + 1}\right)^3 + c - 1},\ \sqrt[3]{c\left(\sqrt[3]{x_2^3 + y_2^3 + 2}\right)^3 + 2c - 2} \right)\\
		&=\left( \sqrt[3]{c\left(x_1^3 + y_1^3 + 1\right) + c - 1},\ \sqrt[3]{c\left(x_2^3 + y_2^3 + 2\right) + 2c - 2} \right)\\
		&=\left( \sqrt[3]{cx_1^3 + cy_1^3 + 2c - 1},\ \sqrt[3]{cx_2^3 + cy_2^3 + 4c - 2} \right)\\
		&=\left( \sqrt[3]{cx_1^3+c + cy_1^3 + 2c - 1},\ \sqrt[3]{cx_2^3 + cy_2^3 + 4c - 2} \right),		
	\end{align*}
	and
	\begin{align*}
		c\odot u\oplus c\odot v=&\left( \sqrt[3]{cx_1^3 + c - 1},\ \sqrt[3]{cx_2^3 + 2c - 2} \right)\oplus\left( \sqrt[3]{cy_1^3 + c - 1},\ \sqrt[3]{cy_2^3 + 2c - 2} \right)\\
		=&\bigg( \sqrt[3]{\left(\sqrt[3]{cx_1^3 + c - 1}\right)^3 + \left( \sqrt[3]{cy_1^3 + c - 1}\right)^3 + 1},\\
		&\sqrt[3]{\left(\sqrt[3]{cx_2^3 + 2c - 2}\right)^3 + \left(\sqrt[3]{cy_2^3 + 2c - 2}\right)^3 + 2} \bigg)\\
		=&\left(\sqrt[3]{cx_1^3 + c - 1 + cy_1^3 + c - 1 + 1}, \sqrt[3]{cx_2^3 + 2c - 2 + cy_2^3 + 2c - 2 + 2} \right)\\
		&=\left( \sqrt[3]{cx_1^3+c + cy_1^3 + 2c - 1},\ \sqrt[3]{cx_2^3 + cy_2^3 + 4c - 2} \right)=c\odot(u\oplus v).
	\end{align*}
	\item Let $u=(x_1, x_2)$, $a,b\in\R$ and $c=a+b$. Then,
	\begin{align*}
		(a+b)\odot u&=\left( \sqrt[3]{cx_1^3 + c - 1},\ \sqrt[3]{cx_2^3 + 2c - 2} \right)\\
		&=\left( \sqrt[3]{(a+b)x_1^3 + (a+b) - 1},\ \sqrt[3]{(a+b)x_2^3 + 2(a+b) - 2} \right),
	\end{align*}
	and
	\begin{align*}
		(a\cdot u)\oplus (b\cdot u)=&\left( \sqrt[3]{ax_1^3 + a - 1},\ \sqrt[3]{ax_2^3 + 2a - 2} \right)\odot \left( \sqrt[3]{bx_1^3 + b - 1},\ \sqrt[3]{bx_2^3 + 2b - 2} \right)\\
		=&\bigg( \sqrt[3]{\left(\sqrt[3]{ax_1^3 + a - 1}\right)^3 + \left(\sqrt[3]{bx_1^3 + b - 1}\right)^3 + 1},\\ 
		=&\sqrt[3]{\left(\sqrt[3]{ax_2^3 + 2a - 2}\right)^3 + \left(\sqrt[3]{bx_2^3 + 2b - 2}\right)^3 + 2} \bigg)\\
		=&\left( \sqrt[3]{ax_1^3 + a - 1 + bx_1^3 + b - 1 + 1},\sqrt[3]{ax_2^3 + 2a - 2 + bx_2^3 + 2b - 2 + 2} \right)\\
		=&\left( \sqrt[3]{(a+b)x_1^3 + (a+b) - 1},\ \sqrt[3]{(a+b)x_2^3 + 2(a+b) - 2} \right)=(a+b)\odot u.
	\end{align*}
	\item Let $u=(x_1, x_2)$, $a,b\in\R$, and $ab=c$. Then,
	\begin{align*}
		(ab)\odot u&=\left( \sqrt[3]{cx_1^3 + c - 1},\ \sqrt[3]{cx_2^3 + 2c - 2} \right)\\
		&=\left( \sqrt[3]{(ab)x_1^3 + (ab) - 1},\ \sqrt[3]{(ab)x_2^3 + 2(ab) - 2} \right),
	\end{align*}
	and
	\begin{align*}
		a\odot(b\odot u)=&a\odot \left( \sqrt[3]{bx_1^3 + b - 1},\ \sqrt[3]{bx_2^3 + 2b - 2} \right)\\
		&=\left( \sqrt[3]{a\left(\sqrt[3]{bx_1^3 + b - 1}\right)^3 + a - 1},\ \sqrt[3]{a\left(\sqrt[3]{bx_2^3 + 2b - 2}\right)^3 + 2a - 2} \right)\\
		&=\left( \sqrt[3]{a\left(bx_1^3 + b - 1\right) + a - 1},\ \sqrt[3]{a\left(bx_2^3 + 2b - 2\right) + 2a - 2} \right)\\
		&=\left( \sqrt[3]{(ab)x_1^3 + (ab) - a + a - 1},\ \sqrt[3]{(ab)x_2^3 + 2ab - 2a + 2a - 2} \right)\\
		&=\left( \sqrt[3]{(ab)x_1^3 + (ab) - 1},\ \sqrt[3]{(ab)x_2^3 + 2ab - 2} \right)=(ab)\odot u.
	\end{align*}
	\item Let $u=(x_1,x_2)$. Then,
	\begin{align*}
		1\odot u&=\left( \sqrt[3]{x_1^3 + 1 - 1},\ \sqrt[3]{x_2^3 + 2 - 2} \right)\\
		&=\left( \sqrt[3]{x_1^3 },\ \sqrt[3]{x_2^3}  \right)\\
		&=(x_1,x_2).
	\end{align*}
\end{enumerate}


\stepcounter{section}
\section*{Problem 2}
Let $U = \operatorname{span}\{(1, 2, 3, 4), (-1, 2, 0, 5), (-3, -6, -9, -12), (-1, 10, 6, 23), (1, 1, 1, 1)\}$. Find a basis of $U$ in two different ways:
\begin{enumerate}
    \item[(a)] By row reducing the matrix whose rows are the vectors listed in the span.\\
    Taking the transpose of the matrix $A$ from part 1 we obtain the following
	\[
	\mathrm{r.r.e.f}(A^T) = 
	\begin{bmatrix}
		1 & -1 & -3 & -1 & 1 \\
		2 & 2 & -6 & 10 & 1 \\
		3 & 0 & -9 & 6 & 1 \\
		4 & 5 & -12 & 23 & 1
	\end{bmatrix}=
	\begin{bmatrix}
		1 & 0 & -3 & 2 & 0 \\
		0 & 1 & 0 & 3 & 0 \\
		0 & 0 & 0 & 0 & 1 \\
		0 & 0 & 0 & 0 & 0
	\end{bmatrix}
	\]
	So the basis vectors are $\{(1, 2, 3, 4), (-1, 2, 0, 5), (1, 1, 1, 1)\}$
    \item[(b)] 
	By selecting the basis from among the spanning vectors of $U$.\\
    Writing out the vectors of $U$ in a matrix, say $A$ we obtain
	\[
	\mathrm{r.r.e.f}(A) = \mathrm{r.r.e.f}
	\begin{bmatrix}
		1 & 2 & 3 & 4 \\
		-1 & 2 & 0 & 5 \\
		-3 & -6 & -9 & -12 \\
		-1 & 10 & 6 & 23 \\
		1 & 1 & 1 & 1
	\end{bmatrix}
	= \begin{bmatrix}
		1 & 0 & 0 & -1.4 \\
		0 & 1 & 0 & 1.8 \\
		0 & 0 & 1 & 0.6 \\
		0 & 0 & 0 & 0 \\
		0 & 0 & 0 & 0
	\end{bmatrix}.
	\]
	By this method we have that the vectors $\{(1, 2, 3, 4), (-1, 2, 0, 5), (-3, -6, -9, -12)\}$ form a basis of $U$.
\end{enumerate}


\stepcounter{section}
\section*{Problem 3}
Complete the set $\{(1, 2, 3, 4), (2, 2, 3, 4)\}$ to a basis of $\R^4$.\\
We need to find two vectors that are linearly idependant with those in the above listed set, we can do so by computing the following determinant
\[
\begin{vmatrix}
	1&2&a&e\\
	2&2&b&f\\
	3&3&c&g\\
	4&4&d&h\\
\end{vmatrix}=2(ah-de)-2(bh-df)+2(bg-cf)-2(ag-ce).
\]
Now we need to pick values of $a,b,c,d,e,f,g,h$ such that the determinant is non-zero. Take $a=h=d=e=0$, $b=g=1$ and $c=f=2$. Then we get
\[
	2(0-0)-2(0-0)+2(1-4)-2(0-0)=-6.
\]
So the vectors $(0,1,2,0)$ and $(0,2,1,0)$ complete the set into a basis of $\R^4$.

\stepcounter{section}
\section*{Problem 4}
Let $V$ be the set of all $2 \times 2$ matrices with real entries of the form $v = \begin{pmatrix} 1 & v \\ 0 & 1 \end{pmatrix}$. Define vector addition by
\[
v + w = \begin{pmatrix} 1 & v \\ 0 & 1 \end{pmatrix} \begin{pmatrix} 1 & w \\ 0 & 1 \end{pmatrix}
\]
and scalar multiplication by $cv = \begin{pmatrix} 1 & cv \\ 0 & 1 \end{pmatrix}$. Show that $V$ together with these operations is a real vector space. In particular state what is zero and $-v$ in this space\\
\textit{Solution.}
\[
	\begin{pmatrix} 1 & v \\ 0 & 1 \end{pmatrix} \begin{pmatrix} 1 & w \\ 0 & 1 \end{pmatrix}=\begin{pmatrix} 1 & v+w \\ 0 & 1 \end{pmatrix}.
\]
Each of the vectors is identified with its upper-right entry $a \in \mathbb{R}$. For the following assume that $v,w,u\in V$ are arbitrary.
\begin{enumerate}
	\item Addition over reals is commutative so we have $v+w=w+v$.
	\item Addition over reals is associative so we have $(v+w)+u=v+(w+u)$.
	\item From our defined notation we have that $0_V=\begin{pmatrix}1&0\\0&1\end{pmatrix}=0$. 
	\[
		v+0=\begin{pmatrix}1&v\\0&1\end{pmatrix}\begin{pmatrix}1&0\\0&1\end{pmatrix}=\begin{pmatrix}1&v+0\\0&1\end{pmatrix}=\begin{pmatrix}1&v\\0&1\end{pmatrix}=v,
	\]
	$0+v=v$ si given by commutativity.
	\item For $v=\begin{pmatrix}1&v\\0&1\end{pmatrix}$ define $-v$ as $-v=\begin{pmatrix}1&-v\\0&1\end{pmatrix}$.
	\[
		v+(-v)=\begin{pmatrix}1&v\\0&1\end{pmatrix}\begin{pmatrix}1&-v\\0&1\end{pmatrix}=\begin{pmatrix}1&v-v\\0&1\end{pmatrix}=\begin{pmatrix}1&0\\0&1\end{pmatrix}=0,
	\]
	$-v+v=0$ is given by commutativity.
	\item Let $c\in\R$. Then
	\begin{align*}
		c(v+w)&=c\left(\begin{pmatrix}1&v\\0&1\end{pmatrix}\begin{pmatrix}1&w\\0&1\end{pmatrix}\right)\\
		&=c\left(\begin{pmatrix}1&v+w\\0&1\end{pmatrix}\right)\\
		&=\begin{pmatrix}1&c(v+w)\\0&1\end{pmatrix}\\
		&=\begin{pmatrix}1&cv+cw\\0&1\end{pmatrix}\\
		&=\begin{pmatrix}1&cv\\0&1\end{pmatrix}\begin{pmatrix}1&cw\\0&1\end{pmatrix}\\
		&=cv+cw.
	\end{align*}
	\item Let $a,b\in\R$ and $(a+b)=c$. Then,
	\begin{align*}
		(a+b)v&=cv\\
		&=\begin{pmatrix}1&cv\\0&1\end{pmatrix}\\
		&=\begin{pmatrix}1&(a+b)v\\0&1\end{pmatrix}\\
		&=\begin{pmatrix}1&av+bv\\0&1\end{pmatrix}\\
		&=av+bv.
	\end{align*}
	\item Let $a,b\in\R$ and $(ab)=c$. Then,
	\begin{align*}
		(ab)u&=cv\\
		&=\begin{pmatrix}1&cv\\0&1\end{pmatrix}\\
		&=\begin{pmatrix}1&a(bv)\\0&1\end{pmatrix}\\
		&=a\begin{pmatrix}1&bv\\0&1\end{pmatrix}\\
		&=a(bv).
	\end{align*}
	\item Given that $1\in\R$ we have
	\begin{align*}
		1v=1\begin{pmatrix}1&v\\0&1\end{pmatrix}=\begin{pmatrix}1&1v\\0&1\end{pmatrix}=\begin{pmatrix}1&v\\0&1\end{pmatrix}=v.
	\end{align*}
\end{enumerate}

\stepcounter{section}
\section*{Problem 5}
Suppose that $A$ is a $2 \times 2$ invertible matrix with real entries. Show that $U = \operatorname{span}\{A^n : n \in \Z\} \neq M_{2\times 2}(\R)$.

\noindent\textbf{Hint:} Let $B \in U$. Do we have $AB = BA$? If so...\\
\begin{proof}
	An element $B\in U$ looks like
	\[
		B=\sum_{n}c_nA^n
	\]
	where $c_n$ is a finite coefficient determined by the matrix in $U$ in question. We also have
	\[
		AB=A\sum_{n}c_nA^{n+1},
	\]
	\[
		BA=\left(\sum_{n}c_nA^n\right)A=\sum_{n}c_nA^{n+1}.
	\]
	Since $AB=BA$ we have that 
	\[
		U\subseteq \{X\in M_{2\times 2}(\R):XA=AX\}.
	\]
	That is, $U$ is a subset of the centralizer of $A$. The only way the centralizer of $A$ is equal to $M_{2\times 2}(\R)$ is if it is a scalar multiple of $I_2$. Assume that $A\neq cI_n$, then $C(A)\neq M_{n\times n}$. Now suppose that $A=cI_2$ for some $c\in\R$, then $U$ would be the span of the identity matrix. If htis were the case then the matrix
	\[
		\begin{bmatrix}
			1&2\\
			3&4
		\end{bmatrix},
	\]
	does not appear in $U$.
\end{proof} 
\end{document}