\documentclass[12pt]{article}

% Import preambles and macros for homework
% Essential packages
\usepackage{amsmath, amsfonts, amssymb, amsthm}
\usepackage{mathtools}
\usepackage{enumitem}
\usepackage{graphicx}
\usepackage{wrapfig}
\usepackage{systeme}
\usepackage{caption}
\usepackage{soul}
\usepackage[dvipsnames]{xcolor}
\usepackage{fancyhdr}
\allowdisplaybreaks

% Page layout
\usepackage[
  top=2cm,
  bottom=2cm,
  left=2cm,
  right=2cm,
  headheight=17pt,
  includehead,includefoot,
  heightrounded,
]{geometry}


% pgfornament for title page decorations
\usepackage[object=vectorian]{pgfornament}

% Fancy header/footer setup
\pagestyle{fancy}
\setlength{\headheight}{14.49998pt}
\addtolength{\topmargin}{-2.49998pt}
\renewcommand{\footrulewidth}{0.4pt}
\setlength\parindent{15pt}
% Math notation shortcuts
\newcommand{\R}{\mathbb{R}}
\newcommand{\Q}{\mathbb{Q}}
\newcommand{\Z}{\mathbb{Z}}
\newcommand{\N}{\mathbb{N}}
\newcommand{\C}{\mathbb{C}}
\newcommand{\X}{\mathcal{X}}

% Theorem environments
\newtheorem{mainthm}{Theorem}[section]
\newtheorem{theorem}{Theorem}[section]  
\newtheorem{lemma}[theorem]{Lemma}
\newtheorem{proposition}[theorem]{Proposition}
\newtheorem{corollary}[theorem]{Corollary}
\newtheorem{definition}[theorem]{Definition}
\newtheorem{claim}[theorem]{Claim}

% Calculus
\newcommand{\diff}{\mathop{}\!\mathrm{d}}
\newcommand{\deriv}[2]{\frac{\mathrm{d}#1}{\mathrm{d}#2}}
\newcommand{\pderiv}[2]{\frac{\partial #1}{\partial #2}}

% Linear Algebra
\newcommand{\inner}[2]{\langle #1, #2 \rangle}
\newcommand{\norm}[1]{\| #1 \|}
\newcommand{\tr}{\operatorname{tr}}
\newcommand{\spn}{\operatorname{span}}
\newcommand{\rank}{\operatorname{rank}}
\newcommand{\nullity}{\operatorname{nullity}}

% Logic
\newcommand{\contra}{\Rightarrow\Leftarrow}

% Custom commands for notes
\newcommand{\todo}[1]{\textcolor{red}{[TODO: #1]}}
\newcommand{\important}[1]{\textbf{\textcolor{blue}{#1}}}

%Number Theory
\DeclareMathOperator{\Li}{Li}
\newcommand{\floor}[1]{\left\lfloor #1 \right\rfloor}
\newcommand{\fract}[1]{\left\{ #1 \right\}}




\newcommand{\maketitlepage}{
    \begin{titlepage}
        \centering
        \vspace*{2.0cm}
        \pgfornament{84}\\
        {\LARGE \textsc{\coursename}\par}
        \vspace{0.5cm}
        {\large\coursecode\par}
        \vspace{0.5cm}
        {\large\instructor\par}
        \vspace{1.5cm}
        {\huge\bfseries\assignment\par}
        \vspace{1cm}
        {\LARGE\itshape\author\par}
        \vspace{2cm}
        {\large\bfseries Due Date:\par}
        \vspace{0.5cm}
        {\Large \duedate}\\
        \pgfornament{84}
    \end{titlepage}
}
% =============================================
% HOMEWORK CONFIGURATION - EDIT THESE VALUES!
% =============================================

% Your personal info
\renewcommand{\author}{Deepak Jassal}
\newcommand{\authorlast}{Jassal}

% Course info
\newcommand{\coursename}{Course Name}
\newcommand{\coursecode}{Course code}
\newcommand{\instructor}{Instructor}

% Assignment-specific info (CHANGE THESE FOR EACH HOMEWORK)
\newcommand{\assignment}{Assignment }
\newcommand{\duedate}{Month Day\textsuperscript{th}, 20XX}

% Header configuration
\fancyhead[l]{\assignment}
\fancyhead[c]{\coursecode}
\fancyhead[r]{\monthyear}
\fancyfoot[c]{\authorlast{ }\thepage}

\renewcommand{\author}{Deepak Jassal}
\renewcommand{\authorlast}{Jassal}
\renewcommand{\coursename}{Advanced Linear Algebra}
\renewcommand{\coursecode}{MATH 326}
\renewcommand{\assignment}{Assignment 1}
\renewcommand{\instructor}{Dr. Edward Dobrowolski}
\renewcommand{\duedate}{January 23\textsuperscript{th}, 2026}


\begin{document}
\begin{titlepage}
	\centering
	\vspace*{2.0cm}	
	\pgfornament{84}\\
	{\LARGE \textsc{\coursename}\par}
	\vspace{0.5cm}
	{\large\coursecode\par}
    \vspace{0.5cm}
    {\large\instructor\par}
	\vspace{1.5cm}
	{\huge\bfseries\assignment\par}
	\vspace{1cm}  
	{\LARGE\itshape\author\par}
    \vspace{2cm}
	{\large\bfseries Due Date:\par}
	\vspace{0.5cm}
	{\Large \duedate}\\
	\pgfornament{84}
\end{titlepage}
\stepcounter{section}
\section*{Problem 1}
Let $V = \R^2$. Define on $V$ addition of vectors by
\[
(x_1, x_2) \oplus (y_1, y_2) = \left( \sqrt[3]{x_1^3 + y_1^3 + 1},\ \sqrt[3]{x_2^3 + y_2^3 + 2} \right),
\]
and multiplication by scalar by
\[
c \odot (x_1, x_2) = \left( \sqrt[3]{cx_1^3 + c - 1},\ \sqrt[3]{cx_2^3 + 2c - 2} \right).
\]
Show that $V$ with addition of vectors $\oplus$ and multiplication by scalars $\odot$ is a vector space over $\R$.


Closure\\
Both $\oplus$ and $\odot$ output real numbers, so closure holds.

Commutativity of addition\\
Since $x_1^3 + y_1^3 = y_1^3 + x_1^3$, we have $(x_1, x_2) \oplus (y_1, y_2) = (y_1, y_2) \oplus (x_1, x_2)$.

Associativity of addition\\
Let $u = (x_1, x_2)$, $v = (y_1, y_2)$, $w = (z_1, z_2)$. Then
\[
(u \oplus v) \oplus w = \left( \sqrt[3]{x_1^3 + y_1^3 + z_1^3 + 2},\ \sqrt[3]{x_2^3 + y_2^3 + z_2^3 + 4} \right)
\]
and
\[
u \oplus (v \oplus w) = \left( \sqrt[3]{x_1^3 + y_1^3 + z_1^3 + 2},\ \sqrt[3]{x_2^3 + y_2^3 + z_2^3 + 4} \right),
\]
so associativity holds.

Additive identity\\
Let $0_V = (a, b)$. We need
\[
(x_1, x_2) \oplus (a, b) = (x_1, x_2) \quad \forall (x_1, x_2).
\]
\[\sqrt[3]{x_1^3 + a^3 + 1} = x_1 \Rightarrow a^3 + 1 = 0 \Rightarrow a = -1.\]
\[\sqrt[3]{x_2^3 + b^3 + 2} = x_2 \Rightarrow b^3 + 2 = 0 \Rightarrow b = -\sqrt[3]{2}.\]
Thus $0_V = \left(-1, -\sqrt[3]{2}\right)$. 
\[
(x_1, x_2) \oplus (-1, -\sqrt[3]{2}) = (x_1, x_2).
\]

Additive inverses\\
For $u = (x_1, x_2)$, define $v = (y_1, y_2)$ by
\[
y_1 = \sqrt[3]{-x_1^3 - 2}, \quad y_2 = \sqrt[3]{-x_2^3 - 4}.
\]
Then $u \oplus v = 0_V$.

Scalar multiplication\\
Let $u = (x_1, x_2)$, $c, d \in \mathbb{R}$. Then
\[
(cd) \odot u = \left( \sqrt[3]{cd\, x_1^3 + cd - 1},\ \sqrt[3]{cd\, x_2^3 + 2cd - 2} \right),
\]
\[
c \odot (d \odot u) = \left( \sqrt[3]{c(d x_1^3 + d - 1) + c - 1},\ \sqrt[3]{c(d x_2^3 + 2d - 2) + 2c - 2} \right) = (cd) \odot u.
\]

Scalar multiplicative identity\\
\[
1 \odot (x_1, x_2) = \left( \sqrt[3]{x_1^3 + 0},\ \sqrt[3]{x_2^3 + 0} \right) = (x_1, x_2). 
\]

Distributivity over vector addition\\
For $c \in \mathbb{R}$, $u = (x_1, x_2)$, $v = (y_1, y_2)$,
\[
c \odot (u \oplus v) = \left( \sqrt[3]{c(x_1^3 + y_1^3 + 1) + c - 1},\ \sqrt[3]{c(x_2^3 + y_2^3 + 2) + 2c - 2} \right),
\]
\[
(c \odot u) \oplus (c \odot v) = \left( \sqrt[3]{c x_1^3 + c y_1^3 + 2c - 1},\ \sqrt[3]{c x_2^3 + c y_2^3 + 4c - 2} \right),
\]
which match.

Distributivity over scalar addition\\
For $c, d \in \mathbb{R}$, $u = (x_1, x_2)$,
\[
(c+d) \odot u = \left( \sqrt[3]{(c+d)x_1^3 + (c+d) - 1},\ \sqrt[3]{(c+d)x_2^3 + 2(c+d) - 2} \right),
\]
\[
(c \odot u) \oplus (d \odot u) = \left( \sqrt[3]{c x_1^3 + d x_1^3 + c + d - 1},\ \sqrt[3]{c x_2^3 + d x_2^3 + 2c + 2d - 2} \right),
\]
which are equivalent.

All axioms are satisfied, so $(V, \oplus, \odot)$ is a vector space over $\mathbb{R}$.

\stepcounter{section}
\section*{Problem 2}
Let $U = \operatorname{span}\{(1, 2, 3, 4), (-1, 2, 0, 5), (-3, -6, -9, -12), (-1, 10, 6, 23), (1, 1, 1, 1)\}$. Find a basis of $U$ in two different ways:
\begin{enumerate}
    \item[(a)] By selecting the basis from among the spanning vectors of $U$.
    \item[(b)] By row reducing the matrix whose rows are the vectors listed in the span.
\end{enumerate}


\stepcounter{section}
\section*{Problem 3}
Complete the set $\{(1, 2, 3, 4), (2, 2, 3, 4)\}$ to a basis of $\R^4$.



\stepcounter{section}
\section*{Problem 4}
Let $V$ be the set of all $2 \times 2$ matrices with real entries of the form $v = \begin{pmatrix} 1 & v \\ 0 & 1 \end{pmatrix}$. Define vector addition by
\[
v + w = \begin{pmatrix} 1 & v \\ 0 & 1 \end{pmatrix} \begin{pmatrix} 1 & w \\ 0 & 1 \end{pmatrix}
\]
and scalar multiplication by $cv = \begin{pmatrix} 1 & cv \\ 0 & 1 \end{pmatrix}$. Show that $V$ together with these operations is a real vector space. In particular state what is zero and $-v$ in this space.



\stepcounter{section}
\section*{Problem 5}
Suppose that $A$ is a $2 \times 2$ invertible matrix with real entries. Show that $U = \operatorname{span}\{A^n : n \in \Z\} \neq M_{2\times 2}(\R)$.

\noindent\textbf{Hint:} Let $B \in U$. Do we have $AB = BA$? If so...\\
\textit{Solution.} Notice that $M_{2\times 2}(\R)$ contains matrices that are not invertible, that is matrices with determinant equal to 0. Since it is given that $\det A\neq 0$, by the following property of determinants 
\[
	\det(AB)=\det(A)\det(B),
\] 
we have that there can be no matrix in $U$ with determinant equal to zero. Therefor, $U\neq M_{2\times 2}(\R)$.
\end{document}