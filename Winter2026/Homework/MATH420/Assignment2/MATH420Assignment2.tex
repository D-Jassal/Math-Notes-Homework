\documentclass[12pt]{article}

% Import preambles and macros for homework
% Essential packages
\usepackage{amsmath, amsfonts, amssymb, amsthm}
\usepackage{mathtools}
\usepackage{enumitem}
\usepackage{graphicx}
\usepackage{wrapfig}
\usepackage{systeme}
\usepackage{caption}
\usepackage{soul}
\usepackage[dvipsnames]{xcolor}
\usepackage{fancyhdr}
\allowdisplaybreaks

% Page layout
\usepackage[
  top=2cm,
  bottom=2cm,
  left=2cm,
  right=2cm,
  headheight=17pt,
  includehead,includefoot,
  heightrounded,
]{geometry}


% pgfornament for title page decorations
\usepackage[object=vectorian]{pgfornament}

% Fancy header/footer setup
\pagestyle{fancy}
\setlength{\headheight}{14.49998pt}
\addtolength{\topmargin}{-2.49998pt}
\renewcommand{\footrulewidth}{0.4pt}
\setlength\parindent{15pt}
% Math notation shortcuts
\newcommand{\R}{\mathbb{R}}
\newcommand{\Q}{\mathbb{Q}}
\newcommand{\Z}{\mathbb{Z}}
\newcommand{\N}{\mathbb{N}}
\newcommand{\C}{\mathbb{C}}
\newcommand{\X}{\mathcal{X}}

% Theorem environments
\newtheorem{mainthm}{Theorem}[section]
\newtheorem{theorem}{Theorem}[section]  
\newtheorem{lemma}[theorem]{Lemma}
\newtheorem{proposition}[theorem]{Proposition}
\newtheorem{corollary}[theorem]{Corollary}
\newtheorem{definition}[theorem]{Definition}
\newtheorem{claim}[theorem]{Claim}

% Calculus
\newcommand{\diff}{\mathop{}\!\mathrm{d}}
\newcommand{\deriv}[2]{\frac{\mathrm{d}#1}{\mathrm{d}#2}}
\newcommand{\pderiv}[2]{\frac{\partial #1}{\partial #2}}

% Linear Algebra
\newcommand{\inner}[2]{\langle #1, #2 \rangle}
\newcommand{\norm}[1]{\| #1 \|}
\newcommand{\tr}{\operatorname{tr}}
\newcommand{\spn}{\operatorname{span}}
\newcommand{\rank}{\operatorname{rank}}
\newcommand{\nullity}{\operatorname{nullity}}

% Logic
\newcommand{\contra}{\Rightarrow\Leftarrow}

% Custom commands for notes
\newcommand{\todo}[1]{\textcolor{red}{[TODO: #1]}}
\newcommand{\important}[1]{\textbf{\textcolor{blue}{#1}}}

%Number Theory
\DeclareMathOperator{\Li}{Li}
\newcommand{\floor}[1]{\left\lfloor #1 \right\rfloor}
\newcommand{\fract}[1]{\left\{ #1 \right\}}




\newcommand{\maketitlepage}{
    \begin{titlepage}
        \centering
        \vspace*{2.0cm}
        \pgfornament{84}\\
        {\LARGE \textsc{\coursename}\par}
        \vspace{0.5cm}
        {\large\coursecode\par}
        \vspace{0.5cm}
        {\large\instructor\par}
        \vspace{1.5cm}
        {\huge\bfseries\assignment\par}
        \vspace{1cm}
        {\LARGE\itshape\author\par}
        \vspace{2cm}
        {\large\bfseries Due Date:\par}
        \vspace{0.5cm}
        {\Large \duedate}\\
        \pgfornament{84}
    \end{titlepage}
}
% =============================================
% HOMEWORK CONFIGURATION - EDIT THESE VALUES!
% =============================================

% Your personal info
\renewcommand{\author}{Deepak Jassal}
\newcommand{\authorlast}{Jassal}

% Course info
\newcommand{\coursename}{Course Name}
\newcommand{\coursecode}{Course code}
\newcommand{\instructor}{Instructor}

% Assignment-specific info (CHANGE THESE FOR EACH HOMEWORK)
\newcommand{\assignment}{Assignment }
\newcommand{\duedate}{Month Day\textsuperscript{th}, 20XX}

% Header configuration
\fancyhead[l]{\assignment}
\fancyhead[c]{\coursecode}
\fancyhead[r]{\monthyear}
\fancyfoot[c]{\authorlast{ }\thepage}

\renewcommand{\author}{Deepak Jassal}
\renewcommand{\authorlast}{Jassal}
\renewcommand{\coursename}{Structure of Groups and Rings}
\renewcommand{\coursecode}{MATH 420}
\renewcommand{\assignment}{Assignment 2}
\renewcommand{\instructor}{Dr. Stanley Yao Xiao}
\renewcommand{\duedate}{February 22\textsuperscript{nd}, 2026}


\begin{document}
\begin{titlepage}
	\centering
	\vspace*{2.0cm}	
	\pgfornament{84}\\
	{\LARGE \textsc{\coursename}\par}
	\vspace{0.5cm}
	{\large\coursecode\par}
    \vspace{0.5cm}
    {\large\instructor\par}
	\vspace{1.5cm}
	{\huge\bfseries\assignment\par}
	\vspace{1cm}  
	{\LARGE\itshape\author\par}
    \vspace{2cm}
	{\large\bfseries Due Date:\par}
	\vspace{0.5cm}
	{\Large \duedate}\\
	\pgfornament{84}
\end{titlepage}
\stepcounter{section}
\section*{Problem 1}
\begin{enumerate}[label=(\alph*)]
    \item Let $R$ be a ring. Suppose that $I \subseteq R$ is an ideal. Prove that if $I$ contains a unit, then $I = R$.
    \begin{proof}
        Since we have that $I$ contains a unit, that means that there exists a $u\in I$ such that $u\cdot u^{-1}=1_R$, where $u^{-1}$ is the multiplicative inverse of $u$. Since $I$ is an ideal and we have $u\in I$ we now have that $1_R\in I$. Furthermore, by the same property of ideals we have that for all $r\in R$, we have $1_R\cdot r=r\in I$. Thus, if an ideal contains a unit, the ideal is the entire ring.
    \end{proof}
    \item Prove that a field $F$ has no proper ideals (that is, the only ideals are the trivial ideal $\{0\}$ and $F$).
    \begin{proof}
        Assume that $I$ is an ideal of a field $F$ such that $I\neq\{0\}$. Since we know that all non-zero elements of a field have a multiplicative inverse for $u\in I$ we have $u\cdot u^{-1}=1_F\in I$. Then by the property of ideals we have that for all $f\in F$, we have $1_F\cdot f=f\in I$. Thus, $I=F$. Therefore the only non-zero ideals in a field $F$ are the entire field. 
    \end{proof}
\end{enumerate}

\stepcounter{section}
\section*{Problem 2}For each of the following statements, either prove that it is true, or provide a counterexample. In each case $R$ is a ring, with addition and multiplication understood.
\begin{enumerate}[label=(\alph*)]
    \item If $R$ is an infinite ring and $I$ is an ideal of $R$, then $I$ is either trivial or infinite.\\
    \textit{Counterexample.} Let $R=\Z_2\times\Z$, then $I=\Z_2\times\{0\}$ is a ifnite ideal of $R$.
    \item If $R$ is an infinite ring, and $I$ is an ideal of $R$, then $R/I$ is infinite.\\
    \textit{Counterexample.} Let $R=\Z_2\times\Z$ and $I=\{0\}\times\Z$ then $R/I\cong\Z_2$, which is finite. 
    \item There exists a ring $R$ and proper, non-trivial ideals $I$, $J$ of $R$ such that $R/I$ is infinite and $R/J$ is finite.
    \begin{proof}
        Let $R=\Z_2\times\Z$, $I=\{0\}\times \Z_2$, and $J=\Z\times\{0\}$. $R/I$ is infinite and $R/I$ is finite.
    \end{proof}
\end{enumerate}

\stepcounter{section}
\section*{Problem 3}Let $R$ be a ring. An ideal $I \subseteq R$ is said to be irreducible if whenever $A$, $B$ are ideals of $R$ such that $I = A \cap B$, then $I = A$ or $I = B$.
\begin{enumerate}[label=(\alph*)]
    \item Prove that every prime ideal $I$ of $R$ is irreducible.
    \begin{proof}
        Assume that $I$ is a prime ideal of $R$ and that $A,B$ are ideals of $R$. Assume for the sake of a contradiction that $I=A\cap B$ but $I\neq A$ and $I\neq B$. Now pick $a\in A/I$ and $b\in B/I$. Since $A,B$ are ideals we have $ab\in A$ and $ab\in B$, then we have $ab\in A\cap B=I$. Since $I$ is prime this implies that $a\in I$ or $b\in I$. But since $a\in A/I$ and $b\in B/I$ this implies that $a\neq I$ and $b\neq I$. This is a contradiction. Thus all prime ideals are irreducible.
    \end{proof}
    \item Is it necessarily true that an irreducible ideal is a prime ideal? Prove or find a counterexample.\\
    \textit{Counterexample.} Let $R=\Z$, and $I=(4)$.
\end{enumerate}

\stepcounter{section}
\section*{Problem 4}We say a ring $R$ has $\Z$-rank $r$ if the group $(R, +)$ is isomorphic to the group $(\Z^r, +)$. Let $S$ be the ring of $2 \times 2$ matrices over the integers $\Z$.
\begin{enumerate}[label=(\alph*)]
    \item Prove that $S$ has $\Z$-rank 4.
    \item Determine all subrings of $S$ of $\Z$-rank equal to 3. Are they isomorphic?
    \item Find two subrings of $S$ of $\Z$-rank 1 which are not isomorphic.
\end{enumerate}

\stepcounter{section}
\section*{Problem 5}
\begin{enumerate}[label=(\alph*)]
    \item Let $A \in M_{n\times n}(\C)$ be a nilpotent matrix. That is, we have $A^n$ is the zero-matrix. Prove that $I_{n\times n} + A$ is a unit in $M_{n\times n}(\C)$.
    \item Prove, more generally, that if $R$ is a unital ring and $x \in R$ is nilpotent, i.e., there exists an integer $n \geq 1$ such that $x^n = 0$, then $1 + x$ is a unit in $R$.
\end{enumerate}

\stepcounter{section}
\section*{Problem 6}A ring $R$ is said to be a division ring if every non-zero element of $R$ has a multiplicative inverse. A commutative division ring is evidently a field.
\begin{enumerate}[label=(\alph*)]
    \item Let $R$ be a finite division ring. Prove that $R$ must in fact be a field.
    \item Prove that the finiteness assumption is essential: find an example of an infinite division ring $R$ which is not a field.
\end{enumerate}



\end{document}