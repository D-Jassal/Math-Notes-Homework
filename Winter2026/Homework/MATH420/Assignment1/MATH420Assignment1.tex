\documentclass[12pt]{article}

% Import preambles and macros for homework
% Essential packages
\usepackage{amsmath, amsfonts, amssymb, amsthm}
\usepackage{mathtools}
\usepackage{enumitem}
\usepackage{graphicx}
\usepackage{wrapfig}
\usepackage{systeme}
\usepackage{caption}
\usepackage{soul}
\usepackage[dvipsnames]{xcolor}
\usepackage{fancyhdr}
\allowdisplaybreaks

% Page layout
\usepackage[
  top=2cm,
  bottom=2cm,
  left=2cm,
  right=2cm,
  headheight=17pt,
  includehead,includefoot,
  heightrounded,
]{geometry}


% pgfornament for title page decorations
\usepackage[object=vectorian]{pgfornament}

% Fancy header/footer setup
\pagestyle{fancy}
\setlength{\headheight}{14.49998pt}
\addtolength{\topmargin}{-2.49998pt}
\renewcommand{\footrulewidth}{0.4pt}
\setlength\parindent{15pt}
% Math notation shortcuts
\newcommand{\R}{\mathbb{R}}
\newcommand{\Q}{\mathbb{Q}}
\newcommand{\Z}{\mathbb{Z}}
\newcommand{\N}{\mathbb{N}}
\newcommand{\C}{\mathbb{C}}
\newcommand{\X}{\mathcal{X}}

% Theorem environments
\newtheorem{mainthm}{Theorem}[section]
\newtheorem{theorem}{Theorem}[section]  
\newtheorem{lemma}[theorem]{Lemma}
\newtheorem{proposition}[theorem]{Proposition}
\newtheorem{corollary}[theorem]{Corollary}
\newtheorem{definition}[theorem]{Definition}
\newtheorem{claim}[theorem]{Claim}

% Calculus
\newcommand{\diff}{\mathop{}\!\mathrm{d}}
\newcommand{\deriv}[2]{\frac{\mathrm{d}#1}{\mathrm{d}#2}}
\newcommand{\pderiv}[2]{\frac{\partial #1}{\partial #2}}

% Linear Algebra
\newcommand{\inner}[2]{\langle #1, #2 \rangle}
\newcommand{\norm}[1]{\| #1 \|}
\newcommand{\tr}{\operatorname{tr}}
\newcommand{\spn}{\operatorname{span}}
\newcommand{\rank}{\operatorname{rank}}
\newcommand{\nullity}{\operatorname{nullity}}

% Logic
\newcommand{\contra}{\Rightarrow\Leftarrow}

% Custom commands for notes
\newcommand{\todo}[1]{\textcolor{red}{[TODO: #1]}}
\newcommand{\important}[1]{\textbf{\textcolor{blue}{#1}}}

%Number Theory
\DeclareMathOperator{\Li}{Li}
\newcommand{\floor}[1]{\left\lfloor #1 \right\rfloor}
\newcommand{\fract}[1]{\left\{ #1 \right\}}




\newcommand{\maketitlepage}{
    \begin{titlepage}
        \centering
        \vspace*{2.0cm}
        \pgfornament{84}\\
        {\LARGE \textsc{\coursename}\par}
        \vspace{0.5cm}
        {\large\coursecode\par}
        \vspace{0.5cm}
        {\large\instructor\par}
        \vspace{1.5cm}
        {\huge\bfseries\assignment\par}
        \vspace{1cm}
        {\LARGE\itshape\author\par}
        \vspace{2cm}
        {\large\bfseries Due Date:\par}
        \vspace{0.5cm}
        {\Large \duedate}\\
        \pgfornament{84}
    \end{titlepage}
}
% =============================================
% HOMEWORK CONFIGURATION - EDIT THESE VALUES!
% =============================================

% Your personal info
\renewcommand{\author}{Deepak Jassal}
\newcommand{\authorlast}{Jassal}

% Course info
\newcommand{\coursename}{Course Name}
\newcommand{\coursecode}{Course code}
\newcommand{\instructor}{Instructor}

% Assignment-specific info (CHANGE THESE FOR EACH HOMEWORK)
\newcommand{\assignment}{Assignment }
\newcommand{\duedate}{Month Day\textsuperscript{th}, 20XX}

% Header configuration
\fancyhead[l]{\assignment}
\fancyhead[c]{\coursecode}
\fancyhead[r]{\monthyear}
\fancyfoot[c]{\authorlast{ }\thepage}

\renewcommand{\author}{Deepak Jassal}
\renewcommand{\authorlast}{Jassal}
\renewcommand{\coursename}{Structure of Groups and Rings}
\renewcommand{\coursecode}{MATH 420}
\renewcommand{\assignment}{Assignment 1}
\renewcommand{\instructor}{Dr. Stanley Yao Xiao}
\renewcommand{\duedate}{February 1\textsuperscript{st}, 2026}


\begin{document}
\begin{titlepage}
	\centering
	\vspace*{2.0cm}	
	\pgfornament{84}\\
	{\LARGE \textsc{\coursename}\par}
	\vspace{0.5cm}
	{\large\coursecode\par}
    \vspace{0.5cm}
    {\large\instructor\par}
	\vspace{1.5cm}
	{\huge\bfseries\assignment\par}
	\vspace{1cm}  
	{\LARGE\itshape\author\par}
    \vspace{2cm}
	{\large\bfseries Due Date:\par}
	\vspace{0.5cm}
	{\Large \duedate}\\
	\pgfornament{84}
\end{titlepage}
\stepcounter{section}
\section*{Problem 1}
\begin{enumerate}[label=(\alph*)]
	\item Let \((\mathbb{Z}, + ,\times)\) be the ring of integers with the standard arithmetic operations. Prove that every subring of \(\mathbb{Z}\) is also an ideal.
	\item Consider the polynomial ring \(\mathbb{R}[x]\) . Prove that every subring of \(\mathbb{R}[x]\) which forms an \(\mathbb{R}\) - vector space and is not equal to \(\mathbb{R}\) is an ideal of \(\mathbb{R}[x]\).
\end{enumerate}


\stepcounter{section}
\section*{Problem 2}

For each of the following statements, either prove that it is true, or provide a counterexample. In each case \(R\) is a ring, with addition and multiplication understood.
\begin{enumerate}[label=(\alph*)]
	\item If \(R\) is unital, then the additive inverse of the multiplicative identity is also a unit.

	\item If \(R\) is unital, then \(R\) has at least two units.

	\item If \(R\) is a unital ring which is infinite, then \(R\) has infinitely many units.	
\end{enumerate}


\stepcounter{section}
\section*{Problem 3}

Consider the matrices

\[
\bf P=\left\{\begin{pmatrix}
1 & 0 & 0 \\
0 & 1 & 0 \\
0 & 0 & 1
\end{pmatrix},
\begin{pmatrix}
0 & 1 & 0 \\
1 & 0 & 0 \\
0 & 0 & 1
\end{pmatrix}, 
\begin{pmatrix}
0 & 0 & 1 \\
0 & 1 & 0 \\
1 & 0 & 0
\end{pmatrix},
\begin{pmatrix}
1 & 0 & 0 \\
0 & 0 & 1 \\
0 & 1 & 0
\end{pmatrix},
\begin{pmatrix}
0 & 1 & 0 \\
0 & 0 & 1 \\
1 & 0 & 0
\end{pmatrix},
\begin{pmatrix}
0 & 0 & 1 \\
1 & 0 & 0 \\
0 & 1 & 0
\end{pmatrix}\right\}.
\]

Call the elements of \(\mathbf{P}\) as \(p_{e},p_{(12)},p_{(13)},p_{(23)},p_{(123)},p_{(132)}\) respectively (as labelled above). Consider the set

\[
\mathbb{Z}[\mathbf{P}] = \{a_0p_e + a_1p_{(12)} + a_2p_{(13)} + a_3p_{(23)} + a_4p_{(123)} + a_5p_{(132)}:a_0,\dots ,a_5\in \mathbb{Z}\}.
\]
\begin{enumerate}[label=(\alph*)]
	\item Prove that \(\mathbb{Z}[\mathbf{P}]\) is closed under (matrix) addition.
	\item Prove that \(\mathbb{Z}[\mathbf{P}]\) is closed under (matrix) multiplication.
	\item Prove that \(\mathbb{Z}[\mathbf{P}]\) is a ring with respect to matrix addition and matrix multiplication. Is it commutative?
\end{enumerate}


\stepcounter{section}
\section*{Problem 4}

Suppose that \(R\) is a principal ideal domain (that is, \(R\) has no zero divisors, is commutative, and every ideal is a principal ideal). Consider the set of \(n\times n\) matrices with coefficients in \(R\) , namely \(M_{n\times n}(R)\) .
\begin{enumerate}
	\item Prove that \(M_{n\times n}(R)\) is a ring with respect to matrix addition and matrix multiplication.

	\item Let \(J\) be a two-sided ideal of \(M_{n\times n}(R)\) . Prove that there exists an ideal \(I\) of \(R\) such that \(J = M_{n\times n}(I)\).
\end{enumerate}


\stepcounter{section}
\section*{Problem 5}

We consider the polynomial ring \(\mathbb{Z}[x]\) . We note that \(\mathbb{Z}\) is a principal ideal domain.
\begin{enumerate}[label=(\alph*)]
	\item Let \(f(x)\in \mathbb{Z}[x]\) be an irreducible polynomial. Prove that the ideal \((f(x))\) is a prime ideal.
	\item Let \(p\) be a rational prime. Prove that the principal ideal \((p)\) is still a prime ideal in \(\mathbb{Z}[x]\) .
	\item Let \(p\) be a rational prime and \(f(x)\) a polynomial in \(\mathbb{Z}[x]\) which is irreducible over the finite field \(\mathbb{F}_p\) (such a polynomial exists for every prime \(p\) ). Determine when the ideal \((p,f(x))\) is prime.
	\item Prove that the principal ideals in part (a) and part (b) are never maximal.
	\item Prove that the ideal \((3,x^{2} + 1)\) is a prime ideal in \(\mathbb{Z}[x]\) . Is it a maximal ideal?
\end{enumerate}


\stepcounter{section}
\section*{Problem 6}

Find a ring \(R\) (meaning a set with suitable addition and multiplication operations) with the following properties:
\begin{enumerate}[label=(\alph*)]
	\item \(R\) is uncountable.
	\item \(R\) has uncountably many units.
	\item \(R\) has a unique maximal ideal, which is principal.
\end{enumerate}


Note: any uncountable field like \(\mathbb{R}\) would satisfy the first two conditions. However, fields do not satisfy the third condition. Rings like \(\mathbb{R}[x]\) do not have unique maximal ideals.



\end{document}