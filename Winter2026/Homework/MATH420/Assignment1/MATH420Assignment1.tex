\documentclass[12pt]{article}

% Import preambles and macros for homework
% Essential packages
\usepackage{amsmath, amsfonts, amssymb, amsthm}
\usepackage{mathtools}
\usepackage{enumitem}
\usepackage{graphicx}
\usepackage{wrapfig}
\usepackage{systeme}
\usepackage{caption}
\usepackage{soul}
\usepackage[dvipsnames]{xcolor}
\usepackage{fancyhdr}
\allowdisplaybreaks

% Page layout
\usepackage[
  top=2cm,
  bottom=2cm,
  left=2cm,
  right=2cm,
  headheight=17pt,
  includehead,includefoot,
  heightrounded,
]{geometry}


% pgfornament for title page decorations
\usepackage[object=vectorian]{pgfornament}

% Fancy header/footer setup
\pagestyle{fancy}
\setlength{\headheight}{14.49998pt}
\addtolength{\topmargin}{-2.49998pt}
\renewcommand{\footrulewidth}{0.4pt}
\setlength\parindent{15pt}
% Math notation shortcuts
\newcommand{\R}{\mathbb{R}}
\newcommand{\Q}{\mathbb{Q}}
\newcommand{\Z}{\mathbb{Z}}
\newcommand{\N}{\mathbb{N}}
\newcommand{\C}{\mathbb{C}}
\newcommand{\X}{\mathcal{X}}

% Theorem environments
\newtheorem{mainthm}{Theorem}[section]
\newtheorem{theorem}{Theorem}[section]  
\newtheorem{lemma}[theorem]{Lemma}
\newtheorem{proposition}[theorem]{Proposition}
\newtheorem{corollary}[theorem]{Corollary}
\newtheorem{definition}[theorem]{Definition}
\newtheorem{claim}[theorem]{Claim}

% Calculus
\newcommand{\diff}{\mathop{}\!\mathrm{d}}
\newcommand{\deriv}[2]{\frac{\mathrm{d}#1}{\mathrm{d}#2}}
\newcommand{\pderiv}[2]{\frac{\partial #1}{\partial #2}}

% Linear Algebra
\newcommand{\inner}[2]{\langle #1, #2 \rangle}
\newcommand{\norm}[1]{\| #1 \|}
\newcommand{\tr}{\operatorname{tr}}
\newcommand{\spn}{\operatorname{span}}
\newcommand{\rank}{\operatorname{rank}}
\newcommand{\nullity}{\operatorname{nullity}}

% Logic
\newcommand{\contra}{\Rightarrow\Leftarrow}

% Custom commands for notes
\newcommand{\todo}[1]{\textcolor{red}{[TODO: #1]}}
\newcommand{\important}[1]{\textbf{\textcolor{blue}{#1}}}

%Number Theory
\DeclareMathOperator{\Li}{Li}
\newcommand{\floor}[1]{\left\lfloor #1 \right\rfloor}
\newcommand{\fract}[1]{\left\{ #1 \right\}}




\newcommand{\maketitlepage}{
    \begin{titlepage}
        \centering
        \vspace*{2.0cm}
        \pgfornament{84}\\
        {\LARGE \textsc{\coursename}\par}
        \vspace{0.5cm}
        {\large\coursecode\par}
        \vspace{0.5cm}
        {\large\instructor\par}
        \vspace{1.5cm}
        {\huge\bfseries\assignment\par}
        \vspace{1cm}
        {\LARGE\itshape\author\par}
        \vspace{2cm}
        {\large\bfseries Due Date:\par}
        \vspace{0.5cm}
        {\Large \duedate}\\
        \pgfornament{84}
    \end{titlepage}
}
% =============================================
% HOMEWORK CONFIGURATION - EDIT THESE VALUES!
% =============================================

% Your personal info
\renewcommand{\author}{Deepak Jassal}
\newcommand{\authorlast}{Jassal}

% Course info
\newcommand{\coursename}{Course Name}
\newcommand{\coursecode}{Course code}
\newcommand{\instructor}{Instructor}

% Assignment-specific info (CHANGE THESE FOR EACH HOMEWORK)
\newcommand{\assignment}{Assignment }
\newcommand{\duedate}{Month Day\textsuperscript{th}, 20XX}

% Header configuration
\fancyhead[l]{\assignment}
\fancyhead[c]{\coursecode}
\fancyhead[r]{\monthyear}
\fancyfoot[c]{\authorlast{ }\thepage}

\renewcommand{\author}{Deepak Jassal}
\renewcommand{\authorlast}{Jassal}
\renewcommand{\coursename}{Structure of Groups and Rings}
\renewcommand{\coursecode}{MATH 420}
\renewcommand{\assignment}{Assignment 1}
\renewcommand{\instructor}{Dr. Stanley Yao Xiao}
\renewcommand{\duedate}{February 1\textsuperscript{st}, 2026}


\begin{document}
\begin{titlepage}
	\centering
	\vspace*{2.0cm}	
	\pgfornament{84}\\
	{\LARGE \textsc{\coursename}\par}
	\vspace{0.5cm}
	{\large\coursecode\par}
    \vspace{0.5cm}
    {\large\instructor\par}
	\vspace{1.5cm}
	{\huge\bfseries\assignment\par}
	\vspace{1cm}  
	{\LARGE\itshape\author\par}
    \vspace{2cm}
	{\large\bfseries Due Date:\par}
	\vspace{0.5cm}
	{\Large \duedate}\\
	\pgfornament{84}
\end{titlepage}
\stepcounter{section}
\section*{Problem 1}
\begin{enumerate}[label=(\alph*)]
	\item Let \((\Z, + ,\times)\) be the ring of integers with the standard arithmetic operations. Prove that every subring of \(\Z\) is also an ideal.
	\begin{proof}
		Let $S$ be a subring of $\Z$. We need to show that $S$ is an ideal, that is, for all $s \in S$ and all $r \in \Z$, the product $rs$ belongs to $S$.

		If $S = \{0\}$, then $S$ is trivially an ideal. Now assume $S \neq \{0\}$.

		Since $S$ is a subring, it contains $0$ and is closed under subtraction. Let $m$ be the smallest positive integer in $S$ (such an $m$ exists because $S$ is non-zero and closed under taking negatives, so it contains a positive integer). 

		\textbf{Claim.} $S = m\Z = \{ m t \mid t \in \Z \}$.

		\begin{itemize}
			\item[$(\supseteq)$] Because $S$ is closed under addition and contains additive inverses, all integer multiples of $m$ lie in $S$. Thus $m\Z \subseteq S$.
			\item[$(\subseteq)$] Take any $x \in S$. By the division algorithm, write $x = mq + r$ with $q, r \in \Z$ and $0 \le r < m$. Since $x \in S$ and $mq \in S$ (as $m \in S$), we have $r = x - mq \in S$. The minimality of $m$ forces $r = 0$, so $x = mq \in m\Z$.
		\end{itemize}

		Hence $S = m\Z$ for some $m \ge 0$.

		Now let $a \in S$ and $r \in \Z$ be arbitrary. Write $a = m t$ with $t \in \Z$. Then
		\[
		ra = r(m t) = m(rt) \in m\Z = S.
		\]
		Therefore $S$ is an ideal of $\Z$.
	\end{proof}
	\item Consider the polynomial ring \(\R[x]\). Prove that every ideal of $\R[x]$ is principal.
	\begin{proof}
		Let $I$ be an ideal of $\R[x]$. We need to show that $I=(f)$ for some $f\in I$. Let $g\in I$. Then by the polynomial division we have 
		\[
			g(x)=q(x)f(x)+r(x)
		\]
		with $deg(r)<deg(f)$. Then we have
		\[
			r(x)=g(x)-q(x)f(x)
		\]
		which implies that $r(x)=0$ else we would have $deg(r)>deg(f)$. Thus $g=(f)$. Therefore we have that each ideal of $\R[x]$ is principal.
	\end{proof}
\end{enumerate}


\stepcounter{section}
\section*{Problem 2}

For each of the following statements, either prove that it is true, or provide a counterexample. In each case \(R\) is a ring, with addition and multiplication understood.
\begin{enumerate}[label=(\alph*)]
	\item If \(R\) is unital, then the additive inverse of the multiplicative identity is also a unit.
	\begin{proof}
		We denote the multiplicative identity by $1_R$. The additive inverse of the multiplicative identity would then be $-1_R$. By the nature of $1_R$ we have $1_R\times1_R=1_R$ that being it is its own multiplicative identity. We need to find $x\in R$ such that
		\[
			-1\times x=1.
		\]
		We know that for all $r\in R$ $-1\times r=-r$. Thus we have
		\[
			-1\times x=-x=1\Rightarrow x=-1.
		\]
		To check this see that
		\begin{align*}
			(-1)^2+(-1)&=(-1)(-1+1)\\
			&=-1\times0\\
			&=0.
		\end{align*}
	\end{proof}

	\item If \(R\) is unital, then \(R\) has at least two units.\\
	\textit{Counterexample.} The zero ring $R=\{0\}$ is a ring where the additive and multiplicative identity are the same element.

	\item If \(R\) is a unital ring which is infinite, then \(R\) has infinitely many units.	\\
	\textit{Counterexample.} The ring $(\Z, + ,\times)$  is an infiinte ring, but it only have two units. Those being $1$ and $-1$.
\end{enumerate}
\newpage

\stepcounter{section}
\section*{Problem 3}

Consider the matrices

\[
\bf P=\left\{\begin{pmatrix}
1 & 0 & 0 \\
0 & 1 & 0 \\
0 & 0 & 1
\end{pmatrix},
\begin{pmatrix}
0 & 1 & 0 \\
1 & 0 & 0 \\
0 & 0 & 1
\end{pmatrix}, 
\begin{pmatrix}
0 & 0 & 1 \\
0 & 1 & 0 \\
1 & 0 & 0
\end{pmatrix},
\begin{pmatrix}
1 & 0 & 0 \\
0 & 0 & 1 \\
0 & 1 & 0
\end{pmatrix},
\begin{pmatrix}
0 & 1 & 0 \\
0 & 0 & 1 \\
1 & 0 & 0
\end{pmatrix},
\begin{pmatrix}
0 & 0 & 1 \\
1 & 0 & 0 \\
0 & 1 & 0
\end{pmatrix}\right\}.
\]

Call the elements of \(\mathbf{P}\) as \(p_{e},p_{(12)},p_{(13)},p_{(23)},p_{(123)},p_{(132)}\) respectively (as labelled above). Consider the set

\[
\Z[\mathbf{P}] = \{a_0p_e + a_1p_{(12)} + a_2p_{(13)} + a_3p_{(23)} + a_4p_{(123)} + a_5p_{(132)}:a_0,\dots ,a_5\in \Z\}.
\]
\begin{enumerate}[label=(\alph*)]
	\item Prove that \(\Z[\mathbf{P}]\) is closed under (matrix) addition.
	\begin{proof}
		Let $A,B\in\Z[\mathbf{P}]$. Then 
		\[
			A=a_0p_e + a_1p_{(12)} + a_2p_{(13)} + a_3p_{(23)} + a_4p_{(123)} + a_5p_{(132)}
		\]
		\[
			B=b_0p_e + b_1p_{(12)} + b_2p_{(13)} + b_3p_{(23)} + b_4p_{(123)} + b_5p_{(132)}
		\]
		for some $a_0,\dots,a_5,b_0,\dots,b_5\in\Z$. We now need to compute $A+B$. Since matrix addition is commutative we have
		\begin{align*}
			A+B=&a_0p_e + a_1p_{(12)} + a_2p_{(13)} + a_3p_{(23)} + a_4p_{(123)} + a_5p_{(132)}\\
			&+b_0p_e + b_1p_{(12)} + b_2p_{(13)} + b_3p_{(23)} + b_4p_{(123)} + b_5p_{(132)}\\
			=&a_0p_e +b_0p_e+ a_1p_{(12)}+ b_1p_{(12)}+ a_2p_{(13)}+ b_2p_{(13)}\\
			&+ a_3p_{(23)}+ b_3p_{(23)}+ a_4p_{(123)}+ b_4p_{(123)}+ a_5p_{(132)}+ b_5p_{(132)}\\
			=&(a_0+b_0)p_e + (a_1+b_1)p_{(12)} + (a_2+b_2)p_{(13)} + (a_3+b3)p_{(23)} + (a_4+b_4)p_{(123)} + (a_5+b_5)p_{(132)}.
		\end{align*}
		Since addition in $\Z$ is closed we let $a_i+b_i=c_i\in\Z$ for $0\leq i\leq5$. Then
		\[
			A+B=c_0p_e +c_1p_{(12)} + c_2p_{(13)} + c_3p_{(23)} + c_4p_{(123)} + c_5p_{(132)}=C,
		\]
		and $C\in\Z[\mathbf{P}]$. Thus $\Z[\mathbf{P}]$ is closed under matrix addition.
	\end{proof}
	\newpage
	\item Prove that \(\Z[\mathbf{P}]\) is closed under (matrix) multiplication.
	\begin{proof}
		Let $A,B\in\Z[\mathbf{P}]$. Then 
		\[
			A=a_0p_e + a_1p_{(12)} + a_2p_{(13)} + a_3p_{(23)} + a_4p_{(123)} + a_5p_{(132)}
		\]
		\[
			B=b_0p_e + b_1p_{(12)} + b_2p_{(13)} + b_3p_{(23)} + b_4p_{(123)} + b_5p_{(132)}
		\]
		for some $a_0,\dots,a_5,b_0,\dots,b_5\in\Z$. Next we will compute the matrix $AB$. Instead of explicitly computing the prducts of these matrices we will ntoe that these matrices represent permutations and that
		\[
			p_\sigma p_\tau=p_{\sigma\tau}
		\]
		Then the multiplication
		\[
			(a_0 p_e + a_1 p_{(12)} + a_2 p_{(13)} + a_3 p_{(23)} + a_4 p_{(123)} + a_5 p_{(132)})(b_0 p_e + b_1 p_{(12)} + b_2 p_{(13)} + b_3 p_{(23)} + b_4 p_{(123)} + b_5 p_{(132)})
		\]
		expands to
		\[
			c_0 p_e + c_1 p_{(12)} + c_2 p_{(13)} + c_3 p_{(23)} + c_4 p_{(123)} + c_5 p_{(132)},
		\]
		where
		\[
			\begin{aligned}
				c_0 &= a_0 b_0 + a_1 b_1 + a_2 b_2 + a_3 b_3 + a_4 b_5 + a_5 b_4, \\
				c_1 &= a_0 b_1 + a_1 b_0 + a_2 b_4 + a_3 b_5 + a_4 b_3 + a_5 b_2, \\
				c_2 &= a_0 b_2 + a_1 b_5 + a_2 b_0 + a_3 b_4 + a_4 b_1 + a_5 b_3, \\
				c_3 &= a_0 b_3 + a_1 b_4 + a_2 b_5 + a_3 b_0 + a_4 b_2 + a_5 b_1, \\
				c_4 &= a_0 b_4 + a_1 b_3 + a_2 b_1 + a_3 b_2 + a_4 b_0 + a_5 b_5, \\
				c_5 &= a_0 b_5 + a_1 b_2 + a_2 b_3 + a_3 b_1 + a_4 b_4 + a_5 b_0.
			\end{aligned}
		\]

		Explicitly we have,
		\[
			\begin{aligned}
				&(a_0 p_e + a_1 p_{(12)} + a_2 p_{(13)} + a_3 p_{(23)} + a_4 p_{(123)} + a_5 p_{(132)}) \\
				&\times (b_0 p_e + b_1 p_{(12)} + b_2 p_{(13)} + b_3 p_{(23)} + b_4 p_{(123)} + b_5 p_{(132)})\\
				= &\quad (a_0 b_0 + a_1 b_1 + a_2 b_2 + a_3 b_3 + a_4 b_5 + a_5 b_4) p_e \\
				&+ (a_0 b_1 + a_1 b_0 + a_2 b_4 + a_3 b_5 + a_4 b_3 + a_5 b_2) p_{(12)} \\
				&+ (a_0 b_2 + a_1 b_5 + a_2 b_0 + a_3 b_4 + a_4 b_1 + a_5 b_3) p_{(13)} \\
				&+ (a_0 b_3 + a_1 b_4 + a_2 b_5 + a_3 b_0 + a_4 b_2 + a_5 b_1) p_{(23)} \\
				&+ (a_0 b_4 + a_1 b_3 + a_2 b_1 + a_3 b_2 + a_4 b_0 + a_5 b_5) p_{(123)} \\
				&+ (a_0 b_5 + a_1 b_2 + a_2 b_3 + a_3 b_1 + a_4 b_4 + a_5 b_0) p_{(132)}.
			\end{aligned}
		\]
		Thus we have $AB=C\in\Z[\mathbf{P}]$, and $\Z[\mathbf{P}]$ is closed under matrix multiplication.
	\end{proof}
	\newpage
	\item Prove that \(\Z[\mathbf{P}]\) is a ring with respect to matrix addition and matrix multiplication. Is it commutative?
	\begin{proof}
		Since addition of matrices it commutative we have that $(\Z[\mathbf{P}],+)$ is an Abelian group. Since elements in $\Z[\mathbf{P}]$ are closed under matrix multiplication and matrix multiplication over $\Z$ we have that matrix  multiplication is associative in $\Z[\mathbf{P}]$. Since matrices are also distributive we have $A(B+C)=AB+AC$ and $(B+C)A=BA+CA$. Thus $(\Z[\mathbf{P}],+,\times)$ is a ring under matrix adidtion and multiplication.\\
		To show that $\Z[\mathbf{P}]$ is commutative see that
		\begin{align*}
			BA=&(a_0 p_e + a_1 p_{(12)} + a_2 p_{(13)} + a_3 p_{(23)} + a_4 p_{(123)} + a_5 p_{(132)})\\
			&(b_0 p_e + b_1 p_{(12)} + b_2 p_{(13)} + b_3 p_{(23)} + b_4 p_{(123)} + b_5 p_{(132)})\\
			=&c_0 p_e + c_1 p_{(12)} + c_2 p_{(13)} + c_3 p_{(23)} + c_4 p_{(123)} + c_5 p_{(132)},
		\end{align*}
		with
		\[
			\begin{aligned}
				c_0 &= a_0 b_0 + a_1 b_1 + a_2 b_2 + a_3 b_3 + a_4 b_5 + a_5 b_4, \\
				c_1 &= a_0 b_1 + a_1 b_0 + a_2 b_4 + a_3 b_5 + a_4 b_3 + a_5 b_2, \\
				c_2 &= a_0 b_2 + a_1 b_5 + a_2 b_0 + a_3 b_4 + a_4 b_1 + a_5 b_3, \\
				c_3 &= a_0 b_3 + a_1 b_4 + a_2 b_5 + a_3 b_0 + a_4 b_2 + a_5 b_1, \\
				c_4 &= a_0 b_4 + a_1 b_3 + a_2 b_1 + a_3 b_2 + a_4 b_0 + a_5 b_5, \\
				c_5 &= a_0 b_5 + a_1 b_2 + a_2 b_3 + a_3 b_1 + a_4 b_4 + a_5 b_0,
			\end{aligned}
		\] and that
		\begin{align*}
			AB=&(b_0 p_e + b_1 p_{(12)} + b_2 p_{(13)} + b_3 p_{(23)} + b_4 p_{(123)} + b_5 p_{(132)})\\
			&(a_0 p_e + a_1 p_{(12)} + a_2 p_{(13)} + a_3 p_{(23)} + a_4 p_{(123)} + a_5 p_{(132)})\\
			=&c'_0 p_e + c'_1 p_{(12)} + c'_2 p_{(13)} + c'_3 p_{(23)} + c'_4 p_{(123)} + c'_5 p_{(132)},
		\end{align*}
		with
		\[
			\begin{aligned}
				c'_0 &= b_0 a_0 + b_1 a_1 + b_2 b_2 + b_3 a_3 + b_4 a_5 + b_5 a_4, \\
				c'_1 &= b_0 a_1 + b_1 a_0 + b_2 a_4 + b_3 a_5 + b_4 a_3 + b_5 a_2, \\
				c'_2 &= b_0 a_2 + b_1 a_5 + b_2 a_0 + b_3 a_4 + b_4 a_1 + b_5 a_3, \\
				c'_3 &= b_0 a_3 + b_1 a_4 + b_2 a_5 + b_3 a_0 + b_4 a_2 + b_5 a_1, \\
				c'_4 &= b_0 a_4 + b_1 a_3 + b_2 a_1 + b_3 a_2 + b_4 a_0 + b_5 a_5, \\
				c'_5 &= b_0 a_5 + b_1 a_2 + b_2 a_3 + b_3 a_1 + b_4 a_4 + b_5 a_0.
			\end{aligned}
		\]
		Since multiplication over $\Z$ is commutative we have $c_i=c_i'$ for $0\leq i\leq 5$. Thus $\Z[\mathbf{P}]$ is a commutative ring.
	\end{proof}
\end{enumerate}

\newpage
\stepcounter{section}
\section*{Problem 4}

Suppose that \(R\) is a principal ideal domain (that is, \(R\) has no zero divisors, is commutative, and every ideal is a principal ideal). Consider the set of \(n\times n\) matrices with coefficients in \(R\) , namely \(M_{n\times n}(R)\).
\begin{enumerate}[label=(\alph*)]
	\item Prove that \(M_{n\times n}(R)\) is a ring with respect to matrix addition and matrix multiplication.
	\begin{proof}
		Since $R$ is a principal ideal domain we have that elements in $R$ are closed under addition and multiplication, that multiplication of elements in $R$ is associative and commutative and that $R$ has the distributive property.\\
		Let $A,B\in M_{n\times n}(R)$ such that $A+B=C$. Then
		\[
			\begin{pmatrix}
			a_{11} & a_{12} & \cdots & a_{1n} \\
			a_{21} & a_{22} & \cdots & a_{2n} \\
			\vdots & \vdots & \ddots & \vdots \\
			a_{n1} & a_{n2} & \cdots & a_{nn}
			\end{pmatrix}+
			\begin{pmatrix}
			b_{11} & b_{12} & \cdots & b_{1n} \\
			b_{21} & b_{22} & \cdots & b_{2n} \\
			\vdots & \vdots & \ddots & \vdots \\
			b_{n1} & b_{n2} & \cdots & b_{nn}
			\end{pmatrix}
			=
			\begin{pmatrix}
			c_{11} & c_{12} & \cdots & c_{1n} \\
			c_{21} & c_{22} & \cdots & c_{2n} \\
			\vdots & \vdots & \ddots & \vdots \\
			c_{n1} & c_{n2} & \cdots & c_{nn}
			\end{pmatrix}.
		\]
		With $c_{ij}=a_{ij}+b_{ij}$. We also have
		\[
			\begin{pmatrix}
			b_{11} & b_{12} & \cdots & b_{1n} \\
			b_{21} & b_{22} & \cdots & b_{2n} \\
			\vdots & \vdots & \ddots & \vdots \\
			b_{n1} & b_{n2} & \cdots & b_{nn}
			\end{pmatrix}+
			\begin{pmatrix}
			a_{11} & a_{12} & \cdots & a_{1n} \\
			a_{21} & a_{22} & \cdots & a_{2n} \\
			\vdots & \vdots & \ddots & \vdots \\
			a_{n1} & a_{n2} & \cdots & a_{nn}
			\end{pmatrix}
			=
			\begin{pmatrix}
			c'_{11} & c'_{12} & \cdots & c'_{1n} \\
			c'_{21} & c'_{22} & \cdots & c'_{2n} \\
			\vdots & \vdots & \ddots & \vdots \\
			c'_{n1} & c'_{n2} & \cdots & c'_{nn}
			\end{pmatrix}.
		\]
		With $c'_{ij}=b_{ij}+a_{ij}$. Since $(R,+)$ is an Abelian group we have $A+B=B+A=C$. Therefore, $(M_{n\times n},+)$ is an Abelian group.
		Let $A,B\in M_{n\times n}(R)$ such that $AB=C'$.\\\\
		For matrices \( A = (a_{ij}), B = (b_{jk}), C = (c_{kl}) \in M_{n \times n}(R) \), matrix multiplication is defined by:

		\[
			(AB)_{ik} = \sum_{j=1}^n a_{ij} b_{jk}, \qquad (BC)_{jl} = \sum_{k=1}^n b_{jk} c_{kl}.
		\]

		We want to prove that
		\[
			(A B) C = A (B C).
		\]
		Consider the \( (i,l) \)-entry of \( (AB)C \),

		\[
			[(AB)C]_{il} = \sum_{k=1}^n (AB)_{ik} \, c_{kl} 
			= \sum_{k=1}^n \left( \sum_{j=1}^n a_{ij} b_{jk} \right) c_{kl}.
		\]

		Since \( R \) is a commutative ring,

		\[
			[(AB)C]_{il} = \sum_{k=1}^n \sum_{j=1}^n a_{ij} b_{jk} c_{kl}.
		\]

		Now consider the \( (i,l) \)-entry of \( A(BC) \),

		\[
			[A(BC)]_{il} = \sum_{j=1}^n a_{ij} (BC)_{jl} 
			= \sum_{j=1}^n a_{ij} \left( \sum_{k=1}^n b_{jk} c_{kl} \right).
		\]

		Again using distributivity and commutativity of \( R \),

		\[
			[A(BC)]_{il} = \sum_{j=1}^n \sum_{k=1}^n a_{ij} b_{jk} c_{kl}.
		\]

		Since addition in \( R \) is commutative, we can swap the order of summation,

		\[
			\sum_{k=1}^n \sum_{j=1}^n a_{ij} b_{jk} c_{kl} 
			= \sum_{j=1}^n \sum_{k=1}^n a_{ij} b_{jk} c_{kl}.
		\]

		Thus,
		\[
			[(AB)C]_{il} = [A(BC)]_{il} \quad \text{for all } 1 \le i,l \le n.
		\]

		Therefore,
		\[
			(AB)C = A(BC).
		\]
		Thus we have that matrix multiplication is associative in $M_{n\times n}(R)$.\\
		\begin{align*}
			[A(B + C)]_{ik} &= \sum_{j=1}^n a_{ij} (B + C)_{jk}\\
			&= \sum_{j=1}^n a_{ij} (b_{jk} + c_{jk})\\
			&= \sum_{j=1}^n \left( a_{ij} b_{jk} + a_{ij} c_{jk} \right)\\
			&= \sum_{j=1}^n a_{ij} b_{jk} + \sum_{j=1}^n a_{ij} c_{jk}\\
			&= (AB)_{ik} + (AC)_{ik}.
		\end{align*}
		This holds for all $i,k$ so we have $A(B+C)=AB+AC$.\\
		\begin{align*}
			[(A + B)C]_{il} &= \sum_{k=1}^n (A + B)_{ik} c_{kl}\\
			&= \sum_{k=1}^n (a_{ik} + b_{ik}) c_{kl}\\
			&= \sum_{k=1}^n \left( a_{ik} c_{kl} + b_{ik} c_{kl} \right)\\
			&= \sum_{k=1}^n a_{ik} c_{kl} + \sum_{k=1}^n b_{ik} c_{kl}\\
			&= (AC)_{il} + (BC)_{il}.
		\end{align*}
		This holds for all $i,l$ so we have $(A+B)C=AC+BC$.\\
		Therefore $M_{n\times n}(R)$ is a ring under matrix addition and multiplication.
	\end{proof}

	\item Let \(J\) be a two-sided ideal of \(M_{n\times n}(R)\). Prove that there exists an ideal \(I\) of \(R\) such that \(J = M_{n\times n}(I)\).
	\begin{proof}
		Let 
		\[
			I=\{a\in R:aE_{11}\in J\}.
		\]
		Where $E_{11}\in M_{n\times n}(R)$ be the matrix with 1,1 entry 1 and 0 everywhere else. In general $E_{ij}$ is the matrix in $M_{n\times n}(R)$ with $ij$ entry 1 and 0 everywhere else.\\
		\textbf{Claim.} $I$ is an ideal of $R$.\\
		Since $J$ is an ideal we have that $I$ is non empty. For $a,b\in I$ let $A$ be any matrix with 1,1 entry $a$ and $B$ be any matrix with 1,1 entry $b$, both in $J$. Then since $A+B\in J$ we have $(a+b)\in I$. By commutativity of matrix addition we also have $(b+a)\in I$. Thus, $(I,+)$ is an Abelian group. For any $a\in I$ and $r\in R$ we have $rE_{11}(AE_{11})\in J$. This matrix has 1,1 entry $ra$. Thus $ar\in I$ for all $a\in I$ and $r\in R$. By commutativity of $R$ we also have $ra\in I$. Thus, $I$ is an ideal of $R$. \\\\
		For any $a\in I$ we have $aE_{11}\in J$. Now 
		\[
			E_{i1}(aE_{11})E_{1j}=aE_{ij}\in J.
		\]
		Thus for all $a\in I$ and $1\leq i,j\leq n$ we have $aE_{ij}\in J$. Since $J$ is closed under addition we have that each matrix in $M_{n\times n}(I)$ is a sum of $a_{ij}E_{ij}$ with $a_{ij}\in I$. Since $a_{ij}E_{ij}\in J$ we have $M_{n\times n}(I)\subseteq J$.\\ 
		For any $A\in J$ we have $A_{ij}=E_{1j}AE_{i1}\in J$, and more importantly the 1,1 entry of $A_{ij}$ is the $i,j$ entry of $A$, that is $a_{ij}$. By definition of $I$ this means that every entry of $A$ is in $I$. So every entry of every matrix in $J$ is in $I$. Thus we have $J\subseteq M_{n\times n}(I)$.\\
		Combining these two inclusions we see that $J=M_{n\times n}(I)$.
	\end{proof}
\end{enumerate}


\stepcounter{section}
\section*{Problem 5}

We consider the polynomial ring \(\Z[x]\). We note that \(\Z\) is a principal ideal domain.
\begin{enumerate}[label=(\alph*)]
	\item Let \(f(x)\in \Z[x]\) be an irreducible polynomial. Prove that the ideal \((f(x))\) is a prime ideal.
	\begin{proof}
		Let $F$ be the ideal generated by $(f(x))$. Let $A,B\subseteq \Z[x]$ be such that $AB\subseteq F$. Suppose for contradiction that $A\not\subseteq F$ and $B\not\subseteq F$. That means that there exists $a(x)\in A$ and $b(x)\in B$ such that $a\not\in F$ and $b\not\in F$. But $ab\in F$. Since $F$ is generated by the irreducible polynomial $f(x)$, for some $m\in\Z[x]$ we have
		\[
			a(x)b(x)=mf(x).
		\]
		Thus $f(x)\mid a(x)b(x)$, this implies that $f(x)\mid a(x)$ or $f(x)\mid b(x)$. But since $a(x),b(x)\not\in F\Rightarrow\,f(x)\not\vert a(x)$ or $f(x)\not\vert b(x)$. This gives a contradiction. Therefore, $F=(f(x))$ is a prime ideal. 
	\end{proof}
	\item Let \(p\) be a rational prime. Prove that the principal ideal \((p)\) is still a prime ideal in \(\Z[x]\).
	\begin{proof}
		A rational prime $p$ is a zero degree polynomial in $\Z[x]$. By primality it is irreducible in $\Z[x]$. Thus, by the last result we have that $(p)$ is a prime ideal in $\Z[x]$. 
	\end{proof}
	\item Let \(p\) be a rational prime and \(f(x)\) a polynomial in \(\Z[x]\) which is irreducible over the finite field \(\mathbb{F}_p\) (such a polynomial exists for every prime \(p\) ). Determine when the ideal \((p,f(x))\) is prime.\\
	\textit{Solution.} By the third isomorphism theorem we have 
	\[
		\Z[x]/(p,f(x))\cong (\Z[x]/(p))/((p,f(x))/(p)).
	\]
	Furthermore,
	\[
			\Z[x]/(p)\cong \mathbb{F}_p[x]
	\]
	and let
	\[
		\overline{f}(x)=(p,f(x))/(p) \mod \mathbb{F}_p[x].
	\]
	Thus, we have
	\[
		\Z[x]/(p,f(x))\cong \mathbb{F}_p[x]/(\overline{f}(x)).
	\]
	This reduces our problem to finding when $\mathbb{F}_p[x]/(\overline{f}(x))$ is an integral domain. That is because in unital commutative rings $R$ (which both are) for an ideal $I\subseteq R$, $R/I$ is an integral domain when $I$ is prime and vice versa. In this case $R=\mathbb{F}_p[x]$ and $I=\bar{f}(x)$. Since $f(x)$ is irreducible in $\Z[x]$ we have that $\bar{f}(x)$ is irreducible in $\Z[x]$. Thus the ideal $(p,f(x))$ is always prime given that $p\in\Z$ is a prime and $f(x)$ is irreducible in $\Z[x]$.
	\newpage
	\item Prove that the principal ideals in part (a) and part (b) are never maximal.
	\begin{proof}
		For the ideals in part (a) and (b) we have $f(x)\subseteq (f(x),p)$ and $(p)\subseteq (f(x),p)$. Since $(f(x),p)$ is an ideal we have that $f(x)$ and $(p)$ are never maximal ideals.
	\end{proof}
	\item Prove that the ideal \((3,x^{2} + 1)\) is a prime ideal in \(\Z[x]\). Is it a maximal ideal?
	\begin{proof}
		3 is a rational prime and $x^2+1$ is irreducible in $\Z[x]$. Therefore, by the earlier parts we have that $(3,x^2+1)$ is a prime ideal in $\Z[x]$.\\
		This ideal is maximal. This is beacause from part (c) we know that $\mathbb{F}_3/(\overline{x^2+1})$ is an integral domain, and therefore $(\overline{x^2+1}$) is maximal. Since $(\overline{x^2+1})$ comes from $(3,x^2+1)$ mod $\mathbb{F}_3$ we have that $(3,x^2+1)$ is a maximal ideal.
	\end{proof}
\end{enumerate}


\stepcounter{section}
\section*{Problem 6}

Find a ring \(R\) (meaning a set with suitable addition and multiplication operations) with the following properties:
\begin{enumerate}[label=(\alph*)]
	\item \(R\) is uncountable.\\
	\textit{Solution.} $(\R,+,\times)$
	\item \(R\) has uncountably many units.\\
	\textit{Solution.} $(\R,+,\times)$.
	\item \(R\) has a unique maximal ideal, which is principal.\\
	\textit{Solution.} The $p-$addic integers $\Z_p$ for some prime $p$ satisfies this condition.
\end{enumerate}


Note: any uncountable field like \(\R\) would satisfy the first two conditions. However, fields do not satisfy the third condition. Rings like \(\R[x]\) do not have unique maximal ideals.



\end{document}