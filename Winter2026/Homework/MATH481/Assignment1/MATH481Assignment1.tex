\documentclass[12pt]{article}

% Import preambles and macros for homework
% Essential packages
\usepackage{amsmath, amsfonts, amssymb, amsthm}
\usepackage{mathtools}
\usepackage{enumitem}
\usepackage{graphicx}
\usepackage{wrapfig}
\usepackage{systeme}
\usepackage{caption}
\usepackage{soul}
\usepackage[dvipsnames]{xcolor}
\usepackage{fancyhdr}
\allowdisplaybreaks

% Page layout
\usepackage[
  top=2cm,
  bottom=2cm,
  left=2cm,
  right=2cm,
  headheight=17pt,
  includehead,includefoot,
  heightrounded,
]{geometry}


% pgfornament for title page decorations
\usepackage[object=vectorian]{pgfornament}

% Fancy header/footer setup
\pagestyle{fancy}
\setlength{\headheight}{14.49998pt}
\addtolength{\topmargin}{-2.49998pt}
\renewcommand{\footrulewidth}{0.4pt}
\setlength\parindent{15pt}
% Math notation shortcuts
\newcommand{\R}{\mathbb{R}}
\newcommand{\Q}{\mathbb{Q}}
\newcommand{\Z}{\mathbb{Z}}
\newcommand{\N}{\mathbb{N}}
\newcommand{\C}{\mathbb{C}}
\newcommand{\X}{\mathcal{X}}

% Theorem environments
\newtheorem{mainthm}{Theorem}[section]
\newtheorem{theorem}{Theorem}[section]  
\newtheorem{lemma}[theorem]{Lemma}
\newtheorem{proposition}[theorem]{Proposition}
\newtheorem{corollary}[theorem]{Corollary}
\newtheorem{definition}[theorem]{Definition}
\newtheorem{claim}[theorem]{Claim}

% Calculus
\newcommand{\diff}{\mathop{}\!\mathrm{d}}
\newcommand{\deriv}[2]{\frac{\mathrm{d}#1}{\mathrm{d}#2}}
\newcommand{\pderiv}[2]{\frac{\partial #1}{\partial #2}}

% Linear Algebra
\newcommand{\inner}[2]{\langle #1, #2 \rangle}
\newcommand{\norm}[1]{\| #1 \|}
\newcommand{\tr}{\operatorname{tr}}
\newcommand{\spn}{\operatorname{span}}
\newcommand{\rank}{\operatorname{rank}}
\newcommand{\nullity}{\operatorname{nullity}}

% Logic
\newcommand{\contra}{\Rightarrow\Leftarrow}

% Custom commands for notes
\newcommand{\todo}[1]{\textcolor{red}{[TODO: #1]}}
\newcommand{\important}[1]{\textbf{\textcolor{blue}{#1}}}

%Number Theory
\DeclareMathOperator{\Li}{Li}
\newcommand{\floor}[1]{\left\lfloor #1 \right\rfloor}
\newcommand{\fract}[1]{\left\{ #1 \right\}}




\newcommand{\maketitlepage}{
    \begin{titlepage}
        \centering
        \vspace*{2.0cm}
        \pgfornament{84}\\
        {\LARGE \textsc{\coursename}\par}
        \vspace{0.5cm}
        {\large\coursecode\par}
        \vspace{0.5cm}
        {\large\instructor\par}
        \vspace{1.5cm}
        {\huge\bfseries\assignment\par}
        \vspace{1cm}
        {\LARGE\itshape\author\par}
        \vspace{2cm}
        {\large\bfseries Due Date:\par}
        \vspace{0.5cm}
        {\Large \duedate}\\
        \pgfornament{84}
    \end{titlepage}
}
% =============================================
% HOMEWORK CONFIGURATION - EDIT THESE VALUES!
% =============================================

% Your personal info
\renewcommand{\author}{Deepak Jassal}
\newcommand{\authorlast}{Jassal}

% Course info
\newcommand{\coursename}{Course Name}
\newcommand{\coursecode}{Course code}
\newcommand{\instructor}{Instructor}

% Assignment-specific info (CHANGE THESE FOR EACH HOMEWORK)
\newcommand{\assignment}{Assignment }
\newcommand{\duedate}{Month Day\textsuperscript{th}, 20XX}

% Header configuration
\fancyhead[l]{\assignment}
\fancyhead[c]{\coursecode}
\fancyhead[r]{\monthyear}
\fancyfoot[c]{\authorlast{ }\thepage}

\renewcommand{\author}{Deepak Jassal}
\renewcommand{\authorlast}{Jassal}
\renewcommand{\coursename}{Analytic Number Theory}
\renewcommand{\coursecode}{MATH 481}
\renewcommand{\assignment}{Assignment 1}
\renewcommand{\instructor}{Dr. Alia Hamieh}
\renewcommand{\duedate}{February 3\textsuperscript{rd}, 2026}


\begin{document}
\begin{titlepage}
	\centering
	\vspace*{2.0cm}	
	\pgfornament{84}\\
	{\LARGE \textsc{\coursename}\par}
	\vspace{0.5cm}
	{\large\coursecode\par}
    \vspace{0.5cm}
    {\large\instructor\par}
	\vspace{1.5cm}
	{\huge\bfseries\assignment\par}
	\vspace{1cm}  
	{\LARGE\itshape\author\par}
    \vspace{2cm}
	{\large\bfseries Due Date:\par}
	\vspace{0.5cm}
	{\Large \duedate}\\
	\pgfornament{84}
\end{titlepage}
\stepcounter{section}
\section*{Problem 1 [5 marks]} Let $p(x)$ be a polynomial with integer coefficients and degree $k\geq1$. Show that $p(n)$ is composite for infinitely many integers $n$.

\stepcounter{section}
\section*{Problem 2 [5 marks]} An arithmetic function $f$ is called periodic if there exists a positive integer $k$ such that $f(n+k)=f(n)$ for every $n\in\mathbb{N}$; the integer $k$ is called a period for $f$. Show that if $f$ is completely multiplicative and periodic with period $k$, then the values of $f$ are either $0$ or roots of unity. (An root of unity is a complex number $z$ such that $z^n=1$ for some $n\in\mathbb{N}$.)

\stepcounter{section}
\section*{Problem 3 [15 marks]}  Let $f,g$ be positive functions defined on $\mathbb{R}$. Suppose $f(x)\sim g(x)$. Prove that
\begin{enumerate}
	\item $f(x)^r \sim g(x)^r$ for any  real $r$,
	\begin{proof}
		\begin{align*}
			\lim_{x\to\infty}\frac{f^r(x)}{g^r(x)}&=\lim_{x\to\infty}\left(\frac{f(x)}{g(x)}\right)^r\\
			&=\left(\lim_{x\to\infty}\frac{f(x)}{g(x)}\right)^r\\
			&=1^r\\
			&=1.
		\end{align*}
		Thus $f(x)^r \sim g(x)^r$.
	\end{proof}
	\item $\log (f (x)) \sim \log(g(x))$ if  $\ \lim_{x\to\infty} g(x) =\infty$,
	\begin{proof}
		Write 
		\[
			f(x)=g(x)(1+h(x))
		\]
		where $h(x)>1$, $h(x)\to 0$ as $x\to\infty$. Then,
		\begin{align*}
			\log(f(x))&=\log(g(x))+\log(1+h(x))\\
			\frac{\log(f(x))}{\log(g(x))}&=1+\frac{\log(1+h(x))}{\log(g(x))}.
		\end{align*}
		From here we see that
		\begin{align*}
			\lim_{x\to\infty}\frac{\log(f(x))}{\log(g(x))}&=\lim_{x\to\infty}1+\frac{\log(1+h(x))}{\log(g(x))}\\
			&=1+\lim_{x\to\infty}\frac{\log(1+h(x))}{\log(g(x))}\\
			&=1.
		\end{align*}
		Thus, $\log (f (x)) \sim \log(g(x))$ if  $\ \lim_{x\to\infty} g(x) =\infty$.
	\end{proof}
	\item it is not necessarily true that $\exp(f(x)) \sim \exp(g(x))$.
	\begin{proof}
		\begin{align*}
			\lim_{x\to\infty}\frac{e^{f(x)}}{e^{g(x)}}&=\lim_{x\to\infty}e^{f(x)-g(x)}=e^{\lim_{x\to\infty}(f(x)-g(x))}.
		\end{align*}
		The above limit is equal to $1$ if we have $f(x)-g(x)=0$, which is not necessarily true with out given assumption that $\lim_{x\to\infty}\frac{f(x)}{g(x)}=1$. This is because if we have $\lim_{x\to\infty}f(x)=\infty$ and $\lim_{x\to\infty}f(x)=\infty$ then we cannot say anything about $\lim_{x\to\infty}f(x)-g(x)$. If the limits were however finite then we would be able to say $\exp(f(x)) \sim \exp(g(x))$, because we would have $\lim_{x\to\infty}f(x)=L$ and $\lim_{x\to\infty}f(x)=L$.
	\end{proof}
\end{enumerate}

\stepcounter{section}
\section*{Problem 4 [5 marks]} Show that if $f(x)$ satisfies $f(x) = x^2+O(x)$, and $f$ is differentiable with nondecreasing derivative $f'(x)$ for sufficiently large $x$, then $f'(x) = 2x + O(\sqrt{x})$.
\begin{proof}
	We want to show that 
	\[
		|f'(x)-2x|\leq C\sqrt{x}
	\]
	for some constant $C\in\R$.\\
	We know that $|f(x)-x^2|\leq Mx$ for some $x_0>0$, $M\in\R$ and all $x\geq x_0$. Comparing this bound for $f(x)$ and $f(x+h)$ ($h>0$) we obtain
	\begin{align*}
		f(x+h)-f(x)&=[(x+h)^2-x2]+[f(x+h)-(x+h)2]-[f(x)-x^2]\\
		&=2xh+h^2+O(x)
	\end{align*}
	\begin{align*}
		|f(x+h)-f(x)-2xh-h^2|&\leq M(x+h)+Mx\\
		&\leq 2Mx+Mh.
	\end{align*}
	We know by the mean value there exists $c\in[x,x+h]$ such that $f(x+h)-f(x)=hf'(c)$.
	\begin{align*}
		\left|hf'(c)-f(x+h)-f(x)\right|&=\left|hf'(c)-2xh-h^2\right|\\
		\left|hf'(c)-2xh-h^2\right|&\leq2Mx+Mh\\
		\left|f'(c)-2x-h\right|&\leq \frac{2Mx}{h}+M.
	\end{align*}
	Thus we have
	\[
		f'(c)=2x+h+C_1,
	\]
	with $C_1\leq \frac{2Mx}{h}+M$.\\
	Since $f'$ is non-decreasing for $c\in[x,x+h]$ we have
	\[
		f'(x)\leq f'(c)\leq f'(x+h).
	\]
	Using the previous bound for $f(c)$ we Obtain
	\[
		f(x)\leq 2x+h+\frac{2Mx}{h}+M
	\]
	for any $h>0$.\\
	We also need a lower bound to obtain a bound for the absolute value.\\
	Again by monotonicity for $c\in[x-h,x]$, $h>0$ we have
	\[
		f'(x-h)\leq f'(c)\leq f'(x).
	\]
	Here we proceed in a differnt manner than the first bound
	\[
		hf'(c)=f(x)-f(x-h)=[x^2-(x-h)^2]=2xh-h^2+O(x)
	\]
	\[
		|f(x)-f(x-h)-(2xh-h^2)|\leq Mx+M(x-h)\leq2Mx
	\]
	\[
		|hf'(c)-(2xh-h^2)|\leq2Mx
	\]
	\[
		|f'(c)-2x+h|\leq\frac{2Mx}{h}.
	\]
	Thus, by the monotonicity of $f'$ we have
	\[
		f'(x)\geq 2x-h-\frac{2Mx}{h}.
	\]
	Comparing the two bounds we obtain
	\[
		 2x-h-\frac{2Mx}{h}\leq f'(x)\leq 2x+h+\frac{2Mx}{h}+M,
	\]
	\[
		|f'(x)-2x|\leq h+\frac{2Mx}{h}+M.
	\]
	Now we need to pick an $h>0$ that gives us the desired error term.\\
	Pick $h=\sqrt{2M}\sqrt{x}$, then we have
	\begin{align*}
		|f'(x)-2x|&\leq h+\frac{2Mx}{h}+M\\
		&\leq \sqrt{2M}\sqrt{x}+\frac{2Mx}{\sqrt{2M}\sqrt{x}}+M\\
		&\leq 2\sqrt{2M}\sqrt{x}+M,
	\end{align*}
	as $x\to\infty$ we get $2\sqrt{2M}\sqrt{x}>M$ so we have
	\[
		|f'(x)-2x|\leq 2\sqrt{2M}\sqrt{x},
	\]
	if we take $C=2\sqrt{2M}$ we get 
	\[
		f'(x)=2x+O(\sqrt{x}). \qedhere
	\]
\end{proof}

\stepcounter{section}
\section*{Problem 5 [10 marks]} The Ramanujan sum is defined as 
\[
    c_{n}(m)=\sum_{\substack{h\leq n\\(h,n)=1}}\exp\left(2\pi i\frac{hm}{n}\right).
\] 
\begin{enumerate}\item Show that $c_{n}(m)=\sum_{d|(m,n)}d\mu(n/d)$. 
    \item Deduce that $\mu(n)=\sum_{h\leq n;(h,n)=1}\exp(2\pi i h/n)$.
\end{enumerate}

\stepcounter{section}
\section*{Problem 6 [5 marks]} For $x\geq e$, define $I(x)=\int_{e}^{x}\log\log t \;dt$. Obtain an estimate for $I(x)$ to within an error term $O(x/\log^2 x)$.
\begin{proof}
	We first apply integration by parts
	\begin{align*}
		I(x)&=\int_{e}^{x}\log\log t \;dt\\
		u=\log\log t,\quad dv=dt,\quad& du =\frac{1}{t\log t},\quad v=t\\
		&=\left[t\log\log t\right]^x_e-\int_{e}^{x}t\frac{1}{t\log t}\,dt\\
		&=\log\log x-\int_{e}^{x}\frac{1}{\log t}\,dt\\
		&=\log\log x-\int_{2}^{x}\frac{1}{\log t}\,dt+\int_{2}^{e}\frac{1}{\log t}\,dt\\
		&=\log\log x+\int_{2}^{e}\frac{1}{\log t}\,dt-\mathrm{Li}(x)\\
		&=\log\log x+\int_{2}^{e}\frac{1}{\log t}\,dt-\frac{2}{\log x}+O\left(\frac{x}{\log^2x}\right).\qedhere
	\end{align*}
\end{proof}

\stepcounter{section}
\section*{Problem 7 [5 marks]} Obtain an estimate for the sum \[\sum_{n\leq x}\frac{\log n}{n}\] with error term $O\left(\frac{\log x}{x}\right)$.\\
\textit{Solution.}\\
Let $a(x)=1$, and $f(x)=\frac{\log x}{x}$, then $f'(x)=\frac{1-\log x}{x^2}$, and $A(x)=\sum_{n\leq x}1=\floor{x}$. Using Abel's summation formula we can write the sum as
\[
	S(x)=\sum_{n\leq x}\frac{\log n}{n}=A(x)f(x)-\int_{1}^{x}A(t)f'(t)\,dt.
\]
Focusing on the integral
\[
	I=\int_{1}^{x}A(t)f'(t)\,dt=\int_{1}^{x}\floor{t}\frac{1-\log t}{t^2}\,dt
\]
writing $\floor{t}=t-\fract{t}$
\begin{align*}
	I&=\int_{1}^{x}(t-\fract{t})\frac{1-\log t}{t^2}\,dt\\
	&=\int_{1}^{x}\frac{1-\log t}{t}\,dt+\int_{1}^{x}\fract{t}\frac{\log t-1}{t^2}\,dt.
\end{align*}
\begin{align*}
	I_1&=\int_{1}^{x}\frac{1-\log t}{t}\,dt\\
	&=\int_{1}^{x}\frac{1}{t}\,dt-\int_{1}^{x}\frac{\log t}{t}\,dt
\end{align*}
\begin{align*}
	I_1'&=-\int_{1}^{x}\frac{\log t}{t}\,dt\\
	&=-\left[\frac{(\log t)^2}{2}\right]_1^x=-\frac{\log^2x}{2}.
\end{align*}
\[
	I_1=\log x-\frac{\log^2x}{2}.
\]
\begin{align*}
	I_2&=\int_{1}^{x}\fract{t}\frac{\log t-1}{t^2}\,dt\\
	&\leq\int_{1}^{x}\frac{\log t-1}{t^2}\,dt\\
	&\leq\int_{1}^{x}\frac{\log t}{t^2}\,dt
\end{align*}
\[
	\int_{1}^{x}\frac{\log t}{t^2}\,dt=\int_{1}^{\infty}\frac{\log t}{t^2}\,dt-\int_{x}^{\infty}\frac{\log t}{t^2}\,dt
\]
\begin{align*}
	\int_{1}^{\infty}\frac{\log t}{t^2}\,dt&=\left[-\frac{\log t+1}{t}\right]_1^\infty\\
	&=0+\frac{\log1+1}{1}\\
	&=1.
\end{align*}
\begin{align*}
	-\int_{x}^{\infty}\frac{\log t}{t^2}\,dt&=-\left[-\frac{\log t+1}{t}\right]_x^\infty\\
	&=0-\frac{\log x+1}{x}\\
	&=-\frac{\log x+1}{x}.
\end{align*}
As $x\to\infty$ $\log x+1\sim\log x$. Thus we have
\begin{align*}
	S(x)&=A(x)f(x)+I_1+I_2\\
	&\leq\floor{x}\frac{\log x}{x}+\log x-\frac{\log^2x}{2}+1-\frac{\log x+1}{x}\\
	&=\floor{x}\frac{\log x}{x}+\log x-\frac{\log^2x}{2}+O\left(\frac{\log x}{x}\right).
\end{align*}
\stepcounter{section}
\section*{Problem 8 [5 marks]} Prove
\[
    \sum_{n \le x} \frac{d(n)}{n} = \frac{1}{2} (\log  x)^2 + 2\gamma \log x  + O(1),
\]
where $\gamma$ is Euler's constant. 

{\it{Hint: Use the estimate proven in class for $\sum_{n\le x} d(n) $.}}
\begin{proof}
	Let $a(x)=d(x)$, $f(x)=x^{-1}$, then $f'(x)=-x^{-2}$, and $A(x)=\sum_{n\leq x}d(n)=x\log x+(2\gamma-1)x+O(\sqrt{x})$. Using Abel's summation formula to write the sum
	\begin{align*}
		S(x)=\sum_{n \le x} \frac{d(n)}{n}&=A(x)f(x)-\int_{1}^{x}A(t)f'(t)\,dt\\
		&=\frac{\sum_{n\leq x}d(n)}{x}+\int_{1}^{x}\sum_{n\leq x}d(t)t^{-2}\,dt\\
		&=\frac{x\log x+(2\gamma-1)x+O(\sqrt{x})}{x}+\int_{1}^{x}\frac{t\log t+(2\gamma-1)t+O(\sqrt{t})}{t^2}\,dt\\
		&=\log x+(2\gamma-1)+O\left(\frac{1}{\sqrt{x}}\right)+\int_{1}^{x}\frac{\log t}{t}+\frac{(2\gamma-1)}{t}+\frac{O(\sqrt{t})}{t^2}\,dt.
	\end{align*}
	Computing the first of these integrals
	\begin{align*}
		\int_{1}^{x}\frac{\log t}{t}\,dt&=\left[\frac{1}{2}\log^2t\right]^x_1\\
		&=\frac{1}{2}\log^2x.
	\end{align*}
	The second integral
	\begin{align*}
		\int_{1}^{x}\frac{(2\gamma-1)}{t}\,dt&=(2\gamma-1)\left[\log t\right]^x_1\\
		&=(2\gamma-1)\log x.
	\end{align*}
	For the third integral note that
	\[
		\frac{O(\sqrt{t})}{t^2}\leq\frac{\sqrt{t}}{t}=t^{-\frac{3}{2}}.
	\]
	This yields
	\begin{align*}
		\int_{1}^{x}\frac{O(\sqrt{t})}{t^2}&\leq\int_{1}^{x}t^{-\frac{3}{2}}\,dt\\
		&\leq\left[-2t^{-\frac{1}{2}}\right]_1^x\\
		&\leq2-2x^{-\frac{1}{2}}.
	\end{align*}
	Thus we have an error of $O(2-2x^{-\frac{1}{2}})$. As $x\to\infty$ $2-2x^{-\frac{1}{2}}\to2$, so our error term is $O(1)$. This also absorbs the $O\left(\frac{1}{\sqrt{x}}\right)$ error. Putting together all of the terms 
	\begin{align*}
		S(a)&=\log x+(2\gamma-1)+\frac{1}{2}\log^2x+(2\gamma-1)\log x+O(1)\\
		&=\frac{1}{2}\log^2x+2\gamma\log x+O(1)-1.
	\end{align*}
	The error absorbs the $-1$ and thus we have
	\[
		\sum_{n \le x} \frac{d(n)}{n} = \frac{1}{2} (\log  x)^2 + 2\gamma \log x  + O(1).\qedhere
	\]
\end{proof}

\stepcounter{section}
\section*{Problem 9 [10 marks]} Prove that 
\[
    \sum_{n\leq x; (n,k)=1}\frac{1}{n}\sim \frac{\phi(k)}{k}\log x,
\] 
as $x\to\infty$.

\stepcounter{section}
\section*{Problem 10 [5 marks]} Obtain an estimate for the sum
\[
    S(x) =\sum_{2\leq n\leq x}\frac{1}{n\log n}
\] 
with error term $O(1/(x \log x))$.\\
\textit{Solution.}\\
Let $a(n)=1$ and $f(n)=\frac{1}{n\log n}$, then by Abel's summation formula we have
\[
	S(x) =\sum_{2\leq n\leq x}\frac{1}{n\log n}=\sum_{2< n\leq x}\frac{1}{n\log n}+f(2)=\frac{\floor{x}}{x\log x}-\frac{2}{2\log 2}+\frac{1}{2\log2}+\int_{2}^{x}\floor{t}\left(\frac{\log t+1}{t^2\log^2t}\right)\,dt.
\]
Let
\[
	I=\int_{2}^{x}\floor{t}\left(\frac{\log t+1}{t^2\log^2t}\right)\,dt,
\]
and write $\floor{t}=t-\fract{t}$. Then, we obtain
\[
	I=\int_{2}^{x}(t-\fract{t})\left(\frac{\log t+1}{t^2\log^2t}\right)\,dt.
\]
Observe that
\[
	\frac{\log t+1}{t^2\log^2t}=\frac{1}{t^2\log t}+\frac{1}{t^2\log^2t},
\]
this gives
\[
	I=\int_{2}^{x}(t-\fract{t})\left(\frac{1}{t^2\log t}+\frac{1}{t^2\log^2t}\right)\,dt.
\]
Splitting the two integrals
\begin{align*}
	\int_{2}^{x}t\left(\frac{1}{t^2\log t}+\frac{1}{t^2\log^2t}\right)\,dt&=\int_{2}^{x}\frac{1}{t\log t}+\frac{1}{t\log^2t}\,dt\\
	&=\left[\log\log t -\frac{1}{\log t}\right]_2^x\\
	&=\log\log x-\log\log 2-\frac{1}{\log x}+\frac{1}{\log2}.
\end{align*}
As for the second integral
\[
	-\int_{2}^{x}\fract{t}\frac{\log t+1}{t^2\log^2t}\,dt=-\int_{2}^{\infty}\fract{t}\frac{\log t+1}{t^2\log^2t}\,dt+\int_{x}^{\infty}\fract{t}\frac{\log t+1}{t^2\log^2t}\,dt,
\]
the first of these two integrals is convergent since 
\[
	\int_{2}^{\infty}\fract{t}\frac{\log t+1}{t^2\log^2t}\,dt=\int_{2}^{5}\fract{t}\frac{\log t+1}{t^2\log^2t}\,dt+\int_{5}^{\infty}\fract{t}\frac{\log t+1}{t^2\log^2t}\,dt
\]
and $\frac{\log t+1}{t^2\log^2t}<\frac{1}{t^{\frac{3}{2}}}$ for $t\geq5$. We will call this constant $C$.\\
As for the second integral notice that for large enough $t$ we have $\log t+1\leq 2\log t$, this gives
\[
	\fract{t}\frac{\log t+1}{t^2\log^2t}\leq\fract{t}\frac{2\log t}{t^2\log^2t}=\frac{2}{t^2\log t}.
\]
Also note that for $t\in[x,\infty]$ we have
\[
	\frac{1}{\log t}\leq\frac{1}{\log x}.
\]
Combining these two facts we obtain
\begin{align*}
	\int_{x}^{\infty}\fract{t}\frac{\log t+1}{t^2\log^2t}\,dt&\leq \int_{x}^{\infty}\fract{t}\frac{2}{t^2\log x}\,dt\\
	&\leq \frac{1}{\log x}int_{x}^{\infty}\fract{t}\frac{2}{t^2}\,dt\\
	&\leq\frac{1}{\log x}\left[\frac{1}{t}\right]^\infty_x\\
	&\leq\frac{1}{x\log x}\\
	&=O\left(\frac{1}{x\log x}\right).
\end{align*}
Adding all of the terms
\begin{align*}
	S(x)&=\frac{\floor{x}}{x\log x}-\frac{1}{2\log 2}+\log\log x-\log\log 2-\frac{1}{\log x}+\frac{1}{\log2}-C+O\left(\frac{1}{x\log x}\right)\\
	\sum_{2\leq n\leq x}\frac{1}{n\log n}&=\log\log x+\frac{1}{2\log2}-\log\log2-C+O\left(\frac{1}{x\log x}\right)
\end{align*}
\end{document}