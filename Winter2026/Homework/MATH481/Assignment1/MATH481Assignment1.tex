\documentclass[12pt]{article}

% Import preambles and macros for homework
% Essential packages
\usepackage{amsmath, amsfonts, amssymb, amsthm}
\usepackage{mathtools}
\usepackage{enumitem}
\usepackage{graphicx}
\usepackage{wrapfig}
\usepackage{systeme}
\usepackage{caption}
\usepackage{soul}
\usepackage[dvipsnames]{xcolor}
\usepackage{fancyhdr}
\allowdisplaybreaks

% Page layout
\usepackage[
  top=2cm,
  bottom=2cm,
  left=2cm,
  right=2cm,
  headheight=17pt,
  includehead,includefoot,
  heightrounded,
]{geometry}


% pgfornament for title page decorations
\usepackage[object=vectorian]{pgfornament}

% Fancy header/footer setup
\pagestyle{fancy}
\setlength{\headheight}{14.49998pt}
\addtolength{\topmargin}{-2.49998pt}
\renewcommand{\footrulewidth}{0.4pt}
\setlength\parindent{15pt}
% Math notation shortcuts
\newcommand{\R}{\mathbb{R}}
\newcommand{\Q}{\mathbb{Q}}
\newcommand{\Z}{\mathbb{Z}}
\newcommand{\N}{\mathbb{N}}
\newcommand{\C}{\mathbb{C}}
\newcommand{\X}{\mathcal{X}}

% Theorem environments
\newtheorem{mainthm}{Theorem}[section]
\newtheorem{theorem}{Theorem}[section]  
\newtheorem{lemma}[theorem]{Lemma}
\newtheorem{proposition}[theorem]{Proposition}
\newtheorem{corollary}[theorem]{Corollary}
\newtheorem{definition}[theorem]{Definition}
\newtheorem{claim}[theorem]{Claim}

% Calculus
\newcommand{\diff}{\mathop{}\!\mathrm{d}}
\newcommand{\deriv}[2]{\frac{\mathrm{d}#1}{\mathrm{d}#2}}
\newcommand{\pderiv}[2]{\frac{\partial #1}{\partial #2}}

% Linear Algebra
\newcommand{\inner}[2]{\langle #1, #2 \rangle}
\newcommand{\norm}[1]{\| #1 \|}
\newcommand{\tr}{\operatorname{tr}}
\newcommand{\spn}{\operatorname{span}}
\newcommand{\rank}{\operatorname{rank}}
\newcommand{\nullity}{\operatorname{nullity}}

% Logic
\newcommand{\contra}{\Rightarrow\Leftarrow}

% Custom commands for notes
\newcommand{\todo}[1]{\textcolor{red}{[TODO: #1]}}
\newcommand{\important}[1]{\textbf{\textcolor{blue}{#1}}}

%Number Theory
\DeclareMathOperator{\Li}{Li}
\newcommand{\floor}[1]{\left\lfloor #1 \right\rfloor}
\newcommand{\fract}[1]{\left\{ #1 \right\}}




\newcommand{\maketitlepage}{
    \begin{titlepage}
        \centering
        \vspace*{2.0cm}
        \pgfornament{84}\\
        {\LARGE \textsc{\coursename}\par}
        \vspace{0.5cm}
        {\large\coursecode\par}
        \vspace{0.5cm}
        {\large\instructor\par}
        \vspace{1.5cm}
        {\huge\bfseries\assignment\par}
        \vspace{1cm}
        {\LARGE\itshape\author\par}
        \vspace{2cm}
        {\large\bfseries Due Date:\par}
        \vspace{0.5cm}
        {\Large \duedate}\\
        \pgfornament{84}
    \end{titlepage}
}
% =============================================
% HOMEWORK CONFIGURATION - EDIT THESE VALUES!
% =============================================

% Your personal info
\renewcommand{\author}{Deepak Jassal}
\newcommand{\authorlast}{Jassal}

% Course info
\newcommand{\coursename}{Course Name}
\newcommand{\coursecode}{Course code}
\newcommand{\instructor}{Instructor}

% Assignment-specific info (CHANGE THESE FOR EACH HOMEWORK)
\newcommand{\assignment}{Assignment }
\newcommand{\duedate}{Month Day\textsuperscript{th}, 20XX}

% Header configuration
\fancyhead[l]{\assignment}
\fancyhead[c]{\coursecode}
\fancyhead[r]{\monthyear}
\fancyfoot[c]{\authorlast{ }\thepage}

\renewcommand{\author}{Deepak Jassal}
\renewcommand{\authorlast}{Jassal}
\renewcommand{\coursename}{Analytic Number Theory}
\renewcommand{\coursecode}{MATH 481}
\renewcommand{\assignment}{Assignment 1}
\renewcommand{\instructor}{Dr. Alia Hamieh}
\renewcommand{\duedate}{January 29\textsuperscript{th}, 2026}


\begin{document}
\begin{titlepage}
	\centering
	\vspace*{2.0cm}	
	\pgfornament{84}\\
	{\LARGE \textsc{\coursename}\par}
	\vspace{0.5cm}
	{\large\coursecode\par}
    \vspace{0.5cm}
    {\large\instructor\par}
	\vspace{1.5cm}
	{\huge\bfseries\assignment\par}
	\vspace{1cm}  
	{\LARGE\itshape\author\par}
    \vspace{2cm}
	{\large\bfseries Due Date:\par}
	\vspace{0.5cm}
	{\Large \duedate}\\
	\pgfornament{84}
\end{titlepage}
\stepcounter{section}
\section*{Problem 1 [5 marks]} Let $p(x)$ be a polynomial with integer coefficients and degree $k\geq1$. Show that $p(n)$ is composite for infinitely many integers $n$.

\stepcounter{section}
\section*{Problem 2 [5 marks]} An arithmetic function $f$ is called periodic if there exists a positive integer $k$ such that $f(n+k)=f(n)$ for every $n\in\mathbb{N}$; the integer $k$ is called a period for $f$. Show that if $f$ is completely multiplicative and periodic with period $k$, then the values of $f$ are either $0$ or roots of unity. (An root of unity is a complex number $z$ such that $z^n=1$ for some $n\in\mathbb{N}$.)

\stepcounter{section}
\section*{Problem 3 [15 marks]}  Let $f,g$ be positive functions defined on $\mathbb{R}$. Suppose $f(x)\sim g(x)$. Prove that
\begin{enumerate}
	\item $f(x)^r \sim g(x)^r$ for any  real $r$,
	\begin{proof}
		\begin{align*}
			\lim_{x\to\infty}\frac{f^r(x)}{g^r(x)}&=\lim_{x\to\infty}\left(\frac{f(x)}{g(x)}\right)^r\\
			&=\left(\lim_{x\to\infty}\frac{f(x)}{g(x)}\right)^r\\
			&=1^r\\
			&=1.
		\end{align*}
		Thus $f(x)^r \sim g(x)^r$.
	\end{proof}
	\item $\log (f (x)) \sim \log(g(x))$ if  $\ \lim_{x\to\infty} g(x) =\infty$,
	\begin{proof}
		Write 
		\[
			f(x)=g(x)(1+h(x))
		\]
		where $h(x)\to 0$ as $x\to\infty$. Then,
		\begin{align*}
			\log(f(x))&=\log(g)+\log(1+h(x))\\
			\frac{\log(f(x))}{\log(g)}&=1+\frac{\log(1+h(x))}{\log(g)}.
		\end{align*}
		From here we see that
		\begin{align*}
			\lim_{x\to\infty}\frac{\log(f(x))}{\log(g(x))}&=\lim_{x\to\infty}1+\frac{\log(1+h(x))}{\log(g)}\\
			&=1+\lim_{x\to\infty}\frac{\log(1+h(x))}{\log(g)}\\
			&=1.
		\end{align*}
		Thus, $\log (f (x)) \sim \log(g(x))$ if  $\ \lim_{x\to\infty} g(x) =\infty$.
	\end{proof}
	\item it is not necessarily true that $\exp(f(x)) \sim \exp(g(x))$.
	\begin{proof}
		\begin{align*}
			\lim_{x\to\infty}\frac{e^{f(x)}}{e^{g(x)}}&=\lim_{x\to\infty}e^{f(x)-g(x)}.
		\end{align*}
		The above limit is equal to $1$ if we have $f(x)-g(x)=0$, which is not necessarily true with out given assumption that $\lim_{x\to\infty}\frac{f(x)}{g(x)}=1$. This is because if we have $\lim_{x\to\infty}f(x)=\infty$ and $\lim_{x\to\infty}f(x)=\infty$ then we cannot say anything about $\lim_{x\to\infty}f(x)-g(x)$. If the limits were however finite then we would be able to say $\exp(f(x)) \sim \exp(g(x))$, because we would have $\lim_{x\to\infty}f(x)=L$ and $\lim_{x\to\infty}f(x)=L$.
	\end{proof}
\end{enumerate}

\stepcounter{section}
\section*{Problem 4 [5 marks]} Show that if $f(x)$ satisfies $f(x) = x^2+O(x)$, and $f$ is differentiable with nondecreasing derivative $f'(x)$ for sufficiently large $x$, then $f'(x) = 2x + O(\sqrt{x})$.
\begin{proof}
	We want to show that 
	\[
		|f'(x)-2x|\leq C\sqrt{x}
	\]
	for some constant $C\in\R$.\\
	We know that $|f(x)-x^2|\leq Mx$ for some $x_0>0$, $M\in\R$ and all $x\geq x_0$. Comparing this bound for $f(x)$ and $f(x+h)$ ($h>0$) we obtain
	\begin{align*}
		f(x+h)-f(x)&=[(x+h)^2-x2]+[f(x+h)-(x+h)2]-[f(x)-x^2]\\
		&=2xh+h^2+O(x)
	\end{align*}
	\begin{align*}
		|f(x+h)-f(x)-2xh-h^2|&\leq M(x+h)+Mx\\
		&\leq 2Mx+Mh.
	\end{align*}
	We know by the mean value there exists $c\in[x,x+h]$ such that $f(x+h)-f(x)=hf'(c)$.
	\begin{align*}
		\left|hf'(c)-f(x+h)-f(x)\right|&=\left|hf'(c)-2xh-h^2\right|\\
		\left|hf'(c)-2xh-h^2\right|&\leq2Mx+Mh\\
		\left|f'(c)-2x-h\right|&\leq \frac{2Mx}{h}+M.
	\end{align*}
	Thus we have
	\[
		f'(c)=2x+h+C_1,
	\]
	with $C_1\leq \frac{2Mx}{h}+M$.\\
	Since $f'$ is non-decreasing for $c\in[x,x+h]$ we have
	\[
		f'(x)\leq f'(c)\leq f'(x+h).
	\]
	Using the previous bound for $f(c)$ we Obtain
	\[
		f(x)\leq 2x+h+\frac{2Mx}{h}+M
	\]
	for any $h>0$.\\
	We also need a lower bound to obtain a bound for the absolute value.\\
	Again by monotonicity for $c\in[x-h,x]$, $h>0$ we have
	\[
		f'(x-h)\leq f'(c)\leq f'(x).
	\]
	Here we proceed in a differnt manner than the first bound
	\[
		hf'(c)=f(x)-f(x-h)=[x^2-(x-h)^2]=2xh-h^2+O(x)
	\]
	\[
		|f(x)-f(x-h)-(2xh-h^2)|\leq Mx+M(x-h)\leq2Mx
	\]
	\[
		|hf'(c)-(2xh-h^2)|\leq2Mx
	\]
	\[
		|f'(c)-2x+h|\leq\frac{2Mx}{h}.
	\]
	Thus, by the monotonicity of $f'$ we have
	\[
		f'(x)\geq 2x-h-\frac{2Mx}{h}.
	\]
	Comparing the two bounds we obtain
	\[
		 2x-h-\frac{2Mx}{h}\leq f'(x)\leq 2x+h+\frac{2Mx}{h}+M,
	\]
	\[
		|f'(x)-2x|\leq h+\frac{2Mx}{h}+M.
	\]
	Now we need to pick an $h>0$ that gives us the desired error term.\\
	Pick $h=\sqrt{2M}\sqrt{x}$, then we have
	\begin{align*}
		|f'(x)-2x|&\leq h+\frac{2Mx}{h}+M\\
		&\leq \sqrt{2M}\sqrt{x}+\frac{2Mx}{\sqrt{2M}\sqrt{x}}+M\\
		&\leq 2\sqrt{2M}\sqrt{x}+M,
	\end{align*}
	as $x\to\infty$ we get $2\sqrt{2M}\sqrt{x}>M$ so we have
	\[
		|f'(x)-2x|\leq 2\sqrt{2M}\sqrt{x},
	\]
	if we take $C=2\sqrt{2M}$ we get 
	\[
		f'(x)=2x+O(\sqrt{x}). \qedhere
	\]
\end{proof}

\stepcounter{section}
\section*{Problem 5 [10 marks]} The Ramanujan sum is defined as 
\[
    c_{n}(m)=\sum_{\substack{h\leq n\\(h,n)=1}}\exp\left(2\pi i\frac{hm}{n}\right).
\] 
\begin{enumerate}\item Show that $c_{n}(m)=\sum_{d|(m,n)}d\mu(n/d)$. 
    \item Deduce that $\mu(n)=\sum_{h\leq n;(h,n)=1}\exp(2\pi i h/n)$.
\end{enumerate}

\stepcounter{section}
\section*{Problem 6 [5 marks]} For $x\geq e$, define $I(x)=\int_{e}^{x}\log\log t \;dt$. Obtain an estimate for $I(x)$ to within an error term $O(x/\log^2 x)$.

\stepcounter{section}
\section*{Problem 7 [5 marks]} Obtain an estimate for the sum \[\sum_{n\leq x}\frac{\log n}{n}\] with error term $O\left(\frac{\log x}{x}\right)$.

\stepcounter{section}
\section*{Problem 8 [5 marks]} Prove
\[
    \sum_{n \le x} \frac{d(n)}{n} = \frac{1}{2} (\log  x)^2 + 2\gamma \log x  + O(1),
\]
where $\gamma$ is Euler's constant. 

{\it{Hint: Usethe estimate proven in class for $\sum_{n\le x} d(n) $.}}

\stepcounter{section}
\section*{Problem 9 [10 marks]} Prove that 
\[
    \sum_{n\leq x; (n,k)=1}\frac{1}{n}\sim \frac{\phi(k)}{k}\log x,
\] 
as $x\to\infty$.

\stepcounter{section}
\section*{Problem 10 [5 marks]} Obtain an estimate for the sum
\[
    S(x) =\sum_{2\leq n\leq x}\frac{1}{n\log n}
\] 
with error term $O(1/(x \log x))$.


\end{document}