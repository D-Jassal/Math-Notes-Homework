\documentclass[12pt]{article}

% Import preambles and macros for homework
% Essential packages
\usepackage{amsmath, amsfonts, amssymb, amsthm}
\usepackage{mathtools}
\usepackage{enumitem}
\usepackage{graphicx}
\usepackage{wrapfig}
\usepackage{systeme}
\usepackage{caption}
\usepackage{soul}
\usepackage[dvipsnames]{xcolor}
\usepackage{fancyhdr}
\allowdisplaybreaks

% Page layout
\usepackage[
  top=2cm,
  bottom=2cm,
  left=2cm,
  right=2cm,
  headheight=17pt,
  includehead,includefoot,
  heightrounded,
]{geometry}


% pgfornament for title page decorations
\usepackage[object=vectorian]{pgfornament}

% Fancy header/footer setup
\pagestyle{fancy}
\setlength{\headheight}{14.49998pt}
\addtolength{\topmargin}{-2.49998pt}
\renewcommand{\footrulewidth}{0.4pt}
\setlength\parindent{15pt}
% Math notation shortcuts
\newcommand{\R}{\mathbb{R}}
\newcommand{\Q}{\mathbb{Q}}
\newcommand{\Z}{\mathbb{Z}}
\newcommand{\N}{\mathbb{N}}
\newcommand{\C}{\mathbb{C}}
\newcommand{\X}{\mathcal{X}}

% Theorem environments
\newtheorem{mainthm}{Theorem}[section]
\newtheorem{theorem}{Theorem}[section]  
\newtheorem{lemma}[theorem]{Lemma}
\newtheorem{proposition}[theorem]{Proposition}
\newtheorem{corollary}[theorem]{Corollary}
\newtheorem{definition}[theorem]{Definition}
\newtheorem{claim}[theorem]{Claim}

% Calculus
\newcommand{\diff}{\mathop{}\!\mathrm{d}}
\newcommand{\deriv}[2]{\frac{\mathrm{d}#1}{\mathrm{d}#2}}
\newcommand{\pderiv}[2]{\frac{\partial #1}{\partial #2}}

% Linear Algebra
\newcommand{\inner}[2]{\langle #1, #2 \rangle}
\newcommand{\norm}[1]{\| #1 \|}
\newcommand{\tr}{\operatorname{tr}}
\newcommand{\spn}{\operatorname{span}}
\newcommand{\rank}{\operatorname{rank}}
\newcommand{\nullity}{\operatorname{nullity}}

% Logic
\newcommand{\contra}{\Rightarrow\Leftarrow}

% Custom commands for notes
\newcommand{\todo}[1]{\textcolor{red}{[TODO: #1]}}
\newcommand{\important}[1]{\textbf{\textcolor{blue}{#1}}}

%Number Theory
\DeclareMathOperator{\Li}{Li}
\newcommand{\floor}[1]{\left\lfloor #1 \right\rfloor}
\newcommand{\fract}[1]{\left\{ #1 \right\}}




\newcommand{\maketitlepage}{
    \begin{titlepage}
        \centering
        \vspace*{2.0cm}
        \pgfornament{84}\\
        {\LARGE \textsc{\coursename}\par}
        \vspace{0.5cm}
        {\large\coursecode\par}
        \vspace{0.5cm}
        {\large\instructor\par}
        \vspace{1.5cm}
        {\huge\bfseries\assignment\par}
        \vspace{1cm}
        {\LARGE\itshape\author\par}
        \vspace{2cm}
        {\large\bfseries Due Date:\par}
        \vspace{0.5cm}
        {\Large \duedate}\\
        \pgfornament{84}
    \end{titlepage}
}
% =============================================
% HOMEWORK CONFIGURATION - EDIT THESE VALUES!
% =============================================

% Your personal info
\renewcommand{\author}{Deepak Jassal}
\newcommand{\authorlast}{Jassal}

% Course info
\newcommand{\coursename}{Course Name}
\newcommand{\coursecode}{Course code}
\newcommand{\instructor}{Instructor}

% Assignment-specific info (CHANGE THESE FOR EACH HOMEWORK)
\newcommand{\assignment}{Assignment }
\newcommand{\duedate}{Month Day\textsuperscript{th}, 20XX}

% Header configuration
\fancyhead[l]{\assignment}
\fancyhead[c]{\coursecode}
\fancyhead[r]{\monthyear}
\fancyfoot[c]{\authorlast{ }\thepage}

\renewcommand{\author}{Deepak Jassal}
\renewcommand{\authorlast}{Jassal}
\renewcommand{\coursename}{Analytic Number Theory}
\renewcommand{\coursecode}{MATH 481}
\renewcommand{\assignment}{Assignment 2}
\renewcommand{\instructor}{Dr. Alia Hamieh}
\renewcommand{\duedate}{February 24\textsuperscript{th}, 2026}


\begin{document}
\begin{titlepage}
	\centering
	\vspace*{2.0cm}	
	\pgfornament{84}\\
	{\LARGE \textsc{\coursename}\par}
	\vspace{0.5cm}
	{\large\coursecode\par}
    \vspace{0.5cm}
    {\large\instructor\par}
	\vspace{1.5cm}
	{\huge\bfseries\assignment\par}
	\vspace{1cm}  
	{\LARGE\itshape\author\par}
    \vspace{2cm}
	{\large\bfseries Due Date:\par}
	\vspace{0.5cm}
	{\Large \duedate}\\
	\pgfornament{84}
\end{titlepage}
\stepcounter{section}
\section*{Problem 1 [10 Marks]} Let $k\geq 2$ be an integer. We define $d_{k}(n)$ as the number of ways $n$ can be written as the product of $k$ positive integers.
\begin{enumerate}
    \item Verify that $d_{k}(n)=\sum_{d|n}d_{k-1}(d)$\\
    \textit{Solution.} Let $a_1,\dots,a_k\in\N$ be such that $a_1a_2\cdots a_k=n$. Set $d=a_1a_2\cdots a_{k-1}$. Then $d\mid n$ because $a_k=\frac{n}{d}$. Once $d$ is fixed $a_k$ is determined, and for each divisor $d$ of $n$ we have $d_{k-1}(d)$ factorizations of the ordered tuple $(a_1,\dots,a_{k-1},\frac{n}{d})$. When summing over $d\mid n$ we get
	\[
		d_{k}(n)=\sum_{d|n}d_{k-1}(d).
	\]
    \item Show that 
    \[
        \sum_{n\leq x}d_{k}(n)=\frac{x(\log x)^{k-1}}{(k-1)!}+O\left(x(\log x)^{k-2}\right)
    \]
	\begin{proof}
		In part (a) we showed that $d_k=d_{k-1}\ast1$, ($d_k=\underbrace{1\ast\cdots\ast1}_{k-\text{times}}$).
		\begin{align*}
			\sum_{n\leq x}d_k(n)&=\sum_{n\leq x}\sum_{m\mid n}d_{k-1}(m)\\
			&=\sum_{m\leq x}d_{k-1}(m)\sum_{\substack{n\leq x\\m\mid n}}1\\
			&=\sum_{m\leq x}d_{k-1}(m)\floor{\frac{x}{m}}\\
			&=\sum_{m\leq x}d_{k-1}(m)\left(\frac{x}{m}-\fract{\frac{x}{m}}\right)\\
			&=x\sum_{m\leq x}\frac{d_{k-1}(m)}{m}-\sum_{m\leq x}d_{k-1}(m)\fract{\frac{x}{m}}.
		\end{align*}
		Since
		\[
			\left|\sum_{m\leq x}d_{k-1}(m)\fract{\frac{x}{m}}\right|\leq\sum_{m\leq x}d_{k-1}(m)
		\]
		we can rewrite the above as
		\[
			\sum_{n\leq x}d_k(n)=x\sum_{m\leq x}\frac{d_{k-1}(m)}{m}-O\left(\sum_{m\leq x}d_{k-1}(m)\right).
		\]
		Let $a(n)=d_{k-1}(n)$, $A(x)=\sum_{m\leq x}d_{k-1}(m)$ and $f(n)=\frac{1}{n}$. By Abels summation formula we have
		\[
			\sum_{m\leq x}\frac{d(m)}{m}=\frac{A(x)}{x}+\int_{1}^{x}\frac{A(t)}{t^2}\,dt.
		\]
		\begin{claim}
			\[
				A_m(t)=\frac{t(\log t)^{m-1}}{(m-1)!}+O(t(\log t)^{m-2})
			\]
			for $m>1$.
		\end{claim}
		\begin{proof}[Proof of the Claim]
			Case of $m=2$. From class we have
			\[
				\sum_{n\leq x}d_2(n)=x\log x+(2\gamma-1)x+O(\sqrt{x})=\frac{x(\log x)^1}{1!}+O(x(\log x)^0)
			\]
			Assume the formula holds for $k-1\ge 2$, i.e.
			\[
			A_{k-1}(x) := \sum_{n\le x} d_{k-1}(n) = \frac{x(\log x)^{k-2}}{(k-2)!} + O\bigl(x(\log x)^{k-3}\bigr).
			\]
			We prove it for $k$.

			From part (a), $d_k(n)=\sum_{d|n} d_{k-1}(d)$. Hence
			\[
			A_k(x) = \sum_{n\le x}\sum_{d|n} d_{k-1}(d) = \sum_{d\le x} d_{k-1}(d) \sum_{\substack{n\le x\\ d|n}} 1.
			\]
			The inner sum equals $\lfloor x/d\rfloor = x/d + O(1)$. Therefore
			\[
			A_k(x) = x\sum_{d\le x}\frac{d_{k-1}(d)}{d} + O\Bigl(\sum_{d\le x} d_{k-1}(d)\Bigr).
			\]
			By the inductive hypothesis $\sum_{d\le x}d_{k-1}(d) = O(x(\log x)^{k-2})$, so
			\[
			A_k(x) = x\sum_{d\le x}\frac{d_{k-1}(d)}{d} + O\bigl(x(\log x)^{k-2}\bigr).
			\]

			Now evaluate $\sum_{d\le x}\frac{d_{k-1}(d)}{d}$ by Abel's formula. Let $B(t)=\sum_{d\le t} d_{k-1}(d)$. Then
			\[
			\sum_{d\le x}\frac{d_{k-1}(d)}{d} = \frac{B(x)}{x} + \int_1^x \frac{B(t)}{t^2}\,dt.
			\]
			By the inductive hypothesis,
			\[
			B(t) = \frac{t(\log t)^{k-2}}{(k-2)!} + O\bigl(t(\log t)^{k-3}\bigr).
			\]
			Hence
			\[
			\frac{B(x)}{x} = \frac{(\log x)^{k-2}}{(k-2)!} + O\bigl((\log x)^{k-3}\bigr),
			\]
			and
			\[
			\int_1^x \frac{B(t)}{t^2}\,dt = \frac{1}{(k-2)!}\int_1^x \frac{(\log t)^{k-2}}{t}\,dt + O\Bigl(\int_1^x \frac{(\log t)^{k-3}}{t}\,dt\Bigr).
			\]
			Substituting $u=\log t$ gives
			\[
			\int_1^x \frac{(\log t)^{k-2}}{t}\,dt = \int_0^{\log x} u^{k-2}\,du = \frac{(\log x)^{k-1}}{k-1},
			\]
			\[
			\int_1^x \frac{(\log t)^{k-3}}{t}\,dt = \frac{(\log x)^{k-2}}{k-2} = O\bigl((\log x)^{k-2}\bigr).
			\]
			Thus
			\[
			\int_1^x \frac{B(t)}{t^2}\,dt = \frac{(\log x)^{k-1}}{(k-1)!} + O\bigl((\log x)^{k-2}\bigr).
			\]
			Adding $\frac{B(x)}{x}$,
			\[
			\sum_{d\le x}\frac{d_{k-1}(d)}{d} = \frac{(\log x)^{k-1}}{(k-1)!} + O\bigl((\log x)^{k-2}\bigr).
			\]

			Multiplying by $x$ and adding the earlier error term $O(x(\log x)^{k-2})$,
			\[
			A_k(x) = \frac{x(\log x)^{k-1}}{(k-1)!} + O\bigl(x(\log x)^{k-2}\bigr).
			\]
			This completes the induction.
		\end{proof}
		From the induction we have
		\[
			\frac{A(x)}{x}=\frac{(\log x)^{k-2}}{(k-2)!}+O((\log x)^{k-3}).	
		\]
		\[
			\int_{1}^{x}\frac{A(t)}{t^2}\,dt=\int_{1}^{x}\frac{\frac{(t\log t)^{k-2}}{(k-2)!}+O((t\log t)^{k-3})}{t^2}\,dt=\frac{1}{(k-2)!}\int_{1}^{x}\frac{(t\log t)^{k-2}}{t^2}\,dt+\int_{1}^{x}\frac{O((t\log t)^{k-3})}{t^2}\,dt.
		\]
		Evaluating the first of these integrals with the substitution
		\[
			u=\log t,\quad du=\frac{dt}{t},\quad u(1)=0,\quad u(x)=\log x
		\]
		\begin{align*}
			I_1&=\frac{1}{(k-2)!}\int_{1}^{x}\frac{(\log t)^{k-2}}{t}\,dt\\
			&=\frac{1}{(k-2)!}\int_{0}^{\log x}u^{k-2}\,du\\
			&=\frac{1}{(k-2)!}\frac{(\log x)^{k-1}}{k-1}\\
			&=\frac{(\log x)^{k-1}}{(k-1)!}.
		\end{align*}
		Evaluating the second of these integrals
		\begin{align*}
			\int_{1}^{x}\frac{O((t\log t)^{k-3})}{t^2}\,dt&=O\left(\int_{1}^{x}\frac{(t\log t)^{k-3}}{t^2}\,dt\right)\\
			&=O((\log x)^{k-2}).
		\end{align*}
		The error term from the original summation can also be evaluated using the formula derived by induction.
		\begin{align*}
			O\left(\sum_{m\leq x}d_{k-1}(m)\right)&=O\left(\frac{x(\log x)^{k-2}}{(k-2)!}+O(x(\log x)^{k-3})\right)\\
			&=O(x(\log x)^{k-2}).
		\end{align*}
		Thus, combining all of the terms
		\[
			\sum_{n\leq x}d_{k}(n)=\frac{x(\log x)^{k-1}}{(k-1)!}+O\left(x(\log x)^{k-2}\right).\qedhere
		\]
	\end{proof}
\end{enumerate}
\newpage
\stepcounter{section}
\section*{Problem 2 [10 Marks]} In this question, any constants arising in the main term of the estimate should be worked out explicitly. 
\begin{enumerate}
    \item  Obtain an estimate for the sum $S(x)=\displaystyle{\sum_{n\leq x;\; n\;\text{odd}}\frac{1}{n}}$ with error term $O(1/x)$.\\
    \textit{Solution.} We can rewrite the odd condition for $n$ by $n=2k-1$ for $k\geq 1$. $n\leq 2k-1\Rightarrow \leq\frac{x+1}{2}$. Let $K=\floor{\frac{x+1}{2}}$ Rewriting the sum
	\[
		S(x)=\sum_{k=1}^{K}\frac{1}{2k-1}.
	\]
	This can be related back to the harmonic series
	\[
		H_{2k}=\sum_{m=1}^{2K}\frac{1}{m}
	\]
	and then splitting the sum
	\[
		H_{2k}=\sum_{k=1}^{K}\frac{1}{2k-1}+\sum_{k=1}^{K}\frac{1}{2K}.
	\]
	Rearranging we get
	\[
		S(x)=H_{2K}-\frac{1}{2}H_K.
	\]
	Subbing in the approximation
	\[
		\sum_{n\leq x}\frac{1}{x}=\log x+\gamma+O\left(\frac{1}{x}\right)
	\]
	we get
	\[
		S(x)=\log 2K+\gamma-\frac{1}{2}(\log K+\gamma)+O\left(\frac{1}{K}\right).
	\]
	Collecting terms
	\[
		S(x)=\log2+\frac{1}{2}\log K+\frac{\gamma}{2}+O\left(\frac{1}{K}\right).
	\]
	Furthermore, 
	\[
		K=\floor{\frac{x+1}{2}}=\frac{x+1}{2}-\fract{\frac{x+1}{2}}=\frac{x+1}{2}+O(1)
	\]
	subbing this in
	\[
		S(x)=\frac{1}{2}\log x+\frac{\log2+\gamma}{2}+O\left(\frac{1}{x}\right).
	\]
    \item Let $D(x) =\displaystyle{\sum_{n\leq x;\;n\;\text{odd}}d(n)}$, where $d(n)$ is the divisor function. Give an estimate for $D(x)$
    with error term $O(\sqrt{x})$.\\
	{\it Hint: Use Dirichlet's hyperbola method and the result of the previous part}.\\
	\textit{Solution.} 
	\[
		D(x)=\sum_{\substack{ab\leq x\\a,b\text{ odd}}}1.
	\]
	Applying Dirichlet's hyperbola method to the above sum with $y=\sqrt{x}$ and $a,b\leq y$.
	\[
		D(x)=\sum_{\substack{a\leq y\\ a\text{ odd}}}\sum_{\substack{b\leq \frac{x}{a}\\ b\text{ odd}}}1+\sum_{\substack{b\leq y\\ b\text{ odd}}}\sum_{\substack{a\leq \frac{x}{b}\\ a\text{ odd}}}1-\sum_{\substack{a\leq y\\ a\text{ odd}}}\sum_{\substack{b\leq \frac{x}{y}\\ b\text{ odd}}}1,
	\]
	by symmetry we have
	\[
		D(x)=2\sum_{\substack{a\leq y\\ a\text{ odd}}}\sum_{\substack{b\leq \frac{x}{a}\\ b\text{ odd}}}1-\left(\sum_{\substack{a\leq y\\ a\text{ odd}}}1\right)^2.
	\]
	Working on the first sum
	\begin{align*}
		\sum_{\substack{a\leq y\\ a\text{ odd}}}\sum_{\substack{b\leq \frac{x}{a}\\ b\text{ odd}}}1&=\sum_{\substack{a\leq y\\ a\text{ odd}}}\floor{\frac{x}{2a}}\\
		&=\sum_{\substack{a\leq y\\ a\text{ odd}}}\left(\frac{x}{2a}-\fract{\frac{x}{2a}}\right)\\
		&=\sum_{\substack{a\leq y\\ a\text{ odd}}}\frac{x}{2a}-\sum_{\substack{a\leq y\\ a\text{ odd}}}\fract{\frac{x}{2a}}\\
		&=\frac{x}{2}\sum_{\substack{a\leq y\\ a\text{ odd}}}\frac{1}{a}-\sum_{\substack{a\leq y\\ a\text{ odd}}}O\left(1\right).
	\end{align*}
	From part (a) we have
	\begin{align*}
		\sum_{\substack{a\leq y\\ a\text{ odd}}}\sum_{\substack{b\leq \frac{x}{a}\\ b\text{ odd}}}1&=\frac{x}{2}\left(\frac{1}{2}\log y+\frac{\log2+\gamma}{2}+O\left(\frac{1}{y}\right)\right)+O(y)\\
		&=\frac{x}{4}\log y+\frac{\log2+\gamma}{4}x+O\left(\frac{x}{y}\right)+O(y)\\
		&=\frac{x}{4}\log \sqrt{x}+\frac{\log2+\gamma}{4}x+O\left(\sqrt{x}\right).
	\end{align*}
	Now working on the second of the sums
	\begin{align*}
		\left(\sum_{\substack{a\leq y\\ a\text{ odd}}}1\right)^2&=\left(\floor{\frac{y}{2}}\right)^2\\
		&=\left(\frac{y}{2}-\fract{\frac{y}{2}}\right)^2\\
		&=\left(\frac{y}{2}-O(1)\right)^2\\
		&=\frac{y^2}{4}+O(y)\\
		&=\frac{x}{4}+O(\sqrt{x}).
	\end{align*}
	Combining all of the sums
	\[
		D(x)=\frac{x}{4}\log x+\frac{\log4+2\gamma-1}{4}x+O\left(\sqrt{x}\right).
	\]
\end{enumerate}


\stepcounter{section}
\section*{Problem 3 [10 Marks]} Use the Dirichlet hyperbola method (or some other method) to obtain an estimate for the sum $\displaystyle{\sum_{n\leq x}\frac{d(n)}{n}}$ with an error term $O\left(\frac{\log x}{\sqrt{x}}\right)$\\
\textit{Solution.} First we rewrite the divisor function using
\[
	d(n)=\sum_{ab=n}1.
\]
Substituting this into the sum
\begin{align*}
	S(x)=\sum_{n\leq x}\frac{d(n)}{n}&=\sum_{n\leq x}\frac{1}{n}\sum_{ab=n}1\\
	&=\sum_{ab\leq x}\frac{1}{ab}\\
	&=\sum_{a\leq x}\sum_{b\leq\frac{x}{a}}\frac{1}{ab}.
\end{align*}
We can now apply the Dirichlet hyperbola method with $y=\sqrt{x}$, $a,b\leq y$.
\begin{align*}
	S(x)&=\sum_{a\leq y}\sum_{b\leq\frac{x}{a}}\frac{1}{ab}+\sum_{b\leq y}\sum_{a\leq\frac{x}{b}}\frac{1}{ab}-\sum_{a\leq y}\sum_{b\leq\frac{x}{y}}\frac{1}{ab}\\
	&=2\sum_{a\leq y}\sum_{b\leq\frac{x}{a}}\frac{1}{ab}-\sum_{a\leq y}\sum_{b\leq y}\frac{1}{ab}\\
	&=2\sum_{a\leq y}\frac{1}{a}\sum_{b\leq\frac{x}{a}}\frac{1}{b}-\left(\sum_{a\leq y}\frac{1}{a}\right)^2.
\end{align*}
Working on the first of these sums
\begin{align*}
	\sum_{a\leq y}\frac{1}{a}\sum_{b\leq\frac{x}{a}}\frac{1}{b}&=\sum_{a\leq y}\frac{1}{a}\left(\log\left(\frac{x}{a}\right)+\gamma+O\left(\frac{a}{x}\right)\right)\\
	&=\sum_{a\leq y}\frac{\log x}{a}-\sum_{a\leq y}\frac{\log a}{a}+\gamma\sum_{a\leq y}\frac{1}{a}+O\left(\frac{1}{x}\sum_{a\leq y}1\right)\\
	&=\sum_{a\leq \sqrt{x}}\frac{\log x}{a}-\sum_{a\leq \sqrt{x}}\frac{\log a}{a}+\gamma\sum_{a\leq \sqrt{x}}\frac{1}{a}+O\left(\frac{1}{x}\sum_{a\leq \sqrt{x}}1\right)\\
	&=(\log x+\gamma)\sum_{a\leq \sqrt{x}}\frac{1}{a}-\sum_{a\leq \sqrt{x}}\frac{\log a}{a}+O\left(\frac{1}{\sqrt{x}}\right)\\
	&=(\log x+\gamma)\left(\log\sqrt{x}+\gamma+O\left(\frac{1}{\sqrt{x}}\right)\right)-\sum_{a\leq \sqrt{x}}\frac{\log a}{a}+O\left(\frac{1}{\sqrt{x}}\right)\\
	&=(\log x+\gamma)\left(\frac{1}{2}\log x+\gamma+O\left(\frac{1}{\sqrt{x}}\right)\right)-\sum_{a\leq \sqrt{x}}\frac{\log a}{a}+O\left(\frac{1}{\sqrt{x}}\right)\\
	&=\frac{1}{2}\log^2 x+\frac{3\gamma}{2}\log x+\gamma^2-\sum_{a\leq \sqrt{x}}\frac{\log a}{a}+O\left(\frac{\log x}{\sqrt{x}}\right).
\end{align*}
Using Abel's summation formula for the sum with $a(n)=1$, and $f(n)=\frac{\log n}{n}$.
\begin{align*}
	S_1(x)&=\sum_{a\leq \sqrt{x}}\frac{\log a}{a}\\
	&=\floor{\sqrt{x}}\frac{\log \sqrt{x}}{\sqrt{x}}-\int_{1}^{\sqrt{x}}\floor{t}\frac{1-\log t}{t^2}\,dt\\
	&=\log \sqrt{x}-\fract{\sqrt{x}}\frac{\log \sqrt{x}}{\sqrt{x}}-\int_{1}^{\sqrt{x}}(t-\fract{t})\frac{1-\log t}{t^2}\,dt\\
	&=\log \sqrt{x}-\int_{1}^{\sqrt{x}}(t-\fract{t})\frac{1-\log t}{t^2}\,dt+O(1).
\end{align*}
Working on the integral
\begin{align*}
	I&=\int_{1}^{\sqrt{x}}(t-\fract{t})\frac{1-\log t}{t^2}\,dt\\
	&=\int_{1}^{\sqrt{x}}t\frac{1-\log t}{t^2}\,dt-\int_{1}^{\sqrt{x}}\fract{t}\frac{1-\log t}{t^2}\,dt\\
	I_1&=\int_{1}^{\sqrt{x}}\frac{1-\log t}{t}\,dt\\
	u=\log t,\quad du=\frac{dt}{t},&\quad u(1)=0,\quad u(\sqrt{x})=\log\sqrt x\\
	I_1&=\int_{0}^{\log\sqrt{x}}1-u\,du\\
	I_1&=\left(u-\frac{u^2}{2}\right)_0^{\log\sqrt{x}}\\
	I_1&=\log\sqrt{x}-\frac{\log^2\sqrt{x}}{2}.
\end{align*}
Working on the second integral
\begin{align*}
	|I_2|&\leq\int_{1}^{\sqrt{x}}\left|\fract{t}\frac{1-\log t}{t^2}\right|\,dt\\
	&\leq\int_{1}^{\sqrt{x}}\left|\frac{\log t}{t^2}\right|\,dt\\
	\int_{1}^{\sqrt{x}}\frac{\log t}{t^2}\,dt&=-\left(\frac{\log t}{t}\right)_{1}^{\sqrt{x}}+\int_{1}^{\sqrt{x}}\frac{1}{t^2}\,dt\\
	&=-\left(\frac{\log t}{t}+\frac{1}{t}\right)_{1}^{\sqrt{x}}\\
	&=-\left(\frac{\log t+1}{t}\right)_{1}^{\sqrt{x}}\\
	&=1-\frac{\log x+2}{2\sqrt{x}}.
\end{align*}
With this we can bound $I_2$ with
\[
	I_2=O\left(\frac{\log x}{\sqrt{x}}\right).
\]
Working on the last sum from the hyperbola method
\begin{align*}
	\left(\sum_{a\leq \sqrt{x}}\frac{1}{a}\right)^2&=\left(\frac{1}{2}\log x+\gamma+O\left(\frac{1}{\sqrt{x}}\right)\right)^2\\
	&=\frac{1}{4}(\log x)^2+\gamma\log x+\gamma^2+O\left(\frac{\log x}{\sqrt{x}}\right).
\end{align*}
Then we have
\begin{align*}
	S_1(x)&=\log\sqrt{x}-I_1+I_2+O(1)\\
	&=\log\sqrt{x}-\log\sqrt{x}+\frac{\log^2\sqrt{x}}{2}+O\left(\frac{\log x}{\sqrt{x}}\right)\\
	&=\frac{\log^2\sqrt{x}}{2}+O\left(\frac{\log x}{\sqrt{x}}\right).\\
\end{align*}
All in all
\begin{align*}
	S(x)&=2\left(\frac{1}{2}\log^2 x+\frac{3\gamma}{2}\log x+\gamma^2-\frac{\log^2\sqrt{x}}{2}\right)+\frac{1}{4}(\log x)^2+\gamma\log x+\gamma^2+O\left(\frac{\log x}{\sqrt{x}}\right)\\
	&=\log^2 x+3\gamma\log x+2\gamma^2-\log^2\sqrt{x}+\frac{1}{4}(\log x)^2+\gamma\log x+\gamma^2+O\left(\frac{\log x}{\sqrt{x}}\right)\\
	&=\frac{5}{4}\log^2 x+3\gamma\log x-\frac{1}{4}\log^2 x+\gamma\log x+3\gamma^2+O\left(\frac{\log x}{\sqrt{x}}\right)\\
	&=\log^2 x+4\gamma\log x+3\gamma^2+O\left(\frac{\log x}{\sqrt{x}}\right).
\end{align*}
\stepcounter{section}
\section*{Problem 4 [10 Marks]} Obtain an estimate, similar to the estimate for $\displaystyle{\sum_{n\leq x}\frac{1}{n}}$ for the sum $\displaystyle{\sum_{n\leq x}\frac{1}{\varphi(n)}}$.\\
{\it Hint: Use the convolution method.}\\
\textit{Solution.} We start by rewriting $\frac{1}{\varphi(n)}$. 
\begin{align*}
	\frac{1}{\varphi(n)}&=\frac{1}{n}\sum_{d\mid n}\frac{\mu^2(d)}{\varphi(d)}\\
	&=\sum_{d\mid n}\frac{d}{n}\cdot\frac{\mu^2(d)}{d\varphi(d)}\\
	&=\sum_{d\mid n}\frac{1}{\frac{n}{d}}\cdot\frac{\mu^2(d)}{d\varphi(d)}\\
	&=\left(\frac{1}{id}\ast\frac{\mu^2}{id\varphi}\right)(n).
\end{align*}
Now rewriting our sum
\begin{align*}
	S(x)&=\sum_{n\leq x}\sum_{d\mid n}\frac{1}{\frac{n}{d}}\cdot\frac{\mu^2(d)}{d\varphi(d)}\\
	&=\sum_{d\leq x}\frac{\mu^2(d)}{d\varphi(d)}\sum_{\substack{n\leq x\\d\mid n}}\frac{d}{n}.
\end{align*}
Let $n=dk$ then $n\leq x\Rightarrow k\leq\frac{x}{d}$.
\begin{align*}
	S(x)&=\sum_{d\leq x}\frac{\mu^2(d)}{d\varphi(d)}\sum_{k\leq\frac{x}{d}}\frac{1}{k}\\
	&=\sum_{d\leq x}\frac{\mu^2(d)}{d\varphi(d)}\left(\log\left(\frac{x}{d}\right)+\gamma+O\left(\frac{d}{x}\right)\right)\\
	&=\log x\sum_{d\leq x}\frac{\mu^2(d)}{d\varphi(d)}-\sum_{d\leq x}\frac{\log d\mu^2(d)}{d\varphi(d)}+\gamma\sum_{d\leq x}\frac{\mu^2(d)}{d\varphi(d)}+O\left(\frac{1}{x}\sum_{d\leq x}\frac{\mu^2(d)}{\varphi(d)}\right).
\end{align*}
To evaluate the sum $\sum_{d\leq x}\frac{\mu^2(d)}{d\varphi(d)}$ please note that 
\[
	\sum_{d\geq 1}\frac{\mu^2(d)}{d\varphi(d)}=\sum_{\substack{d\geq 1\\d\text{ squarefree}}}\frac{1}{d\varphi(d)}.
\]
$\frac{1}{d\varphi(d)}$ is multiplicative because for $(a,b)=1$ we have $\frac{1}{ab\varphi(ab)}=\frac{1}{a\varphi(a)}\frac{1}{b\varphi(b)}$. So we can take the Euler product giving
\[
	\sum_{\substack{d\geq 1\\d\text{ squarefree}}}\frac{1}{d\varphi(d)}=\prod_{p}\left(1+\frac{1}{p(p-1)}\right)
\]
and since 
\[
	1+\frac{1}{p(p-1)}=1+\frac{1}{p^2}+O\left(\frac{1}{p^3}\right)
\]
and the sum $\frac{1}{p^2}$ converges we have that the original sum also converges to some value, say $C$. To find the error see that for the tail of the sum
\[
	\phi(n)\gg \frac{n}{\log\log n}\Rightarrow \frac{1}{\phi(n)}\ll\frac{\log\log n}{n}
\]
we have
\begin{align*}
	\sum_{d\geq 1}\frac{\mu^2(d)}{d\varphi(d)}&\ll\sum_{\substack{d\leq x\\d\text{ squarefree}}}\frac{\log\log d}{d^2}\\
	&\ll\int_{1}^{\infty}\frac{\log\log t}{t^2}\,dt\\
	\int_{1}^{\infty}\frac{\log\log t}{t^2}\,dt&=\left(-\frac{\log\log t}{t}\right)^\infty_{x}+\int_{x}^{\infty}\frac{1}{t^2\log t}\,dt\\
	\sum_{d\geq 1}\frac{\mu^2(d)}{d\varphi(d)}&\ll\frac{\log\log x}{x}.
\end{align*}
Thus we have
\[
	\sum_{d\geq 1}\frac{\mu^2(d)}{d\varphi(d)}=C+O\left(\frac{\log\log x}{x}\right)
\]
And this sum converges to some finite constant, Say $C$.\\
A similar argument gives that 
\[
	\sum_{d\geq 1}\frac{\log d\mu^2(d)}{d\varphi(d)}=A+O\left(\frac{\log x\log\log x}{x}\right)
\]
for some constant $A$.\\
For the error term we have
\begin{align*}
	\sum_{d\leq x}\frac{1}{\phi(d)}&\ll\sum_{d\leq x}\frac{\log\log d}{d}\\
	\sum_{d\leq x}\frac{\log\log d}{d}&\leq\int_{2}^{x}\frac{\log\log t}{t}\,dt\\
	\int_{2}^{x}\frac{\log\log t}{t}\,dt&\leq\frac{1}{2}(\log\log x)^2\\
	\sum_{d\leq x}\frac{1}{\phi(d)}&\ll(\log\log x)^2\\
	\frac{1}{x}\sum_{d\leq x}\frac{1}{\phi(d)}&\ll\frac{(\log\log x)^2}{x}\\
	O\left(\frac{1}{x}\sum_{d\leq x}\frac{1}{\phi(d)}\right)&=O\left(\frac{(\log\log x)^2}{x}\right).
\end{align*}
Adding all the components
\begin{align*}
	S(x)&=\log x\sum_{d\leq x}\frac{\mu^2(d)}{d\varphi(d)}-\sum_{d\leq x}\frac{\log d\mu^2(d)}{d\varphi(d)}+\gamma\sum_{d\leq x}\frac{\mu^2(d)}{d\varphi(d)}+O\left(\frac{1}{x}\sum_{d\leq x}\frac{\mu^2(d)}{\varphi(d)}\right)\\
	&=(\log x+\gamma)\left(C+O\left(\frac{\log\log x}{x}\right)\right)-A+O\left(\frac{\log x\log\log x}{x}\right)+O\left(\frac{(\log\log x)^2}{x}\right)\\
	&=C(\log x+\gamma)-A+O\left(\frac{\log x\log\log x}{x}\right).
\end{align*}
\stepcounter{section}
\section*{Problem 5 [10 Marks]} Using only Mertens' type estimates (but not the PNT), obtain an asymptotic estimate for the partial sums $\displaystyle{S(x) =\sum_{p\leq x;\; p\;\text{prime} }\frac{1}{p\log p}}$ with as good an error term as you can get using only results at the level of Mertens.\\
\textit{Solution.} In the following, let $p\leq n$ mean all primes lesser than or equal to $n$. Let $a(n)=\frac{1}{n}$ and $f(n)=\frac{1}{\log n}$, then by Abels summation formula we have
\[
	S(x)=\frac{1}{\log x}\sum_{p\leq x}\frac{1}{p}+\int_{2}^{x}\frac{\displaystyle\sum_{p\leq t}\frac{1}{p}}{t\log^2 t}\,dt.
\]
From Mertens' theorem we know that
\[
	\sum_{p\leq x}\frac{1}{p}=\log\log x+M+O\left(\frac{1}{\log x}\right).
\]
Substituting
\[
	S(x)=\frac{\log\log x+M+O\left(\frac{1}{\log x}\right)}{\log x}+\int_{2}^{x}\frac{\log\log t+M+O\left(\frac{1}{\log t}\right)}{t\log^2 t}\,dt.
\]
Working on the integral
\begin{align*}
	I&=\int_{2}^{x}\frac{\log\log t+M+O\left(\frac{1}{\log t}\right)}{t\log^2 t}\,dt\\
	&=\int_{2}^{x}\frac{\log\log t}{t\log^2t}\,dt+\int_{2}^{x}\frac{M}{t\log^2t}\,dt+\int_{2}^{x}O\left(\frac{1}{t\log^3t}\right)\,dt.
\end{align*}
Let $u=\log t$, $du=\frac{dt}{t}$, $u(2)=\log 2$, $u(x)=\log x$
\begin{align*}
	\int_{2}^{x}\frac{\log\log t}{t\log^2t}\,dt&=\int_{\log2}^{\log x}\frac{\log u}{u^2}\,dt\\
	&=\left(-\frac{\log u+1}{u}\right)_{\log2}^{\log x}\\
	&=\frac{\log\log2+1}{\log2}-\frac{\log\log x+1}{\log x}.
\end{align*}
\begin{align*}
	M\int_{2}^{x}\frac{1}{t\log^2t}\,dt&=M\left(-\frac{1}{\log t}\right)_{2}^{x}\\
	&=M\left(\frac{1}{\log2}-\frac{1}{\log x}\right).
\end{align*}
\[
	\int_{2}^{x}\frac{1}{t\log^3t}\\,dt=\int_{2}^{x}O\left(\frac{1}{u^3}\right)\,du,
\]
so the third integral converges to some number giving
\[
	O\left(\int_{2}^{x}\frac{1}{t\log^3t}\right)\,dt=O(1).
\]
Collecting terms
\begin{align*}
	S(x)&=\frac{\log\log x}{\log x}+\frac{M}{\log x}+O\left(\frac{1}{\log^2 x}\right)+\frac{\log\log2+1}{\log2}-\frac{\log\log x+1}{\log x}+\frac{M}{\log2}-\frac{M}{\log x}+O(1)\\
	&=-\frac{1}{\log x}+\frac{\log\log2+1+M}{\log2}+O\left(\frac{1}{\log^2x}\right).
\end{align*}
\stepcounter{section}
\section*{Problem 6 [30 Marks]} Let $q\in\mathbb{N}$, and consider the group $\left(\mathbb{Z}_{q}\right)^{\times}$ of invertible elements modulo $q$. A homomorphism $\chi:\left(\mathbb{Z}_{q}\right)^{\times}\leftarrow\mathbb{C}^{\times}$ is called a Dirichlet character modulo $q$. By Euler's theorem $a^{\varphi(q)}\equiv 1\mod q$ for any $a\in \left(\mathbb{Z}_{q}\right)^{\times}$, and so $\left(\chi(a)\right)^{\varphi(q)}=1$ for all $a\in \left(\mathbb{Z}_{q}\right)^{\times}$. In other words, each $\chi(a)$ is a $\varphi(q)$-th root of unity. 
We extend the definition of $\chi$ to all integers by setting \[\chi(n)=\begin{cases}\chi(n\;\mathrm{mod}\; q)&\;\; \text{if}\; (n,q)=1\\ 0&\;\;\text{otherwise.}\end{cases}\]
If $\chi$ and $\chi$ are characters modulo $q$, then so is the product $\chi\chi$ as well as $\overline{\chi}$ given by $\overline{\chi}(a)=\overline{\chi(a)}$. In fact, the set of all characters modulo $q$ forms a group of order $\varphi(q)$.

The character $\chi_{0}$ satisfying $\chi_0(a)=1$ for all $a\in \left(\mathbb{Z}_{q}\right)^{\times}$ and $\chi_0(a)=0$ if $(a,q)\neq1$ is called the trivial character modulo $q$. 

\begin{enumerate}[label=(\alph*)]
    \item Show that $\chi$ is a completely multiplicative arithmetic function.\\
    \textit{Solution.} We need to show that $\chi(1)=1$ and $\chi(ab)=\chi(a)\chi(b)$ for any $a,b$. Since $(1,q)=1$ we have $\chi(1)=\chi(1\mod q)=1$. For any $a,b$ assume without loss of generality that $(a,q)\neq1$ then $(ab,q)\neq1$ so we have $\chi(ab)=0$ and $\chi(a)\chi(b)=0\times\chi(b)=0=\chi(ab)$. If $(a,q)=1$ and $(b,q)=1$ we have $(ab,q)=1$ and $ab\in(\Z_q)^\times$. Since $\chi$ acts as a group homomorphism we have $\chi(ab)=\chi(a)\chi(b)$.
	\newpage
    \item If $\chi\neq \chi_0$, show that $\displaystyle{\sum_{a(\mathrm{mod}\; q)}\chi(a)}=0$.
    \begin{proof}
		Since if $(a,q)\neq1$ $\chi(a)=0$ we can rewrite the sum as 
		\[
			S(x)=\sum_{a\in(\Z_q^\times)}\chi(a).
		\]
		Find $b\in(\Z_q)^\times$ such that $\chi(b)\neq1$, this is possible since $\chi\neq\chi_0$. Then 
		\begin{align*}
			S(x)&=\sum_{a\in(\Z_q^\times)}\chi(ab)\\
			&=\chi(b)\sum_{a\in(\Z_q^\times)}\chi(a)\\
			S(x)&=\chi(b)S(x)\\
			S(x)(1-\chi(b))&=0.
		\end{align*}
		Since $\chi(b)\neq1$ it must be that $S(x)=0$.
	\end{proof}
    \item Show that \[\sum_{\chi(\mathrm{mod}q)}\chi(n)=\begin{cases}\varphi(q)&\;\;\text{if}\;n\equiv1\mathrm{mod}\; q\\ 0&\;\;\text{otherwise.}\end{cases}\]
	\begin{proof}
		Let $G = (\mathbb{Z}_q)^\times$. Then $G$ is a finite abelian group of order $|G|=\varphi(q)$. The set of Dirichlet characters modulo $q$ is the set of group homomorphisms
		\[
			\widehat{G}=\mathrm{Hom}(G,\mathbb{C}^\times),
		\]
		and it is a group under pointwise multiplication. Also, $|\widehat{G}|=\varphi(q)$. We look at the sum in cases.
		Let $n\in\mathbb{Z}$.\\
		Case 1: $(n,q)\neq 1$. By definition of Dirichlet characters, $\chi(n)=0$ for every character $\chi$. Hence,
		\[
			\sum_{\chi (\mathrm{mod}\, q)} \chi(n)=0.
		\]
		Case 2: $(n,q)=1$. Then $n$ determines an element $a\in G$, namely $a\equiv n\pmod q$. We look at 
		\[
			S(a)=\sum_{\chi\in\widehat{G}} \chi(a)
		\]
		in cases.
		Subcase 2a: $a=1$. If $n\equiv 1\pmod q$, then $\chi(1)=1$ for every character $\chi$. Therefore,
		\[
			S(1)=\sum_{\chi\in\widehat{G}} 1 =|\widehat{G}| =\varphi(q).
		\]
		Subcase 2b: $a\neq1$. Find $b\in(\Z_q)^\times$ such that $\chi(b)\neq1$, this is possible since $\chi\neq\chi_0$. Then 
		\begin{align*}
			S(x)&=\sum_{a\in(\Z_q^\times)}\chi(ab)\\
			&=\chi(b)\sum_{a\in(\Z_q^\times)}\chi(a)\\
			S(x)&=\chi(b)S(x)\\
			S(x)(1-\chi(b))&=0.
		\end{align*}
		Since $\chi(b)\neq1$ it must be that $S(x)=0$.
		Because $\chi(a)\neq 1$, we have $S(a)=0$. Thus, we have
		\[
			\sum_{\chi(\mathrm{mod}\, q)} \chi(n) = 
			\begin{cases} 
				\varphi(q) & \text{if } n\equiv 1 \pmod q,\\ 
				0 & \text{otherwise.}
			\end{cases}.\qedhere
		\]
	\end{proof}
    \item For $\chi\neq\chi_0$, show that $\displaystyle{\left|\sum_{n\leq x}\chi(n)\right|\leq q}$. Deduce that \[L(s,\chi)=\sum_{n=1}^{\infty}\frac{\chi(n)}{n^s}\] is convergent for $\Re(s)>0$.
    \begin{proof}
		We know that
		\[
			\sum_{n\leq q}\chi(n)=0.
		\]
		Write $x=mq+r$ with $0\leq r< q$ then 
		\[
			\sum_{n\leq x}\chi(n)=\sum_{k=0}^{m-1}\sum_{n=kq+1}^{(k+1)q}\chi(n)+\sum_{n=mq+1}^{mq+r}\chi(n)
		\]
		the inner sums are 0 so
		\[
			\left|\sum_{n\leq x}\chi(n)\right|=\left|\sum_{n=mq+1}^{mq+r}\chi(n)\right|\leq r<q.
		\]
		So we have 
		\[
			\left|\sum_{n\leq x}\chi(n)\right|\leq q.
		\]
		Let $A(x)=\sum_{n\leq x}\chi(n)$ and $L(s,\chi)=\sum_{n\geq1}\frac{\chi(n)}{n^s}$ with $\Re(s)>0$. Then 
		\[
			\sum_{n=1}^{N}\frac{\chi(n)}{n^s}=\sum_{n=1}^{N-1}A(n)(n^{-s}-(n+1)^{-s})+A(N)N^{-s}.
		\]
		Clearly $A(N)N^{-s}\to0$ and $N\to\infty$. Let $\sigma=\Re(s)>0$ then 
		\[
			|n^{-s}-(n+1)^{-s}|=O(n^{-\sigma-1}) 
		\]
		so
		\[
			\left|\sum_{n=1}^{N-1}A(n)(n^{-s}-(n+1)^{-s})\right|\leq\sum_{n=1}^{N-1}qCn^{-\sigma-1}
		\]
		which is also approches 0 as $N\to\infty$. Thus, $L(s,\chi)$ converges for $\Re(s)>0$.
	\end{proof}
    \item Define $f(n)=\sum_{d|n}\chi(d)$.  Suppose that $\chi$ is a real Dirichlet character modulo $q$ (i.e. $\chi(a)=0,\pm1$ for all $a$). Show that $f(1)=1$ and $f(n)\geq0$. Moreover, show that $f(n)\geq 1$ whenever $n$ is a perfect square.
    \begin{proof}
		From part (a) we know that $f(1)=\chi(1)=1$ and that $\chi$ is multiplicative. This gives that $f$ is also multiplicative. Write $n=p_1^{e_1}\cdots p_n^{e_n}$ for unique primes $p_i$ $1\leq i\leq n$ and each $e_j\in\N$ with $1\leq j\leq n$. So $f(n)=\prod_{i=1}^{n}\chi(p_i)^{e_i}$. Next we compute $f$ at prime powers. $f(p^j)=\sum_{i=0}^{j}\chi(p^i)$, but $\chi(p^i)=\chi(p)^i$ and $\chi(p)\in\{-1,0,1\}$. 
		\begin{enumerate}
			\item[Case 1.] $\chi(p)=0$ then $f(p^i)=1$ 
			\item[Case 2.] $\chi(p)=1$ then $f(p^i)=i+1\geq1$
			\item[Case 3.] $\chi(p)=-1$ then $f(p^i)=\sum_{j=0}^{i}(-1)^j$ with if $i$ is odd we have $f(p^i)=1$ and if $i$ is even we have $f(p^i)=1$. 
		\end{enumerate}
		So $f(n)\geq0$. If $n$ is a perfect square each $e_i$ is even so we have $f(p_i^{e_i})\geq1$.
	\end{proof}
	\newpage
    \item Suppose $\chi\neq\chi_0$. Using Dirichlet's hyperbola method, prove that \[\sum_{n\leq x}\frac{f(n)}{\sqrt{n}}=2L(1,\chi)\sqrt{x}+O(1).\]
    \begin{proof}
		Let
		\[
			S(x) = \sum_{n \le x} \frac{f(n)}{\sqrt{n}}, \qquad 
			f(n) = \sum_{d|n} \chi(d), \quad \chi \neq \chi_0.
		\]
		\begin{align*}
			S(x) &= \sum_{n \le x} \frac{1}{\sqrt{n}} \sum_{d|n} \chi(d)\\
			&= \sum_{dm \le x} \frac{\chi(d)}{\sqrt{dm}}\\
			&= \sum_{d \le x} \frac{\chi(d)}{\sqrt{d}} \sum_{m \le x/d} \frac{1}{\sqrt{m}}.
		\end{align*}
		Define
		\[
			A(t) = \sum_{m \le t} \frac{1}{\sqrt{m}}.
		\]
		By comparing with the integral \(\int_1^t u^{-1/2}\,du\), we have
		\[
			A(t) = 2\sqrt{t} + C + O(t^{-1/2}), \quad C = \zeta(1/2).
		\]
		Take \(y = \sqrt{x}\). 
		\[
			S(x)=\sum{d\leq\sqrt{x}}\frac{\chi(d)}{\sqrt{d}}\sum_{m\leq\frac{x}{d}}\frac{1}{\sqrt{m}}+\sum_{m\leq x}\frac{1}{\sqrt{m}}\sum{d\leq\frac{x}{m}}\frac{\chi(d)}{\sqrt{d}}-\sum_{d\leq\sqrt{x}}\sum_{m\leq\sqrt{x}}\frac{\chi(d)}{\sqrt{dm}}.
		\]
		\begin{align*}
			S_1(x) &= \sum_{d \le y} \frac{\chi(d)}{\sqrt{d}} A\left( \frac{x}{d} \right)\\
			&=\sum_{d \le y} \frac{\chi(d)}{\sqrt{d}} \left[ 2\sqrt{\frac{x}{d}} + C + O\left( \sqrt{\frac{d}{x}} \right) \right]\\
			&=2\sqrt{x} \sum_{d \le y} \frac{\chi(d)}{d}+ C \sum_{d \le y} \frac{\chi(d)}{\sqrt{d}}+ \sum_{d \le y} \frac{\chi(d)}{\sqrt{d}} \, O\!\left( \sqrt{\frac{d}{x}} \right)
		\end{align*}
		\begin{itemize}
			\item First term: \(\displaystyle \sum_{d \le y} \frac{\chi(d)}{d} = L(1,\chi) + O(1/y)\) by partial summation from bounded \(\sum_{d \le t} \chi(d)\). Hence,
			\[
				2\sqrt{x} \sum_{d \le y} \frac{\chi(d)}{d} = 2\sqrt{x} L(1,\chi) + O(1).
			\]
			\item Second term: \(\displaystyle \sum_{d \le y} \frac{\chi(d)}{\sqrt{d}}\) converges again it is \(O(1)\). Hence, \(C \sum_{d \le y} \frac{\chi(d)}{\sqrt{d}} = O(1)\).
			\item Third term: \(|\chi(d)| \le 1\) and \(\sqrt{d/x} \le 1\) for \(d \le y = \sqrt{x}\), so
			\[
				\sum_{d \le y} \frac{\chi(d)}{\sqrt{d}} \, O\!\left( \sqrt{\frac{d}{x}} \right)
				= O\!\left( \frac{1}{\sqrt{x}} \sum_{d \le y} 1 \right) = O(1).
			\]
		\end{itemize}
		Therefore,
		\[
			S_1(x) = 2\sqrt{x} L(1,\chi) + O(1).
		\]
		For \(d > y\) we have \(m \le x/d < x/y = y\).
		\[
			S_2(x) = \sum_{m < y} \frac{1}{\sqrt{m}} \sum_{d \le x/m} \frac{\chi(d)}{\sqrt{d}}.
		\]
		Let \(K = \lim_{t \to \infty} \sum_{d \le t} \frac{\chi(d)}{\sqrt{d}}\). Then
		\[
			\sum_{d \le x/m} \frac{\chi(d)}{\sqrt{d}} = K + O\!\left( \frac{1}{\sqrt{x/m}} \right).
		\]
		Hence,
		\[
			S_2(x) = \sum_{m < y} \frac{1}{\sqrt{m}} \left[ K + O\!\left( \sqrt{\frac{m}{x}} \right) \right]
			= K \sum_{m < y} \frac{1}{\sqrt{m}} + O\!\left( \frac{1}{\sqrt{x}} \sum_{m < y} 1 \right).
		\]
		Now \(\sum_{m < y} 1 = O(\sqrt{x})\) and \(\sum_{m < y} 1/\sqrt{m} = 2\sqrt{y} + C + O(1/\sqrt{y}) = 2 x^{1/4} + O(1)\). Thus,
		\[
			S_2(x) = 2K x^{1/4} + O(1).
		\]
		\[
			S_3(x) = \sum_{d \le y} \sum_{m \le y} \frac{\chi(d)}{\sqrt{dm}}
			= \left( \sum_{d \le y} \frac{\chi(d)}{\sqrt{d}} \right) \left( \sum_{m \le y} \frac{1}{\sqrt{m}} \right) = O(1) \cdot O(x^{1/4}) = O(x^{1/4}).
		\]
		\[
			S(x) = S_1(x) + S_2(x) - S_3(x) = \big( 2\sqrt{x} L(1,\chi) + O(1) \big) + \big( 2K x^{1/4} + O(1) \big) - O(x^{1/4}).
		\]
		Thus,
		\[
			\sum_{n \le x} \frac{f(n)}{\sqrt{n}}  = 2\sqrt{x} L(1,\chi) + O(1).\qedhere
		\]
	\end{proof}
    \item If $\chi\neq \chi_0$ is a real character, show that $L(1,\chi)\neq0$.
    \begin{proof}
		Assume for the sake of contradiction that $L(1,\chi)=0$, then from (f) we have
		\[
			\sum_{n\leq x}\frac{f(n)}{\sqrt{n}}=2L(1,\chi)\sqrt{x}+O(1)=O(1).
		\]
		Now, take $x=N^2$ and consider $n=m^2\leq N^2$. We have $f(m^2)\geq1$ and $\frac{1}{\sqrt{n}}=\frac{1}{m}$. Then
		\[
			\sum_{n\leq N^2}\frac{f(n)}{\sqrt{n}}\geq\sum_{m\leq N}\frac{1}{m}\sim \log N.
		\]
		But as $N\to\infty$ $\log N\to\infty$. This is a contradiction since the left hand side tends to $\infty$ but the right hand side if $O(1)$. Thus, $L(1,\chi)\neq0$.
	\end{proof}
\end{enumerate}


\end{document}