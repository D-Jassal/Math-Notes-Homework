\documentclass[12pt]{article}

% Import preambles and macros for homework
% Essential packages
\usepackage{amsmath, amsfonts, amssymb, amsthm}
\usepackage{mathtools}
\usepackage{enumitem}
\usepackage{graphicx}
\usepackage{wrapfig}
\usepackage{systeme}
\usepackage{caption}
\usepackage{soul}
\usepackage[dvipsnames]{xcolor}
\usepackage{fancyhdr}
\allowdisplaybreaks

% Page layout
\usepackage[
  top=2cm,
  bottom=2cm,
  left=2cm,
  right=2cm,
  headheight=17pt,
  includehead,includefoot,
  heightrounded,
]{geometry}


% pgfornament for title page decorations
\usepackage[object=vectorian]{pgfornament}

% Fancy header/footer setup
\pagestyle{fancy}
\setlength{\headheight}{14.49998pt}
\addtolength{\topmargin}{-2.49998pt}
\renewcommand{\footrulewidth}{0.4pt}
\setlength\parindent{15pt}
% Math notation shortcuts
\newcommand{\R}{\mathbb{R}}
\newcommand{\Q}{\mathbb{Q}}
\newcommand{\Z}{\mathbb{Z}}
\newcommand{\N}{\mathbb{N}}
\newcommand{\C}{\mathbb{C}}
\newcommand{\X}{\mathcal{X}}

% Theorem environments
\newtheorem{mainthm}{Theorem}[section]
\newtheorem{theorem}{Theorem}[section]  
\newtheorem{lemma}[theorem]{Lemma}
\newtheorem{proposition}[theorem]{Proposition}
\newtheorem{corollary}[theorem]{Corollary}
\newtheorem{definition}[theorem]{Definition}
\newtheorem{claim}[theorem]{Claim}

% Calculus
\newcommand{\diff}{\mathop{}\!\mathrm{d}}
\newcommand{\deriv}[2]{\frac{\mathrm{d}#1}{\mathrm{d}#2}}
\newcommand{\pderiv}[2]{\frac{\partial #1}{\partial #2}}

% Linear Algebra
\newcommand{\inner}[2]{\langle #1, #2 \rangle}
\newcommand{\norm}[1]{\| #1 \|}
\newcommand{\tr}{\operatorname{tr}}
\newcommand{\spn}{\operatorname{span}}
\newcommand{\rank}{\operatorname{rank}}
\newcommand{\nullity}{\operatorname{nullity}}

% Logic
\newcommand{\contra}{\Rightarrow\Leftarrow}

% Custom commands for notes
\newcommand{\todo}[1]{\textcolor{red}{[TODO: #1]}}
\newcommand{\important}[1]{\textbf{\textcolor{blue}{#1}}}

%Number Theory
\DeclareMathOperator{\Li}{Li}
\newcommand{\floor}[1]{\left\lfloor #1 \right\rfloor}
\newcommand{\fract}[1]{\left\{ #1 \right\}}




\newcommand{\maketitlepage}{
    \begin{titlepage}
        \centering
        \vspace*{2.0cm}
        \pgfornament{84}\\
        {\LARGE \textsc{\coursename}\par}
        \vspace{0.5cm}
        {\large\coursecode\par}
        \vspace{0.5cm}
        {\large\instructor\par}
        \vspace{1.5cm}
        {\huge\bfseries\assignment\par}
        \vspace{1cm}
        {\LARGE\itshape\author\par}
        \vspace{2cm}
        {\large\bfseries Due Date:\par}
        \vspace{0.5cm}
        {\Large \duedate}\\
        \pgfornament{84}
    \end{titlepage}
}
% =============================================
% HOMEWORK CONFIGURATION - EDIT THESE VALUES!
% =============================================

% Your personal info
\renewcommand{\author}{Deepak Jassal}
\newcommand{\authorlast}{Jassal}

% Course info
\newcommand{\coursename}{Course Name}
\newcommand{\coursecode}{Course code}
\newcommand{\instructor}{Instructor}

% Assignment-specific info (CHANGE THESE FOR EACH HOMEWORK)
\newcommand{\assignment}{Assignment }
\newcommand{\duedate}{Month Day\textsuperscript{th}, 20XX}

% Header configuration
\fancyhead[l]{\assignment}
\fancyhead[c]{\coursecode}
\fancyhead[r]{\monthyear}
\fancyfoot[c]{\authorlast{ }\thepage}

\renewcommand{\author}{Deepak Jassal}
\renewcommand{\authorlast}{Jassal}
\renewcommand{\coursename}{Analytic Number Theory}
\renewcommand{\coursecode}{MATH 481}
\renewcommand{\assignment}{Assignment 2}
\renewcommand{\instructor}{Dr. Alia Hamieh}
\renewcommand{\duedate}{February 24\textsuperscript{th}, 2026}


\begin{document}
\begin{titlepage}
	\centering
	\vspace*{2.0cm}	
	\pgfornament{84}\\
	{\LARGE \textsc{\coursename}\par}
	\vspace{0.5cm}
	{\large\coursecode\par}
    \vspace{0.5cm}
    {\large\instructor\par}
	\vspace{1.5cm}
	{\huge\bfseries\assignment\par}
	\vspace{1cm}  
	{\LARGE\itshape\author\par}
    \vspace{2cm}
	{\large\bfseries Due Date:\par}
	\vspace{0.5cm}
	{\Large \duedate}\\
	\pgfornament{84}
\end{titlepage}
\stepcounter{section}
\section*{Problem 1 [10 Marks]} Let $k\geq 2$ be an integer. We define $d_{k}(n)$ as the number of ways $n$ can be written as the product of $k$ positive integers.
\begin{enumerate}
    \item Verify that $d_{k}(n)=\sum_{d|n}d_{k-1}(d)$
    \item Show that 
    \[
        \sum_{n\leq x}d_{k}(n)=\frac{x(\log x)^{k-1}}{(k-1)!}+O\left(x(\log x)^{k-2}\right)
    \]
\end{enumerate}

\stepcounter{section}
\section*{Problem 2 [10 Marks]} In this question, any constants arising in the
main term of the estimate should be worked out explicitly. 
\begin{enumerate}
    \item  Obtain an estimate for the sum
    $S(x)=\displaystyle{\sum_{n\leq x;\; n\;\text{odd}}\frac{1}{n}}$
    with error term $O(1/x)$.
    \item Let $D(x) =\displaystyle{\sum_{n\leq x;\;n\;\text{odd}}d(n)}$, where $d(n)$ is the divisor function. Give an estimate for $D(x)$
    with error term $O(\sqrt{x})$.  

{\it Hint: Use Dirichlet's
hyperbola method and the result of the previous part}.
\end{enumerate}


\stepcounter{section}
\section*{Problem 3 [10 Marks]} Use the Dirichlet hyperbola method (or some other method) to obtain
an estimate for the sum $\displaystyle{\sum_{n\leq x}\frac{d(n)}{n}}$ with an error term $O\left(\frac{\log x}{\sqrt{x}}\right)$

\stepcounter{section}
\section*{Problem 4 [10 Marks]} Obtain an estimate, similar to the estimate for
$\displaystyle{\sum_{n\leq x}\frac{1}{n}}$ for the sum $\displaystyle{\sum_{n\leq x}\frac{1}{\phi(n)}}$.

 {\it Hint: Use the convolution method.}

\stepcounter{section}
\section*{Problem 5 [10 Marks]} Using only Mertens' type estimates (but not the PNT), obtain an
asymptotic estimate for the partial sums
$\displaystyle{S(x) =\sum_{p\leq x;\; p\;\text{prime} }\frac{1}{p\log p}}$ with as good an error term as you can get using only results at the
level of Mertens.


\stepcounter{section}
\section*{Problem 6 [30 Marks]} Let $q\in\mathbb{N}$, and consider the group $\left(\mathbb{Z}_{q}\right)^{\times}$ of invertible elements modulo $q$. A homomorphism $\chi:\left(\mathbb{Z}_{q}\right)^{\times}\leftarrow\mathbb{C}^{\times}$ is called a Dirichlet character modulo $q$. By Euler's theorem $a^{\phi(q)}\equiv 1\mod q$ for any $a\in \left(\mathbb{Z}_{q}\right)^{\times}$, and so $\left(\chi(a)\right)^{\phi(q)}=1$ for all $a\in \left(\mathbb{Z}_{q}\right)^{\times}$. In other words, each $\chi(a)$ is a $\phi(q)$-th root of unity. 
We extend the definition of $\chi$ to all integers by setting \[\chi(n)=\begin{cases}\chi(n\;\mathrm{mod}\; q)&\;\; \text{if}\; (n,q)=1\\ 0&\;\;\text{otherwise.}\end{cases}\]
If $\chi$ and $\psi$ are characters modulo $q$, then so is the product $\chi\psi$ as well as $\overline{\chi}$ given by $\overline{\chi}(a)=\overline{\chi(a)}$. In fact, the set of all characters modulo $q$ forms a group of order $\phi(q)$.

The character $\chi_{0}$ satisfying $\chi_0(a)=1$ for all $a\in \left(\mathbb{Z}_{q}\right)^{\times}$ and $\chi_0(a)=0$ if $(a,q)\neq1$ is called the trivial character modulo $q$. 

\begin{enumerate}[label=(\alph*)]
    \item Show that $\chi$ is a completely multiplicative arithmetic function.
    \item If $\chi\neq \chi_0$, show that $\displaystyle{\sum_{a(\mathrm{mod}\; q)}\chi(a)}=0$.
    \item Show that \[\sum_{\chi(\mathrm{mod}q)}\chi(n)=\begin{cases}\phi(q)&\;\;\text{if}\;n\equiv1\mathrm{mod}\; q\\ 0&\;\;\text{otherwise.}\end{cases}\]
    \item For $\chi\neq\chi_0$, show that $\displaystyle{\left|\sum_{n\leq x}\chi(n)\right|\leq q}$. Deduce that \[L(s,\chi)=\sum_{n=1}^{\infty}\frac{\chi(n)}{n^s}\] is convergent for $\Re(s)>0$.
    \item Define $f(n)=\sum_{d|n}\chi(d)$.  Suppose that $\chi$ is a real Dirichlet character modulo $q$ (i.e. $\chi(a)=0,\pm1$ for all $a$). Show that $f(1)=1$ and $f(n)\geq0$. Moreover, show that $f(n)\geq 1$ whenever $n$ is a perfect square.
    \item Suppose $\chi\neq\chi_0$. Using Dirichlet's hyperbola method, prove that \[\sum_{n\leq x}\frac{f(n)}{\sqrt{n}}=2L(1,\chi)\sqrt{x}+O(1).\]
    \item If $\chi\neq \chi_0$ is a real character, show that $L(1,\chi)\neq0$.
\end{enumerate}

\stepcounter{section}
\section*{Problem 7 [10 Marks]} Let $\chi_0$ be the principal character modulo $q$ (i.e. $\chi(a)=0$ if $(a,q)>1$ and $\chi(a)=1$ if $(a,q)=1$). Prove the Merten's type estimate
    \[
        \sum_{p\leq x}\frac{\chi_0(p)}{p}=\log\log x+b_{\chi_0}+O_{\chi_0}(1/\log x),
    \] where 
    \[
        b_{\chi_0}=A+\sum_{p|q}\log(1-1/p)-\sum_{k=2}^{\infty}\sum_{p\;\text{prime}}\frac{\chi_0(p^k)}{kp^k}
    \] 
    and $A$ is some absolue constant that is independent of $\chi_0$.

\stepcounter{section}
\section*{Problem 8 [10 Marks]} Let $q \in \mathbb{N}$.
\begin{enumerate}
	\item Prove that 
	$$
	\sum_{d \mid q} |\mu(d)| = 2^{\omega(q)} 
	$$
	\item Use the identity
	$$
	\sum_{d \mid q} \frac{\mu(d) \log d}{d} = - \frac{\phi(q)}{q} \sum_{p \mid q} \frac{\log p}{p-1},
	$$
	to prove that 
	$$
	\sum_{ \substack{ n \le x \\ (n,q)=1}} \frac{1}{n} = \frac{\phi(q)}{q} 
	\Big( 
	\log x  + \gamma + \sum_{p \mid q} \frac{\log p}{p-1} \Big) + O\Big( \frac{2^{\omega(q)}}{x} \Big),
	$$
	where $\gamma$ is Euler's constant.\\
	
\end{enumerate}	


\end{document}