\documentclass[12pt]{article}

% Import preambles and macros for homework
% Essential packages
\usepackage{amsmath, amsfonts, amssymb, amsthm}
\usepackage{mathtools}
\usepackage{enumitem}
\usepackage{graphicx}
\usepackage{wrapfig}
\usepackage{systeme}
\usepackage{caption}
\usepackage{soul}
\usepackage[dvipsnames]{xcolor}
\usepackage{fancyhdr}
\allowdisplaybreaks

% Page layout
\usepackage[
  top=2cm,
  bottom=2cm,
  left=2cm,
  right=2cm,
  headheight=17pt,
  includehead,includefoot,
  heightrounded,
]{geometry}


% pgfornament for title page decorations
\usepackage[object=vectorian]{pgfornament}

% Fancy header/footer setup
\pagestyle{fancy}
\setlength{\headheight}{14.49998pt}
\addtolength{\topmargin}{-2.49998pt}
\renewcommand{\footrulewidth}{0.4pt}
\setlength\parindent{15pt}
% Math notation shortcuts
\newcommand{\R}{\mathbb{R}}
\newcommand{\Q}{\mathbb{Q}}
\newcommand{\Z}{\mathbb{Z}}
\newcommand{\N}{\mathbb{N}}
\newcommand{\C}{\mathbb{C}}
\newcommand{\X}{\mathcal{X}}

% Theorem environments
\newtheorem{mainthm}{Theorem}[section]
\newtheorem{theorem}{Theorem}[section]  
\newtheorem{lemma}[theorem]{Lemma}
\newtheorem{proposition}[theorem]{Proposition}
\newtheorem{corollary}[theorem]{Corollary}
\newtheorem{definition}[theorem]{Definition}
\newtheorem{claim}[theorem]{Claim}

% Calculus
\newcommand{\diff}{\mathop{}\!\mathrm{d}}
\newcommand{\deriv}[2]{\frac{\mathrm{d}#1}{\mathrm{d}#2}}
\newcommand{\pderiv}[2]{\frac{\partial #1}{\partial #2}}

% Linear Algebra
\newcommand{\inner}[2]{\langle #1, #2 \rangle}
\newcommand{\norm}[1]{\| #1 \|}
\newcommand{\tr}{\operatorname{tr}}
\newcommand{\spn}{\operatorname{span}}
\newcommand{\rank}{\operatorname{rank}}
\newcommand{\nullity}{\operatorname{nullity}}

% Logic
\newcommand{\contra}{\Rightarrow\Leftarrow}

% Custom commands for notes
\newcommand{\todo}[1]{\textcolor{red}{[TODO: #1]}}
\newcommand{\important}[1]{\textbf{\textcolor{blue}{#1}}}

%Number Theory
\DeclareMathOperator{\Li}{Li}
\newcommand{\floor}[1]{\left\lfloor #1 \right\rfloor}
\newcommand{\fract}[1]{\left\{ #1 \right\}}




\newcommand{\maketitlepage}{
    \begin{titlepage}
        \centering
        \vspace*{2.0cm}
        \pgfornament{84}\\
        {\LARGE \textsc{\coursename}\par}
        \vspace{0.5cm}
        {\large\coursecode\par}
        \vspace{0.5cm}
        {\large\instructor\par}
        \vspace{1.5cm}
        {\huge\bfseries\assignment\par}
        \vspace{1cm}
        {\LARGE\itshape\author\par}
        \vspace{2cm}
        {\large\bfseries Due Date:\par}
        \vspace{0.5cm}
        {\Large \duedate}\\
        \pgfornament{84}
    \end{titlepage}
}
% =============================================
% HOMEWORK CONFIGURATION - EDIT THESE VALUES!
% =============================================

% Your personal info
\renewcommand{\author}{Deepak Jassal}
\newcommand{\authorlast}{Jassal}

% Course info
\newcommand{\coursename}{Course Name}
\newcommand{\coursecode}{Course code}
\newcommand{\instructor}{Instructor}

% Assignment-specific info (CHANGE THESE FOR EACH HOMEWORK)
\newcommand{\assignment}{Assignment }
\newcommand{\duedate}{Month Day\textsuperscript{th}, 20XX}

% Header configuration
\fancyhead[l]{\assignment}
\fancyhead[c]{\coursecode}
\fancyhead[r]{\monthyear}
\fancyfoot[c]{\authorlast{ }\thepage}

\usepackage{booktabs}


\renewcommand{\author}{Deepak Jassal}
\renewcommand{\authorlast}{Jassal}
\renewcommand{\coursename}{Mathematical Statistics}
\renewcommand{\coursecode}{STAT 372}
\renewcommand{\assignment}{Assignment 2}
\renewcommand{\instructor}{Dr. Dan Ryan}
\renewcommand{\duedate}{February 14\textsuperscript{th}, 2026}


\begin{document}
\begin{titlepage}
	\centering
	\vspace*{2.0cm}	
	\pgfornament{84}\\
	{\LARGE \textsc{\coursename}\par}
	\vspace{0.5cm}
	{\large\coursecode\par}
    \vspace{0.5cm}
    {\large\instructor\par}
	\vspace{1.5cm}
	{\huge\bfseries\assignment\par}
	\vspace{1cm}  
	{\LARGE\itshape\author\par}
    \vspace{2cm}
	{\large\bfseries Due Date:\par}
	\vspace{0.5cm}
	{\Large \duedate}\\
	\pgfornament{84}
\end{titlepage}



\section*{Part I: Maximum Likelihood Estimation (MLE)}
\setcounter{section}{1}
\renewcommand{\theequation}{1.\arabic{equation}}

\subsection*{Step 1: Specify the distribution/model}
We assume that our data $X_1, X_2, \ldots, X_n$ are independent and identically distributed (i.i.d.) following a Normal distribution with unknown mean $\mu$ and known variance $\sigma^2$:
\begin{equation}
X_i \sim \mathcal{N}(\mu, \sigma^2), \quad i = 1, 2, \ldots, n
\end{equation}
The probability density function for a single observation is:
\begin{equation}
f(x_i \mid \mu) = \frac{1}{\sqrt{2\pi\sigma^2}} \exp\left( -\frac{(x_i - \mu)^2}{2\sigma^2} \right)
\end{equation}

\subsection*{Step 2: Construct the Likelihood function}
The likelihood function is the joint density of all observations, viewed as a function of $\mu$:
\begin{equation}
L(\mu \mid x) = \prod_{i=1}^n f(x_i \mid \mu) = \left(\frac{1}{\sqrt{2\pi\sigma^2}}\right)^n \exp\left( -\frac{1}{2\sigma^2} \sum_{i=1}^n (x_i - \mu)^2 \right)
\end{equation}

\subsection*{Step 3: Construct the Log-Likelihood}
Taking the natural logarithm simplifies the expression:
\begin{equation}
\ell(\mu \mid x) = \ln(L(\mu \mid x)) = -\frac{n}{2}\ln(2\pi\sigma^2) - \frac{1}{2\sigma^2} \sum_{i=1}^n (x_i - \mu)^2
\end{equation}

\subsection*{Step 4: Maximize the Log-Likelihood}
We differentiate with respect to $\mu$ and set the derivative equal to zero:
\begin{align}
\frac{\partial \ell}{\partial \mu} &= -\frac{1}{2\sigma^2} \sum_{i=1}^n 2(x_i - \mu)(-1) \\
&= \frac{1}{\sigma^2} \sum_{i=1}^n (x_i - \mu) = 0
\end{align}
Notice that $\sigma^2$ cancels out, so the MLE for $\mu$ does not depend on the known variance.
This implies:
\begin{equation}
\colorbox{yellow}{$\mu = \dfrac{1}{n}\displaystyle\sum_{i=1}^n x_i=\bar{x}$}
\end{equation}

\subsection*{Step 5: Solve for $\hat{\mu}_{\text{MLE}}$}
Solving for $\mu$ yields the sample mean. First, calculate the sample mean from the data:
\begin{equation}
x = \{11.2, 10.7, 9.9, 10.4, 12.1, 11.5, 10.9, 9.6, 10.1, 11.0\}
\end{equation}
\begin{align}
\sum_{i=1}^{10} x_i &= 11.2 + 10.7 + 9.9 + 10.4 + 12.1 + 11.5 + 10.9 + 9.6 + 10.1 + 11.0 \\
&= 107.4
\end{align}
\begin{equation}
n = 10
\end{equation}
\begin{equation}
\bar{x} = \frac{107.4}{10} = 10.74
\end{equation}

Therefore:
\begin{equation}
\colorbox{yellow}{$\hat{\mu}_{\text{MLE}} = \bar{x} = 10.74$}
\end{equation}

\subsection*{Likelihood Plots}

\begin{figure}[h]
\centering
\includegraphics[width=0.9\textwidth]{partI_likelihood_plot.png}
\caption{Likelihood and log-likelihood functions for the normal mean. \textbf{Left panel:} The likelihood function $L(\mu \mid x)$ is maximized at $\hat{\mu}_{\text{MLE}} = 10.74$, the sample mean. \textbf{Right panel:} The log-likelihood $\ell(\mu \mid x)$ is quadratic in $\mu$, with its maximum at the same point. Sample variance $s^2=0.58488889$ was used in place of $\sigma^2$.}
\label{fig:likelihood_loglikelihood}
\end{figure}
\newpage

\subsection*{Validity of the MLE approach}
The MLE approach is valid for the following reasons:
\begin{itemize}
    \item \textbf{Regularity Conditions:} The Normal distribution satisfies all regularity conditions—the support of the distribution does not depend on $\mu$, and the log-likelihood is twice differentiable with respect to $\mu$.
    \item \textbf{Consistency:} $\hat{\mu}_{\text{MLE}}$ is a consistent estimator, meaning it converges in probability to the true parameter value $\mu$ as $n \to \infty$.
    \item \textbf{Asymptotic Normality:} The MLE is asymptotically Normal:
    \begin{equation}
    \hat{\mu}_{\text{MLE}} \overset{\text{approx}}{\sim} \mathcal{N}\left(\mu, \frac{\sigma^2}{n}\right)
    \end{equation}
    \item \textbf{Sufficiency:} The MLE $\bar{x}$ is a function of the sufficient statistic $\sum_{i=1}^n x_i$, meaning it captures all information in the data relevant to $\mu$.
    \item \textbf{Invariance:} The MLE of a function of $\mu$ is the function of the MLE.
\end{itemize}

\newpage

\section*{Part II: Bayesian Estimation (Normal Priors)}
\setcounter{equation}{0}
\renewcommand{\theequation}{2.\arabic{equation}}

We maintain the same data model: $X_i \mid \mu \overset{\text{i.i.d.}}{\sim} \mathcal{N}(\mu, \sigma^2)$ with $\sigma^2$ known. The sample mean follows:
\begin{equation}
\bar{x} \mid \mu \sim \mathcal{N}\left(\mu, \frac{\sigma^2}{n}\right)
\end{equation}

\subsection*{General Framework: Conjugate Normal-Normal Model}

When the likelihood is Normal with known variance and the prior is Normal, the posterior is also Normal due to conjugacy.


\subsubsection*{Posterior Derivation (Completing the Square):}

Starting with Bayes' theorem:
\begin{equation}
p(\mu \mid x) \propto p(\mu) \times L(\mu \mid x)
\end{equation}
(Here we used $\propto$ instead of = because the denominator in this context is a normailizing constant.)

The prior is:
\begin{equation}
p(\mu) \propto \exp\left( -\frac{1}{2\tau^2} (\mu - \mu_0)^2 \right)
\end{equation}

The likelihood is:
\begin{equation}
L(\mu \mid x) \propto \exp\left( -\frac{1}{2\sigma^2/n} (\mu - \bar{x})^2 \right) = \exp\left( -\frac{n}{2\sigma^2} (\mu - \bar{x})^2 \right)
\end{equation}

Multiplying the exponents:
\begin{align}
p(\mu \mid x) &\propto \exp\left( -\frac{1}{2\tau^2} (\mu - \mu_0)^2 - \frac{n}{2\sigma^2} (\mu - \bar{x})^2 \right) \\
&= \exp\left( -\frac{1}{2} \left[ \frac{1}{\tau^2} (\mu - \mu_0)^2 + \frac{n}{\sigma^2} (\mu - \bar{x})^2 \right] \right)
\end{align}

Expand the quadratic terms:
\begin{align}
\frac{1}{\tau^2} (\mu - \mu_0)^2 &= \frac{1}{\tau^2} (\mu^2 - 2\mu_0\mu + \mu_0^2) \\
\frac{n}{\sigma^2} (\mu - \bar{x})^2 &= \frac{n}{\sigma^2} (\mu^2 - 2\bar{x}\mu + \bar{x}^2)
\end{align}

Summing these expressions:
\begin{align}
&\frac{1}{\tau^2} (\mu^2 - 2\mu_0\mu + \mu_0^2) + \frac{n}{\sigma^2} (\mu^2 - 2\bar{x}\mu + \bar{x}^2) \\
&= \left( \frac{1}{\tau^2} + \frac{n}{\sigma^2} \right) \mu^2 - 2\left( \frac{\mu_0}{\tau^2} + \frac{n\bar{x}}{\sigma^2} \right) \mu + \left( \frac{\mu_0^2}{\tau^2} + \frac{n\bar{x}^2}{\sigma^2} \right)
\end{align}

Now complete the square for $\mu$. For a quadratic of the form $A\mu^2 - 2B\mu + C$, we can write:
\begin{equation}
A\mu^2 - 2B\mu + C = A\left( \mu - \frac{B}{A} \right)^2 + \left( C - \frac{B^2}{A} \right)
\end{equation}

In our expression:
\begin{align}
A &= \frac{1}{\tau^2} + \frac{n}{\sigma^2} \\
B &= \frac{\mu_0}{\tau^2} + \frac{n\bar{x}}{\sigma^2} \\
C &= \frac{\mu_0^2}{\tau^2} + \frac{n\bar{x}^2}{\sigma^2}
\end{align}

Therefore:
\begin{align}
&\left( \frac{1}{\tau^2} + \frac{n}{\sigma^2} \right) \mu^2 - 2\left( \frac{\mu_0}{\tau^2} + \frac{n\bar{x}}{\sigma^2} \right) \mu + \left( \frac{\mu_0^2}{\tau^2} + \frac{n\bar{x}^2}{\sigma^2} \right) \\
&= \left( \frac{1}{\tau^2} + \frac{n}{\sigma^2} \right) \left( \mu - \frac{ \frac{\mu_0}{\tau^2} + \frac{n\bar{x}}{\sigma^2} }{ \frac{1}{\tau^2} + \frac{n}{\sigma^2} } \right)^2 + \text{constant (not involving } \mu\text{)}
\end{align}

Substituting back into the exponential:
\begin{align}
p(\mu \mid x) &\propto \exp\left( -\frac{1}{2} \left[ \left( \frac{1}{\tau^2} + \frac{n}{\sigma^2} \right) \left( \mu - \frac{ \frac{\mu_0}{\tau^2} + \frac{n\bar{x}}{\sigma^2} }{ \frac{1}{\tau^2} + \frac{n}{\sigma^2} } \right)^2 + \text{constant} \right] \right) \\
&\colorbox{yellow}{$\propto \exp\left( -\dfrac{1}{2} \left( \dfrac{1}{\tau^2} + \dfrac{n}{\sigma^2} \right) \left( \mu - \dfrac{ \dfrac{\mu_0}{\tau^2} + \dfrac{n\bar{x}}{\sigma^2} }{ \dfrac{1}{\tau^2} + \dfrac{n}{\sigma^2} } \right)^2 \right)$}
\end{align}

(We can remove the constant because it woudl only scale the value by some constant, and we are looking at what the distribution is proportional to.)


Thus, the posterior is:
\begin{equation}
\colorbox{yellow}{$\mu \mid x \sim \mathcal{N}(\mu_n, \tau_n^2)$}
\end{equation}

where:
\begin{align}
\mu_n &= \frac{ \frac{\mu_0}{\tau^2} + \frac{n\bar{x}}{\sigma^2} }{ \frac{1}{\tau^2} + \frac{n}{\sigma^2} } \\
\frac{1}{\tau_n^2} &= \frac{1}{\tau^2} + \frac{n}{\sigma^2}
\end{align}


\subsubsection*{Weighted Average Form:}
Equation (2.20) can be rewritten as a weighted average of the prior mean $\mu_0$ and the sample mean $\bar{x}$:
\begin{equation}
\mu_n = \frac{1/\tau^2}{1/\tau^2 + n/\sigma^2} \mu_0 + \frac{n/\sigma^2}{1/\tau^2 + n/\sigma^2} \bar{x} 
\end{equation}
\begin{equation}
\mu_n = \frac{\text{Prior Precision}}{\text{Prior Precision} + \text{Data Precision}} \mu_0 + \frac{\text{Data Precision}}{\text{Prior Precision} + \text{Data Precision}} \bar{x}
\end{equation}
The weights are the precisions (inverse variances), reflecting the relative information in the prior versus the data.

\subsection*{Data Specifications}

From our dataset:
\begin{align}
n &= 10 \quad \text{(sample size)} \\
\bar{x} &= 10.74 \quad \text{(sample mean from observed data)}
\end{align}
The data variance $\sigma^2$ is known but not specified numerically; it remains in the formulas as a constant.

\newpage

\subsection*{Part II(A): Concentrated Informative Prior}
\setcounter{equation}{0}
\renewcommand{\theequation}{2.A.\arabic{equation}}

\subsubsection*{Prior Specification:}
\begin{equation}
\mu \sim \mathcal{N}(\mu_0 = 9, \tau^2 = 0.25)
\end{equation}
Standard deviation: $\tau = 0.5$ \\
Prior precision: 
\begin{equation}
\frac{1}{\tau^2} = 4
\end{equation}

\subsubsection*{Posterior Mean (in terms of $\sigma^2$):}
\begin{align}
\mu_n &= \frac{4}{4 + \frac{n}{\sigma^2}}(9) + \frac{\frac{n}{\sigma^2}}{4 + \frac{n}{\sigma^2}}(10.74) \\
&= \frac{4}{4 + \frac{10}{\sigma^2}}(9) + \frac{\frac{10}{\sigma^2}}{4 + \frac{10}{\sigma^2}}(10.74)
\end{align}

\subsubsection*{Posterior Precision:}
\begin{align}
\frac{1}{\tau_n^2} &= \frac{1}{\tau^2} + \frac{n}{\sigma^2} \\
&= 4 + \frac{10}{\sigma^2}
\end{align}

\subsubsection*{Posterior Variance:}
\begin{equation}
\tau_n^2 = \frac{1}{4 + \frac{10}{\sigma^2}}
\end{equation}

\subsubsection*{Result:}
\begin{equation}
\colorbox{yellow}{$\hat{\mu}_A = \mu_n = \dfrac{4}{4 + \dfrac{10}{\sigma^2}}(9) + \dfrac{\dfrac{10}{\sigma^2}}{4 + \dfrac{10}{\sigma^2}}(10.74), \quad \tau_n^2 = \dfrac{1}{4 + \dfrac{10}{\sigma^2}}$}
\end{equation}
The posterior distribution is:
\begin{equation}
\colorbox{yellow}{$\mu \mid x \sim \mathcal{N}\left(\dfrac{4}{4 + \dfrac{10}{\sigma^2}}(9) + \dfrac{\dfrac{10}{\sigma^2}}{4 + \dfrac{10}{\sigma^2}}(10.74), \dfrac{1}{4 + \dfrac{10}{\sigma^2}}\right)$}
\end{equation}




\newpage

\subsection*{Part II(B): Diffuse Informative Prior (Same Mean, Large Variance)}
\setcounter{equation}{0}
\renewcommand{\theequation}{2.B.\arabic{equation}}

\subsubsection*{Prior Specification:}
\begin{equation}
\mu \sim \mathcal{N}(\mu_0 = 9, \tau^2 = 25)
\end{equation}
Standard deviation: $\tau = 5$ \\
Prior precision: 
\begin{equation}
\frac{1}{\tau^2} = 0.04
\end{equation}

\subsubsection*{Posterior Mean (in terms of $\sigma^2$):}
\begin{align}
\mu_n &= \frac{0.04}{0.04 + \frac{n}{\sigma^2}}(9) + \frac{\frac{n}{\sigma^2}}{0.04 + \frac{n}{\sigma^2}}(10.74) \\
&= \frac{0.04}{0.04 + \frac{10}{\sigma^2}}(9) + \frac{\frac{10}{\sigma^2}}{0.04 + \frac{10}{\sigma^2}}(10.74)
\end{align}

\subsubsection*{Posterior Precision:}
\begin{align}
\frac{1}{\tau_n^2} &= \frac{1}{\tau^2} + \frac{n}{\sigma^2} \\
&= 0.04 + \frac{10}{\sigma^2}
\end{align}

\subsubsection*{Posterior Variance:}
\begin{equation}
\tau_n^2 = \frac{1}{0.04 + \frac{10}{\sigma^2}}
\end{equation}

\subsubsection*{Result:}
\begin{equation}
\colorbox{yellow}{$\hat{\mu}_B = \mu_n = \dfrac{0.04}{0.04 + \dfrac{10}{\sigma^2}}(9) + \dfrac{\dfrac{10}{\sigma^2}}{0.04 + \dfrac{10}{\sigma^2}}(10.74), \quad \tau_n^2 = \dfrac{1}{0.04 + \dfrac{10}{\sigma^2}}$}
\end{equation}
The posterior distribution is:
\begin{equation}
\colorbox{yellow}{$\mu \mid x \sim \mathcal{N}\left(\dfrac{0.04}{0.04 + \dfrac{10}{\sigma^2}}(9) + \dfrac{\dfrac{10}{\sigma^2}}{0.04 + \dfrac{10}{\sigma^2}}(10.74), \dfrac{1}{0.04 + \dfrac{10}{\sigma^2}}\right)$}
\end{equation}


\newpage

\subsection*{Part II(C): Non-informative Prior}
\setcounter{equation}{0}
\renewcommand{\theequation}{2.C.\arabic{equation}}

\subsubsection*{Prior Specification:}
\begin{equation}
p(\mu) \propto 1, \quad -\infty < \mu < \infty
\end{equation}
This is an \textit{improper prior} since it does not integrate to 1. It can be viewed as the limiting case of a Normal prior as $\tau^2 \to \infty$, making the prior precision approach zero:
\begin{equation}
\lim_{\tau^2 \to \infty} \frac{1}{\tau^2} = 0
\end{equation}

\subsubsection*{Posterior Derivation:}
Applying Bayes' theorem with a flat prior:
\begin{align}
p(\mu \mid x) &\propto 1 \times L(\mu \mid x) \\
&\propto \exp\left( -\frac{1}{2(\sigma^2/n)} (\mu - 10.74)^2 \right)
\end{align}

\subsubsection*{Posterior Mean:}
Substituting $1/\tau^2 = 0$ into the weighted average formula:
\begin{align}
\mu_n &= \frac{0}{0 + \frac{n}{\sigma^2}} \mu_0 + \frac{\frac{n}{\sigma^2}}{0 + \frac{n}{\sigma^2}} \bar{x} \\
&= 0 \times \mu_0 + 1 \times 10.74 \\
&= 10.74
\end{align}

\subsubsection*{Posterior Variance:}
\begin{align}
\tau_n^2 &= \frac{1}{0 + \frac{n}{\sigma^2}} \\
&= \frac{\sigma^2}{n}
\end{align}

\subsubsection*{Result:}
\begin{equation}
\colorbox{yellow}{$\hat{\mu}_C = \mu_n = \bar{x} = 10.74, \quad \tau_n^2 = \dfrac{\sigma^2}{n} = \dfrac{\sigma^2}{10}$}
\end{equation}
The posterior distribution is:
\begin{equation}
\colorbox{yellow}{$\mu \mid x \sim \mathcal{N}\left(10.74, \dfrac{\sigma^2}{10}\right)$}
\end{equation}
which is exactly the sampling distribution of the MLE.



\newpage

\section*{Part III: Graphical Comparison}
\setcounter{equation}{0}
\renewcommand{\theequation}{3.\arabic{equation}}

For each of the three prior choices (A, B, C), we produce a single plot that includes:
\begin{itemize}
    \item the prior density for $\mu$,
    \item the posterior density for $\mu \mid x$,
    \item and a rescaled version of the likelihood (scaled to have approximately the same maximum height as the posterior for visual comparison).
\end{itemize}

Note: For actual plotting, a specific numerical value of $\sigma^2$ must be chosen. Since we are given that $\sigma^2$ is known i calculated the sample variance to be $s^2=0.58488889$ and used that.

\subsection*{III.1 Plots}

\begin{figure}[h!]
\centering
\includegraphics[width=0.8\textwidth]{PartIIA.png}
\caption{Case A: Concentrated prior $\mathcal{N}(9, 0.25)$, posterior (depends on $\sigma^2$), and rescaled likelihood. Point estimates: $\hat{\mu}_{\text{MLE}} = 10.74$, $\hat{\mu}_A=10.41$.}
\label{fig:plotA}
\end{figure}

\begin{figure}[h!]
\centering
\includegraphics[width=0.8\textwidth]{PartIIB.png}
\caption{Case B: Diffuse prior $\mathcal{N}(9, 25)$, posterior (depends on $\sigma^2$), and rescaled likelihood. Point estimates: $\hat{\mu}_{\text{MLE}} = 10.74$, $\hat{\mu}_B=10.74$.}
\label{fig:plotB}
\end{figure}

\begin{figure}[h!]
\centering
\includegraphics[width=0.8\textwidth]{PartIIC.png}
\caption{Case C: Flat prior (improper), posterior $\mathcal{N}(10.74, \sigma^2/10)$, and rescaled likelihood. Point estimates: $\hat{\mu}_{\text{MLE}} = 10.74$, $\hat{\mu}_C = 10.74$.}
\label{fig:plotC}
\end{figure}

\newpage

\section*{Part IV: Statistical Insight Questions}
\setcounter{equation}{0}
\renewcommand{\theequation}{4.\arabic{equation}}

\subsection*{IV.1 Prior spread and estimator behavior}
As the prior variance increases from $\tau^2 = 0.25$ (Case A) to $\tau^2 = 25$ (Case B), the posterior mean shifts from being pulled toward the prior mean to being dominated by the data. In Case A, with a concentrated prior (small variance), the posterior mean $\mu_n = \frac{4}{4 + n/\sigma^2}(9) + \frac{n/\sigma^2}{4 + n/\sigma^2}(10.74)$ is substantially shrunk from the data mean of 10.74 toward the prior mean of 9. The prior receives a weight of $4/(4 + n/\sigma^2)$ in this case. In Case B, with a diffuse prior (large variance), the posterior mean $\mu_n = \frac{0.04}{0.04 + n/\sigma^2}(9) + \frac{n/\sigma^2}{0.04 + n/\sigma^2}(10.74)$ is very close to the sample mean, as the prior receives only $0.04/(0.04 + n/\sigma^2)$ weight. This demonstrates that the prior's influence on the posterior mean is directly proportional to its precision relative to the data precision.

\subsection*{IV.2 Shrinkage}
``Shrinkage'' refers to the phenomenon where the Bayesian posterior estimate is pulled away from the sample mean ($\bar{x} = 10.74$) toward the prior mean ($\mu_0 = 9$). In Case A, the estimate shows clear shrinkage toward 9. In Case B, the estimate shows very little shrinkage. Shrinkage is strongest in case A, where the concentrated prior exerts the most influence, pulling the estimate toward the prior mean of 9. The amount of shrinkage is given by:
\begin{equation}
\text{Shrinkage} = \bar{x} - \mu_n = \frac{1/\tau^2}{1/\tau^2 + n/\sigma^2}(\bar{x} - \mu_0)
\end{equation}

\subsection*{IV.3 When does the prior matter most?}
Based on the posterior mean formula:
\begin{equation}
\mu_n = \frac{1/\tau^2}{1/\tau^2 + n/\sigma^2} \mu_0 + \frac{n/\sigma^2}{1/\tau^2 + n/\sigma^2} \bar{x}
\end{equation}
the prior has the strongest influence when:

\begin{enumerate}
    \item The prior precision ($1/\tau^2$) is large relative to the data precision ($n/\sigma^2$). In this problem, Case A has prior precision 4, which will have substantial weight unless the data precision $n/\sigma^2$ is very large. If the prior were even more concentrated (smaller $\tau^2$), its influence would be greater.
    
    \item The sample size ($n$) is small or the data variance ($\sigma^2$) is large. The data precision is $n/\sigma^2$, so it decreases with smaller $n$ or larger $\sigma^2$. With our sample size of 10, if $\sigma^2$ is large, the data precision could be small, giving the prior more influence. Conversely, with a very large sample or very small $\sigma^2$, the prior's influence diminishes regardless of its precision.
\end{enumerate}

\subsection*{IV.4 Non-informative prior}
Using $p(\mu) \propto 1$ implies that we have no prior preference for any value of $\mu$. In this case, the posterior distribution is proportional to the likelihood function alone. Consequently, the posterior mean ($\hat{\mu}_C = 10.74$) is exactly equal to the MLE ($\hat{\mu}_{\text{MLE}} = 10.74$). Statistically, this demonstrates that maximum likelihood estimation can be viewed as a special case of Bayesian inference under a non-informative prior, where the inference is driven entirely by the observed data with no prior influence. The posterior variance ($\tau_n^2 = \sigma^2/n$) also matches the variance of the sampling distribution of the MLE:
\begin{equation}
\text{Var}(\hat{\mu}_{\text{MLE}}) = \frac{\sigma^2}{n}
\end{equation}

\subsection*{IV.5 Conclusion}
This assignment illuminates the fundamental role of priors in Bayesian inference. A concentrated prior (small variance) represents strong prior beliefs and acts as a powerful ``shrinkage target,'' pulling the posterior estimate toward the prior mean even when the data suggest otherwise. In our analysis, the concentrated prior in Case A pulls the estimate toward 9, with the exact amount depending on the known data variance $\sigma^2$. In contrast, a diffuse prior (large variance) represents weak prior knowledge and allows the data to dominate the inference, as seen in Case B where the posterior estimate is very close to the sample mean. The key insight is that the prior's influence is not absolute but relative—it depends on the ratio of prior precision to data precision, where data precision is $n/\sigma^2$. This ratio encapsulates the trade-off between prior information and sample evidence, showing that even a strong prior can be overwhelmed by sufficient data (large $n$) or highly precise data (small $\sigma^2$), while even a weak prior can have noticeable influence with small samples or noisy data. The flat prior case serves as a bridge between Bayesian and frequentist approaches, demonstrating that MLE is simply Bayesian inference with no prior information.

%\newpage

\section*{Summary Table of Results (in terms of $\sigma^2$)}

\begin{table}[h]
\centering
\begin{tabular}{lccc}
\toprule
\textbf{Case} & \textbf{Prior} & $\hat{\mu}_{\text{Bayes}}$ & $\tau_n^2$ \\
\midrule
MLE & — & 10.74 & $\sigma^2/10$ \\
A & $\mathcal{N}(9, 0.25)$ & $\frac{4}{4 + 10/\sigma^2}(9) + \frac{10/\sigma^2}{4 + 10/\sigma^2}(10.74)$ & $\frac{1}{4 + 10/\sigma^2}$ \\
B & $\mathcal{N}(9, 25)$ & $\frac{0.04}{0.04 + 10/\sigma^2}(9) + \frac{10/\sigma^2}{0.04 + 10/\sigma^2}(10.74)$ & $\frac{1}{0.04 + 10/\sigma^2}$ \\
C & Flat ($\propto 1$) & 10.74 & $\sigma^2/10$ \\
\bottomrule
\end{tabular}
\caption{Complete summary of all estimates with $\sigma^2$ known but unspecified}
\label{tab:final_summary}
\end{table}

\end{document}