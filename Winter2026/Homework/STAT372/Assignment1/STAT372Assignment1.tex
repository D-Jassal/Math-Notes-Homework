\documentclass[12pt]{article}

% Import preambles and macros for homework
% Essential packages
\usepackage{amsmath, amsfonts, amssymb, amsthm}
\usepackage{mathtools}
\usepackage{enumitem}
\usepackage{graphicx}
\usepackage{wrapfig}
\usepackage{systeme}
\usepackage{caption}
\usepackage{soul}
\usepackage[dvipsnames]{xcolor}
\usepackage{fancyhdr}
\allowdisplaybreaks

% Page layout
\usepackage[
  top=2cm,
  bottom=2cm,
  left=2cm,
  right=2cm,
  headheight=17pt,
  includehead,includefoot,
  heightrounded,
]{geometry}


% pgfornament for title page decorations
\usepackage[object=vectorian]{pgfornament}

% Fancy header/footer setup
\pagestyle{fancy}
\setlength{\headheight}{14.49998pt}
\addtolength{\topmargin}{-2.49998pt}
\renewcommand{\footrulewidth}{0.4pt}
\setlength\parindent{15pt}
% Math notation shortcuts
\newcommand{\R}{\mathbb{R}}
\newcommand{\Q}{\mathbb{Q}}
\newcommand{\Z}{\mathbb{Z}}
\newcommand{\N}{\mathbb{N}}
\newcommand{\C}{\mathbb{C}}
\newcommand{\X}{\mathcal{X}}

% Theorem environments
\newtheorem{mainthm}{Theorem}[section]
\newtheorem{theorem}{Theorem}[section]  
\newtheorem{lemma}[theorem]{Lemma}
\newtheorem{proposition}[theorem]{Proposition}
\newtheorem{corollary}[theorem]{Corollary}
\newtheorem{definition}[theorem]{Definition}
\newtheorem{claim}[theorem]{Claim}

% Calculus
\newcommand{\diff}{\mathop{}\!\mathrm{d}}
\newcommand{\deriv}[2]{\frac{\mathrm{d}#1}{\mathrm{d}#2}}
\newcommand{\pderiv}[2]{\frac{\partial #1}{\partial #2}}

% Linear Algebra
\newcommand{\inner}[2]{\langle #1, #2 \rangle}
\newcommand{\norm}[1]{\| #1 \|}
\newcommand{\tr}{\operatorname{tr}}
\newcommand{\spn}{\operatorname{span}}
\newcommand{\rank}{\operatorname{rank}}
\newcommand{\nullity}{\operatorname{nullity}}

% Logic
\newcommand{\contra}{\Rightarrow\Leftarrow}

% Custom commands for notes
\newcommand{\todo}[1]{\textcolor{red}{[TODO: #1]}}
\newcommand{\important}[1]{\textbf{\textcolor{blue}{#1}}}

%Number Theory
\DeclareMathOperator{\Li}{Li}
\newcommand{\floor}[1]{\left\lfloor #1 \right\rfloor}
\newcommand{\fract}[1]{\left\{ #1 \right\}}




\newcommand{\maketitlepage}{
    \begin{titlepage}
        \centering
        \vspace*{2.0cm}
        \pgfornament{84}\\
        {\LARGE \textsc{\coursename}\par}
        \vspace{0.5cm}
        {\large\coursecode\par}
        \vspace{0.5cm}
        {\large\instructor\par}
        \vspace{1.5cm}
        {\huge\bfseries\assignment\par}
        \vspace{1cm}
        {\LARGE\itshape\author\par}
        \vspace{2cm}
        {\large\bfseries Due Date:\par}
        \vspace{0.5cm}
        {\Large \duedate}\\
        \pgfornament{84}
    \end{titlepage}
}
% =============================================
% HOMEWORK CONFIGURATION - EDIT THESE VALUES!
% =============================================

% Your personal info
\renewcommand{\author}{Deepak Jassal}
\newcommand{\authorlast}{Jassal}

% Course info
\newcommand{\coursename}{Course Name}
\newcommand{\coursecode}{Course code}
\newcommand{\instructor}{Instructor}

% Assignment-specific info (CHANGE THESE FOR EACH HOMEWORK)
\newcommand{\assignment}{Assignment }
\newcommand{\duedate}{Month Day\textsuperscript{th}, 20XX}

% Header configuration
\fancyhead[l]{\assignment}
\fancyhead[c]{\coursecode}
\fancyhead[r]{\monthyear}
\fancyfoot[c]{\authorlast{ }\thepage}

\renewcommand{\author}{Deepak Jassal}
\renewcommand{\authorlast}{Jassal}
\renewcommand{\coursename}{Mathematical Statistics}
\renewcommand{\coursecode}{STAT 372}
\renewcommand{\assignment}{Assignment 1}
\renewcommand{\instructor}{Dr. Dan Ryan}
\renewcommand{\duedate}{February 6\textsuperscript{th}, 2026}


\begin{document}
\begin{titlepage}
	\centering
	\vspace*{2.0cm}	
	\pgfornament{84}\\
	{\LARGE \textsc{\coursename}\par}
	\vspace{0.5cm}
	{\large\coursecode\par}
    \vspace{0.5cm}
    {\large\instructor\par}
	\vspace{1.5cm}
	{\huge\bfseries\assignment\par}
	\vspace{1cm}  
	{\LARGE\itshape\author\par}
    \vspace{2cm}
	{\large\bfseries Due Date:\par}
	\vspace{0.5cm}
	{\Large \duedate}\\
	\pgfornament{84}
\end{titlepage}
\section*{Context}
This assignment is about setting up the maximum likelihood framework for a realistic manufacturing-style problem. You will receive the data values (x1, . . . , xn) in class on the next day. For now, your job is to build the model, write the likelihood, and determine the MLE in terms of the observed data.
\section*{Scenario}
You are working in quality control at a manufacturing plant. A machine produces long strips of material with a target length $\theta$, but the value of $\theta$ is unknown.\\
To inspect the process, the plant uses a destructive test that produces a single scrap piece from each strip. Only that scrap piece is recorded and stored.\\
The paper strips you are given simulate the data produced by this manufacturing process.
\section*{How the data are generated}
Each observed strip is produced in the following way:
\begin{enumerate}
    \item Start with a strip of material of total length $\theta$ (unknown).
    \item The strip is cut once at a location chosen uniformly along the strip.
    \item Two pieces are created.
    \item For consistency, the plant keeps only the shorter of the two pieces as the recorded scrap.
    \item This process is repeated independently many times.
\end{enumerate}
You do not see the original strip, the cut location, or the longer piece.
\section*{What you observe}
You receive one or more of the recorded scrap pieces.\\
Measure the length of each piece and record the values as
\[
    x_1,x_2,\dots,x_n.
\]
These measurements are your data.
\section*{Your task}
Using only the measured data and the description of the manufacturing process:
\begin{enumerate}
    \item Determine what values of $\theta$ are possible given the observed data.
    \item Propose a statistical model for a single observed measurement X.
    \item Write down the likelihood function
    \[
        L(\theta\vert x_1, x_2,\dots , x_n).
    \]
    \item Find the maximum likelihood estimator $\widehat{\theta}$ of the original strip length.
\end{enumerate}
Be prepared to explain how the physical constraints of the process shape both the likelihood and the estimator.

\stepcounter{section}
\section*{Solutions}
\subsection*{Question 1}
Let the observed scrap lengths be $x_1, x_2, \dots, x_n > 0$.
The data-generation process implies that each shorter piece satisfies
\[
0 < X \leq \frac{\theta}{2}.
\]
Thus, for every observation,
\[
x_i \leq \frac{\theta}{2} \quad \text{for } i = 1,\dots,n.
\]
Equivalently,
\[
\theta \geq 2x_i \quad \forall i.
\]
Let $x_{\text{max}} := \max\{x_1,\dots,x_n\}$.  
The strongest constraint is
\[
\theta \geq 2x_{\text{max}}.
\]
Hence, given the data and setup a possible parameter is
\[
    \colorbox{yellow}{$\theta \in [\,2x_{\text{max}},\infty).$}
\]
\newpage
\subsection*{Question 2}
Define the cut position (distance from one end) as
\[
    Y \sim \text{Uniform}(0,\theta).
\]
The two pieces are $Y$ and $\theta-Y$.  
The recorded scrap length is the shorter piece:
\[
    X = \min\{Y,\; \theta-Y\}.
\]

\subsection*{Support of $X$ given $\theta$}
Since $\min\{u,v\} \leq \frac{\theta}{2}$ for $u+v=\theta$, we have
\[
    0 < X \leq \frac{\theta}{2}.
\]

\subsection*{Derivation of the distribution}
For $0 < t < \theta/2$,
\begin{align*}
F_X(t) &= P(X \leq t) \\
       &= 1 - P(X > t) \\
       &= 1 - P\big(\min\{Y,\theta-Y\} > t\big) \\
       &= 1 - P(t < Y < \theta-t) \\
       &= 1 - \frac{(\theta-t) - t}{\theta} \\
       &= 1 - \frac{\theta - 2t}{\theta} \\
       &= \frac{2t}{\theta}.
\end{align*}
For $t \geq \theta/2$, $F_X(t) = 1$.

Differentiating, the probability density function is
\[
    \colorbox{yellow}{$f_X(x \mid \theta) = \frac{2}{\theta}, \quad 0 < x < \frac{\theta}{2}$,}
\]
and $0$ elsewhere.  

\subsection*{Summary}
\[
    \colorbox{yellow}{$X \sim \text{Uniform}\big(0,\;\theta/2\big)$.}
\]
\newpage
\subsection*{Question 3}
Given $n$ independent observations $x_1,\dots,x_n$,
\begin{align*}
L(\theta \mid x_1,\dots,x_n) 
&= \prod_{i=1}^n f_X(x_i \mid \theta) \\
&= \prod_{i=1}^n \frac{2}{\theta}, \quad 0 < x < \frac{\theta}{2}.
\end{align*}
The indicators require $\theta > 2x_i$ for all $i$, i.e., $\theta > 2x_{\text{max}}$.  
Therefore,
\[ 
    \colorbox{yellow}{$L(\theta) = 
    \begin{cases}
    \displaystyle\left(\frac{2}{\theta}\right)^n, & \theta \geq 2x_{\text{max}}, \\[10pt]
    0, & \theta < 2x_{\text{max}}.
    \end{cases}
    $
    }
\]
\subsection*{Question 4}
Instead of taking the derivative of the likelihood function we can observe that it is an exponential decay function (when we have $\theta \geq 2x_{\text{max}},$, the other situation clearly doesn't maximize $\theta$). Since it is an exponential decay function we have the max at the smallest possible value. In our situation that would be when $\theta=2\max\{x_1,\dots,x_n\}$. Thus, (using the notation from question 1s) we have
\[
    \colorbox{yellow}{$\widehat{\theta}=2x_{\text{max}}$.}
\]
\subsection*{Question 5}
Determine $\widehat{\theta}$ based on the values collected in class.\\
\subsection*{Collected Data (inches):}
\[
    \left\{1\frac{1}{16},3.0,2\frac{3}{4},3.0,1\frac{13}{16},1\frac{5}{16},2\frac{1}{2},3\frac{9}{32},\frac{27}{32},2\frac{1}{32},3\frac{6}{32}\right\}
\]
From the data we see that $x_{\text{max}}=3\frac{6}{32}$. Then
\[
    \colorbox{yellow}{$\widehat{\theta}=2\times3\frac{6}{32}=6\frac{3}{8}$}
\]
\end{document}