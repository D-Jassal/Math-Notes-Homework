\documentclass[12pt]{article}

% Import preambles and macros for homework
% Essential packages
\usepackage{amsmath, amsfonts, amssymb, amsthm}
\usepackage{mathtools}
\usepackage{enumitem}
\usepackage{graphicx}
\usepackage{wrapfig}
\usepackage{systeme}
\usepackage{caption}
\usepackage{soul}
\usepackage[dvipsnames]{xcolor}
\usepackage{fancyhdr}
\allowdisplaybreaks

% Page layout
\usepackage[
  top=2cm,
  bottom=2cm,
  left=2cm,
  right=2cm,
  headheight=17pt,
  includehead,includefoot,
  heightrounded,
]{geometry}


% pgfornament for title page decorations
\usepackage[object=vectorian]{pgfornament}

% Fancy header/footer setup
\pagestyle{fancy}
\setlength{\headheight}{14.49998pt}
\addtolength{\topmargin}{-2.49998pt}
\renewcommand{\footrulewidth}{0.4pt}
\setlength\parindent{15pt}
% Math notation shortcuts
\newcommand{\R}{\mathbb{R}}
\newcommand{\Q}{\mathbb{Q}}
\newcommand{\Z}{\mathbb{Z}}
\newcommand{\N}{\mathbb{N}}
\newcommand{\C}{\mathbb{C}}
\newcommand{\X}{\mathcal{X}}

% Theorem environments
\newtheorem{mainthm}{Theorem}[section]
\newtheorem{theorem}{Theorem}[section]  
\newtheorem{lemma}[theorem]{Lemma}
\newtheorem{proposition}[theorem]{Proposition}
\newtheorem{corollary}[theorem]{Corollary}
\newtheorem{definition}[theorem]{Definition}
\newtheorem{claim}[theorem]{Claim}

% Calculus
\newcommand{\diff}{\mathop{}\!\mathrm{d}}
\newcommand{\deriv}[2]{\frac{\mathrm{d}#1}{\mathrm{d}#2}}
\newcommand{\pderiv}[2]{\frac{\partial #1}{\partial #2}}

% Linear Algebra
\newcommand{\inner}[2]{\langle #1, #2 \rangle}
\newcommand{\norm}[1]{\| #1 \|}
\newcommand{\tr}{\operatorname{tr}}
\newcommand{\spn}{\operatorname{span}}
\newcommand{\rank}{\operatorname{rank}}
\newcommand{\nullity}{\operatorname{nullity}}

% Logic
\newcommand{\contra}{\Rightarrow\Leftarrow}

% Custom commands for notes
\newcommand{\todo}[1]{\textcolor{red}{[TODO: #1]}}
\newcommand{\important}[1]{\textbf{\textcolor{blue}{#1}}}

%Number Theory
\DeclareMathOperator{\Li}{Li}
\newcommand{\floor}[1]{\left\lfloor #1 \right\rfloor}
\newcommand{\fract}[1]{\left\{ #1 \right\}}




\newcommand{\maketitlepage}{
    \begin{titlepage}
        \centering
        \vspace*{2.0cm}
        \pgfornament{84}\\
        {\LARGE \textsc{\coursename}\par}
        \vspace{0.5cm}
        {\large\coursecode\par}
        \vspace{0.5cm}
        {\large\instructor\par}
        \vspace{1.5cm}
        {\huge\bfseries\assignment\par}
        \vspace{1cm}
        {\LARGE\itshape\author\par}
        \vspace{2cm}
        {\large\bfseries Due Date:\par}
        \vspace{0.5cm}
        {\Large \duedate}\\
        \pgfornament{84}
    \end{titlepage}
}
% =============================================
% HOMEWORK CONFIGURATION - EDIT THESE VALUES!
% =============================================

% Your personal info
\renewcommand{\author}{Deepak Jassal}
\newcommand{\authorlast}{Jassal}

% Course info
\newcommand{\coursename}{Course Name}
\newcommand{\coursecode}{Course code}
\newcommand{\instructor}{Instructor}

% Assignment-specific info (CHANGE THESE FOR EACH HOMEWORK)
\newcommand{\assignment}{Assignment }
\newcommand{\duedate}{Month Day\textsuperscript{th}, 20XX}

% Header configuration
\fancyhead[l]{\assignment}
\fancyhead[c]{\coursecode}
\fancyhead[r]{\monthyear}
\fancyfoot[c]{\authorlast{ }\thepage}

\renewcommand{\author}{Deepak Jassal}
\renewcommand{\authorlast}{Jassal}
\renewcommand{\coursename}{Introductory Complex Analysis}
\renewcommand{\coursecode}{MATH 301}
\renewcommand{\assignment}{Assignment 1}
\renewcommand{\instructor}{Dr. Edward Dobrowolski}
\renewcommand{\duedate}{January 23\textsuperscript{th}, 2026}


\begin{document}
\begin{titlepage}
	\centering
	\vspace*{2.0cm}	
	\pgfornament{84}\\
	{\LARGE \textsc{\coursename}\par}
	\vspace{0.5cm}
	{\large\coursecode\par}
    \vspace{0.5cm}
    {\large\instructor\par}
	\vspace{1.5cm}
	{\huge\bfseries\assignment\par}
	\vspace{1cm}  
	{\LARGE\itshape\author\par}
    \vspace{2cm}
	{\large\bfseries Due Date:\par}
	\vspace{0.5cm}
	{\Large \duedate}\\
	\pgfornament{84}
\end{titlepage}
\stepcounter{section}
\section*{Problem 1}
Express in the form $a+bi$, $a,b\in\R$
\begin{enumerate}[label=(\alph*)]
    \item $(2+i)^4$
    \begin{align*}
		(2+4i)^4&=((2+4i)^2)^2\\
		&=(4+4i+i^2)^2\\
		&=(3+4i)^2\\
		&=9+24i^16i^2\\
		&=-7+24i.
	\end{align*}
    \item $\frac{3+4i}{4+3i}$
    \begin{align*}
		\frac{3+4i}{4+3i}&=\frac{3+4i}{4+3i}\cdot\frac{4-3i}{4-3i}\\
		&=\frac{12+7i-12i^2}{4^2+3^2}\\
		&=\frac{24+7i}{25}\\
		&=\frac{24}{25}+\frac{7}{25}i.
	\end{align*}
\end{enumerate}
\stepcounter{section}
\section*{Problem 2}
Find the three roots of $z^3-4=0$.
\[
	z^3-4=0\Leftrightarrow z^3=4.
\]
\begin{align*}
	z^3&=4\cdot e^{i2\pi n},\quad n \in\Z\\
	z_n&=4^{\frac{1}{3}}\cdot e^{\frac{i2\pi n}{3}},\quad k=0,1,2\\
	&=2^{\frac{2}{3}}\cdot e^{\frac{i2\pi n}{3}}\\
	\intertext{For $n=0$}
	z_0&=2^{\frac{2}{3}}\cdot e^{0i}=2^{\frac{2}{3}}\\
	\intertext{For $n=1$}
	z_1&=2^{\frac{2}{3}}\cdot e^{\frac{i2\pi 1}{3}}=2^{\frac{2}{3}}\left(-\frac{1}{2}+i\frac{\sqrt{3}}{2}\right)\\
	\intertext{For $n=2$}
	z_2&=2^{\frac{2}{3}}\cdot e^{\frac{i2\pi 2}{3}}=2^{\frac{2}{3}}\left(-\frac{1}{2}-i\frac{\sqrt{3}}{2}\right).
\end{align*}
\stepcounter{section}
\section*{Problem 3}
Assuming that either $|z| = 1$ or $|w| = 1$ and $\overline{z}w\neq1$, prove that
\[
    \left|\frac{z-w}{1-\overline{z}w}\right|=1.
\]
Let $E=\frac{z-w}{1-\overline{z}w}$. Then we have
\begin{align*}
	|E|^2&=E\overline{E}\\
	&=\frac{z-w}{1-\overline{z}w}\cdot\frac{\overline{z}-\overline{w}}{1-z\overline{w}}\\
	&=\frac{z\overline{z}-z\overline{w}-w\overline{z}+w\overline{w}}{1-z\overline{w}-\overline{z}w+\overline{z}w\cdot z\overline{w}}\\
	&=\frac{|z|^2-z\overline{w}-\overline{z}w+|w|^2}{1+|z|^2|w|^2-z\overline{w}-\overline{z}w}.
\end{align*}
Since both the numerator and the denominator are both positive real numbers we can take there difference, and if the difference is 0 then then we have $|E|^2=1$.
\[
	|z|^2-z\overline{w}-\overline{z}w+|w|^2-(1+|z|^2|w|^2-z\overline{w}-\overline{z}w)
\]
\[
	(|z|^2+|w|^2)-(1+|z|^2|w|^2)=|z|^2+|w|^2-1-|z|^2|w|^2.
\]
Case 1: $|z|=1$
\begin{align*}
	|z|^2+|w|^2-1-|z|^2|w|^2&=1+|w|^2-1-|w|^2\\
	&=0.
\end{align*}
Case 2: $|w|=1$
\begin{align*}
	|z|^2+|w|^2-1-|z|^2|w|^2&=1+|z|^2-1-|z|^2\\
	&=0.
\end{align*}
Since the difference between the numerator and denominator is 0 in both cases we have
\[
	|E|^2=1.
\]
Since $|E|\geq 0$ we have 
\[
	|E|=\left|\frac{z-w}{1-\overline{z}w}\right|=1.
\]
\stepcounter{section}
\section*{Problem 4}
Use complex numbers to prove identity:
\[
    1+\cos(\theta)+\cos(2\theta)+\cdots+\cos(n\theta)=\frac{1}{2}+\frac{\sin\left(n+\frac{1}{2}\theta\right)}{2\sin\left(\frac{\theta}{2}\right)}.
\]
\stepcounter{section}
\section*{Problem 5}
Solve $\sin(z)=\frac{3}{4}+i/4$.\\
\[
	\sin(z)=\frac{e^{ix}-e^{-ix}}{2i}=\frac{3}{4}+i/4.
\]
Set $w=e^{ix}$. Then,
\[
	\frac{w-w^{-1}}{2i}=\frac{3}{4}+i/4.
\]
Multiply both sides by $2i$.
\[
w-\frac{1}{w}=2i\left(\frac{3}{4} + \frac{i}{4}\right) = \frac{3i}{2} - \frac{1}{2}.
\]
Let $b=\frac{3i}{2} - \frac{1}{2}$ and multiply both sides by $w$.
\[
w^2-1=bw
\]
\[
	w^2-bw-1=0.
\]
Solving for $w$
\[
	w=\frac{b \pm \sqrt{b^2 + 4}}{2}.
\]
\[
	b^2=\left(\frac{3i}{2} - \frac{1}{2}\right)^2= \frac{1}{4} - \frac{3i}{2} + \frac{9i^2}{4}= \frac{1}{4} - \frac{3i}{2} - \frac{9}{4} = -2 - \frac{3i}{2}.
\]
\[
	b^2+4=2 - \frac{3i}{2}=\frac{4-3i}{2}.
\]
\stepcounter{section}
\section*{Problem 6}
Find all values of $\log(-i)$.
\begin{align*}
	\log(-i)&=\log|-i|-i\frac{\pi}{2}+2\pi ni,\quad n\in\Z,\\
	&=-i\frac{\pi}{2}+2\pi ni,\quad n\in\Z.
\end{align*}
\stepcounter{section}
\section*{Problem 7}
Find all values of $1^i$.\\
$1=e^{0i+2\pi ni},$ $n\in\Z$. Thus, we can rewrite $1^i$ as 
\[
	1^i=e^{i(0i+2\pi ni)}=e^{-2\pi n}
\]
where $n\in\Z$.
\stepcounter{section}
\section*{Problem 8}
Prove that $1-\cos(2z)=2\sin^2(z)$.
\stepcounter{section}
\section*{Problem 9}
Use De Moivre's formula to express $\cos(7\theta)$ in terms of $\cos(\theta)$ and $sin(\theta)$.\\
Let $z=\cos(\theta)+i\sin(\theta)$. Then by De Moivre's theorem:
\begin{align*}
    z^7 &= \cos(7\theta) + i\sin(7\theta) \\
    &= (\cos(\theta) + i\sin(\theta))^7.
\end{align*}
Using the binomial expansion:
\begin{align*}
    (\cos\theta + i\sin\theta)^7 &= \sum_{k=0}^{7} \binom{7}{k} (i\sin\theta)^k (\cos\theta)^{7-k} \\
    &= \cos^7\theta + 7i\cos^6\theta\sin\theta - 21\cos^5\theta\sin^2\theta - 35i\cos^4\theta\sin^3\theta \\
    &\quad + 35\cos^3\theta\sin^4\theta + 21i\cos^2\theta\sin^5\theta - 7\cos\theta\sin^6\theta - i\sin^7\theta.
\end{align*}
$\cos(7\theta)$ consists of the real parts of the above equation, so we have
\[
	\cos(7\theta)= \cos^7\theta- 21\cos^5\theta\sin^2\theta + 35\cos^3\theta\sin^4\theta - 7\cos\theta\sin^6\theta.
\]

\end{document}