\documentclass[12pt]{article}

% Import preambles and macros for homework
% Essential packages
\usepackage{amsmath, amsfonts, amssymb, amsthm}
\usepackage{mathtools}
\usepackage{enumitem}
\usepackage{graphicx}
\usepackage{wrapfig}
\usepackage{systeme}
\usepackage{caption}
\usepackage{soul}
\usepackage[dvipsnames]{xcolor}
\usepackage{fancyhdr}
\allowdisplaybreaks

% Page layout
\usepackage[
  top=2cm,
  bottom=2cm,
  left=2cm,
  right=2cm,
  headheight=17pt,
  includehead,includefoot,
  heightrounded,
]{geometry}


% pgfornament for title page decorations
\usepackage[object=vectorian]{pgfornament}

% Fancy header/footer setup
\pagestyle{fancy}
\setlength{\headheight}{14.49998pt}
\addtolength{\topmargin}{-2.49998pt}
\renewcommand{\footrulewidth}{0.4pt}
\setlength\parindent{15pt}
% Math notation shortcuts
\newcommand{\R}{\mathbb{R}}
\newcommand{\Q}{\mathbb{Q}}
\newcommand{\Z}{\mathbb{Z}}
\newcommand{\N}{\mathbb{N}}
\newcommand{\C}{\mathbb{C}}
\newcommand{\X}{\mathcal{X}}

% Theorem environments
\newtheorem{mainthm}{Theorem}[section]
\newtheorem{theorem}{Theorem}[section]  
\newtheorem{lemma}[theorem]{Lemma}
\newtheorem{proposition}[theorem]{Proposition}
\newtheorem{corollary}[theorem]{Corollary}
\newtheorem{definition}[theorem]{Definition}
\newtheorem{claim}[theorem]{Claim}

% Calculus
\newcommand{\diff}{\mathop{}\!\mathrm{d}}
\newcommand{\deriv}[2]{\frac{\mathrm{d}#1}{\mathrm{d}#2}}
\newcommand{\pderiv}[2]{\frac{\partial #1}{\partial #2}}

% Linear Algebra
\newcommand{\inner}[2]{\langle #1, #2 \rangle}
\newcommand{\norm}[1]{\| #1 \|}
\newcommand{\tr}{\operatorname{tr}}
\newcommand{\spn}{\operatorname{span}}
\newcommand{\rank}{\operatorname{rank}}
\newcommand{\nullity}{\operatorname{nullity}}

% Logic
\newcommand{\contra}{\Rightarrow\Leftarrow}

% Custom commands for notes
\newcommand{\todo}[1]{\textcolor{red}{[TODO: #1]}}
\newcommand{\important}[1]{\textbf{\textcolor{blue}{#1}}}

%Number Theory
\DeclareMathOperator{\Li}{Li}
\newcommand{\floor}[1]{\left\lfloor #1 \right\rfloor}
\newcommand{\fract}[1]{\left\{ #1 \right\}}




\newcommand{\maketitlepage}{
    \begin{titlepage}
        \centering
        \vspace*{2.0cm}
        \pgfornament{84}\\
        {\LARGE \textsc{\coursename}\par}
        \vspace{0.5cm}
        {\large\coursecode\par}
        \vspace{0.5cm}
        {\large\instructor\par}
        \vspace{1.5cm}
        {\huge\bfseries\assignment\par}
        \vspace{1cm}
        {\LARGE\itshape\author\par}
        \vspace{2cm}
        {\large\bfseries Due Date:\par}
        \vspace{0.5cm}
        {\Large \duedate}\\
        \pgfornament{84}
    \end{titlepage}
}
% =============================================
% HOMEWORK CONFIGURATION - EDIT THESE VALUES!
% =============================================

% Your personal info
\renewcommand{\author}{Deepak Jassal}
\newcommand{\authorlast}{Jassal}

% Course info
\newcommand{\coursename}{Course Name}
\newcommand{\coursecode}{Course code}
\newcommand{\instructor}{Instructor}

% Assignment-specific info (CHANGE THESE FOR EACH HOMEWORK)
\newcommand{\assignment}{Assignment }
\newcommand{\duedate}{Month Day\textsuperscript{th}, 20XX}

% Header configuration
\fancyhead[l]{\assignment}
\fancyhead[c]{\coursecode}
\fancyhead[r]{\monthyear}
\fancyfoot[c]{\authorlast{ }\thepage}

\renewcommand{\author}{Deepak Jassal}
\renewcommand{\authorlast}{Jassal}
\renewcommand{\coursename}{Introductory Complex Analysis}
\renewcommand{\coursecode}{MATH 301}
\renewcommand{\assignment}{Assignment 3}
\renewcommand{\instructor}{Dr. Edward Dobrowolski}
\renewcommand{\duedate}{February 25\textsuperscript{th}, 2026}


\begin{document}
\begin{titlepage}
	\centering
	\vspace*{2.0cm}	
	\pgfornament{84}\\
	{\LARGE \textsc{\coursename}\par}
	\vspace{0.5cm}
	{\large\coursecode\par}
    \vspace{0.5cm}
    {\large\instructor\par}
	\vspace{1.5cm}
	{\huge\bfseries\assignment\par}
	\vspace{1cm}  
	{\LARGE\itshape\author\par}
    \vspace{2cm}
	{\large\bfseries Due Date:\par}
	\vspace{0.5cm}
	{\Large \duedate}\\
	\pgfornament{84}
\end{titlepage}
\stepcounter{section}
\section*{Page 1 Question 1} DIfferntiate and give the appropriate region of analyticity for each of the following:
\begin{enumerate}
    \item[(c)]$f(z)=\sin z/\cos z$
    \begin{align*}
        \deriv{f(z)}{z}&=\deriv{\tan z}{z}\\
        &=\sec^2z\\
        &=\frac{1}{\cos^2z}.
    \end{align*}
    The region of analyticity is whenever $\cos z\neq0$. That being $\C\setminus\left\{\frac{\pi}{2}+n\pi:n\in\Z\right\}$.
    \item[(d)] $f(z)=\exp\left(\frac{z^3+1}{z-1}\right)$. 
    \begin{align*}
        \deriv{f(z)}{z}&=\exp\left(\frac{z^3+1}{z-1}\right)\cdot\frac{(3z^2)(z-1)-(z^3+1)}{(z-1)^2}\\
        &=\exp\left(\frac{z^3+1}{z-1}\right)\cdot\frac{2z^2-3z^2-1}{(z-1)^2}.
    \end{align*}
    The exponential function is entire, but the radical has a pole due to $z-1$ in the denominator. So the region of analyticity is $\C\setminus\{1\}$.
\end{enumerate}
\stepcounter{section}
\section*{Page 1 Question 2}
\begin{enumerate}
    \item[(a)] $f(z)=3^z$.
    \[
        f(z)=e^{z\log3}.
    \]
    The exponential function is entire, so the region of analyticity is $\C$.
    \item[(c)] $f(z)=z^{(1+i)}$
    \begin{align*}
        f(z)&=e^{(1+i)\log z}\\
        \deriv{f(z)}{z}&=e^{(1+i)\log z}\cdot\frac{1+i}{z}\\
        &=\frac{z^{1+i}}{z}\cdot(1+i)\\
        &=(1+i)e^{i\log z}.
    \end{align*}
    The region of analyticity is $\C\setminus (-\infty,0]$, due to the branch cut for the priciple branch of $\log z$.
    \item[(e)] $f(z)=\sqrt[3]{z}$
    \begin{align*}
        f(z)&=z^{\frac{1}{3}}\\
        &=e^{\frac{1}{3}\log z}.
    \end{align*}
    Again, the region of analyticity is $\C\setminus (-\infty,0]$.
\end{enumerate}
\stepcounter{section}
\section*{Page 1 Question 4} Determine whether the following complex limits exist and find their value if they do:
\begin{enumerate}
    \item[(a)]
    \[
        \lim_{z\to0}\frac{e^z-1}{z}.
    \]
    Since $|z|\leq\infty$ in the limite we can rewrite the function as 
    \[
        \frac{e^z-1}{z}=1+\frac{z}{2!}+\frac{z^2}{3!}+\cdots
    \]
    this function is entire, so the limit does exist. It is clear to see that the limit approaches 1 as $z$ approaches 0.
    \item[(b)] 
    \[
        \lim_{z\to0}\frac{\sin|z|}{z}.
    \]
    First we check the limit with $z=a,a\in\R$.
    \begin{align*}
        \frac{\sin|z|}{z}&=\frac{\sin a}{a}\\
        &=1,
    \end{align*}
    as $a\to0$
    Next we check with $z=ib$.
        \begin{align*}
        \frac{\sin|z|}{z}&=\frac{\sin b}{i}\\
        &=\frac{1}{i}\\
        &=-i,
    \end{align*}
    as $t\to0$. Since, two paths do not agree the limit does not exist.
\end{enumerate}
\stepcounter{section}
\section*{Page 1 Question 5} Is it true that $|\sin z|\leq1$ for all $z\in\C$?\\
No, 
\begin{align*}
    \sin z&=\frac{e^{iz}-e^{-iz}}{2i}\\
    \sin i&=\frac{e^{i\cdot i}-e^{-i\cdot i}}{2i}\\
    |\sin i|&=\frac{e-e^{-1}}{2}\\
    &\approx 1.1752>1.
\end{align*}
\stepcounter{section}
\section*{Page 2 Question 10} Find the region of analyticity and the derivative of each of the following functions:
\begin{enumerate}
    \item[(a)] $f(z)=\sqrt{z^3-1}$. Let $g(z)=z^3-1$ then
    \begin{align*}
        \deriv{f(z)}{z}&=\frac{1}{2}g(z)^{-\frac{1}{2}}\cdot g(z)'\\
        &=\frac{3z^2}{\sqrt{z^3-1}}.
    \end{align*}
    We need to ensure that $\sqrt{z^3-1}\neq0$, so we cannot have $z$ equal to cube roots of unity, those being
    \[
        \C\setminus\{2,e^{\frac{2\pi i}{3}},e^{\frac{4\pi i}{3}}\}
    \]
\end{enumerate}
\stepcounter{section}
\section*{Page 2 Question 11} Find the minimum of of $|e^{z^2}|$ for those $z$ with $|z|\leq 1$.\\
\textit{Solution.} Let $z=a+bi$. Now note that 
\[
    |e^{z^2}|=e^{\Re(z^2)}
\]
and 
\[
    z^2=(a+bi)^2=a^2-b^2+2abi.
\]
Then
\[
    \Re(z^2)=a^2-b^2.
\]
Now we have
\[
    |e^{z^2}|=e^{a^2-b^2}.
\]
Since $e^x$ is increasing we need to find the minimum value of $a^2-b^2$ such that $a^2+b^2\leq1$. Set $r^2=a^2+b^2$
\[
    a^2-b^2=a^2-(r^2-a^2)=2a^2-r^2\Rightarrow a^2-b^2=2a^2-r^2.
\]
For fixed $r$, the minimum occurs when $a^2$ is minimal: $a^2\geq0$. Minimum $a^2$ is 0, giving $a^2-b^2=-r^2$ at $a=0$, $y=\pm r$. So for fixed $r$, $\min\{a^2-b^2\}=-r^2$. This minimum occurs at $r=1$ and gives -1, and this happens when $a=0$ and $b=\pm1$, or $z=\pm i$. So the minimums article
\[
    |e^{i^2}|=|e^{-i^2}|=e^{-1}=\frac{1}{e}.
\]
\stepcounter{section}
\section*{Page 3 Question 2} For what values of $z$ is $\log z^2=2\log z$ if the principal branch of the logarithm is used on both sides of the equation.\\
\textit{Solution.}
\begin{align*}
    \log z^2&=2\log z\\
    \log |z^2|+i\arg z^2&=2\log |z|+2i\arg z\\
    2\log |z|+i\arg z^2&=2\log |z|+2i\arg z\\
    i\arg z^2&=2i\arg z\\
    \arg z^2&=2\arg z.
\end{align*}
So we have $\arg z^2\equiv 2\arg z\mod 2\pi$. This gives $-\pi<2\arg z\leq\pi$.
\begin{align*}
    -\pi<2\arg z\leq\pi\\
    -\frac{\pi}{2}<\arg z\leq\frac{\pi}{2}\\
    \arg z\in \left(-\frac{\pi}{2},\frac{\pi}{2}\right].
\end{align*}
This corresponds to $z$ with $\Re(z)\geq0$ and $\Im(z)\geq0$. So the set of values of $z$ is
\[
    \{z\in\C:\Re(z)\geq0\}\cup\{z\in\C:\Re(z)=0,\Im(z)\geq0\}.
\]
\stepcounter{section}
\section*{Page 3 Question 7} Describe geometrically the set of points $z\in\C$ satisfying
\begin{enumerate}
    \item[(b)] $|z-1|=3|z-2|$. Let $z=a+bi$ $a,b\in\R$
    \begin{align*}
        |z-1|&=3|z-2|\\
        \sqrt{(a-1)^2+b^2}&=3\sqrt{(a-2)^2+b^2}\\
        x^2-2x+1+y^2&=9x^2-36x+36+9y^2\\
        x^2-2x+1+y^2-9x^2+36x-36-9y^2&=0\\
        8x^2+8y^2-34x+35&=0\\
        x^2-\frac{17}{4}x+y^2+\frac{35}{8}&=0\\
        \left(x-\frac{17}{8}\right)^2-\frac{289}{64}+y^2+\frac{35}{8}&=0\\
        \left(x-\frac{17}{8}\right)^2+y^2-\frac{9}{64}&=0\\
        \left(x-\frac{17}{8}\right)^2+y^2&=\frac{9}{64}.
    \end{align*}
    Geometrically this is a circle in $\C$ with centre $(\frac{17}{8},0)$ and radius $\frac{3}{8}$.
\end{enumerate}
\stepcounter{section}
\section*{Page 4 Question 13} Can a single-valued (analytic) branch of $\log z$ be defined on the following sets?
\begin{enumerate}[label=(\alph*)]
    \item $\{z\mid1<|z|<2\}$\\
    \textit{Solution.} This is an annulus and is not simply connected, so you cannot define a single-valued branch of $\log z$.
    \item $\{z\mid\Re z>0\}$\\
    \textit{Solution.} This is the right half plane not including the imaginary axis. It is simply connected so we can define a single-valued branch of $\log z$.
    \item $\{z\mid\Re z>\Im z\}$\\
    \textit{Solution.} This is the half plane under the diaganol line $a=b$, $(z=a+bi)$. It is simply connected so we can define a single-valued branch of $\log z$.
\end{enumerate}

\stepcounter{section}
\section*{Page 4 Question 15} Let $f$ be analytic on $A$. Define $g:A\to\C$ by $g(z)=\overline{f(z)}$. When is $g$ analytic?\\
\textit{Solution.} Let $z=x+yi$ $x,y\in\R$. Write $f(z)=u(x,y)+iv(x,y)$. So $g(z)=u(x,y)-iv(x,y)$. From the Cauchy-Riemann equations we have
\[
    u_x=v_y,\quad u_y=-v_x.
\]
For $g(z)$ we have
\[
    U(x,y)=u(x,y),\quad V(x,y)=-v(x,y).
\]
We need
\[
    U_x=V_y,\quad U_y=-V_x.
\]
\[
    u_x=-v_y\Rightarrow u_x=-u_x\Rightarrow u_x=0,
\]
\[
    u_y=-(-v_x)\Rightarrow u_y=v_x\Rightarrow u_y=-u_y\Rightarrow u_y=0.
\]
$u_x=0$ and $u_y=0$ implies that $u(x,y)$ is constant on $A$. From the Cauchy-Riemann equations we have $u_x=v_y=0$ and $-v_x=u_y=0$. Then also $v(x,y)$ is constant. So we have that $f(x,y)=u(x,y)+iv(x,y)=C$ for some $C\in\C$. So $g(z)$ is analytic when $f(z)$ is constant.

\stepcounter{section}
\section*{Page 4 Question 19} Suppose that $f:A\subset\C\to\C$ is analytic on the open connected set $A$ and that $f(z)$ is real for all $z\in A$. Show that $f$ is constant.\\
\textit{Solution.} Let $z=x+iy$, $x,y\in\R$, write $f(x,y)=u(x,y)+iv(x,y)$. Since we have that $f(z)\in\R$ for all $z\in A$, we have that $v=0$. Then since $f$ is analytic by the Cauchy-Riemann equations we have
\[
    u_x=v_y,\quad u_y=-v_x
\]
\[
    v_y=v_x=0,\Rightarrow u_x=u_y=0.
\]
So $u$ and $v$ are constant meaning that $f$ is constant.
\section*{Page 5 Question 8} Use the inverse function theorem to show that if $f:A\to\C$ is analytic and $f'(z)\neq0$ for all $z\in A$, then $f$ maps open sets in $A$ to open sets.
\begin{proof}
    We have that $f$ is analytic on $A$ meaning that it is continuous. We want to show that for all open sets $U\subset A$ we have that $f(U)$ is also an open set. Consider $u\in A$ such that $f(u)\in f(U)$. Then by the given and the inverse function theorem we have that $f'(u)\neq0$ and that there exists an open neighbourhood of $u$, say $V$ such that $f(u)\in f(V)\subset f(U)$ since the mapping is bijective and $V\subset U$. Thus, we have that for all $u\in U$, $f(u)$ is an interior point of $f(U)$, meaning $f(U)$ is open.
\end{proof}
\stepcounter{section}
\section*{Page 5 Question 15} Suppose that $f$ is an analytic function on the disk $D=\{z:|z|<1\}$ and that $\Re f(z)=3$ for all $z\in D$. Show that $f$ is constant on $D$.\\
\textit{Solution.} Let $z=x+iy$ $x,y\in\R$. Write $f(x,y)=u(x,y)+iv(x,y)$. 
\[
    u_x=u_y=0.
\] 
By the Cauchy-Riemann equations we have
\[
    u_x=v_y=0,\quad u_y=-v_x=0.
\]
Both the imaginary and real part of $f$ are constant so we have that $f$ is constant.
\end{document}