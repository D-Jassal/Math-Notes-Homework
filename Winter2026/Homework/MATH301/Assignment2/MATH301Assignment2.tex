\documentclass[12pt]{article}

% Import preambles and macros for homework
% Essential packages
\usepackage{amsmath, amsfonts, amssymb, amsthm}
\usepackage{mathtools}
\usepackage{enumitem}
\usepackage{graphicx}
\usepackage{wrapfig}
\usepackage{systeme}
\usepackage{caption}
\usepackage{soul}
\usepackage[dvipsnames]{xcolor}
\usepackage{fancyhdr}
\allowdisplaybreaks

% Page layout
\usepackage[
  top=2cm,
  bottom=2cm,
  left=2cm,
  right=2cm,
  headheight=17pt,
  includehead,includefoot,
  heightrounded,
]{geometry}


% pgfornament for title page decorations
\usepackage[object=vectorian]{pgfornament}

% Fancy header/footer setup
\pagestyle{fancy}
\setlength{\headheight}{14.49998pt}
\addtolength{\topmargin}{-2.49998pt}
\renewcommand{\footrulewidth}{0.4pt}
\setlength\parindent{15pt}
% Math notation shortcuts
\newcommand{\R}{\mathbb{R}}
\newcommand{\Q}{\mathbb{Q}}
\newcommand{\Z}{\mathbb{Z}}
\newcommand{\N}{\mathbb{N}}
\newcommand{\C}{\mathbb{C}}
\newcommand{\X}{\mathcal{X}}

% Theorem environments
\newtheorem{mainthm}{Theorem}[section]
\newtheorem{theorem}{Theorem}[section]  
\newtheorem{lemma}[theorem]{Lemma}
\newtheorem{proposition}[theorem]{Proposition}
\newtheorem{corollary}[theorem]{Corollary}
\newtheorem{definition}[theorem]{Definition}
\newtheorem{claim}[theorem]{Claim}

% Calculus
\newcommand{\diff}{\mathop{}\!\mathrm{d}}
\newcommand{\deriv}[2]{\frac{\mathrm{d}#1}{\mathrm{d}#2}}
\newcommand{\pderiv}[2]{\frac{\partial #1}{\partial #2}}

% Linear Algebra
\newcommand{\inner}[2]{\langle #1, #2 \rangle}
\newcommand{\norm}[1]{\| #1 \|}
\newcommand{\tr}{\operatorname{tr}}
\newcommand{\spn}{\operatorname{span}}
\newcommand{\rank}{\operatorname{rank}}
\newcommand{\nullity}{\operatorname{nullity}}

% Logic
\newcommand{\contra}{\Rightarrow\Leftarrow}

% Custom commands for notes
\newcommand{\todo}[1]{\textcolor{red}{[TODO: #1]}}
\newcommand{\important}[1]{\textbf{\textcolor{blue}{#1}}}

%Number Theory
\DeclareMathOperator{\Li}{Li}
\newcommand{\floor}[1]{\left\lfloor #1 \right\rfloor}
\newcommand{\fract}[1]{\left\{ #1 \right\}}




\newcommand{\maketitlepage}{
    \begin{titlepage}
        \centering
        \vspace*{2.0cm}
        \pgfornament{84}\\
        {\LARGE \textsc{\coursename}\par}
        \vspace{0.5cm}
        {\large\coursecode\par}
        \vspace{0.5cm}
        {\large\instructor\par}
        \vspace{1.5cm}
        {\huge\bfseries\assignment\par}
        \vspace{1cm}
        {\LARGE\itshape\author\par}
        \vspace{2cm}
        {\large\bfseries Due Date:\par}
        \vspace{0.5cm}
        {\Large \duedate}\\
        \pgfornament{84}
    \end{titlepage}
}
% =============================================
% HOMEWORK CONFIGURATION - EDIT THESE VALUES!
% =============================================

% Your personal info
\renewcommand{\author}{Deepak Jassal}
\newcommand{\authorlast}{Jassal}

% Course info
\newcommand{\coursename}{Course Name}
\newcommand{\coursecode}{Course code}
\newcommand{\instructor}{Instructor}

% Assignment-specific info (CHANGE THESE FOR EACH HOMEWORK)
\newcommand{\assignment}{Assignment }
\newcommand{\duedate}{Month Day\textsuperscript{th}, 20XX}

% Header configuration
\fancyhead[l]{\assignment}
\fancyhead[c]{\coursecode}
\fancyhead[r]{\monthyear}
\fancyfoot[c]{\authorlast{ }\thepage}

\renewcommand{\author}{Deepak Jassal}
\renewcommand{\authorlast}{Jassal}
\renewcommand{\coursename}{Introductory Complex Analysis}
\renewcommand{\coursecode}{MATH 301}
\renewcommand{\assignment}{Assignment 2}
\renewcommand{\instructor}{Dr. Edward Dobrowolski}
\renewcommand{\duedate}{January 30\textsuperscript{th}, 2026}


\begin{document}
\begin{titlepage}
	\centering
	\vspace*{2.0cm}	
	\pgfornament{84}\\
	{\LARGE \textsc{\coursename}\par}
	\vspace{0.5cm}
	{\large\coursecode\par}
    \vspace{0.5cm}
    {\large\instructor\par}
	\vspace{1.5cm}
	{\huge\bfseries\assignment\par}
	\vspace{1cm}  
	{\LARGE\itshape\author\par}
    \vspace{2cm}
	{\large\bfseries Due Date:\par}
	\vspace{0.5cm}
	{\Large \duedate}\\
	\pgfornament{84}
\end{titlepage}
\stepcounter{section}
\section*{Problem 1}
Show that is $w\in\C$ then 
\begin{enumerate}[label=(\alph*)]
    \item $|\Re w|\leq |w|$
    \begin{proof}
        Let $w=a+bi$ with $a,b\in\R$. Then we have
        \begin{align*}
            a^2&\leq a^2+b^2\\
            \sqrt{a^2}&\leq \sqrt{a^2+b^2}\\
            |\Re w|&\leq |w|.\qedhere
        \end{align*}
    \end{proof}
    \item $|\Im w|\leq |w|$
    \begin{proof}
        Let $w=a+bi$ with $a,b\in\R$. Then we have
        \begin{align*}
            b^2&\leq a^2+b^2\\
            \sqrt{b^2}&\leq \sqrt{a^2+b^2}\\
            |\Im w|&\leq |w|.\qedhere
        \end{align*}
    \end{proof}
    \item $|w|\leq |\Re w|+|\Im w|$
    \begin{proof}
        Let $w=a+bi$ with $a,b\in\R$. Then we have
        \begin{align*}
            x+y&\leq x+2\sqrt{xy}+y\\
            x+y&\leq(\sqrt{x}+\sqrt{y})^2\\
            \sqrt{x+y}&\leq\sqrt{x}+\sqrt{y},
        \end{align*}
        set $x=a^2$ and $y=b^2$ and we get
        \begin{align*}
            \sqrt{a^2+b^2}&\leq\sqrt{a^2}+\sqrt{b^2}\\
            |w|&\leq|\Re w|+|\Im w|.\qedhere
        \end{align*}
    \end{proof}
\end{enumerate}
\newpage
\stepcounter{section}
\section*{Problem 2a}
Show that 
\[
    |\Re z_1-\Re z_2|\leq |z_1-z_2|\leq |\Re z_1-\Re z_2|+|\Im z_1-\Im z_2|
\]
\begin{proof}
    Let $z_1=a_1+b_1i$, $z_2=a_2+b_2i$ and $z=z_1-z_2=(a_1-a_2)+(b_1-b_2)i$, with $a_1,a_2,b_1,b_2\in\R$.\\
    Set $p=a_1-a_2$ and $q=b_1-b_2$, then we have from $1c$ we know that 
    \begin{align*}
        \sqrt{p^2+q^2}&\leq\sqrt{p^2}+\sqrt{q^2}\\
        \sqrt{(a_1-a_2)^2+(b_1-b_2)^2}&\leq\sqrt{(a_1-a_2)^2}+\sqrt{(b_1-b_2)^2}\\
        |z|&\leq|\Re z|+|\Im z|\\
        |z_1-z_2|&\leq|\Re z_1-\Re z_2|+|\Im z_1-\Im z_2|.
    \end{align*}
    \begin{align*}
        (a_1-a_2)^2&\leq(a_1-a_2)^2+(b_1-b_2)^2\\
        \sqrt{(a_1-a_2)^2}&\leq\sqrt{(a_1-a_2)^2+(b_1-b_2)^2} \\
        |\Re z|&\leq|z|\\
        |\Re z_1-\Re z_2|&\leq|z_1-z_2|.
    \end{align*}
    Combining the two inequalities we arrive at the desired result.
\end{proof}
\stepcounter{section}
\section*{Problem 3}
Prove: If $f$ is continuous and $f(z_0)\neq 0$ then there exists a neighbourhood of $z_0$ on which $f\neq 0$.
\begin{proof}
    Let $f(z_0)$. Set $r=\frac{|f(z_0)-0|}{2}$. We know that all points $f(z)\in D(f(z_0),r)$ are not equal to 0 since they are at most half the distance between $f(z_0)$ and 0. By the continuity of $f$ we we can find $\delta_r>0$ such that $D(z_0,\delta_r)$ is a neighbourhood of $z_0$, and $z\in D(z_0,\delta_r)\Rightarrow f(z)\in D(f(z_0),r)$. Thus, we have that there exists a neighbourhood of $z_0$ such that $f\neq0$. 
\end{proof}
\stepcounter{section}
\section*{Problem 8}
Show that $f(z)=|z|$ is continuous.
\begin{proof}
    Let $s\in\C$ be an arbitrary complex number. We say that a function $f$ is continuous if for all $\varepsilon>0$ there exists $\delta>0$ such that we have $|f(z)-f(c)|<\varepsilon$ whenever $|z-c|<\delta$.\\
    Observe by the reverse triangle inequality we have
    \[
        |f(z)-f(c)|=|z|-|w|=||z|-|w||\leq|z-w|.
    \]
    Now chose $\delta=\varepsilon$. Then whenever $|z-c|<\delta$ we have $|f(z)-f(c)|\leq|z-w|<\delta=\varepsilon$
\end{proof}
\stepcounter{section}
\section*{Problem 9}
What is the largest set on which the function $f(z)=1/(1-e^z)$ is continuous.\\
\textit{Solution.} $f(z)$ is not continuous for values of $z$ at which $1-e^z=0$. Those values of $z$ lie on the unit circle in $\C$, that is points of the form $e^z=1$. This function is continuous everywhere else. The largest set on which it is continuous if $\C\setminus\{z:|e^z|\leq1\}$. It does not suffice to only remove values on the unit cirle in $\C$ because we would have a discontinuity jump when $z$ goes from inside the unit circle to outside the unit circle.
\stepcounter{section}
\section*{Problem 14}
For each of the following sets, state $(i)$ whether or not the set is open and $(ii)$ whether or not the set is closed.
\begin{enumerate}[label=(\alph*)]
    \item $\{z\,|\, \Im z>2\}$
    \begin{enumerate}[label=(\roman*)]
        \item Yes. There the sequence of $z_n=(2+\frac{1}{10^n})i$ exists in the set, but the limit point of the set $2i$ is not a part of the set.
        \item No. This set contains none of its limit points.
    \end{enumerate}
    \item $\{z\,|\, 1<|z|\leq2\}$
    \begin{enumerate}[label=(\roman*)]
        \item No. The point $z=2+0i$ is in this set. Any disk of arbitrary radius centred at $2+0i$, the disk will always contain a point $z,z'$ with $|z|\leq2$ and $|z'|>2$. 
        \item No. There are limit points of this set which are not a part of the set.
    \end{enumerate}
    \item $\{z\,|\, -1< \Re z\leq2\}$
    \begin{enumerate}[label=(\roman*)]
        \item No. The point $z=2+0i$ is in this set. Any disk of arbitrary radius centred at $2+0i$, the disk will always contain a point $z,z'$ with $\Re(z)\leq2$ and $\Re(z')>2$. 
        \item No. There are limit points of this set which are not a part of the set.
    \end{enumerate}
\end{enumerate}
\newpage
\stepcounter{section}
\section*{Problem 16}
For each of the following sets, state $(i)$ whether or not the set is connected and $(ii)$ whether or not the set is compact.
\begin{enumerate}[label=(\alph*)]
    \item $\{z\,|\, -1< \Re z\leq2\}$
    \begin{enumerate}[label=(\roman*)]
        \item Yes. For any two points in this set there is a line entirely in the set between the two. The set is convex.
        \item No. Not closed.
    \end{enumerate}
    \item $\{z\,|\, 2\leq |z|\leq3\}$
    \begin{enumerate}[label=(\roman*)]
        \item Yes. This set is an annulus in $\C$, therefore it is connected.
        \item Yes. Closed and bounded.
    \end{enumerate}
    \item $\{z\,|\, |z|\leq 5, \Im z\geq 1\}$
    \begin{enumerate}[label=(\roman*)]
        \item Yes. This set is teh intersection of two connected sets and is therefore connected.
        \item Yes. Closed and bounded.
    \end{enumerate}
\end{enumerate}
\stepcounter{section}
\section*{Problem 19}
Show the union of any collection of open subsets of $\C$ is open.\\
We want to show that given any collection of open sets $\{G_\alpha\}$ we have that $\bigcup_\alpha G_\alpha$ is also open.
\begin{proof}
    Let $G=\bigcup_\alpha G_\alpha$. If $x\in G$, then $x\in G_\alpha$ for some $\alpha$. Since $G_\alpha$ is open for all $\alpha$ we have that $x$ is an interior point of $G_\alpha$, then we have that $x$ is an interior point of $G$.
\end{proof}
\stepcounter{section}
\section*{Problem 20}
Show that the intersection of any finite collection of open subsets of $\C$ is open.
\begin{proof}
    Let $G_\alpha$ be some finite collection of open sets. Set $H=\bigcap_{\alpha=1}^n G_{\alpha}$. For any $x\in H$ we know that there exists a neighbourhood $N_i$ with radii $r_i$ such that $N_i\subset G_\alpha$ for some $\alpha$. Put
    \[
        r=\min\{r_1,r_2,\dots,r_n\},
    \]
    and $N$ a neighbourhood of $x$ with radius $r$. Then $N\subset G_\alpha$ for $1\leq\alpha\leq n$. Thus, we have that $H$ is open.
\end{proof}
\end{document}