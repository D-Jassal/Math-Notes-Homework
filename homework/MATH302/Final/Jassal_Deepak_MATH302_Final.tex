\documentclass[12pt]{article}

% Import preambles and macros for homework
% Essential packages
\usepackage{amsmath, amsfonts, amssymb, amsthm}
\usepackage{mathtools}
\usepackage{enumitem}
\usepackage{graphicx}
\usepackage{wrapfig}
\usepackage{systeme}
\usepackage{caption}
\usepackage{soul}
\usepackage[dvipsnames]{xcolor}
\usepackage{fancyhdr}
\allowdisplaybreaks

% Page layout
\usepackage[
  top=2cm,
  bottom=2cm,
  left=2cm,
  right=2cm,
  headheight=17pt,
  includehead,includefoot,
  heightrounded,
]{geometry}


% pgfornament for title page decorations
\usepackage[object=vectorian]{pgfornament}

% Fancy header/footer setup
\pagestyle{fancy}
\setlength{\headheight}{14.49998pt}
\addtolength{\topmargin}{-2.49998pt}
\renewcommand{\footrulewidth}{0.4pt}
\setlength\parindent{15pt}
% Math notation shortcuts
\newcommand{\R}{\mathbb{R}}
\newcommand{\Q}{\mathbb{Q}}
\newcommand{\Z}{\mathbb{Z}}
\newcommand{\N}{\mathbb{N}}
\newcommand{\C}{\mathbb{C}}
\newcommand{\X}{\mathcal{X}}

% Theorem environments
\newtheorem{mainthm}{Theorem}[section]
\newtheorem{theorem}{Theorem}[section]  
\newtheorem{lemma}[theorem]{Lemma}
\newtheorem{proposition}[theorem]{Proposition}
\newtheorem{corollary}[theorem]{Corollary}
\newtheorem{definition}[theorem]{Definition}
\newtheorem{claim}[theorem]{Claim}

% Calculus
\newcommand{\diff}{\mathop{}\!\mathrm{d}}
\newcommand{\deriv}[2]{\frac{\mathrm{d}#1}{\mathrm{d}#2}}
\newcommand{\pderiv}[2]{\frac{\partial #1}{\partial #2}}

% Linear Algebra
\newcommand{\inner}[2]{\langle #1, #2 \rangle}
\newcommand{\norm}[1]{\| #1 \|}
\newcommand{\tr}{\operatorname{tr}}
\newcommand{\spn}{\operatorname{span}}
\newcommand{\rank}{\operatorname{rank}}
\newcommand{\nullity}{\operatorname{nullity}}

% Logic
\newcommand{\contra}{\Rightarrow\Leftarrow}

% Custom commands for notes
\newcommand{\todo}[1]{\textcolor{red}{[TODO: #1]}}
\newcommand{\important}[1]{\textbf{\textcolor{blue}{#1}}}

%Number Theory
\DeclareMathOperator{\Li}{Li}
\newcommand{\floor}[1]{\left\lfloor #1 \right\rfloor}
\newcommand{\fract}[1]{\left\{ #1 \right\}}




\newcommand{\maketitlepage}{
    \begin{titlepage}
        \centering
        \vspace*{2.0cm}
        \pgfornament{84}\\
        {\LARGE \textsc{\coursename}\par}
        \vspace{0.5cm}
        {\large\coursecode\par}
        \vspace{0.5cm}
        {\large\instructor\par}
        \vspace{1.5cm}
        {\huge\bfseries\assignment\par}
        \vspace{1cm}
        {\LARGE\itshape\author\par}
        \vspace{2cm}
        {\large\bfseries Due Date:\par}
        \vspace{0.5cm}
        {\Large \duedate}\\
        \pgfornament{84}
    \end{titlepage}
}
% =============================================
% HOMEWORK CONFIGURATION - EDIT THESE VALUES!
% =============================================

% Your personal info
\renewcommand{\author}{Deepak Jassal}
\newcommand{\authorlast}{Jassal}

% Course info
\newcommand{\coursename}{Course Name}
\newcommand{\coursecode}{Course code}
\newcommand{\instructor}{Instructor}

% Assignment-specific info (CHANGE THESE FOR EACH HOMEWORK)
\newcommand{\assignment}{Assignment }
\newcommand{\duedate}{Month Day\textsuperscript{th}, 20XX}

% Header configuration
\fancyhead[l]{\assignment}
\fancyhead[c]{\coursecode}
\fancyhead[r]{\monthyear}
\fancyfoot[c]{\authorlast{ }\thepage}

\renewcommand{\author}{Deepak Jassal}
\renewcommand{\authorlast}{Jassal}
\renewcommand{\coursename}{Introductory Mathematical Analysis}
\renewcommand{\coursecode}{MATH 302}
\renewcommand{\assignment}{Final}
\renewcommand{\instructor}{Dr. Stanley Yao Xiao}
\renewcommand{\duedate}{December 14\textsuperscript{th}, 2025}


\begin{document}
\begin{titlepage}
	\centering
	\vspace*{2.0cm}	
	\pgfornament{84}\\
	{\LARGE \textsc{\coursename}\par}
	\vspace{0.5cm}
	{\large\coursecode\par}
    \vspace{0.5cm}
    {\large\instructor\par}
	\vspace{1.5cm}
	{\huge\bfseries\assignment\par}
	\vspace{1cm}
	{\LARGE\itshape\author\par}
    \vspace{2cm}
	{\large\bfseries Due Date:\par}
	\vspace{0.5cm}
	{\Large \duedate}\\
	\pgfornament{84}
\end{titlepage}

\stepcounter{section}
\section*{Question 1}
This question is about the Baire Category Theorem.

\begin{enumerate}[label=(\alph*)]
    \item State the Baire Category Theorem, along with all relevant definitions.
    \begin{definition}
        Let $(\X,\tau)$ be a topological space and $S$ a subset
        \begin{enumerate}
            \item We say that $S$ is dense if $\bar{S}=X$;
            \item $S$ is nowhere dense is $(\bar{S})^\circ=\varnothing$;
            \item $S$ is Baire Category I or meagre in $\X$ if $S$ is the countable union of nowhere dense set. Otherwise $S$ is Baire Category II or non-meagre.
        \end{enumerate}
        \begin{theorem}[Baire Category I]
            Let $(\X,d)$ be a complete metric space. If $(S_n)_{n\geq1}$ is a sequence of dense open subsets of $\X$, then 
            \[
                \bigcap_{n=1}^\infty S_n\neq \varnothing.
            \]  
        \end{theorem}
        \begin{theorem}[Baire Category II]
            Let $(\X,d)$ be a complete metric space. Then $\X$ is category II or non-meagre.
        \end{theorem}
    \end{definition}
    \item Prove that \(\mathbb{Q}\) cannot be expressed as a countable intersection of dense open subsets of \(\R\).
    \begin{proof}
        Assume for contradiction that $\Q$ can be written as a countable intersection of dense open subsets of \(\R\), that is
        \[
            \Q=\bigcap_{n=1}^\infty U_n,
        \]
        where each $U_n$ is open and dense in $\R$. 
        Then we have
        \[
            \R\setminus\Q=\bigcap_{n=1}^\infty \R\setminus U_n.
        \]
        Each $\R\setminus U_n$ is closed and nowhere dense (since each $U_n$ is open and dense). So each $\R\setminus \Q$ is the countable union of nowhere dense sets, that is, it is meagre. We have
        \[
        \R=\Q\cup(\R\cap\Q),
        \] 
        we know that $\Q$ is countable, so it is meagre, and from the above we have that $\R\setminus\Q$ is meagre, so $\R$ is also meagre. But this is a contradiction since we know by Baire Category Theorem II that $\R$ is Baire Category II, or non-meagre.
    \end{proof}
\end{enumerate}

\stepcounter{section}
\section*{Question 2}
(This question is worth 20 points) Let \((\X,d)\) be a compact metric space.

\begin{enumerate}[label=(\alph*)]
    \item Prove that \((\X,d)\) is separable, i.e., contains a dense countable subset.
    \begin{proof}
        The set ${B\left(x,\frac{1}{n}\right): x\in\X}$ coverse $\X$. By the compactness of $\X$ we have a finite subcover
        \[
            \X=\bigcup_{i=1}^k B\left(x_{n_i},\frac{1}{n}\right),
        \]
        for some finite $k$.
        Let $D_n=\{x_{n_1},x_{n_2},\dots,x_{n_k}\}$, then $D_n$ is a finite subset of $\X.$\\
        Let 
        \[
            D=\bigcup_{n=1}^\infty D_n.
        \]
        Let $\varepsilon>0$, and $x\in\X$. Choose $n$ such that $\frac{1}{n}<\varepsilon$ (this is possible by the Archimedean property). Since $\bigcup_{i=1}^k B\left(x_{n_i},\frac{1}{n}\right)$ covers $\X$, there exists a $X_{n_i}\in D_n\subset D$ such that $x\in B\left(x_{n_i},\frac{1}{n}\right)\Rightarrow d(x,x_{n_i})<\frac{1}{n}<\varepsilon$. So every open ball of $x$ contains a point of $D$, that is to say that $D$ is dense in $\X$.
    \end{proof}
    \item Prove that \(\mathcal{C}(\X,\R)\), the set of continuous functions from \(\X\) to \(\R\), is separable. \\ 
    Hint: consider the \(\R\)-algebra generated by \(\{1,f_{1},f_{2},\cdots\}\) where
    \[
    f_{j}(x)=d(x,x_{j})
    \]
    where \(\{x_{1},\cdots,x_{k},\cdots\}\) is a countable dense subset of \(\X\).
    \begin{proof}
        Let $\mathcal{A}=\{1,f_{1},f_{2},\cdots\}$, with
            \[
                f_{j}(x)=d(x,x_{j})
            \]
            as defined above. Let $a,b\in\R$ with $a\neq b$. Then for some $f_j\in\mathcal{A}$ we have $f_j(a)\neq f_j(b)$ (take $f_1$), that is to say that $\mathcal{A}$ is an algebra which separates points. Since $\mathcal{A}$ contains the constant functions, by the Stone-Weierstrass theorem, $\mathcal{A}$ is dense in $\mathcal{C}(\X,\R)$. Let $\mathcal{A}_0\subset\mathcal{A}$ where $\mathcal{A}_0$ only contains the linear combinations of $\{1,f_{1},f_{2},\cdots\}$ with rational coefficients. Then $\mathcal{A}_0$ is countable in $\mathcal{C}(\X,\R)$. Since $\mathcal{A}_0$ contains the constant rational functions and for $a,b\in\R$ with $a\neq b$, then for some $f_k\in\mathcal{A}_0$ we have $f_k(a)\neq f_k(b)$ (take $f_1$), then by the Stone-Weierstrass theorem, $\mathcal{A}_0$ is dense in $\mathcal{A}$. Then $\mathcal{A}_0$ is dense in $\mathcal{C}(\X,\R)$. Therefore, $\mathcal{C}(\X,\R)$ contains a dense countable subset, that is to say $\mathcal{C}(\X,\R)$ is seprable. 
    \end{proof}
    \newpage
    \item Let \(\{U_{i}\}_{i\in\mathcal{I}}\) be an open cover of \(\X\). Prove that there exists \(r>0\) such that whenever \(x,y\in X\) are such that \(d(x,y)<r\), there exists \(U_{\alpha}\) in the cover such that \(x,y\in U_{\alpha}\).
    \begin{proof}
        This proof assumes that $U_\alpha$ can be the union of sets in the covering of $\X$. Let $r>0$ then for some $x,y\in X$ we have $d(x,y)<r$. Let
        \[
            U_\alpha=\bigcup U_{n_i}
        \] 
        where for some $z\in U_{n_i}$ we have $z\in B_r(x)$. Then $x\in B_r(x)$ since $d(x,x)=0<r$, and $y\in B_r(x)$ since $d(x,y)<r$. So whichever sets in the cover contain $x,y$, they also are in $U_\alpha$.  Then $x,y\in U_\alpha$.
    \end{proof}
\end{enumerate}

\stepcounter{section}
\section*{Question 3}
Prove the following statements:

\begin{enumerate}[label=(\alph*)]
    \item Let \(S\subset\R^{n}\) and put
    \[
    T=\{x\in S:B_{\delta}(x)\cap S\text{ is uncountable for all }\delta>0\}.
    \]
    Then \(S\setminus T\) is countable.
    \begin{proof}
        Since $\R^n$ is seprable, we can find a countable dense subset of $S$ $\{q_1,q_1,\dots\}$ where each element is composed of rational numbers. Let $x \in S\setminus T$, then there exists $r\in\Q$ such that $x\in B_r(q_i)$ for some $i\in\N$ where $B_r(q_i)\cap S$ is countable. This is possible since we know $B_\delta(q_i)\cap S$ is countable, then for some $\varepsilon$ set $R_i\{r_i:delta+\varepsilon=r_i\}$, let $R=R_i\cap \Q$. Finally set $r=\inf R$. So we have a countable number of countable sets covering $S\setminus T$, therefore $S\setminus T$ is countable.   
    \end{proof}
    \item Suppose \(\X\) is a set. Then the set of functions \(\mathcal{F}:\X\to\{0,1\}\) is either finite or uncountable.
    \begin{proof}
        If $\X$ is finite, then $|\X|=n$, and the number of functions $\mathcal{F}:\X\to\{0,1\}$ is $2^n$, which is finite.\\
        If $\X$ is countably inifinte then let $\X=\{x_1,x_2,\dots\}$ and each function $f\in\mathcal{F}$ reduces to a binary sequence $(f(x_1),f(x_2),\dots)$. The set $\{0,1\}^\X$ is uncountable. Assume that is it countable, arrange these sequences in an arbitrary list. Then make a new sequence by taking the $n$-th element of the $n$-th sequence and altering the value, this new sequence does not appear in the list. So $|\mathcal{F}|$ is uncountable.\\
        If $\X$ is uncountably infinite, find $A\in\X$ where $|A|=|\N|$, then apply the above procedure. We get the result that $|\mathcal{F}|$ is uncountable.
    \end{proof}
    \newpage
    \item If \(f:\R\to\R\) is continuous and has the property that \(f(S)\in\tau_{\R}\) for all \(S\in\tau_{\R}\), then \(f\) is strictly increasing or strictly decreasing.
    \begin{proof}
        Since elements of $\tau_\R$ are open sets, this questions boils down to showing that a continuous function that maps open sets to other open sets is strictly monotonic.\\
        Suppose that this function $f$ is not monotonic, then for some open interval $(a,c)$ there exists $b\in(a,c)$ with $a<b<c$ where $f(b)>f(a)$ and $f(b)>f(c)$  is a local max, or $f(b)<f(a)$ and $f(b)<f(c)$ is a local min. Let $U=(b-\varepsilon,b+\varepsilon)$ for some $\varepsilon>0$. In the case where $f(b)$ is a local max $f(U)$ contains elements lesser than or equal to $f(b)$, and in the case where $f(b)$ is a local min $f(U)$ contains elements greater than or equal to $f(b)$. This means that $f(U)$ has a maximal or minimal element, which means $f(U)$ is not open, since open sets cannot have maximal or minimal elements.\\
        Suppose that $f$ is not strictly monotonic, then for some interval $(a,b)$ we may have $f((a,b))=\{c\}$. This contradicts that fact that $f$ maps open sets to open sets since $\{c\}$ is not open.\\
        Therefore, the function $f$ from $\R\to\R$ that maps open sets to open sets is strictly monotonic.  
    \end{proof}
\end{enumerate}

\stepcounter{section}
\section*{Question 4}
Let \((f_{n})\) be a family of functions in \(\mathcal{C}([0,1],\R)\). Suppose that each \(f_{n}\) is continuously differentiable on \((0,1)\), and that \((f^{\prime}_{n})\) is uniformly bounded on \([0,1]\). Prove that \((f_{n})\) is equicontinuous.
\begin{proof}
    Since $f_{n}$ is continuously differentiable on \((0,1)\) we can apply the mean value theorem. 
    \[
        |f_n(x)-f_n(y)|=|f'_n(c)||x-y|,
    \]
    for some $c\in(x,y)$.
    Since $(f'_{n})$ is uniformly bounded, we know there exists $M$ such that $|f_n'(x)|<M$ for all $x\in [0,1]$. Then we have
    \[
        |f_n(x)-f_n(y)|\leq M|x-y|.
    \]
    Choose $\delta=\frac{\varepsilon}{M}$. Then whenever $|x-y|<\delta$,
    \[
        |f_n(x)-f_n(y)|\leq M|x-y|<M\delta=M\frac{\varepsilon}{M}=\varepsilon.
    \]
    So we have
    \[
        |f_n(x)-f_n(y)|<\varepsilon,
    \]
    for all $f_n\in\mathcal{F}$.
\end{proof}
\stepcounter{section}
\section*{Question 5}
Prove or disprove the following statements. To disprove the statement, you must prove its negation.

\begin{enumerate}[label=(\alph*)]
    \item Let \((\X,d)\) be a metric space, and \(E\subset \X\) a subset. Then the closure of \(E^{\circ}\) is equal to the closure of \(E\).
    \begin{proof}
        This statement is false. The negation of the statement is\\
        Let \((\X,d)\) be a metric space, and \(E\subset \X\) a subset. Then the closure of \(E^{\circ}\) is not equal to the closure of \(E\).\\
        To show this we need to find an example of a metric space $(\X,d)$ and a subet $E\subset\X$ where $\overline{E^\circ}\neq\overline{E}.$\\
        Let $X=\R$, and $E=\Q$. Then $\Q\subset\R$, and $\Q^\circ=\varnothing$. Then $\overline{\Q^\circ}=\varnothing$ and $\overline{\Q}=\R$. So we have $\overline{E^\circ}\neq\overline{E}$.
    \end{proof}
    \item If \((a_{n}),(b_{n})\in\ell^{2}(\R)\), then \((a_{n}b_{n})\in\ell^{1}(\R)\). \\
    Recall that \(\ell^{p}(\R)\) is the set of real sequences \((a_{n})\) such that
    \[
    \sum_{n=1}^{\infty}|a_{n}|^{p}<\infty.
    \]
    \begin{proof}
        Using the Cauchy-Schwarz inequality we have
        \[
            \sum_{n=1}^{\infty}|a_{n}b_n|\leq \left(\sum_{n=1}^{\infty}|a_{n}|^2\right)^{\frac{1}{2}}\left(\sum_{n=1}^{\infty}|b_{n}|^2\right)^{\frac{1}{2}}.
        \]
        Since $(a_n),(b_n)\in\ell^2$ we have 
        \[
            \left(\sum_{n=1}^{\infty}|a_{n}|\right)^2<\infty,
        \]
        and
        \[
            \left(\sum_{n=1}^{\infty}|b_{n}|\right)^2<\infty.
        \]
        So 
        \[
            \sum_{n=1}^{\infty}|a_{n}b_n|\leq \left(\sum_{n=1}^{\infty}|a_{n}|^2\right)^{\frac{1}{2}}\left(\sum_{n=1}^{\infty}|b_{n}|^2\right)^{\frac{1}{2}}<\left(\sum_{n=1}^{\infty}|a_{n}|\right)^2\left(\sum_{n=1}^{\infty}|b_{n}|\right)^2<\infty.   
        \]
    \end{proof}
    \newpage
    \item If \(f:[0,1]\to\R\) is unbounded, then there exists \(x\in[0,1]\) such that for all \(\delta>0\), \(f\) is unbounded on the interval \((x-\delta,x+\delta)\).
    \begin{proof}
        Since \(f:[0,1]\to\R\) is unbounded we know there exists $y\in[0,1]$ where $f(y)>M$ for all $M\in\N$. Set $\Delta=|x-y|$, and let $\delta>\Delta$. Then $y\in(x-\delta,x+\delta)$. So $f$ is unbounded on $(x-\delta,x+\delta)$ since this interval contains $y$. This works for any $\delta>0$ since we only need to find an $x\in[0,1]$ such that $|x-y|<\delta$.
    \end{proof}
\end{enumerate}

\stepcounter{section}
\section*{Question 6}
Let \(f,g\) be continuous functions on the circle with Fourier series
\[
f(x)\sim\sum_{n\in\mathbb{Z}}a_{n}e^{inx}\quad\text{and}\quad g(x)\sim\sum_{n\in\mathbb{Z}}b_{n}e^{inx}.
\]
Suppose that the series \(\sum|a_{n}|^{2}\) and \(\sum|b_{n}|^{2}\) are finite. Prove:

\begin{enumerate}[label=(\alph*)]
    \item 
    \[
    \sum_{n\in\mathbb{Z}}a_{n}\overline{b_{n}}=\frac{1}{2\pi}\int_{-\pi}^{\pi}f(x)\overline{g(x)}dx,
    \]
    where the bar denotes complex conjugation.
    \begin{proof}
        Since $f$ and $g$ are continuous on the circle we have
        \[
            \frac{1}{2\pi}\int_{-\pi}^{\pi}f(x)\overline{g(x)}dx=\frac{1}{2\pi}\int_{-\pi}^{\pi}\sum_{n\in\mathbb{Z}}a_{n}e^{inx}\sum_{n\in\mathbb{Z}}\overline{b_{n}}e^{-inx}dx.
        \] 
        Since the sums use the same index we can rewrite the sums as
        \[
            \frac{1}{2\pi}\int_{-\pi}^{\pi}f(x)\overline{g(x)}dx=\frac{1}{2\pi}\int_{-\pi}^{\pi}\sum_{n\in\mathbb{Z}}a_{n}\overline{b_{n}}e^{inx}e^{-inx}dx.
        \]
        Simplifying and using the fact $a_n$ and $b_n$ are square summable (I think this is what you meant by \textit{Suppose that the series \(\sum|a_{n}|^{2}\) and \(\sum|b_{n}|^{2}\) are finite} and this is where it is used.)
        \[
            \frac{1}{2\pi}\int_{-\pi}^{\pi}f(x)\overline{g(x)}dx=\frac{1}{2\pi}\sum_{n\in\mathbb{Z}}a_{n}\overline{b_{n}}\int_{-\pi}^{\pi}dx.
        \]
        Computing the integral
        \[
            \frac{1}{2\pi}\int_{-\pi}^{\pi}f(x)\overline{g(x)}dx=\frac{1}{2\pi}\sum_{n\in\mathbb{Z}}a_{n}\overline{b_{n}}2\pi=\sum_{n\in\mathbb{Z}}a_{n}\overline{b_{n}}.\qedhere
        \]       
    \end{proof}
    \newpage
    \item Apply Parseval's identity as follows. Let \(f(x)=x\) on \([-\pi,\pi)\) and extend periodically. Compute the Fourier series of \(f\) and use this to prove that
    \[
    \frac{\pi^{2}}{6}=\sum_{n=1}^{\infty}\frac{1}{n^{2}}.
    \]
    \begin{proof}
        The Fourier series fo $f(x)=x$ is given by 
        \[
            f(x)\sim a_0+\sum_{n=1}^{\infty}\left[a_n\cos(nx)+b_n\sin(nx)\right].
        \]
        Computing the Fourier coefficients
        \[
            a_0=\frac{1}{\pi}\int_{-\pi}^{\pi}x\,dx=0,
        \]
        \[
            a_n=\frac{1}{\pi}\int_{-\pi}^{\pi}x\cos(nx)\,dx=0,
        \]
        \[
            b_n=\frac{1}{\pi}\int_{-\pi}^{\pi}x\sin(nx)\,dx=\frac{2}{\pi}\left[\left(-\frac{x\cos(nx)}{n}\right)_0^\pi+\int_{0}^{\pi}\frac{cos(nx)}{n}\,dx\right]
        \]
        \[
            b_n=\frac{2}{\pi}\left[\frac{\pi(-1)^n}{n}\right]=\frac{2(-1)^n}{n}.
        \]
        \[
            f(x)\sim2\sum_{n=1}^{\infty}\frac{(-1)^n}{n}\sin(nx).
        \]
        We can express this Fourier series as a series of complex values using the substitution $\sin(nx)=\frac{e^{inx}-e^{-inx}}{2i}$
        \[
            f(x)=\sum_{n\in\Z}c_ne^{inx}.
        \]
        \[
            c_n=\frac{a_n-ib_n}{2}=\frac{-ib_n}{2}=-\frac{1}{2}\left[\frac{2(-1)^n}{n}\right]=i\cdot\frac{(-1)^n}{n}.
        \]
        So
        \[
            f(x)\sim\sum_{n\in\Z,n\neq0}i\cdot\frac{(-1)^n}{n}e^{inx}.
        \]
        In this situation we apply Pareseval's identity with $f=g$. So we have
        \[
            \frac{1}{2\pi}\int_{-\pi}^{\pi}x^2\,dx=\sum_{n\in\Z}|c_n|^2.
        \]
        The left hand side becomes
        \[
            \frac{1}{2\pi}\int_{-\pi}^{\pi}x^2\,dx=\frac{1}{2\pi}\cdot\frac{2\pi^3}{3}=\frac{\pi^2}{3}.
        \]
        The right hand side becomes
        \[
            \sum_{n\in\Z}|c_n|^2=\sum_{n\in\Z,n\neq0}\left|i\cdot\frac{(-1)^n}{n}\right|^2=\sum_{n\in\Z,n\neq0}\frac{1}{n^2}=2\sum_{n=1}^{\infty}\frac{1}{n^2}.
        \]
        Equating both sides we obtain
        \[
            \frac{\pi^2}{3}=2\sum_{n=1}^{\infty}\frac{1}{n^2}
        \]
        \[
            \sum_{n=1}^{\infty}\frac{1}{n^2}=\frac{\pi^2}{6}.
        \]
    \end{proof}
\end{enumerate}
\newpage
\stepcounter{section}
\section*{Question 7}
Let
\[
h(x)=\begin{cases}
-1 & \text{if } -\pi<x<0 \\
0 & \text{if } x\in\{-\pi,0,\pi\} \\
1 & \text{if } 0<x<\pi.
\end{cases}
\]
Prove:

\begin{enumerate}[label=(\alph*)]
    \item The Fourier series for \(h(x)\) is
    \[
    \sum_{n=1}^{\infty}\frac{4}{(2n-1)\pi}\sin((2n-1)x).
    \]
    \begin{proof}
        Since $h(-x)=-h(x)$, $a_n=0$ for all $n$.
        \[
            b_n=\frac{1}{\pi}\int_{-\pi}^{\pi}h(x)\sin(nx)\,dx=\frac{2}{\pi}\int_{0}^{\pi}h(x)\sin(nx)\,dx=\frac{2}{\pi}\int_{0}^{\pi}1\sin(nx)\,dx.
        \]
        \[
            b_n=\frac{2}{\pi}\left[-\frac{\cos(nx)}{n}\right]^\pi_0
        \]
        \[
            b_n=\frac{2}{\pi}\left[-\frac{\cos(0)}{n}\right]=\frac{2}{\pi}\left[-\frac{(-1)^n}{n}+\frac{1}{n}\right]
        \]
        \[
            b_n=\frac{2}{n\pi}[1-(-1)^n].
        \]
        Rewrite $n$ as $2k-1$ $k\in\N$, then 
        \[
            b_k=\frac{4}{\pi(2k-1)}.
        \]
        Since the index does not matter we can write $k$ as $n$ and we  obtain the Fourier series
        \[
            \sum_{n=1}^{\infty}\frac{4\sin((2n-1)x)}{(2n-1)\pi}.
        \]
    \end{proof}
    \newpage
    \item The partial sum
    \[
    S_{N}(x)=\sum_{n=1}^{N}\frac{4}{\pi(2n-1)}\sin((2n-1)x)=\frac{2}{\pi}\int_{0}^{x}\frac{\sin(2Nt)}{\sin(t)}dt.
    \]
    \begin{proof}
        Differentiating both sides with respect to $x$. 
        \[
            \frac{d}{dx} SN(x) = \frac{4}{\pi} \sum{n=1}^N \cos((2n-1)x).
        \]
        And
        \[
            \frac{d}{dx} \left[ \frac{2}{\pi} \int{0}^x \frac{\sin(2 N t)}{\sin t} dt \right] = \frac{2}{\pi} \cdot \frac{\sin(2 N x)}{\sin x}.
        \]
        Then we only need to show that
        \[
            2 \sum_{n=1}^N \cos((2n - 1)x) = \frac{\sin(2 N x)}{\sin x}
        \]
        The geometric sum of complex exponentials
        \[
        \sum{n=1}^{N}e^{i(2n-1)x}=e^{ix}\frac{1-e^{i2Nx}}{1-e^{i2x}}.
        \]
        Looking at the real parts
        \[
        \sum{n=1}^{N}\cos((2n-1)x)=\mathfrak{Re}\left[e^{ix}\frac{1-e^{i2Nx}}{1-e^{i2x}}\right]
        =\frac{\sin(2Nx)}{2\sin x}.
        \]
        Multiply by 2. The deriviatives match at $x=0$. So it works
    \end{proof}
    \item The limit
    \[
    \lim_{N\to\infty}S_{N}\left[\frac{\pi}{2N}\right]=\frac{2}{\pi}\int_{0}^{\pi}\frac{\sin(t)dt}{t}
    \]
    holds.
    \begin{proof}
        We know from part (c) that 
        \[
            S_{N}(x)=\frac{2}{\pi}\int_{0}^{x}\frac{\sin(2Nt)}{\sin(t)}dt.
        \]
        So 
        \[
            S_{N}\left(\frac{\pi}{2N}\right)=\frac{2}{\pi}\int_{0}^{\frac{\pi}{2N}}\frac{\sin(2Nt)}{\sin(t)}dt.  
        \]
        Let 
        \[
            u=2Nt,\,\,t=\frac{u}{2N},\,\,dt=\frac{du}{2N},\,\,t=0\Rightarrow u=0,\,\,t=\frac{\pi}{2N}\Rightarrow u=\pi.
        \]
        \[
            S_N\left(\frac{\pi}{2N}\right)=\frac{2}{\pi}\int_{0}^{\pi}\frac{\sin(u)}{\sin\left(\frac{u}{2N}\right)}\cdot\frac{1}{2N}\,du
        \]
        \[
             S_N\left(\frac{\pi}{2N}\right)=\frac{1}{N\pi}\int_{0}^{\pi}\frac{\sin(u)}{\sin\left(\frac{u}{2N}\right)}\,du.
        \]
        As $N\to\infty$, $\frac{u}{2N}\to0$. So $\sin\left(\frac{u}{2N}\right)\to\frac{u}{2N}$. Then 
        \[
            \frac{\sin(u)}{\sin\left(\frac{u}{2N}\right)}=\frac{\sin(u)2N}{u}.
        \]
        \[
           S_N\left(\frac{\pi}{2N}\right)=\frac{2N}{N\pi}\int_{0}^{\pi}\frac{\sin(u)}{u}\,du.
        \]
        Let $t=u$, then
        \[
            \lim_{N\to\infty}S_N\left(\frac{\pi}{2N}\right)=\frac{2}{\pi}\int_{0}^{\pi}\frac{\sin(t)}{t}\,dt.\qedhere
        \]
    \end{proof}
\end{enumerate}




\end{document}