\documentclass[12pt]{article}

% Import preambles and macros for homework
% Essential packages
\usepackage{amsmath, amsfonts, amssymb, amsthm}
\usepackage{mathtools}
\usepackage{enumitem}
\usepackage{graphicx}
\usepackage{wrapfig}
\usepackage{systeme}
\usepackage{caption}
\usepackage{soul}
\usepackage[dvipsnames]{xcolor}
\usepackage{fancyhdr}
\allowdisplaybreaks

% Page layout
\usepackage[
  top=2cm,
  bottom=2cm,
  left=2cm,
  right=2cm,
  headheight=17pt,
  includehead,includefoot,
  heightrounded,
]{geometry}


% pgfornament for title page decorations
\usepackage[object=vectorian]{pgfornament}

% Fancy header/footer setup
\pagestyle{fancy}
\setlength{\headheight}{14.49998pt}
\addtolength{\topmargin}{-2.49998pt}
\renewcommand{\footrulewidth}{0.4pt}
\setlength\parindent{15pt}
% Math notation shortcuts
\newcommand{\R}{\mathbb{R}}
\newcommand{\Q}{\mathbb{Q}}
\newcommand{\Z}{\mathbb{Z}}
\newcommand{\N}{\mathbb{N}}
\newcommand{\C}{\mathbb{C}}
\newcommand{\X}{\mathcal{X}}

% Theorem environments
\newtheorem{mainthm}{Theorem}[section]
\newtheorem{theorem}{Theorem}[section]  
\newtheorem{lemma}[theorem]{Lemma}
\newtheorem{proposition}[theorem]{Proposition}
\newtheorem{corollary}[theorem]{Corollary}
\newtheorem{definition}[theorem]{Definition}
\newtheorem{claim}[theorem]{Claim}

% Calculus
\newcommand{\diff}{\mathop{}\!\mathrm{d}}
\newcommand{\deriv}[2]{\frac{\mathrm{d}#1}{\mathrm{d}#2}}
\newcommand{\pderiv}[2]{\frac{\partial #1}{\partial #2}}

% Linear Algebra
\newcommand{\inner}[2]{\langle #1, #2 \rangle}
\newcommand{\norm}[1]{\| #1 \|}
\newcommand{\tr}{\operatorname{tr}}
\newcommand{\spn}{\operatorname{span}}
\newcommand{\rank}{\operatorname{rank}}
\newcommand{\nullity}{\operatorname{nullity}}

% Logic
\newcommand{\contra}{\Rightarrow\Leftarrow}

% Custom commands for notes
\newcommand{\todo}[1]{\textcolor{red}{[TODO: #1]}}
\newcommand{\important}[1]{\textbf{\textcolor{blue}{#1}}}

%Number Theory
\DeclareMathOperator{\Li}{Li}
\newcommand{\floor}[1]{\left\lfloor #1 \right\rfloor}
\newcommand{\fract}[1]{\left\{ #1 \right\}}




\newcommand{\maketitlepage}{
    \begin{titlepage}
        \centering
        \vspace*{2.0cm}
        \pgfornament{84}\\
        {\LARGE \textsc{\coursename}\par}
        \vspace{0.5cm}
        {\large\coursecode\par}
        \vspace{0.5cm}
        {\large\instructor\par}
        \vspace{1.5cm}
        {\huge\bfseries\assignment\par}
        \vspace{1cm}
        {\LARGE\itshape\author\par}
        \vspace{2cm}
        {\large\bfseries Due Date:\par}
        \vspace{0.5cm}
        {\Large \duedate}\\
        \pgfornament{84}
    \end{titlepage}
}
% =============================================
% HOMEWORK CONFIGURATION - EDIT THESE VALUES!
% =============================================

% Your personal info
\renewcommand{\author}{Deepak Jassal}
\newcommand{\authorlast}{Jassal}

% Course info
\newcommand{\coursename}{Course Name}
\newcommand{\coursecode}{Course code}
\newcommand{\instructor}{Instructor}

% Assignment-specific info (CHANGE THESE FOR EACH HOMEWORK)
\newcommand{\assignment}{Assignment }
\newcommand{\duedate}{Month Day\textsuperscript{th}, 20XX}

% Header configuration
\fancyhead[l]{\assignment}
\fancyhead[c]{\coursecode}
\fancyhead[r]{\monthyear}
\fancyfoot[c]{\authorlast{ }\thepage}

\renewcommand{\author}{Deepak Jassal}
\renewcommand{\authorlast}{Jassal}
\renewcommand{\coursename}{ Introductory Mathematical Analysis }
\renewcommand{\coursecode}{MATH 302}
\renewcommand{\assignment}{Assignment 3}
\renewcommand{\instructor}{Dr. Stanley Yao Xiao}
\renewcommand{\duedate}{December 3\textsuperscript{rd}, 2025}

\begin{document}
\begin{titlepage}
	\centering
	\vspace*{2.0cm}	
	\pgfornament{84}\\
	{\LARGE \textsc{\coursename}\par}
	\vspace{0.5cm}
	{\large\coursecode\par}
    \vspace{0.5cm}
    {\large\instructor\par}
	\vspace{1.5cm}
	{\huge\bfseries\assignment\par}
	\vspace{1cm}
	{\LARGE\itshape\author\par}
    \vspace{2cm}
	{\large\bfseries Due Date:\par}
	\vspace{0.5cm}
	{\Large \duedate}\\
	\pgfornament{84}
\end{titlepage}
\stepcounter{section}
\section*{Question 1}Let $\{f_{n}:n\geq 1\}$ be a sequence in ${\cal C}([0,1],{\mathbb{R}})$. Define a function $F_{n}$ by
\[
    F_{n}(x)=\int_{0}^{x}\sin(f_{n}(t))\,dt,\quad x\in[0,1].
\]
Use the Arzela-Ascoli theorem to prove that $\{F_{n}:n\geq 1\}$ has a uniformly convergent subsequence.
\begin{proof}
    \[
        \Q\cap [0,1]\subset [0,1]
    \]
    \[
        \Q\cap [0,1]\subset \Q
    \]
    Since $\Q$ is countable and dense then so is $\Q\cap [0,1]$. So, $[0,1]$ has a countable dense subset. $\R$ is a complete metric space.\\
    Pick $\delta=\varepsilon$ then whenever $|x-y|<\delta$
    \[
        |F_n(x)-F_n(y)|\leq \left|\int_{0}^{x}\sin\left(f_n(t)\right)\,dt-\int_{0}^{y}\sin\left(f_n(t)\right)\,dt\right|=\left|\int_{x}^{y}\sin\left(f_n(t)\right)\,dt\right|
    \]
    \[
        \left|\int_{x}^{y}\sin\left(f_n(t)\right)\,dt\right|\leq \int_{x}^{y}\left|\sin\left(f_n(t)\right)\right|\,dt\leq\int_{x}^{y}\left|1\right|\,dt=|y-x|<\varepsilon.
    \]
    Thus, $F_n(x),\,x\in[0,1]$ is equicontinuous.\\
    \[
        \left|F_n(x)\right|=\left|\int_{0}^{x}\sin\left(f_n(t)\right)\,dt\right|=\int_{0}^{x}\left|\sin\left(f_n(t)\right)\right|\,dt\leq\int_{0}^{x}\left|\sin\left(f_n(t)\right)\right|\,dt\leq1.
    \]
    Since $F_n(x)$ is bounded we have that the closure of $\left\{F_n(x)\right\}$ is compact.\\
    Therefore, by the general \textit{Arzela-Ascoli} theorem, $\left\{F_n(x):n\geq1\right\}$ has a convergent subsequence. Furthermore, since [0,1] is compact, this convergence is uniform.
\end{proof}

\stepcounter{section}
\section*{Question 2}
\begin{enumerate}[label=(\alph*)]
    \item Prove that every open subset of ${\mathbb{R}}$ (with respect to the standard topology $\tau_{\R}$) is a countable union of disjoint open intervals.
    \begin{proof}
        A special property of $\R$ is that all intervals in $\R$ happen to also be a connected set. We can write any subset $E\subset\R$ as 
        \[
            E=E_1\cup E_2\cdots\cup E_n,
        \]
        where each $E_i$ ($1\leq i\leq n$) is a connected set. Furthermore, each of these connected sets are disjoint, for if they were not, we could simply make a new connected set that is the union of the non-disjoint connected sets. Due to the density of $\Q\in\R$ each of these connected sets (which are also open intervals) has a rational number in them. Since $\Q$ is a countable set, we have that the number of number of disjoint open intervals is countable. 
    \end{proof}
    \item Explain why this statement cannot be true for ${\mathbb{R}}^{n}$ for any $n\geq 2$.
\end{enumerate}

\stepcounter{section}
\section*{Question 3}Let $f$ be an integrable function on the unit circle (so $f$ is $2\pi$-periodic).
\begin{enumerate}[label=(\alph*)]
    \item Suppose that $f(\theta+\pi)=f(\theta)$ for all $\theta\in{\mathbb{R}}$. Prove that $\hat{f}(n)=0$ for all odd integers $n$.
    \begin{proof}
        \[
            \hat{f}(n)=\frac{1}{2\pi}\int_{0}^{2\pi}f(x)e^{-inx}\,dx=\frac{1}{2\pi}\left[\int_{0}^{\pi}f(x)e^{-inx}\,dx+\int_{\pi}^{2\pi}f(x)e^{-inx}\,dx\right]
        \]
        Let 
        \[
            I_1=\int_{0}^{\pi}f(x)e^{-inx}\,dx,
        \]
        \[
            I_2=\int_{\pi}^{2\pi}f(x)e^{-inx}\,dx,
        \]
        \[
            u=x-\pi,\,du=dx,\,x=\pi\Rightarrow0,\,x=2\pi\Rightarrow\pi.
        \]
        then
        \[
            I_2=\int_{0}^{\pi}f(u+\pi)e^{-in(u+\pi)}\,du=e^{-in\pi}\int_{0}^{\pi}f(u)e^{-inu}\,du=(-1)^n I_1.
        \]
        Thus,
        \[
            \hat{f}(n)=I_1+(-1)^nI_1=
            \begin{cases}
                2I_1, & \text{if $x$ is even} \\
                0, & \text{if $x$ is odd} 
            \end{cases}.
        \]
    \end{proof}
    \item Compute the Fourier series of a trigonometric polynomial of the form
    \begin{equation}\label{eqn:trigpoly}
        a_0+a_1\cos(x)+\cdots+a_k\cos(kx).
    \end{equation}
        
    \textit{Solution.} There are no $\sin(nx)$ terms in (\ref{eqn:trigpoly}), so $b_n=0$ for all $n$. It is then easy to see that the Fourier series of (\ref{eqn:trigpoly}) is 
    \[
        a_0+\sum_{n=1}^{k}a_n\cos(nx).
    \]
\end{enumerate}


\stepcounter{section}
\section*{Question 4} Consider the sequence
\[
    f_{n}(x)=\frac{\pi n+\sin(nx)}{2n+\cos(n^{2}x)},\quad x\in[0,1].
\]
\begin{enumerate}[label=(\alph*)]
    \item Prove that $(f_{n})$ converges uniformly on $[0,1]$.
    \begin{claim}
        $f_n(x)$ converges to $\dfrac{\pi}{2}$ uniformly.
    \end{claim}
    \begin{proof}
        \[
            \left|f_n(x)-\frac{\pi}{2}\right|=\left|\frac{\pi n+\sin(nx)}{2n+\cos(n^{2}x)}-\frac{\pi}{2}\right|\leq\left|\frac{\pi n+1}{2n-1}-\frac{\pi}{2}\right|=\left|\frac{2(\pi n+1)-\pi(2n-1)}{2(2n-1)}\right|=\left|\frac{2+\pi}{4n-2}\right|
        \]
        Let $\varepsilon>0$, pick $N=\dfrac{\frac{2+\pi}{\varepsilon}+2}{4}$, then whenever $n>N$ we have 
        \[
             \left|f_n(x)-\frac{\pi}{2}\right|\leq\left|\frac{2+\pi}{4n-2}\right|<\left|\frac{2+\pi}{4N-2}\right|=\left|\frac{2+\pi}{4\left(\dfrac{\frac{2+\pi}{\varepsilon}+2}{4}\right)-2}\right|=\left|\frac{2+\pi}{\frac{2+\pi}{\varepsilon}}\right|=\varepsilon.
        \]
    \end{proof}
    \item Hence, or otherwise, evaluate the limit

    \[
        \lim_{n\to\infty}\int_{0}^{1}\frac{\pi n+\sin(nx)}{2n+\cos(n^{2}x)}dx.
    \]
    \textit{Solution.} We know that $f_n(x)$ converges uniformly to $\frac{\pi}{2}$, due to this we can switch the order of integration and the limit.
    \[
        \lim_{n\to\infty}\int_{0}^{1}\frac{\pi n+\sin(nx)}{2n+\cos(n^{2}x)}\,dx=\int_{0}^{1}\lim_{n\to\infty}\frac{\pi n+\sin(nx)}{2n+\cos(n^{2}x)}\,dx=\int_{0}^{1}\frac{\pi}{2}\,dx=\left[\frac{\pi}{2}x\right]_0^1=\frac{\pi}{2}.
    \]    
\end{enumerate}


\stepcounter{section}
\section*{Question 5}Consider a second-order differential equation of the form
\[
    x^{\prime\prime}(t)+5x(t)=F(t),
\]
where
\[
    F(t)=a_{0}+\sum_{n=1}^{\infty}b_{n}\cos(n\pi t)+\sum_{n=1}^{\infty}c_{n}\sin(n \pi t).
\]
\begin{enumerate}[label=(\alph*)]
    \item Solve this differential equation by considering a potential solution $x(t)$ given as a Fourier series:
    \[
        x(t)=A_{0}+\sum_{n=1}^{\infty}B_{n}\cos(n\pi t)+\sum_{n=1}^{\infty}C_{n}\sin(n \pi t),
    \]
    and obtain relations for $A_{0},B_{n},C_{n}$ in terms of the coefficients $a_{0},b_{n},c_{n}$ of $F(t)$.\\
    \textit{Solution.} 
    \[
        x''(t)=-\left[\sum_{n=1}^{\infty}B_n n^2\pi^2\cos(n\pi t)+C_n n^2\pi^2\sin(n\pi t)\right].
    \]
    \[
        -\left[\sum_{n=1}^{\infty}B_n n^2\pi^2\cos(n\pi t)+C_n n^2\pi^2\sin(n\pi t)\right]+5\left[A_{0}+\sum_{n=1}^{\infty}B_{n}\cos(n\pi t)+\sum_{n=1}^{\infty}C_{n}\sin(n \pi t)\right]=F
    \]
    \[
        a_{0}+\sum_{n=1}^{\infty}b_{n}\cos(n\pi t)+\sum_{n=1}^{\infty}c_{n}\sin(n \pi t)=5A_0+\sum_{n=1}^{\infty}(5-n^2\pi^2)\left[B_n\cos(n\pi t)+C_n\sin(n\pi t)\right].
    \]
    It can be seen that 
    \[
        A_0=\frac{a_0}{5},\quad B_n=\frac{b_n}{5-n^2\pi^2},\quad C_n=\frac{c_n}{5-n^2\pi^2}.
    \]
    Then we have 
    \[
        x(t)=\frac{a_0}{5}+\sum_{n=1}^{\infty}\frac{b_n}{5-n^2\pi^2}\cos(n\pi t)+\sum_{n=1}^{\infty}\frac{c_n}{5-n^2\pi^2}\sin(n \pi t).
    \]
    \item Suppose
    \[
        F(t)=\begin{cases}-1&\text{if }-1<t<0\\ 1&\text{if }0<t<1\end{cases}.
    \] so
    Solve for $x(t)$.\\
    \textit{Solution.} Period $T=2$ so $L=1$.
    \[
        F(t)=a_{0}+\sum_{n=1}^{\infty}b_{n}\cos\left(\frac{n\pi t}{L}\right)+\sum_{n=1}^{\infty}c_{n}\sin\left(\frac{n \pi t}{L}\right)=a_{0}+\sum_{n=1}^{\infty}b_{n}\cos(n\pi t)+\sum_{n=1}^{\infty}c_{n}\sin(n \pi t).
    \]
    Since $F(-t)=-F(t)$ for all $t$, $F(t)$ is odd $a_0=0$ and $b_n=0$ for all $n$.
    \[
        c_n=\frac{2}{T}\int_{-1}^{1}F(t)\sin(n\pi t)\,dt=\int_{-1}^{0}F(t)\sin(n\pi t)\,dt+\int_{0}^{1}F(t)\sin(n\pi t)\,dt
    \]
    \[
        c_n=-1\left[\frac{-\cos(n\pi t)}{n\pi}\right]_{-1}^0+\left[\frac{-\cos(n\pi t)}{n\pi}\right]_0^1
    \]
    \[
        c_n=\frac{1}{n\pi}\left[1-\cos(n\pi)\right]+\frac{1}{n\pi}\left[-\cos(n\pi)+1\right]
    \]
    \[
        c_n=\frac{2}{n\pi}\left[1-\cos(n\pi)\right]
    \]
    \[
        c_n=\frac{2}{n\pi}\left[1-(-1)^n\right].
    \]
    \[
        c_n=\begin{cases}\frac{4}{n\pi}&\text{if $n$ is even}\\ 0&\text{if $n$ is odd}\end{cases}
    \]
    \[
        F(n)=\sum_{n=1}^{\infty}\frac{2(1-(-1)^n)}{(n)\pi}\sin(n\pi t).
    \]
    We know from part a that 
    \[
        C_n=\frac{c_n}{5-n^2\pi^2}=\frac{\frac{2}{n\pi}\left[1-(-1)^n\right]}{5-n^2\pi^2}=\frac{2[1-(-1)^n]}{n\pi[5-n^2\pi^2]}.
    \]
    Then,
    \[
        x(t)=\sum_{n=1}^{\infty}\frac{2[1-(-1)^n]}{n\pi[5-n^2\pi^2]}\sin(n\pi t).
    \]
    Let $n=2k-1$,
    \[
        x(t)=\sum_{k=1}^{\infty}\frac{4}{(2k-1)\pi[5-(2k-1)^2\pi^2]}\sin((2k-1)\pi t).
    \]
    \[
        x(t)=\frac{4}{\pi}\sum_{k=1}^{\infty}\frac{\sin((2k-1)\pi t)}{(2k-1)(5-(2k-1)^2\pi^2)}.
    \]
\end{enumerate}


\end{document}