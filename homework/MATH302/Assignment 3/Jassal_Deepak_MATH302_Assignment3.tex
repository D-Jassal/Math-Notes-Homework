\documentclass[12pt]{article}

% Import preambles and macros for homework
% Essential packages
\usepackage{amsmath, amsfonts, amssymb, amsthm}
\usepackage{mathtools}
\usepackage{enumitem}
\usepackage{graphicx}
\usepackage{wrapfig}
\usepackage{systeme}
\usepackage{caption}
\usepackage{soul}
\usepackage[dvipsnames]{xcolor}
\usepackage{fancyhdr}
\allowdisplaybreaks

% Page layout
\usepackage[
  top=2cm,
  bottom=2cm,
  left=2cm,
  right=2cm,
  headheight=17pt,
  includehead,includefoot,
  heightrounded,
]{geometry}


% pgfornament for title page decorations
\usepackage[object=vectorian]{pgfornament}

% Fancy header/footer setup
\pagestyle{fancy}
\setlength{\headheight}{14.49998pt}
\addtolength{\topmargin}{-2.49998pt}
\renewcommand{\footrulewidth}{0.4pt}
\setlength\parindent{15pt}
% Math notation shortcuts
\newcommand{\R}{\mathbb{R}}
\newcommand{\Q}{\mathbb{Q}}
\newcommand{\Z}{\mathbb{Z}}
\newcommand{\N}{\mathbb{N}}
\newcommand{\C}{\mathbb{C}}
\newcommand{\X}{\mathcal{X}}

% Theorem environments
\newtheorem{mainthm}{Theorem}[section]
\newtheorem{theorem}{Theorem}[section]  
\newtheorem{lemma}[theorem]{Lemma}
\newtheorem{proposition}[theorem]{Proposition}
\newtheorem{corollary}[theorem]{Corollary}
\newtheorem{definition}[theorem]{Definition}
\newtheorem{claim}[theorem]{Claim}

% Calculus
\newcommand{\diff}{\mathop{}\!\mathrm{d}}
\newcommand{\deriv}[2]{\frac{\mathrm{d}#1}{\mathrm{d}#2}}
\newcommand{\pderiv}[2]{\frac{\partial #1}{\partial #2}}

% Linear Algebra
\newcommand{\inner}[2]{\langle #1, #2 \rangle}
\newcommand{\norm}[1]{\| #1 \|}
\newcommand{\tr}{\operatorname{tr}}
\newcommand{\spn}{\operatorname{span}}
\newcommand{\rank}{\operatorname{rank}}
\newcommand{\nullity}{\operatorname{nullity}}

% Logic
\newcommand{\contra}{\Rightarrow\Leftarrow}

% Custom commands for notes
\newcommand{\todo}[1]{\textcolor{red}{[TODO: #1]}}
\newcommand{\important}[1]{\textbf{\textcolor{blue}{#1}}}

%Number Theory
\DeclareMathOperator{\Li}{Li}
\newcommand{\floor}[1]{\left\lfloor #1 \right\rfloor}
\newcommand{\fract}[1]{\left\{ #1 \right\}}




\newcommand{\maketitlepage}{
    \begin{titlepage}
        \centering
        \vspace*{2.0cm}
        \pgfornament{84}\\
        {\LARGE \textsc{\coursename}\par}
        \vspace{0.5cm}
        {\large\coursecode\par}
        \vspace{0.5cm}
        {\large\instructor\par}
        \vspace{1.5cm}
        {\huge\bfseries\assignment\par}
        \vspace{1cm}
        {\LARGE\itshape\author\par}
        \vspace{2cm}
        {\large\bfseries Due Date:\par}
        \vspace{0.5cm}
        {\Large \duedate}\\
        \pgfornament{84}
    \end{titlepage}
}
% =============================================
% HOMEWORK CONFIGURATION - EDIT THESE VALUES!
% =============================================

% Your personal info
\renewcommand{\author}{Deepak Jassal}
\newcommand{\authorlast}{Jassal}

% Course info
\newcommand{\coursename}{Course Name}
\newcommand{\coursecode}{Course code}
\newcommand{\instructor}{Instructor}

% Assignment-specific info (CHANGE THESE FOR EACH HOMEWORK)
\newcommand{\assignment}{Assignment }
\newcommand{\duedate}{Month Day\textsuperscript{th}, 20XX}

% Header configuration
\fancyhead[l]{\assignment}
\fancyhead[c]{\coursecode}
\fancyhead[r]{\monthyear}
\fancyfoot[c]{\authorlast{ }\thepage}

\renewcommand{\author}{Deepak Jassal}
\renewcommand{\authorlast}{Jassal}
\renewcommand{\coursename}{ Introductory Mathematical Analysis }
\renewcommand{\coursecode}{MATH 302}
\renewcommand{\assignment}{Assignment 3}
\renewcommand{\instructor}{Dr. Stanley Yao Xiao}
\renewcommand{\duedate}{December 3\textsuperscript{rd}, 2025}

\begin{document}
\begin{titlepage}
	\centering
	\vspace*{2.0cm}	
	\pgfornament{84}\\
	{\LARGE \textsc{\coursename}\par}
	\vspace{0.5cm}
	{\large\coursecode\par}
    \vspace{0.5cm}
    {\large\instructor\par}
	\vspace{1.5cm}
	{\huge\bfseries\assignment\par}
	\vspace{1cm}
	{\LARGE\itshape\author\par}
    \vspace{2cm}
	{\large\bfseries Due Date:\par}
	\vspace{0.5cm}
	{\Large \duedate}\\
	\pgfornament{84}
\end{titlepage}
\stepcounter{section}
\section*{Question 1}Let $\{f_{n}:n\geq 1\}$ be a sequence in ${\cal C}([0,1],{\mathbb{R}})$. Define a function $F_{n}$ by
\[
    F_{n}(x)=\int_{0}^{x}\sin(f_{n}(t))dt,\quad x\in[0,1].
\]
Use the Arzela-Ascoli theorem to prove that $\{F_{n}:n\geq 1\}$ has a uniformly convergent subsequence.
\begin{proof}
    \[
        \Q\cap [0,1]\subset [0,1]
    \]
    \[
        \Q\cap [0,1]\subset \Q
    \]
    Since $\Q$ is countable and dense then so is $\Q\cap [0,1]$. So, $[0,1]$ has a countable dense subset. $\R$ is a complete metric space. 
\end{proof}

\stepcounter{section}
\section*{Question 2}
\begin{enumerate}[label=(\alph*)]
    \item Prove that every open subset of ${\mathbb{R}}$ (with respect to the standard topology $\tau_{\mathbb{R}}$) is a countable union of disjoint open intervals.
    \item Explain why this statement cannot be true for ${\mathbb{R}}^{n}$ for any $n\geq 2$.
\end{enumerate}

\stepcounter{section}
\section*{Question 3}Let $f$ be an integrable function on the unit circle (so $f$ is $2\pi$-periodic).
\begin{enumerate}[label=(\alph*)]
    \item Suppose that $f(\theta+\pi)=f(\theta)$ for all $\theta\in{\mathbb{R}}$. Prove that $\hat{f}(n)=0$ for all odd integers $n$.
    \item Find an example of a metric space $(X,d)$ and a subset $S\subset X$ such that $S=L(S)$, where $L(S)$ is the set of limit points of $S$.
\end{enumerate}


\stepcounter{section}
\section*{Question 4} Consider the sequence
\[
    f_{n}(x)=\frac{\pi n+\sin(nx)}{2n+\cos(n^{2}x)},\quad x\in[0,1].
\]
\begin{enumerate}[label=(\alph*)]
    \item Prove that $(f_{n})$ converges uniformly on $[0,1]$.
    \item Hence, or otherwise, evaluate the limit
\[
    \lim_{n\to\infty}\int_{0}^{1}\frac{\pi n+\sin(nx)}{2n+\cos(n^{2}x)}dx.
\]    
\end{enumerate}


\stepcounter{section}
\section*{Question 5}Consider a second-order differential equation of the form
\[
    x^{\prime\prime}(t)+5x(t)=F(t),
\]
where
\[
    F(t)=a_{0}+\sum_{n=1}^{\infty}b_{n}\cos(n\pi t)+\sum_{n=1}^{\infty}c_{n}\sin(n \pi t).
\]
\begin{enumerate}[label=(\alph*)]
    \item Solve this differential equation by considering a potential solution $x(t)$ given as a Fourier series:
\[
    x(t)=A_{0}+\sum_{n=1}^{\infty}B_{n}\cos(n\pi t)+\sum_{n=1}^{\infty}C_{n}\sin(n \pi t),
\]
and obtain relations for $A_{0},B_{n},C_{n}$ in terms of the coefficients $a_{0},b_{n},c_{n}$ of $F(t)$.
    \item Suppose
\[
    F(t)=\begin{cases}-1&\text{if }-1<t<0\\ 1&\text{if }0<t<1\end{cases}.
\]
Solve for $x(t)$.
\end{enumerate}


\end{document}