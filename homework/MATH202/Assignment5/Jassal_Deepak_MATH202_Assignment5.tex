\documentclass[12pt]{article}

% Import preambles and macros for notes
% Essential packages
\usepackage{amsmath, amsfonts, amssymb, amsthm}
\usepackage{mathtools}
\usepackage{enumitem}
\usepackage{graphicx}
\usepackage{wrapfig}
\usepackage{systeme}
\usepackage{caption}
\usepackage{soul}
\usepackage[dvipsnames]{xcolor}
\usepackage{fancyhdr}
\allowdisplaybreaks

% Page layout
\usepackage[
  top=2cm,
  bottom=2cm,
  left=2cm,
  right=2cm,
  headheight=17pt,
  includehead,includefoot,
  heightrounded,
]{geometry}


% pgfornament for title page decorations
\usepackage[object=vectorian]{pgfornament}

% Fancy header/footer setup
\pagestyle{fancy}
\setlength{\headheight}{14.49998pt}
\addtolength{\topmargin}{-2.49998pt}
\renewcommand{\footrulewidth}{0.4pt}
\setlength\parindent{15pt}
% Math notation shortcuts
\newcommand{\R}{\mathbb{R}}
\newcommand{\Q}{\mathbb{Q}}
\newcommand{\Z}{\mathbb{Z}}
\newcommand{\N}{\mathbb{N}}
\newcommand{\C}{\mathbb{C}}
\newcommand{\X}{\mathcal{X}}

% Theorem environments
\newtheorem{mainthm}{Theorem}[section]
\newtheorem{theorem}{Theorem}[section]  
\newtheorem{lemma}[theorem]{Lemma}
\newtheorem{proposition}[theorem]{Proposition}
\newtheorem{corollary}[theorem]{Corollary}
\newtheorem{definition}[theorem]{Definition}
\newtheorem{claim}[theorem]{Claim}

% Calculus
\newcommand{\diff}{\mathop{}\!\mathrm{d}}
\newcommand{\deriv}[2]{\frac{\mathrm{d}#1}{\mathrm{d}#2}}
\newcommand{\pderiv}[2]{\frac{\partial #1}{\partial #2}}

% Linear Algebra
\newcommand{\inner}[2]{\langle #1, #2 \rangle}
\newcommand{\norm}[1]{\| #1 \|}
\newcommand{\tr}{\operatorname{tr}}
\newcommand{\spn}{\operatorname{span}}
\newcommand{\rank}{\operatorname{rank}}
\newcommand{\nullity}{\operatorname{nullity}}

% Logic
\newcommand{\contra}{\Rightarrow\Leftarrow}

% Custom commands for notes
\newcommand{\todo}[1]{\textcolor{red}{[TODO: #1]}}
\newcommand{\important}[1]{\textbf{\textcolor{blue}{#1}}}

%Number Theory
\DeclareMathOperator{\Li}{Li}
\newcommand{\floor}[1]{\left\lfloor #1 \right\rfloor}
\newcommand{\fract}[1]{\left\{ #1 \right\}}




\newcommand{\maketitlepage}{
    \begin{titlepage}
        \centering
        \vspace*{2.0cm}
        \pgfornament{84}\\
        {\LARGE \textsc{\coursename}\par}
        \vspace{0.5cm}
        {\large\coursecode\par}
        \vspace{0.5cm}
        {\large\instructor\par}
        \vspace{1.5cm}
        {\huge\bfseries\assignment\par}
        \vspace{1cm}
        {\LARGE\itshape\author\par}
        \vspace{2cm}
        {\large\bfseries Due Date:\par}
        \vspace{0.5cm}
        {\Large \duedate}\\
        \pgfornament{84}
    \end{titlepage}
}
% =============================================
% HOMEWORK CONFIGURATION - EDIT THESE VALUES!
% =============================================

% Your personal info
\renewcommand{\author}{Deepak Jassal}
\newcommand{\authorlast}{Jassal}

% Course info
\newcommand{\coursename}{Course Name}
\newcommand{\coursecode}{Course code}
\newcommand{\instructor}{Instructor}

% Assignment-specific info (CHANGE THESE FOR EACH HOMEWORK)
\newcommand{\assignment}{Assignment }
\newcommand{\duedate}{Month Day\textsuperscript{th}, 20XX}

% Header configuration
\fancyhead[l]{\assignment}
\fancyhead[c]{\coursecode}
\fancyhead[r]{\monthyear}
\fancyfoot[c]{\authorlast{ }\thepage}

\renewcommand{\author}{Deepak Jassal}
\renewcommand{\authorlast}{Jassal}
\renewcommand{\coursename}{Multivariable Calculus I}
\renewcommand{\coursecode}{MATH 202}
\renewcommand{\assignment}{Assignment 5}
\renewcommand{\instructor}{Dr. Stanley Yao Xiao}
\renewcommand{\duedate}{November 15\textsuperscript{th}, 2025}

\begin{document}
\begin{titlepage}
	\centering
	\vspace*{2.0cm}	
	\pgfornament{84}\\
	{\LARGE \textsc{\coursename}\par}
	\vspace{0.5cm}
	{\large\coursecode\par}
    \vspace{0.5cm}
    {\large\instructor\par}
	\vspace{1.5cm}
	{\huge\bfseries\assignment\par}
	\vspace{1cm}
	{\LARGE\itshape\author\par}
    \vspace{2cm}
	{\large\bfseries Due Date:\par}
	\vspace{0.5cm}
	{\Large \duedate}\\
	\pgfornament{84}
\end{titlepage}
\stepcounter{section}
\section*{Question 1}
Compute the volume of a right cylindrical cone where the base radius is \( R \) and the height of the cone is \( h \), using cylindrical coordinates.

\stepcounter{section}
\section*{Question 2}
A spherical cap is the piece of a sphere (ball) sliced off by a plane. Suppose that a sphere has radius 10, and the height of the spherical cap is 6. Determine the volume of the spherical cap.

\stepcounter{section}
\section*{Question 3}
A client spends \( X \) minutes in an insurance agent's waiting room and \( Y \) minutes meeting with the agent (That is, both \( X, Y \) are random variables). The joint probability density function of \( X \) and \( Y \) can be modelled by

\[
f(x, y) = 
\begin{cases} 
ke^{-\frac{x+2y}{40}} & \text{for } x > 0, y > 0 \\
0 & \text{otherwise}.
\end{cases}
\]

Determine the value of \( k \) and the probability that a client spends less than 60 minutes at the agent's office.\\
\textit{Solution.}
\[
	\iint_{\R^2}f(x,y)\,dA=\int_{-\infty}^{\infty}\int_{-\infty}^{\infty}ke^{-\frac{x+2y}{40}}\,dy\,dx=k\int_{-\infty}^{\infty}e^{\frac{-x}{40}}\int_{-\infty}^{\infty}e^{\frac{-y}{20}}\,dy\,dx
\]
Since the probability density function is non-zero only for non-zero values of $x$ and $y$ we can further simplify the double integral
\[
	k\int_{0}^{\infty}e^{\frac{-x}{40}}\int_{0}^{\infty}e^{\frac{-y}{20}}\,dy\,dx
\]
\[
	u=\frac{y}{20},\: 20\,du=dy,\: y=0\Rightarrow u=0,\: y=\infty \Rightarrow u=\infty
\]
\[
	k\int_{0}^{\infty}e^{\frac{-x}{40}}\int_{0}^{\infty}20e^{-u}\,du\,dx=k\int_{0}^{\infty}e^{\frac{-x}{40}}\left[ 20e^{-u} \right]^\infty_0\,dx=-20k\int_{0}^{\infty}e^{\frac{-x}{40}}\,dx
\]
\[
	v=\frac{x}{40},\: 40\,dv=dx,\: x=0\Rightarrow v=0,\: x=\infty \Rightarrow v=\infty
\]
\[
	-20k\int_{0}^{\infty}40e^{-v}\,dv=800k\left[ e^{-v} \right]^\infty_0=800k
\]
We want the total probability to be 1, so we have 
\[
	k=\frac{1}{800}.
\]
We want to find $\mathcal{P}(X+Y<60)$, that is evaluate the integral
\[
	\frac{1}{800}\int_{X=0}^{60}\int_{Y=0}^{60-x}e^{-\frac{x+2y}{40}}\,dy\,dx.
\]
Using the same substitutions as the previous part we get
\[
	\frac{1}{800}\int_{0}^{60}e^{\frac{-x}{40}}\int_{0}^{60-x}e^{\frac{-y}{20}}\,dy\,dx=\frac{1}{800}\int_{0}^{60}e^{\frac{-x}{40}}\left[ 20e^{-u} \right]^{\frac{60-x}{20}}_0\,dx=\frac{1}{800}\int_{0}^{60}e^{\frac{-x}{40}}\left( 20e^{\frac{x-60}{20}} -20 \right)\,dx
\]
Factoring and multiplying through
\[
	\frac{1}{40}\int_{0}^{60}e^{-3}e^{\frac{x}{40}} - e^{\frac{-x}{40}}\,dx
\]
\stepcounter{section}
\section*{Question 4}
(Worth 20 points). In this question, we derive some formulas for volumes of higher dimensional balls. Let \( V_n(R) \) denote the volume of a \( n \)-ball (in \( \mathbb{R}^{n+1} \)) of radius \( R \).

A 0-ball is just an interval. This allows us to compute the volume of a 2-ball, defined by the inequality

\[
x^2 + y^2 + z^2 \leq R^2,
\]

using cylindrical coordinates, we deduce:

\[
V_2(R) = \int_{0}^{2\pi} \int_{0}^{R} \int_{-\sqrt{R^2-r^2}}^{\sqrt{R^2-r^2}} r \, dh \, dr \, d\theta = \int_{0}^{2\pi} \int_{0}^{R} 2\sqrt{R^2 - r^2} \, dr \, d\theta.
\]

\begin{enumerate}[label=(\alph*)]
\item Verify that \( V_2(R) = \frac{4\pi R^3}{3} \), by finishing with the iterated integral above.

\item Using the same idea, deduce that

\[
V_3(R) = \int_{0}^{2\pi} \int_{0}^{R} \pi(R^2 - r^2)r \, dr \, d\theta.
\]

\item Generalize this to obtain the formula

\[
V_{n+2}(R) = \int_{0}^{2\pi} \int_{0}^{R} V_n(\sqrt{R^2 - r^2}) \, dr \, d\theta.
\]

\item Using the recursion in part (c), give formulas for \( V_4(R), V_5(R), V_6(R) \).

\item Evaluate the limit
\[
\lim_{n \to \infty} \frac{V_n(R)}{R^{n+1}}.
\]

\end{enumerate}

\end{document}