\documentclass[12pt]{article}

% Import preambles and macros for homework
% Essential packages
\usepackage{amsmath, amsfonts, amssymb, amsthm}
\usepackage{mathtools}
\usepackage{enumitem}
\usepackage{graphicx}
\usepackage{wrapfig}
\usepackage{systeme}
\usepackage{caption}
\usepackage{soul}
\usepackage[dvipsnames]{xcolor}
\usepackage{fancyhdr}
\allowdisplaybreaks

% Page layout
\usepackage[
  top=2cm,
  bottom=2cm,
  left=2cm,
  right=2cm,
  headheight=17pt,
  includehead,includefoot,
  heightrounded,
]{geometry}


% pgfornament for title page decorations
\usepackage[object=vectorian]{pgfornament}

% Fancy header/footer setup
\pagestyle{fancy}
\setlength{\headheight}{14.49998pt}
\addtolength{\topmargin}{-2.49998pt}
\renewcommand{\footrulewidth}{0.4pt}
\setlength\parindent{15pt}
% Math notation shortcuts
\newcommand{\R}{\mathbb{R}}
\newcommand{\Q}{\mathbb{Q}}
\newcommand{\Z}{\mathbb{Z}}
\newcommand{\N}{\mathbb{N}}
\newcommand{\C}{\mathbb{C}}
\newcommand{\X}{\mathcal{X}}

% Theorem environments
\newtheorem{mainthm}{Theorem}[section]
\newtheorem{theorem}{Theorem}[section]  
\newtheorem{lemma}[theorem]{Lemma}
\newtheorem{proposition}[theorem]{Proposition}
\newtheorem{corollary}[theorem]{Corollary}
\newtheorem{definition}[theorem]{Definition}
\newtheorem{claim}[theorem]{Claim}

% Calculus
\newcommand{\diff}{\mathop{}\!\mathrm{d}}
\newcommand{\deriv}[2]{\frac{\mathrm{d}#1}{\mathrm{d}#2}}
\newcommand{\pderiv}[2]{\frac{\partial #1}{\partial #2}}

% Linear Algebra
\newcommand{\inner}[2]{\langle #1, #2 \rangle}
\newcommand{\norm}[1]{\| #1 \|}
\newcommand{\tr}{\operatorname{tr}}
\newcommand{\spn}{\operatorname{span}}
\newcommand{\rank}{\operatorname{rank}}
\newcommand{\nullity}{\operatorname{nullity}}

% Logic
\newcommand{\contra}{\Rightarrow\Leftarrow}

% Custom commands for notes
\newcommand{\todo}[1]{\textcolor{red}{[TODO: #1]}}
\newcommand{\important}[1]{\textbf{\textcolor{blue}{#1}}}

%Number Theory
\DeclareMathOperator{\Li}{Li}
\newcommand{\floor}[1]{\left\lfloor #1 \right\rfloor}
\newcommand{\fract}[1]{\left\{ #1 \right\}}




\newcommand{\maketitlepage}{
    \begin{titlepage}
        \centering
        \vspace*{2.0cm}
        \pgfornament{84}\\
        {\LARGE \textsc{\coursename}\par}
        \vspace{0.5cm}
        {\large\coursecode\par}
        \vspace{0.5cm}
        {\large\instructor\par}
        \vspace{1.5cm}
        {\huge\bfseries\assignment\par}
        \vspace{1cm}
        {\LARGE\itshape\author\par}
        \vspace{2cm}
        {\large\bfseries Due Date:\par}
        \vspace{0.5cm}
        {\Large \duedate}\\
        \pgfornament{84}
    \end{titlepage}
}
% =============================================
% HOMEWORK CONFIGURATION - EDIT THESE VALUES!
% =============================================

% Your personal info
\renewcommand{\author}{Deepak Jassal}
\newcommand{\authorlast}{Jassal}

% Course info
\newcommand{\coursename}{Course Name}
\newcommand{\coursecode}{Course code}
\newcommand{\instructor}{Instructor}

% Assignment-specific info (CHANGE THESE FOR EACH HOMEWORK)
\newcommand{\assignment}{Assignment }
\newcommand{\duedate}{Month Day\textsuperscript{th}, 20XX}

% Header configuration
\fancyhead[l]{\assignment}
\fancyhead[c]{\coursecode}
\fancyhead[r]{\monthyear}
\fancyfoot[c]{\authorlast{ }\thepage}

\renewcommand{\author}{Deepak Jassal}
\renewcommand{\authorlast}{Jassal}
\renewcommand{\coursename}{Number Theory}
\renewcommand{\coursecode}{MATH 480}
\renewcommand{\assignment}{Assignment 5}
\renewcommand{\instructor}{Dr. Alia Hamieh}
\renewcommand{\duedate}{December 4\textsuperscript{th}, 2025}


\begin{document}
\begin{titlepage}
	\centering
	\vspace*{2.0cm}	
	\pgfornament{84}\\
	{\LARGE \textsc{\coursename}\par}
	\vspace{0.5cm}
	{\large\coursecode\par}
    \vspace{0.5cm}
    {\large\instructor\par}
	\vspace{1.5cm}
	{\huge\bfseries\assignment\par}
	\vspace{1cm}  
	{\LARGE\itshape\author\par}
    \vspace{2cm}
	{\large\bfseries Due Date:\par}
	\vspace{0.5cm}
	{\Large \duedate}\\
	\pgfornament{84}
\end{titlepage}
\stepcounter{section}
\section*{Question 1 [3]}
The Mangoldt function $\Lambda$ is defined for all positive integers as follows:
\[
\Lambda(n) =
\begin{cases}
\log p & \text{if } n = p^k \text{ where } p \text{ is prime and } k \text{ is a positive integer} \\
0 & \text{otherwise.}
\end{cases}
\]
Show that $\Lambda(n) = -\sum_{d|n} \mu(d) \log(d)$.
\begin{proof}
    \begin{claim}
        \[
            \log(n)=\sum_{d\mid n}\Lambda(d)
        \]
    \end{claim}
    \begin{proof}[Proof of Claim]
        If $p$ is prime we have
        \[
            \sum_{d\mid p^k}\Lambda(d)=\Lambda(1)+\Lambda(p)+\cdots+\Lambda(p^k)=k\log(p)=\log(p^k).
        \]
        If $n=p_1^{e_1}p_2^{e_2}\cdots p_k^{e_k}$ we have,
        \[
            \sum_{d\mid p^k}\Lambda(n)=\sum_{i=1}^{k}\left(\sum_{d\mid p_i^{e_i}}\Lambda(n)\right)=\sum_{i=1}^{k}\log(p_i)=\log(n).
        \]
    \end{proof}
    Recall that 
    \[
        \mu(n)=
        \begin{cases}
            1 & \text{if } n=1\\
            0 & \text{if } n=0\\
            (-1^k)& \text{if } n=p_1^{e_1}p_2^{e_2}\cdots p_k^{e_k}.
        \end{cases}
    \]
    So we can see that $\log(n)$ is the summatory function of $\Lambda(n)$. Using Mobius inversion,
    \[
        \Lambda(n)=\sum_{d\vert n} \mu(d)\log\left(\frac{n}{d}\right)=\sum_{d\vert n} \mu(d)\log\left(n\right)-\sum_{d\vert n} \mu(d)\log\left(d\right),
    \] 
    \[
        \sum_{d\vert n} \mu(d)\log\left(n\right)=
        \begin{cases}
            \text{if } n=1,\log(1)=0\\
            \text{if } n>1,\mu(n)=0.
        \end{cases}       
    \]
    \[
        \Lambda(n)=\sum_{d\vert n} \mu(d)\log\left(\frac{n}{d}\right)=-\sum_{d\vert n} \mu(d)\log\left(d\right),        
    \]
\end{proof}

\section*{Question 2 [3]}
Show that $\sum_{e|n} \frac{n\sigma(e)}{e} = \sum_{e|n} e d(e)$ for all positive integers $n$.
\begin{proof}
    \[
        \sum_{e|n} \frac{n\sigma(e)}{e} =(\sigma\ast \mathrm{id})(n).
    \]
    Recall that
    \[
        (1\ast\mathrm{id})(n)=\sigma(n),
    \]
    and
    \[
        d(n)n=((1\ast\mathrm{id})\ast\mathrm{id})(n)=(\sigma\ast\mathrm{id})(n)=\sum_{e|n} \frac{n\sigma(e)}{e}.
    \]
\end{proof}
\section*{Question 3 [3]}
If $n$ is a positive integer, let $\sigma_{-1}(n) = \sum_{\substack{d|n \\ d>0}} \frac{1}{d}$. Prove that $n$ is perfect if and only if $\sigma_{-1}(n) = 2$.
\begin{proof}
    $(\Rightarrow)$ We are given that $\sigma(n)=2n$ then,
    \[
        \sum_{d\mid n}d=2n
    \]
    \[
        \sigma_{-1}(n)\sum=\frac{1}{n}m_{d\mid n}d=\frac{1}{n} 2n=2.
    \]
    $(\Leftarrow)$ We are given that $\sigma_{-1}(n)=2$ then,
    \[
        \sum_{d\mid n}\frac{1}{d}=2
    \]
    \[
        n\sum_{d\mid n}\frac{1}{d}=\sum_{d\mid n}\frac{n}{d}=\sigma(n)=2n.
    \]   
\end{proof}
\newpage
\section*{Question 4 [4]}
Suppose that $n$ is an odd perfect number. Show that $n = p^a m^2$ where $m$ is an integer and $p$ is an odd prime with $p \equiv a \equiv 1 \mod 4$.

\section*{Question 5 [4]}
Let $a$ and $b$ be positive integers. We say $a$ and $b$ are amicable numbers if $\sigma(a) = \sigma(b) = a + b$.

(a) Let $n > 1$ be an integer. Suppose that $p = 3 \times 2^{n-1} - 1$, $q = 3 \times 2^n - 1$, $r = 9 \times 2^{2n-1} - 1$ are prime numbers. Prove that
\[
a = 2^n p q, \quad b = 2^n r
\]
is a pair of amicable numbers.
\begin{proof}
    \[
        \sigma(a)=\sigma(2^n)\sigma(p)\sigma(q)=(2^{n+1}-1)(3\times2^{n-1})(3\times2^n)
    \]
    \[
        (2^{n+1}-1)(9\times2^{2n-1})=9\times2^{3n}-9\times2^{2n-1}
    \]
    \[
        \sigma(a) =9\times2^{n-1}(2^{2n+1}-2^n)
    \]
    \[
        \sigma(b)=\sigma(2^n)\sigma(9\times2^{2n-1}-1)=(2^{n+1}-1)(9\times2^{2n-1})
    \]
    \[
        \sigma(b)=(p\times2^{3m}-9\times2^{2n-1})=9\times2^{n-1}(2^{2n+1}-2^n)
    \]
    \[
        a+b=2^n(3\times2^{n-1}-1)(3\times2^n-1)+2^n(9\times2^{2n-1}-1)
    \]
\end{proof}
(b) Find two pairs of amicable numbers by finding such prime numbers $p$, $q$ and $r$.

\section*{Question 6 [4]}
Let $f$ be an arithmetic function.

(a) Prove that $\sum_{\substack{1 \leq a \leq n \\ (a,n)=1}} f(a) = \sum_{d|n} \mu(d) \sum_{\substack{1 \leq a \leq n \\ d|a}} f(a)$.

(b) Prove that when $n > 1$, we have
\[
\sum_{\substack{1 \leq a \leq n \\ (a,n)=1}} a = \frac{1}{2} n\phi(n).
\]

\section*{Question 7 [3]}
Let $n$ be a positive integer. Prove that
\[
\sum_{\substack{a=1 \\ (a,n)=1}}^n e^{\frac{2\pi i a}{n}} = \mu(n).
\]
Note that here $i$ is the complex number satisfying $i^2 = -1$, and by Euler's formula, we have $e^{2\pi i} = 1$. You may use without proof the fact that for a positive integer $k$, we have the following:
\[
\sum_{a=1}^k e^{\frac{2\pi i a}{k}} = 1 \text{ if } k = 1 \quad \text{and} \quad \sum_{a=1}^k e^{\frac{2\pi i a}{k}} = 0 \text{ if } k \neq 1.
\]

\section*{Question 8 [3]}
Solve the diophantine equation $x^2 + y^2 = 2z^2$.

\section*{Question 9 [3]}
Prove that the equation $x^3 + 2y^3 = 4z^3$ has no solution in the positive integers.

\end{document}