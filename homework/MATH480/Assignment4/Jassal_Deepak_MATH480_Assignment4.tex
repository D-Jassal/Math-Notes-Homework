\documentclass[12pt]{article}

% Import preambles and macros for homework
% Essential packages
\usepackage{amsmath, amsfonts, amssymb, amsthm}
\usepackage{mathtools}
\usepackage{enumitem}
\usepackage{graphicx}
\usepackage{wrapfig}
\usepackage{systeme}
\usepackage{caption}
\usepackage{soul}
\usepackage[dvipsnames]{xcolor}
\usepackage{fancyhdr}
\allowdisplaybreaks

% Page layout
\usepackage[
  top=2cm,
  bottom=2cm,
  left=2cm,
  right=2cm,
  headheight=17pt,
  includehead,includefoot,
  heightrounded,
]{geometry}


% pgfornament for title page decorations
\usepackage[object=vectorian]{pgfornament}

% Fancy header/footer setup
\pagestyle{fancy}
\setlength{\headheight}{14.49998pt}
\addtolength{\topmargin}{-2.49998pt}
\renewcommand{\footrulewidth}{0.4pt}
\setlength\parindent{15pt}
% Math notation shortcuts
\newcommand{\R}{\mathbb{R}}
\newcommand{\Q}{\mathbb{Q}}
\newcommand{\Z}{\mathbb{Z}}
\newcommand{\N}{\mathbb{N}}
\newcommand{\C}{\mathbb{C}}
\newcommand{\X}{\mathcal{X}}

% Theorem environments
\newtheorem{mainthm}{Theorem}[section]
\newtheorem{theorem}{Theorem}[section]  
\newtheorem{lemma}[theorem]{Lemma}
\newtheorem{proposition}[theorem]{Proposition}
\newtheorem{corollary}[theorem]{Corollary}
\newtheorem{definition}[theorem]{Definition}
\newtheorem{claim}[theorem]{Claim}

% Calculus
\newcommand{\diff}{\mathop{}\!\mathrm{d}}
\newcommand{\deriv}[2]{\frac{\mathrm{d}#1}{\mathrm{d}#2}}
\newcommand{\pderiv}[2]{\frac{\partial #1}{\partial #2}}

% Linear Algebra
\newcommand{\inner}[2]{\langle #1, #2 \rangle}
\newcommand{\norm}[1]{\| #1 \|}
\newcommand{\tr}{\operatorname{tr}}
\newcommand{\spn}{\operatorname{span}}
\newcommand{\rank}{\operatorname{rank}}
\newcommand{\nullity}{\operatorname{nullity}}

% Logic
\newcommand{\contra}{\Rightarrow\Leftarrow}

% Custom commands for notes
\newcommand{\todo}[1]{\textcolor{red}{[TODO: #1]}}
\newcommand{\important}[1]{\textbf{\textcolor{blue}{#1}}}

%Number Theory
\DeclareMathOperator{\Li}{Li}
\newcommand{\floor}[1]{\left\lfloor #1 \right\rfloor}
\newcommand{\fract}[1]{\left\{ #1 \right\}}




\newcommand{\maketitlepage}{
    \begin{titlepage}
        \centering
        \vspace*{2.0cm}
        \pgfornament{84}\\
        {\LARGE \textsc{\coursename}\par}
        \vspace{0.5cm}
        {\large\coursecode\par}
        \vspace{0.5cm}
        {\large\instructor\par}
        \vspace{1.5cm}
        {\huge\bfseries\assignment\par}
        \vspace{1cm}
        {\LARGE\itshape\author\par}
        \vspace{2cm}
        {\large\bfseries Due Date:\par}
        \vspace{0.5cm}
        {\Large \duedate}\\
        \pgfornament{84}
    \end{titlepage}
}
% =============================================
% HOMEWORK CONFIGURATION - EDIT THESE VALUES!
% =============================================

% Your personal info
\renewcommand{\author}{Deepak Jassal}
\newcommand{\authorlast}{Jassal}

% Course info
\newcommand{\coursename}{Course Name}
\newcommand{\coursecode}{Course code}
\newcommand{\instructor}{Instructor}

% Assignment-specific info (CHANGE THESE FOR EACH HOMEWORK)
\newcommand{\assignment}{Assignment }
\newcommand{\duedate}{Month Day\textsuperscript{th}, 20XX}

% Header configuration
\fancyhead[l]{\assignment}
\fancyhead[c]{\coursecode}
\fancyhead[r]{\monthyear}
\fancyfoot[c]{\authorlast{ }\thepage}

\renewcommand{\author}{Deepak Jassal}
\renewcommand{\authorlast}{Jassal}
\renewcommand{\coursename}{Number Theory}
\renewcommand{\coursecode}{MATH 480}
\renewcommand{\assignment}{Assignment 4}
\renewcommand{\instructor}{Dr. Alia Hamieh}
\renewcommand{\duedate}{November 20\textsuperscript{th}, 2025}


\begin{document}
\begin{titlepage}
	\centering
	\vspace*{2.0cm}	
	\pgfornament{84}\\
	{\LARGE \textsc{\coursename}\par}
	\vspace{0.5cm}
	{\large\coursecode\par}
    \vspace{0.5cm}
    {\large\instructor\par}
	\vspace{1.5cm}
	{\huge\bfseries\assignment\par}
	\vspace{1cm}  
	{\LARGE\itshape\author\par}
    \vspace{2cm}
	{\large\bfseries Due Date:\par}
	\vspace{0.5cm}
	{\Large \duedate}\\
	\pgfornament{84}
\end{titlepage}
\stepcounter{section}
\section*{Question 1 [2 marks]}
Find all the primitive roots modulo 27.\\
$27=3^3$. $\varphi(3)=2$, by \textit{lemma 2.8.13} the number of integers less than 3 of order 2 do not exceed $\varphi(2)=1$. Also, by \textit{theorem 2.8.9} we know that 3 has a primitive root. Since $2^1\not\equiv 1 \mod 3$ and $2^2\equiv 1 \mod 3$ we have 2 as a primitive root of 3. By \textit{theorem 2.8.15} we know that either $2$ or $2+3=5$ is a primitive root(s) for 9. $\varphi(9)=6$, so we need to check to see if the order of 2 or 5 are 6 in modulo 9. 
\[
	2^1=2,\,2^2=4,\,2^3=8,\,2^4=16\equiv 7,\,2^5=32\equiv 5\mod 9,\,2^6\equiv 5\times2\equiv1\mod 9,
\]
and 
\begin{align*}
	5^1=5,\,5^2=25\equiv 7,\,5^3\equiv 7\times5=&35\equiv8\mod 9,\,5^4\equiv8\times5\equiv 4,\\5^5\equiv4\times5\equiv 2\mod 9,&\,5^6\equiv 2\times5\equiv1\mod 9.	
\end{align*}
So, both 2 and 5 are primitive roots of 9. From \textit{theorem 2.8.16} we know that both 2 and 5 are primitive roots of 27. The other primitive roots of 27 are of the form $2^a$ where $a$ is any integer mod 27 such that $\gcd(a,\phi(27))=\gcd(a,18)=1$. These integers are 1,5,7,11,13,17.
\[
	\begin{aligned}
		2^1 &\equiv 2 \\
		2^5 &\equiv 32 \equiv 5 \\
		2^7 &\equiv 2^5 \cdot 2^2 \equiv 5 \cdot 4 = 20 \\
		2^{11} &= 2^9 \cdot 2^2 \equiv (-1) \cdot 4 \equiv 23 \\
		2^{13} &= 2^9 \cdot 2^4 \equiv (-1) \cdot 16 \equiv 11 \\
		2^{17} &= 2^9 \cdot 2^8 \equiv (-1) \cdot 13 \equiv 14
	\end{aligned}
\]
So the primitive roots of 27 are 2,5,20,23,11,14.


\stepcounter{section}
\section*{Question 2 [2 marks]}
Evaluate $\displaystyle{ \left(\frac{105}{1009}\right)}$.\\
\[
	\left(\frac{105}{1009}\right)=\left(\frac{3}{1009}\right)\left(\frac{5}{1009}\right)\left(\frac{7}{1009}\right)
\] 
By \textit{theorem 3.2.1} we have
\[
	\left(\frac{105}{1009}\right)=\left(\frac{1009}{3}\right)\left(\frac{1009}{5}\right)\left(\frac{1009}{7}\right)
\] 
\[
	\left(\frac{105}{1009}\right)=\left(\frac{1}{3}\right)\left(\frac{4}{5}\right)\left(\frac{1}{7}\right)
\] 
\[
	\left(\frac{105}{1009}\right)=\left(\frac{1}{3}\right)\left(\frac{2}{5}\right)\left(\frac{2}{5}\right)\left(\frac{1}{7}\right)
\] 
\[
	\left(\frac{105}{1009}\right)=\left(\frac{-1}{3}\right)\left(\frac{-1}{3}\right)\left(-1\right)\left(-1\right)\left(\frac{-1}{7}\right)\left(\frac{-1}{7}\right)
\]
\[
	\left(\frac{105}{1009}\right)=\left(-1\right)\left(-1\right)\left(-1\right)\left(-1\right)\left(-1\right)\left(-1\right)=1
\] 


\stepcounter{section}
\section*{Question 3 [3 marks]}
Let $m$ be a positive integer with a primitive root. Suppose that $(a,m)=1$. Prove that then the congruence $x^n\equiv a\mod m$ has exactly $(n,\phi(m))$  solutions if and only if $a^{\frac{\phi(m)}{(\phi(m),n)}}\equiv 1\mod m$.
\begin{proof}
	$(\Rightarrow)$ $x^n\equiv a\mod m$ has $\gcd(n,\varphi(m))$ solutions. Let $y=\gcd(n,\varphi(m))$. We know that $a^{\varphi(m)}\equiv 1 \mod m$. Let $r$ be the primitive root, then $a\equiv r^k\mod m$, and $x^n\equiv r^kl\mod m$ for some $k,l \in \Z$. So 
	\[
		r^{nl}\equiv r^k\mod m
	\]
	then
	\[
		nl\equiv k \mod \varphi(m)
	\]
	this implies that
	\[
		y \vert k
	\]
	then $k/y\in \Z$. So 
	\[
		r^k\equiv r^{k\cdot \frac{\varphi(m)}{y}}\equiv a^{\frac{\phi(m)}{y}}.
	\] 
	$(\Leftarrow)$ $a^{\frac{\varphi{m}}{y}}\equiv 1$. Let $a\equiv r^k\mod m$, and $x^n\equiv r^kl\mod m$ for some $k,l \in \Z$. then
	\[
		r^{\frac{k\varphi(m)}{y}}\equiv 1 \mod m.
	\]
	\[
		r^{\frac{k}{y}\varphi(m)}\equiv 1 \mod m.
	\]
	So $k\vert y$. Then $ln\equiv k \mod m$ has $y$ solutions. Then 
	\[
		r^{ln}\equiv r^k \Leftrightarrow x^n\equiv a \mod m
	\]
\end{proof}
\stepcounter{section}
\section*{Question 4 [3 marks]}
Let $p$ be an odd prime number, and suppose that $h \geq 2$. Denote by $g$ a primitive root modulo $p^{h}$.
\begin{enumerate}[label=(\alph*)]
    \item List all the solutions of the congruence $x^{p} \equiv 1 \pmod{p^{h}}$  using the primitive root $g$ modulo $p^{h}$.
    \begin{proof}
		Let $x\equiv g^k\mod p^h$ for some $k\in\Z.$ Then $g^{kp}\equiv 1 \mod p^h$ and $\varphi(p^h)=p^(h-1)(p-1)$. So 
		\[
			k=t(p^{h-2}(p-1)),\, 1\leq t\leq p-1.
		\]
	\end{proof}
    \item  List all the solutions of the congruence $x^{2p} \equiv 1 \pmod{p^{h}}$ using the primitive root $g$ modulo $p^{h}$.
    \begin{proof}
		Let $x\equiv g^k\mod p^h$ for some $k\in\Z.$ Then $g^{2kp}\equiv 1 \mod p^h$ and $\varphi(p^h)=p^(h-1)(p-1)$. So 
		\[
			k=\frac{t}{2}(p^{h-2}(p-1)),\, 1\leq t\leq p-1.
		\]
	\end{proof}
\end{enumerate}

\stepcounter{section}
\section*{Question 5 [2 marks]}
Let $n$ be a positive integer with a primitive root. Using this primitive root, prove that the product of all positive integers less than $n$ and relatively prime to $n$ is congruent to $-1$ modulo $n$.
\begin{proof}
	There are $\varphi(n)$ integers that are less than $n$ and relatively prime to $n$. Let these integers be
	\[
		\{
			m_1,m_2,\dots,m_{\varphi(n)}.
		\}
	\]
	We can rewrite these using the primitve root
	\[
		\{
			g^{k_1},g^{k_2},\dots,g^{k_{\varphi(m)}}.
		\}
	\]
	These can be further rewritten as 
	\[
		\{
			g^0,g^1,\dots,g^{\varphi(n)-1}.
		\}
	\]
	So this product is 
	\[
		\prod_{r=0}^{\varphi(n)-1}g^r=g^{1+2+\cdots+\varphi(n)-1}=g^{\frac{(\varphi(n)-1)\varphi(n)}{2}}
	\]
	This is an element of order 2, so we have
	\[
		\prod_{r=0}^{\varphi(n)-1}g^r=g^{1+2+\cdots+\varphi(n)-1}=g^{\frac{(\varphi(n)-1)\varphi(n)}{2}}\equiv -1\mod n.	
	\]
\end{proof}

\stepcounter{section}
\section*{Question 6 [2 marks]}
Let $p_{1},p_{2},\ldots,p_{r}$ be distinct prime numbers. Show that there exists an integer $g$ such  that $g$ is a primitive root modulo $p_{i}$ for all $1 \leq i \leq r$.
\begin{proof}
	Let $g_i$ be a primitive root modulo $p_i$ for $1\leq i\leq r$. By CRT there exists
	\[
		g\equiv g_i\mod p_i\quad 1\leq i\leq r.
	\]
	And this $g$ is a primitive root for all moduli because $g\equiv g_i\mod p_i$ for all $i$.
\end{proof}

\stepcounter{section}
\section*{Question 7 [2 marks]}
\begin{enumerate}[label=(\alph*)]
    \item Let $a$ be an integer with $a \geq 2$, and suppose that $q \in \mathbb{N}$. What is the smallest positive integer $d$ satisfying the property that $a^{d} \equiv 1 \pmod{a^{q}-1}$? Deduce that $q$ divides $\varphi(a^{q}-1)$.
    \begin{proof}
		$a^{d} \equiv 1 \pmod{a^{q}-1}\Leftrightarrow a^{d}-1 \equiv 0 \pmod{a^{q}-1}\Leftrightarrow a^q-1\vert a^d-1\Leftrightarrow q\vert d$. The smallest $d$ is $d=q$. $\gcd(a^q-1,a)=1$ so $a^{\varphi(a^q-1)}\equiv 1 \mod a^q-1$ and $a^q\equiv 1 \mod a^q-1$ so $\varphi(a^q-1)=wq$ for some $w\in\Z$. Therefore, $q\vert \varphi(a^q-1)$.
	\end{proof}
    \item Let $q$ be a prime number. By considering the prime factorisation of the integer $N = a^{q}-1$, show that either $N$ is divisible by $q$, or else $N$ is divisible by a prime number $p$ with $p \equiv 1 \pmod{q}$.
    \begin{proof}
		\[\varphi(N)=\prod_{p^e\vert N}p^{e-1}(p-1).\]
		Assume for contradiction that no prime divisor of $N$ is of the form $p=q$ or $p\equiv 1\mod q$. Then for all $p$ that divide $N$ we habe $q\nmid p-1$ which shows that $q\nmid \varphi(N)$. But in 7a we shows that $q\vert \varphi(N)$, this is a contradiction. So either $p=q$ or $p\equiv 1\mod q$.
	\end{proof}
\end{enumerate}

\stepcounter{section}
\section*{Question 8 [3 marks]}
Let $q$ be a prime number. Prove that there are infinitely many prime
numbers $p$ with $p \equiv 1 \pmod{q}$.
\begin{proof}
	Assume there are finitely many primes $p$ with $p\equiv 1 \mod q$. Let these primes be
	\[
		p_1,p_2,\dots,p_n.
	\]
	Let 
	\[
		a=\prod^n_{i=1}p_i,\,N=a^q-1.
	\]
	
\end{proof}

\stepcounter{section}
\section*{Question 9 [3 marks]}
Let $p\geq 5$ be an odd prime, show that
\[ 
    \left(\frac{3}{p}\right)=\begin{cases} 1\:\:\: \text{if}\:\:\: p\equiv \pm1\mod{12}\\ -1\:\:\: \text{if}\:\:\: p\equiv \pm5\mod{12}.\end{cases}
\] 

\stepcounter{section}
\section*{Question 10 [6 marks]}
Let $n>1$ be an odd integer. Write $n=p_1^{e_1}\cdots p_k^{e_k}$. Let $a$ be an integer. We define the {\bf Jacobi symbol} $\left(\frac{a}{n}\right)$ as follows:
\[\left(\frac{a}{n}\right)=\left(\frac{a}{p_1}\right)^{e_1}\cdots \left(\frac{a}{p_k}\right)^{e_k}.\]
Prove the following properties:
\begin{enumerate}[label=(\alph*)]
    \item If $a\equiv b\mod n$, then $\left(\frac{a}{n}\right)=\left(\frac{b}{n}\right)$.
    \item If $a$ and $b$ are integers, then $\left(\frac{a}{n}\right)\left(\frac{b}{n}\right)=\left(\frac{ab}{n}\right)$.
    \item If $x^2\equiv a\mod n$ has a solution, then $\left(\frac{a}{n}\right)=1$. Provide an example that shows that the converse of this statement isn't always true. 
    \item If $m,n$ are relatively prime odd integers, then \[\left(\frac{m}{n}\right)\left(\frac{n}{m}\right)=(-1)^{\frac{m-1}{2}}(-1)^{\frac{n-1}{2}}.\]
\end{enumerate}
\end{document}