\documentclass[12pt]{article}

% Import preambles and macros for homework
% Essential packages
\usepackage{amsmath, amsfonts, amssymb, amsthm}
\usepackage{mathtools}
\usepackage{enumitem}
\usepackage{graphicx}
\usepackage{wrapfig}
\usepackage{systeme}
\usepackage{caption}
\usepackage{soul}
\usepackage[dvipsnames]{xcolor}
\usepackage{fancyhdr}
\allowdisplaybreaks

% Page layout
\usepackage[
  top=2cm,
  bottom=2cm,
  left=2cm,
  right=2cm,
  headheight=17pt,
  includehead,includefoot,
  heightrounded,
]{geometry}


% pgfornament for title page decorations
\usepackage[object=vectorian]{pgfornament}

% Fancy header/footer setup
\pagestyle{fancy}
\setlength{\headheight}{14.49998pt}
\addtolength{\topmargin}{-2.49998pt}
\renewcommand{\footrulewidth}{0.4pt}
\setlength\parindent{15pt}
% Math notation shortcuts
\newcommand{\R}{\mathbb{R}}
\newcommand{\Q}{\mathbb{Q}}
\newcommand{\Z}{\mathbb{Z}}
\newcommand{\N}{\mathbb{N}}
\newcommand{\C}{\mathbb{C}}
\newcommand{\X}{\mathcal{X}}

% Theorem environments
\newtheorem{mainthm}{Theorem}[section]
\newtheorem{theorem}{Theorem}[section]  
\newtheorem{lemma}[theorem]{Lemma}
\newtheorem{proposition}[theorem]{Proposition}
\newtheorem{corollary}[theorem]{Corollary}
\newtheorem{definition}[theorem]{Definition}
\newtheorem{claim}[theorem]{Claim}

% Calculus
\newcommand{\diff}{\mathop{}\!\mathrm{d}}
\newcommand{\deriv}[2]{\frac{\mathrm{d}#1}{\mathrm{d}#2}}
\newcommand{\pderiv}[2]{\frac{\partial #1}{\partial #2}}

% Linear Algebra
\newcommand{\inner}[2]{\langle #1, #2 \rangle}
\newcommand{\norm}[1]{\| #1 \|}
\newcommand{\tr}{\operatorname{tr}}
\newcommand{\spn}{\operatorname{span}}
\newcommand{\rank}{\operatorname{rank}}
\newcommand{\nullity}{\operatorname{nullity}}

% Logic
\newcommand{\contra}{\Rightarrow\Leftarrow}

% Custom commands for notes
\newcommand{\todo}[1]{\textcolor{red}{[TODO: #1]}}
\newcommand{\important}[1]{\textbf{\textcolor{blue}{#1}}}

%Number Theory
\DeclareMathOperator{\Li}{Li}
\newcommand{\floor}[1]{\left\lfloor #1 \right\rfloor}
\newcommand{\fract}[1]{\left\{ #1 \right\}}




\newcommand{\maketitlepage}{
    \begin{titlepage}
        \centering
        \vspace*{2.0cm}
        \pgfornament{84}\\
        {\LARGE \textsc{\coursename}\par}
        \vspace{0.5cm}
        {\large\coursecode\par}
        \vspace{0.5cm}
        {\large\instructor\par}
        \vspace{1.5cm}
        {\huge\bfseries\assignment\par}
        \vspace{1cm}
        {\LARGE\itshape\author\par}
        \vspace{2cm}
        {\large\bfseries Due Date:\par}
        \vspace{0.5cm}
        {\Large \duedate}\\
        \pgfornament{84}
    \end{titlepage}
}
% =============================================
% HOMEWORK CONFIGURATION - EDIT THESE VALUES!
% =============================================

% Your personal info
\renewcommand{\author}{Deepak Jassal}
\newcommand{\authorlast}{Jassal}

% Course info
\newcommand{\coursename}{Course Name}
\newcommand{\coursecode}{Course code}
\newcommand{\instructor}{Instructor}

% Assignment-specific info (CHANGE THESE FOR EACH HOMEWORK)
\newcommand{\assignment}{Assignment }
\newcommand{\duedate}{Month Day\textsuperscript{th}, 20XX}

% Header configuration
\fancyhead[l]{\assignment}
\fancyhead[c]{\coursecode}
\fancyhead[r]{\monthyear}
\fancyfoot[c]{\authorlast{ }\thepage}

\renewcommand{\author}{Deepak Jassal}
\renewcommand{\authorlast}{Jassal}
\renewcommand{\coursename}{Survey of Algebra}
\renewcommand{\coursecode}{MATH 320}
\renewcommand{\assignment}{Assignment 6}
\renewcommand{\instructor}{Dr. Alia Hamieh}
\renewcommand{\duedate}{November 20\textsuperscript{th}, 2025}


\begin{document}
\begin{titlepage}
	\centering
	\vspace*{2.0cm}	
	\pgfornament{84}\\
	{\LARGE \textsc{\coursename}\par}
	\vspace{0.5cm}
	{\large\coursecode\par}
    \vspace{0.5cm}
    {\large\instructor\par}
	\vspace{1.5cm}
	{\huge\bfseries\assignment\par}
	\vspace{1cm}
	{\LARGE\itshape\author\par}
    \vspace{2cm}
	{\large\bfseries Due Date:\par}
	\vspace{0.5cm}
	{\Large \duedate}\\
	\pgfornament{84}
\end{titlepage}
\stepcounter{section}
\section*{Question 1 [2 marks]}
What is the order of the element $14 + \langle8\rangle$ in the factor group $\Z_{24}/ \langle8\rangle$?\\
\[
	|\langle 8 \rangle|=3,
\]
\[
	|\Z_{24}/ \langle8\rangle|=8,
\]
\[
	\langle8\rangle=\{0,8,16\}.
\]
It can be seen that 
\[
	8+\langle8\rangle=0+\langle8\rangle,
\]
that is in the factor group $\Z_{24}/\langle8\rangle$ $8\equiv0$. This tells us that $14+\langle8\rangle=6+\langle8\rangle$, and that $|14+\langle8\rangle|=|6+\langle8\rangle|$. We know that for any two cosets in a factor group that
\[
	(a+H)+(b+H)=(a+b)+H
\]
where $H$ is the identity coset. If $a=b$ this can be generalized to
\[
	\underbrace{(a+H)+\cdots+(a+H)}_{n\text{ times}}=n(a)+H.
\]
(A proper proof of this would be given through the use of the principle of Mathematical induction)\\
Therefore to find the order of $6+\langle 8\rangle$ we need to find the smallest $n$ such that $n6+\langle 8\rangle=\langle 8\rangle$. We can see that if $n=4$, then 
\[
	4(6)+\langle 8\rangle =24+\langle 8\rangle =\langle 8\rangle.  
\]
Now we need to see if $4$ is the smallest number with this property. The only numbers that divide both 4 and 8 are 1,2,4. We can see that $n=1$ does not work, and neither does $n=2$. So,
\[
	|14+\langle8\rangle|=4.
\]

\stepcounter{section}
\section*{Question 2 [2 marks]}
Explain why the correspondence $x \mapsto 3x$ from $\Z_{12}$ to $\Z_{10}$ is not a homomorphism.\\
Let $\varphi:\Z_{12}\mapsto \Z_{10}$ defined as above. Then if $\varphi$ is a homomorphism we would have
\[
	\varphi(a+b)=\varphi(a)+\varphi(b)\quad \forall a,b\in \Z_{12}.
\]
Take $a=6,b=6$, then
\[
	\varphi(6+6)=\varphi(0)=0
\]
\[
	\varphi(6)+\varphi(6)=6.
\]
Thus, $\varphi$ is not a homomorphism.

\stepcounter{section}
\section*{Question 3 [2 marks]}
Let $H$ be a normal subgroup of a finite group $G$, and let $a$ belong to $G$. If the element $aH$ has order 3 in the group $G/H$ and $|H| = 10$, what are the possibilities for the order of $a$ in $G$?\\
Let $|a|=n$. Then $3\vert n$. Let $h=a^3\in H$, then $h^{10}=a^{3(10)}=e$. This shows that $n\vert30$. Putting these two together we see that the possible values of $n$ are 1,3,6,15,30.

\stepcounter{section}
\section*{Question 4 [3 marks]}
Prove that a factor group of a cyclic group is cyclic.
\begin{proof}
	Let $G=\langle g\rangle G=\langle g\rangle$ be cyclic and $N\unlhd G.$. Then every element $n\in G/N$ is of the form $n=xN$, For some $x\in G$. Since $G$ is cyclic we have $x=g^n$, for some $n\in\Z$. That is for all $xN$, we have $xN=g^kN=(gN)^K$ for some $k\in\Z$. Thus every coset in $G/N$ is a power of $gN$, so $G/N=\langle g\rangle$, hence $G/N$ is cyclic.
\end{proof}

\stepcounter{section}
\section*{Question 5 [3 marks]}
Let $H$ and $K$ be normal subgroups of a group $G$. Prove that $HK$ is also a normal subgroup of $G$.
\begin{proof}
	First we must show that $HK$ is a subgroup of $G$. Let $a,b\in HK$, then for each of these elements in either in $H$, $K$ or both (becuase $HK=\left\{ hk: h\in H, k\in K\right\}$). So $ab\in HK$. We also have $a^{-1}\in HK$ becuase $a$ is in either $H$, $K$ or both, and each of those is a subgroup. So $HK\leq G$.
	We know that a subgroup is normal if and only if $xJx^{-1}\subset J$ for all $x\in G.$ Let $x$ be any element of the group $G$. Then since $H$ and $K$ are nomal in $G$ we have
	\[
		xHKx^{-1}=(xHx^{-1})(xKx^{-1})=HK.
	\]
	Therefore, $HK\unlhd G.$
\end{proof}

\stepcounter{section}
\section*{Question 6 [3 marks]}
Let $G$ be a group acting on a set $X$. Suppose that the stabilizer $G_x$ of a certain point $x \in X$ is a proper normal subgroup of $G$. Prove that every element of $G_x$ fixes every element $y \in G.x$.
\begin{proof}
	
\end{proof}


\stepcounter{section}
\section*{Question 7 [5 marks]}
In what follows, you prove the third isomorphism theorem. Let $M, N$ be normal sub-groups of a group $G$ such that $N$ is a subgroup of $M$.
\begin{enumerate}[label=(\alph*)]
	\item Show that $N$ is a normal subgroup of $M$.
	\item Show that $(G/N )/(M/N ) \cong G/M $.
\end{enumerate}


\end{document}