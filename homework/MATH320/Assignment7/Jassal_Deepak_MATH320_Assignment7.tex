\documentclass[12pt]{article}

% Import preambles and macros for homework
% Essential packages
\usepackage{amsmath, amsfonts, amssymb, amsthm}
\usepackage{mathtools}
\usepackage{enumitem}
\usepackage{graphicx}
\usepackage{wrapfig}
\usepackage{systeme}
\usepackage{caption}
\usepackage{soul}
\usepackage[dvipsnames]{xcolor}
\usepackage{fancyhdr}
\allowdisplaybreaks

% Page layout
\usepackage[
  top=2cm,
  bottom=2cm,
  left=2cm,
  right=2cm,
  headheight=17pt,
  includehead,includefoot,
  heightrounded,
]{geometry}


% pgfornament for title page decorations
\usepackage[object=vectorian]{pgfornament}

% Fancy header/footer setup
\pagestyle{fancy}
\setlength{\headheight}{14.49998pt}
\addtolength{\topmargin}{-2.49998pt}
\renewcommand{\footrulewidth}{0.4pt}
\setlength\parindent{15pt}
% Math notation shortcuts
\newcommand{\R}{\mathbb{R}}
\newcommand{\Q}{\mathbb{Q}}
\newcommand{\Z}{\mathbb{Z}}
\newcommand{\N}{\mathbb{N}}
\newcommand{\C}{\mathbb{C}}
\newcommand{\X}{\mathcal{X}}

% Theorem environments
\newtheorem{mainthm}{Theorem}[section]
\newtheorem{theorem}{Theorem}[section]  
\newtheorem{lemma}[theorem]{Lemma}
\newtheorem{proposition}[theorem]{Proposition}
\newtheorem{corollary}[theorem]{Corollary}
\newtheorem{definition}[theorem]{Definition}
\newtheorem{claim}[theorem]{Claim}

% Calculus
\newcommand{\diff}{\mathop{}\!\mathrm{d}}
\newcommand{\deriv}[2]{\frac{\mathrm{d}#1}{\mathrm{d}#2}}
\newcommand{\pderiv}[2]{\frac{\partial #1}{\partial #2}}

% Linear Algebra
\newcommand{\inner}[2]{\langle #1, #2 \rangle}
\newcommand{\norm}[1]{\| #1 \|}
\newcommand{\tr}{\operatorname{tr}}
\newcommand{\spn}{\operatorname{span}}
\newcommand{\rank}{\operatorname{rank}}
\newcommand{\nullity}{\operatorname{nullity}}

% Logic
\newcommand{\contra}{\Rightarrow\Leftarrow}

% Custom commands for notes
\newcommand{\todo}[1]{\textcolor{red}{[TODO: #1]}}
\newcommand{\important}[1]{\textbf{\textcolor{blue}{#1}}}

%Number Theory
\DeclareMathOperator{\Li}{Li}
\newcommand{\floor}[1]{\left\lfloor #1 \right\rfloor}
\newcommand{\fract}[1]{\left\{ #1 \right\}}




\newcommand{\maketitlepage}{
    \begin{titlepage}
        \centering
        \vspace*{2.0cm}
        \pgfornament{84}\\
        {\LARGE \textsc{\coursename}\par}
        \vspace{0.5cm}
        {\large\coursecode\par}
        \vspace{0.5cm}
        {\large\instructor\par}
        \vspace{1.5cm}
        {\huge\bfseries\assignment\par}
        \vspace{1cm}
        {\LARGE\itshape\author\par}
        \vspace{2cm}
        {\large\bfseries Due Date:\par}
        \vspace{0.5cm}
        {\Large \duedate}\\
        \pgfornament{84}
    \end{titlepage}
}
% =============================================
% HOMEWORK CONFIGURATION - EDIT THESE VALUES!
% =============================================

% Your personal info
\renewcommand{\author}{Deepak Jassal}
\newcommand{\authorlast}{Jassal}

% Course info
\newcommand{\coursename}{Course Name}
\newcommand{\coursecode}{Course code}
\newcommand{\instructor}{Instructor}

% Assignment-specific info (CHANGE THESE FOR EACH HOMEWORK)
\newcommand{\assignment}{Assignment }
\newcommand{\duedate}{Month Day\textsuperscript{th}, 20XX}

% Header configuration
\fancyhead[l]{\assignment}
\fancyhead[c]{\coursecode}
\fancyhead[r]{\monthyear}
\fancyfoot[c]{\authorlast{ }\thepage}

\renewcommand{\author}{Deepak Jassal}
\renewcommand{\authorlast}{Jassal}
\renewcommand{\coursename}{Survey of Algebra}
\renewcommand{\coursecode}{MATH 320}
\renewcommand{\assignment}{Assignment 7}
\renewcommand{\instructor}{Dr. Alia Hamieh}
\renewcommand{\duedate}{December 4\textsuperscript{th}, 2025}

\usepackage{float}

\begin{document}
\begin{titlepage}
	\centering
	\vspace*{2.0cm}	
	\pgfornament{84}\\
	{\LARGE \textsc{\coursename}\par}
	\vspace{0.5cm}
	{\large\coursecode\par}
    \vspace{0.5cm}
    {\large\instructor\par}
	\vspace{1.5cm}
	{\huge\bfseries\assignment\par}
	\vspace{1cm}
	{\LARGE\itshape\author\par}
    \vspace{2cm}
	{\large\bfseries Due Date:\par}
	\vspace{0.5cm}
	{\Large \duedate}\\
	\pgfornament{84}
\end{titlepage}

\section*{Question 1 [2 marks]}
Suppose that $\varphi$ is an isomorphism from $\mathbb{Z}_3 \oplus \mathbb{Z}_5$ to $\mathbb{Z}_{15}$ and $\varphi(2, 3) = 2$. Find an element in $\mathbb{Z}_3 \oplus \mathbb{Z}_5$ that maps to 1.\\
\textit{Solution.} The order of 2 in $\Z_3$ is 3, and the order of 3 in $\Z_5$ is 5. So $|(2,3)|=\mathrm{lcm}(3,5)=15$. Thus $\Z_3\oplus\Z_5=\langle(2,3)\rangle$. Then since, 
\[
	2^15\equiv 30\equiv1\mod 15 
\]
we have
\[
	2^{15}=\varphi\left((2,3)^15\right)=\varphi((2^{15},3^{15}))=\varphi((1,1)).
\]
\section*{Question 2 [3 marks]}
Prove that there is no homomorphism from $\mathbb{Z}_8 \oplus \mathbb{Z}_2$ onto $\mathbb{Z}_4 \oplus \mathbb{Z}_4$.
\begin{proof}
	Assume for contradicition that $\varphi:\Z_8\oplus\Z_2\mapsto \Z_4\oplus\Z_4$ is an onto homomorphism. Since $\varphi$ is a homomorphism we have that $\Z_8\oplus\Z_2\setminus\ker(\varphi)=H$. Then by the first isomorphism theorem we have that 
	\[
		\psi:\Z_8\oplus\Z_2\setminus\ker(\varphi)\mapsto\Z_4\oplus\Z_4
	\]  
	is an isomorphism. Since $|\Z_8\oplus\Z_2|=16$ and $|\Z_2\oplus\Z_4|=16$ we have that $|\ker(\varphi)|=1$ so the kernel of the homomorphism $\varphi$ is trivial kernel. This means that $\varphi$ is also an isomorphism, but $\Z_8\oplus\Z_2$ has an element of order 8 and $\Z_4\oplus\Z_4$ has an element with order at most 4, this is a contradicition and no such homomorphism or isomosrphism exists.
\end{proof}
\newpage
\section*{Question 3 [3 marks]}
Show that $\mathbb{Z}_6 \oplus \mathbb{Z}_2$ has subgroups of orders 1, 2, 3, 4, 6, and 12. Suppose that $\varphi$ is a homomorphism from a group $G$ onto $\mathbb{Z}_6 \oplus \mathbb{Z}_2$ and that $|\text{Ker}(\varphi)| = 5$. Show that $G$ has subgroups of order 5, 10, 15, 20, 30, and 60.
\begin{proof}
	The following elements in $\Z_6\oplus\Z_2$ have orders
	\begin{figure}[h!]
		\centering
		\begin{tabular}{c|c}
			Element&order\\
			\hline
			(0,0)&1\\
			(0,1)&2\\
			(2,0)&3\\
			(2,1)&6\\
			(1,1)&12\\
		\end{tabular}
	\end{figure}
	These elements generate subgroups of $\Z_6\oplus\Z_2$ with orders equal to their own orders. For a subgroup of order 4 observe that \\$\langle (3,0), (0,1) \rangle=\{(0,0),(3,0),(0,1),(3,1)\}$ is a subgroup of $\Z_6\oplus\Z_2$ and has order 4.\\
	Since $\varphi:G\mapsto \Z_6\oplus\Z_2$ is a homomorphism we have
	\[
		G\setminus\ker(\varphi)\cong\Z_6\oplus\Z_2.
	\]
	We know that for finite groups (which these are), 
	\[
	\frac{|G|}{|\ker(\varphi)|}=|\Z_6\oplus\Z_2|
	\]
	\[
	|G|=|\Z_6\oplus\Z_2|\ker(\varphi)=12\times5=60.
	\]
	For any subgroup $H$ os $G$ we know that $|H|=5,10,15,20,30,60$. We also know that $\Z_6\oplus\Z_2$ has subgroups of orders 1,2,3,4,6,12. Since $G\setminus\ker(\varphi)\cong\Z_6\oplus\Z_2$ we know that for some $\bar{H}\leq \Z_6\Z_2$, $\varphi(\bar{H})\vert|H|$ where $H\leq G$. This means that $|H|=k|\varphi(\bar{H})|$. Since we know the possible orders of $\varphi(\bar{H})$ we have the possible corresponding values of $|H|$ as
	\begin{figure}[H]
		\centering
		\begin{tabular}{c|c}
			$|\bar{H}|$&$|H|$\\
			\hline
			1&5\\
			2&10\\
			3&15\\
			6&30\\
			12&60\\
		\end{tabular}
	\end{figure}	 
	Thus, $G$ has subgroups of order 5,10,15,20,30,60.
\end{proof}


\section*{Question 4 [3 marks]}
Consider the set $G = \{1, 7, 17, 23, 49, 55, 65, 71\}$ under multiplication modulo 96. This is a group (no need to prove that). Determine the isomorphism class of $G$.\\
\textit{Solution.} Since $|G|=8=2^3$, the isomorphism classes of $G$ are
\[
	\Z_2\oplus\Z_2\oplus\Z_2
\]
\[
	\Z_4\oplus\Z_2
\]
\[
	\Z_8.
\]
\section*{Question 5 [3 marks]}
Determine the possible isomorphism classes of an Abelian group of order 792 that contains at least two elements of order 2 and an element of order 9.\\
\textit{Solution.} $792=2^3\times3^2\times11$. For there to be two elements of order 2 we need the isomorphism classes coming from $\Z_2^3$ to have at least two elements of order two, those are $\Z_4\oplus\Z_2$ or $\Z_2\oplus\Z_2\oplus\Z_2$. For the element of order 9 we need the isomorphism classes coming from $\Z_9$ to have an element of order 9, the only option for this is $\Z_9$. So the possible isomorphism classes of an Abelian group with order 792 are 
\[
	\Z_4\oplus\Z_2\oplus\Z_9\oplus\Z_{11}
\]
and 
\[
	\Z_2\oplus\Z_2\oplus\Z_2\oplus\Z_9\oplus\Z_{11}.
\]
These groups are Abelian by the \textit{Fundamental Theorem of Finite Abelian Groups.}
\section*{Question 6 [3 marks]}
Let $p$ and $q$ be distinct odd prime numbers. How many non-isomorphic Abelian groups of order $4p^3q^4$ are there?
\textit{Solution.} We can break this problem down to finding the number of non-isomorphic Abelian groups of orders $4=2^2$, $p^3$ and $q^4$. For groups of order 4 we have 2, groups of order $p^3$ we have 3, and groups of order $q^4$ we have 5. This gives a total of $2\times3\times5=30$ groups. The values were calculated by looking at the partition of 2, 3, and 5. 
\newpage
\section*{Question 7 [3 marks]}
Prove that a group of order 351 has a normal Sylow $p$-subgroup for some prime $p$ dividing its order.
\begin{proof}
	$351=3^3\times13$. We know that if the number of Sylow $p$-subgroups of a group $G$ is 1 then that subgroup is normal. If the numer of Sylow 13-subgroups of $G$ are one, then we are done. Let $n_{13}>1$ be the number of Sylow 13-subgroups of $G$. Then $n_{13}\equiv1 \mod 13$, and $n_{13}\vert 27$. The divisors of 27 are 1,3,9,27. The only divisors which satify the congruence relationship are 1 or 27. Since $n_{13}>1$ we have $n_{13}=27$. Each of these subgroups have 13 elements in total, and 12 of these elements have order 13. Each of subgroups are disjoint so each elements are disjoint. This means that a total of $27\times12=324$ elements are in the Sylow 13-subgroups (the identity element needs to be in each subgroup). Then 351-324=27=$3^3$. These remaining 27 elements make a subgroup by the first Sylow theorem. So either there is one Sylow 13-subgroup or one Sylow 3-subgroup in this group, in which ever case has one subgroup that subgroup is normal. 
\end{proof}
\section*{Question 8 [5 marks]}
Consider $n \in \mathbb{N}$ and $p$ a prime number. Let $G = GL_n(\mathbb{Z}_p)$, and let $P$ be the subgroup of strictly upper triangular matrices in $G$ (strictly means 1's along the diagonal).

\begin{enumerate}[label=(\alph*)]
    \item Determine the order of $G$.
    \item Show that $P$ is a Sylow $p$-subgroup of $G$.
\end{enumerate}

\end{document}