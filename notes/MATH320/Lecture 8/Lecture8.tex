\documentclass[12pt]{article}
\usepackage{color,soul}
% Import preambles and macros for notes
% Essential packages for notes
\usepackage{amsmath, amssymb, amsthm}
\usepackage{mathtools}  % for \coloneqq, etc.
\usepackage{geometry}   % Better page margins
\usepackage{parskip}    % Better paragraph spacing
\usepackage{microtype}  % Better typography
\usepackage{enumitem}   % Customize lists
\usepackage{hyperref}   % Clickable links
\usepackage{booktabs}   % Better tables
\usepackage{tcolorbox}  % For colored boxes/theorems

% Page layout for notes
\geometry{a4paper, margin=1in}
\setlength{\parskip}{0.8em}

% Theorem environments with shared numbering
\newtheorem{theorem}{Theorem}[subsection]  % Number within sections: 2.3.1, 2.3.2, etc.

% All other environments share the same counter as theorem
\newtheorem{lemma}[theorem]{Lemma}
\newtheorem{proposition}[theorem]{Proposition}
\newtheorem{corollary}[theorem]{Corollary}
\newtheorem{definition}[theorem]{Definition}
\newtheorem{example}[theorem]{Example}
\newtheorem{remark}[theorem]{Remark}
\newtheorem{claim}[theorem]{Claim}

% Custom colors for notes
\usepackage{xcolor}
\definecolor{note-blue}{RGB}{220, 230, 255}
\definecolor{theorem-green}{RGB}{220, 255, 220}
% Math notation shortcuts for notes
\newcommand{\R}{\mathbb{R}}
\newcommand{\C}{\mathbb{C}}
\newcommand{\Q}{\mathbb{Q}}
\newcommand{\Z}{\mathbb{Z}}
\newcommand{\N}{\mathbb{N}}

% Calculus
\newcommand{\diff}{\mathop{}\!\mathrm{d}}
\newcommand{\deriv}[2]{\frac{\mathrm{d}#1}{\mathrm{d}#2}}
\newcommand{\pderiv}[2]{\frac{\partial #1}{\partial #2}}

% Linear Algebra
\newcommand{\inner}[2]{\langle #1, #2 \rangle}
\newcommand{\norm}[1]{\| #1 \|}
\newcommand{\tr}{\operatorname{tr}}
\newcommand{\spn}{\operatorname{span}}
\newcommand{\rank}{\operatorname{rank}}
\newcommand{\nullity}{\operatorname{nullity}}

% Logic
\newcommand{\contra}{\Rightarrow\Leftarrow}

% Custom commands for notes
\newcommand{\todo}[1]{\textcolor{red}{[TODO: #1]}}
\newcommand{\important}[1]{\textbf{\textcolor{blue}{#1}}}
\input{../../../preambles/theorem-system-section.tex}

\title{MATH 320 Lecture 8}
\author{Deepak Jassal}
\date{October 2\textsuperscript{nd}, 2025}

\begin{document}
\maketitle

\newpage 

\setcounter{section}{0}    % This is to set the section across notes
\setcounter{subsection}{0} % This is to set the subsection across notes


\section{Matrix Groups}
In this section we will be looking into non-Abelian groups, namely Matrix Groups. The set
\[
    GL_n(F)=\{A\in M_n(F):\det{A}\neq0\}
\]
of all invertible $n\times n$ matrices with entries in a field $F$ forms a group under matrix multiplication.\\
A field is the smallest algebraic structure in which we can perform the usual arithmetic operations, including division by non-zero elements. In particular, every non-zero element in a field has a multiplicative inverse.
\defnr{Field}{field}{A \textit{field} is a set $F$ with two binary operations $+$ and $\cdot$ such that
\begin{enumerate}[label=(\roman*)]
    \item $(F, +)$ is an abelian group with identity element 0;
    \item $F \times := F \setminus {0}$ is an abelian group under · with identity element 1;
    \item Distributivity holds: for all $a, b, c \in F$ , one has $a(b + c) = ab + ac$.
\end{enumerate}
}
\ex{Common examples of fields include $\Q,\R,\C$, and finite fields $\Z/p\Z$ where $p$ is prime.}
\rmkb{
    \begin{enumerate}
        \item In any field $F$, one has $a\cdot0=0$ for all $a\in F$.
        \item We write $F^\times$ for the multiplicative group of nonzero elements of a field $F$.
    \end{enumerate}
}
\thm{}{If $F$ is a finite field, then $|F|=p^n$, for some prime $p$ and integer $n\geq 1$.}

\prop{Let $F$ be a field. The set $GL_n(F)$ of all invertible $n\times n$ matrices  with entries in $F$ forms a group under matrix multiplication. It is called the general linear group of degree $n$ over $F$.}


\hl{The following material is covered in Chapter 1 Section 6 of the Dummit and Foote textbook}

\section{Group Homomorphisms and Isomorphisms}

\defn{Group Homomorphisms}{ Let $(G,\ast)$ and $(H,\circ)$ be groups. A map $\varphi: G\rightarrow H$ is a \textit{homomorphisms} is 
\[
    \varphi(a\ast b)=\varphi(a)\circ\varphi(b)\quad \forall a,b\in G.
\]
}

\defn{Group Isomorphisms}{A map $\varphi: G\rightarrow H$ is an \textit{isomorphism} if it is a bijective homomorphism. In this case we say that $G$ and $H$ are \textit{isomorphic} and write $G\cong H$. More precisely, an isomorphism $\varphi$ from a group $(G,\ast)$ to a group $(H,\circ)$ is a \textbf{one-to-one} mapping from $G$ \textbf{onto} $H$ \textbf{that preserves the group operation}. That isomorphism
\[
    \varphi(a\ast b)=\varphi(a)\circ\varphi(b).
\]
}

From now on, we shall write $\varphi(ab) = \varphi(a)\varphi(b)$ and it will be understood that for ab we are using the operation of $G$, while for $\varphi(a)\varphi(b)$ we are using the operation of $H$.\\
How do we prove that two groups are isomorphic?
\begin{enumerate}
    \item Find a candidate mapping for the isomorphism. That is a function $\varphi:G\rightarrow H$.
    \item Prove that $\varphi$ is one-to-one.
    \item Prove that $\varphi$ is onto.
    \item Prove that $\varphi$ is operation preserving.
\end{enumerate}

\ex{(Automorphisms). Every group $G$ is isomorphic to itself via the identity map $1_G$. An isomorphism $G \rightarrow G$ is called an \textit{automorphism} of $G$.}

\ex{[The exponention isomorphism of additive and multiplicative groups of $\R$]\\ Consider exp : $(\R,+)\rightarrow (\R,\times).$ then
\[
    \text{exp}(a+b)=\text{exp}(a)\text{exp}(b)\quad \text{(homomorphism).}
\]
It is injective because $\text{exp}(a)=\text{exp}(b)$ implies $a=b$, and surjective onto $\R_{>0}$ since for any $y>0$ there exists $a=\log y$ with $\text{exp}(a)=y$. Thus $(\R,+)\cong(\R,\times)$ 
}
\ex{
    Later we will prove that any non-abelian group of order 6 is isomorphic to the symmetric group S3 giving a first illustration of group classification by structure rather than by presentation. Hence, $D6 \cong S_3$ and $GL_2(\Z/2\Z) \cong S_3$.
}
\end{document}