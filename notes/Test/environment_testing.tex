\documentclass[12pt]{article}

% Import preambles and macros for notes
% Essential packages for notes
\usepackage{amsmath, amssymb, amsthm}
\usepackage{mathtools}  % for \coloneqq, etc.
\usepackage{geometry}   % Better page margins
\usepackage{parskip}    % Better paragraph spacing
\usepackage{microtype}  % Better typography
\usepackage{enumitem}   % Customize lists
\usepackage{hyperref}   % Clickable links
\usepackage{booktabs}   % Better tables
\usepackage{tcolorbox}  % For colored boxes/theorems

% Page layout for notes
\geometry{a4paper, margin=1in}
\setlength{\parskip}{0.8em}

% Theorem environments with shared numbering
\newtheorem{theorem}{Theorem}[subsection]  % Number within sections: 2.3.1, 2.3.2, etc.

% All other environments share the same counter as theorem
\newtheorem{lemma}[theorem]{Lemma}
\newtheorem{proposition}[theorem]{Proposition}
\newtheorem{corollary}[theorem]{Corollary}
\newtheorem{definition}[theorem]{Definition}
\newtheorem{example}[theorem]{Example}
\newtheorem{remark}[theorem]{Remark}
\newtheorem{claim}[theorem]{Claim}

% Custom colors for notes
\usepackage{xcolor}
\definecolor{note-blue}{RGB}{220, 230, 255}
\definecolor{theorem-green}{RGB}{220, 255, 220}
% Math notation shortcuts for notes
\newcommand{\R}{\mathbb{R}}
\newcommand{\C}{\mathbb{C}}
\newcommand{\Q}{\mathbb{Q}}
\newcommand{\Z}{\mathbb{Z}}
\newcommand{\N}{\mathbb{N}}

% Calculus
\newcommand{\diff}{\mathop{}\!\mathrm{d}}
\newcommand{\deriv}[2]{\frac{\mathrm{d}#1}{\mathrm{d}#2}}
\newcommand{\pderiv}[2]{\frac{\partial #1}{\partial #2}}

% Linear Algebra
\newcommand{\inner}[2]{\langle #1, #2 \rangle}
\newcommand{\norm}[1]{\| #1 \|}
\newcommand{\tr}{\operatorname{tr}}
\newcommand{\spn}{\operatorname{span}}
\newcommand{\rank}{\operatorname{rank}}
\newcommand{\nullity}{\operatorname{nullity}}

% Logic
\newcommand{\contra}{\Rightarrow\Leftarrow}

% Custom commands for notes
\newcommand{\todo}[1]{\textcolor{red}{[TODO: #1]}}
\newcommand{\important}[1]{\textbf{\textcolor{blue}{#1}}}

\title{Theorem System Test}
\author{Deepak Jassal}

\begin{document}
\maketitle

\tableofcontents
\newpage 
\setcounter{section}{0}
\setcounter{subsection}{0}

\section{Testing All Theorem Environments}

\subsection{Basic Environments}

% Definition with title
\defn{Group}{A \textbf{group} is a set $G$ with a binary operation $*$ such that:
\begin{enumerate}
    \item \textbf{Associativity}: $(a * b) * c = a * (b * c)$ for all $a,b,c \in G$
    \item \textbf{Identity}: There exists $e \in G$ such that $e * a = a * e = a$ for all $a \in G$
    \item \textbf{Inverses}: For each $a \in G$, there exists $a^{-1} \in G$ such that $a * a^{-1} = a^{-1} * a = e$
\end{enumerate}
}

% Theorem with title
\thm{Lagrange's Theorem}{If $G$ is a finite group and $H$ is a subgroup of $G$, then the order of $H$ divides the order of $G$. That is, $|H|$ divides $|G|$.}

% Lemma with title
\lem{Subgroup Test}{A subset $H$ of a group $G$ is a subgroup if and only if:
\begin{enumerate}
    \item $H$ is non-empty
    \item For all $a, b \in H$, $ab \in H$
    \item For all $a \in H$, $a^{-1} \in H$
\end{enumerate}
}

% Corollary without title
\cor{Every group of prime order is cyclic.}

% Proposition without title
\prop{The intersection of two subgroups is also a subgroup.}

% Claim with title
\clm{Order of Element}{In a finite group, the order of any element divides the order of the group.}

% Fact without title
\fact{The symmetric group $S_n$ has order $n!$.}

\subsection{Environments with Proofs}

% Lemma with proof
\lemp{Cyclic Subgroups}{If $a$ is an element of a group $G$, then the set $\langle a \rangle = \{a^n : n \in \mathbb{Z}\}$ is a subgroup of $G$.}{
Let $x, y \in \langle a \rangle$. Then $x = a^m$ and $y = a^n$ for some $m, n \in \mathbb{Z}$. Then $xy = a^m a^n = a^{m+n} \in \langle a \rangle$. Also, $x^{-1} = a^{-m} \in \langle a \rangle$. Therefore, $\langle a \rangle$ is a subgroup.
}

% Corollary with proof
\corp{If $G$ is a cyclic group of order $n$, then $G$ has exactly $\phi(n)$ generators, where $\phi$ is Euler's totient function.}{
Let $G = \langle a \rangle$ be cyclic of order $n$. Then $a^k$ is a generator if and only if $\gcd(k, n) = 1$. There are exactly $\phi(n)$ such integers $k$ with $1 \leq k \leq n$.
}

% Proposition with proof
\propp{The center of a group $Z(G) = \{g \in G : gh = hg \text{ for all } h \in G\}$ is a subgroup of $G$.}{
Let $x, y \in Z(G)$. For any $h \in G$, we have $(xy)h = x(yh) = x(hy) = (xh)y = (hx)y = h(xy)$, so $xy \in Z(G)$. Also, $x^{-1}h = hx^{-1}$ for all $h \in G$, so $x^{-1} \in Z(G)$.
}

% Claim with proof
\clmp{Abelian Center}{If $G$ is a group, then $Z(G)$ is abelian.}{
Let $x, y \in Z(G)$. Since $x$ commutes with all elements of $G$, it commutes with $y$ in particular. Therefore $xy = yx$, so $Z(G)$ is abelian.
}
\subsection{Regular Proof Environment}

% Regular proof
\pf{Let's prove that every subgroup of a cyclic group is cyclic. Suppose $G = \langle a \rangle$ is cyclic and $H$ is a subgroup of $G$. If $H = \{e\}$, then $H$ is cyclic. Otherwise, let $m$ be the smallest positive integer such that $a^m \in H$. We claim $H = \langle a^m \rangle$. For any $a^n \in H$, by the division algorithm, $n = mq + r$ with $0 \leq r < m$. Then $a^r = a^{n - mq} = a^n (a^m)^{-q} \in H$, so by minimality of $m$, we must have $r = 0$. Thus $a^n = (a^m)^q \in \langle a^m \rangle$.}

\subsection{Examples and Remarks}

% Example
\ex{Consider the group $\mathbb{Z}_6 = \{0,1,2,3,4,5\}$ under addition modulo 6. The subgroups are: $\{0\}$, $\{0,2,4\}$, $\{0,3\}$, and $\mathbb{Z}_6$ itself. Notice that the orders are 1, 3, 2, and 6, which all divide 6.}

% Another example
\ex{The symmetric group $S_3$ has order 6. Its subgroups have orders 1, 2, 3, and 6, which all divide 6. However, $S_3$ is not abelian.}

% Inline remark
This shows that Lagrange's Theorem has a converse that is \rmk{not true} in general.

% Block remark
\rmkb{While Lagrange's Theorem tells us about possible subgroup orders, it doesn't guarantee that subgroups of those orders actually exist. This leads to the study of Sylow theorems.}

\subsection{Referenced Theorems}

% Theorem with reference
\thmr{Cauchy's Theorem}{cauchy}{If $G$ is a finite group and $p$ is a prime dividing $|G|$, then $G$ has an element of order $p$.}
\pf[Proof for Theorem \ref{thm:cauchy}]{
    Example text
}
% Definition with reference  
\defnr{Simple Group}{simple}{A group $G$ is called \textbf{simple} if it has no non-trivial proper normal subgroups.}

We can reference these later: Theorem \ref{thm:cauchy} and Definition \ref{defn:simple}.

\subsection{More Examples}

% Fact with example
\fact{The alternating group $A_n$ is simple for $n \geq 5$.}

% Another claim
\clm{Index Formula}{If $H$ is a subgroup of $G$, then $[G:H] = |G|/|H|$.}

% Final example showing math
\ex{Let $G = \mathbb{Z}_{12}$. The possible subgroup orders are divisors of 12: 1, 2, 3, 4, 6, 12. For example:
\begin{itemize}
    \item Order 2: $\langle 6 \rangle = \{0, 6\}$
    \item Order 3: $\langle 4 \rangle = \{0, 4, 8\}$ 
    \item Order 4: $\langle 3 \rangle = \{0, 3, 6, 9\}$
\end{itemize}
}

\end{document}